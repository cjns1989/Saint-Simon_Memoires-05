\PassOptionsToPackage{unicode=true}{hyperref} % options for packages loaded elsewhere
\PassOptionsToPackage{hyphens}{url}
%
\documentclass[oneside,10pt,french,]{extbook} % cjns1989 - 27112019 - added the oneside option: so that the text jumps left & right when reading on a tablet/ereader
\usepackage{lmodern}
\usepackage{amssymb,amsmath}
\usepackage{ifxetex,ifluatex}
\usepackage{fixltx2e} % provides \textsubscript
\ifnum 0\ifxetex 1\fi\ifluatex 1\fi=0 % if pdftex
  \usepackage[T1]{fontenc}
  \usepackage[utf8]{inputenc}
  \usepackage{textcomp} % provides euro and other symbols
\else % if luatex or xelatex
  \usepackage{unicode-math}
  \defaultfontfeatures{Ligatures=TeX,Scale=MatchLowercase}
%   \setmainfont[]{EBGaramond-Regular}
    \setmainfont[Numbers={OldStyle,Proportional}]{EBGaramond-Regular}      % cjns1989 - 20191129 - old style numbers 
\fi
% use upquote if available, for straight quotes in verbatim environments
\IfFileExists{upquote.sty}{\usepackage{upquote}}{}
% use microtype if available
\IfFileExists{microtype.sty}{%
\usepackage[]{microtype}
\UseMicrotypeSet[protrusion]{basicmath} % disable protrusion for tt fonts
}{}
\usepackage{hyperref}
\hypersetup{
            pdftitle={SAINT-SIMON},
            pdfauthor={Mémoires V},
            pdfborder={0 0 0},
            breaklinks=true}
\urlstyle{same}  % don't use monospace font for urls
\usepackage[papersize={4.80 in, 6.40  in},left=.5 in,right=.5 in]{geometry}
\setlength{\emergencystretch}{3em}  % prevent overfull lines
\providecommand{\tightlist}{%
  \setlength{\itemsep}{0pt}\setlength{\parskip}{0pt}}
\setcounter{secnumdepth}{0}

% set default figure placement to htbp
\makeatletter
\def\fps@figure{htbp}
\makeatother

\usepackage{ragged2e}
\usepackage{epigraph}
\renewcommand{\textflush}{flushepinormal}

\usepackage{indentfirst}
\usepackage{relsize}

\usepackage{fancyhdr}
\pagestyle{fancy}
\fancyhf{}
\fancyhead[R]{\thepage}
\renewcommand{\headrulewidth}{0pt}
\usepackage{quoting}
\usepackage{ragged2e}

\newlength\mylen
\settowidth\mylen{...................}

\usepackage{stackengine}
\usepackage{graphicx}
\def\asterism{\par\vspace{1em}{\centering\scalebox{.9}{%
  \stackon[-0.6pt]{\bfseries*~*}{\bfseries*}}\par}\vspace{.8em}\par}

\usepackage{titlesec}
\titleformat{\chapter}[display]
  {\normalfont\bfseries\filcenter}{}{0pt}{\Large}
\titleformat{\section}[display]
  {\normalfont\bfseries\filcenter}{}{0pt}{\Large}
\titleformat{\subsection}[display]
  {\normalfont\bfseries\filcenter}{}{0pt}{\Large}

\setcounter{secnumdepth}{1}
\ifnum 0\ifxetex 1\fi\ifluatex 1\fi=0 % if pdftex
  \usepackage[shorthands=off,main=french]{babel}
\else
  % load polyglossia as late as possible as it *could* call bidi if RTL lang (e.g. Hebrew or Arabic)
%   \usepackage{polyglossia}
%   \setmainlanguage[]{french}
%   \usepackage[french]{babel} % cjns1989 - 1.43 version of polyglossia on this system does not allow disabling the autospacing feature
\fi

\title{SAINT-SIMON}
\author{Mémoires V}
\date{}

\begin{document}
\maketitle

\hypertarget{chapitre-premier.}{%
\chapter{CHAPITRE PREMIER.}\label{chapitre-premier.}}

1705

~

{\textsc{Mariage du comte d'Harcourt, et ses suites, avec
M\textsuperscript{lle} de Montjeu\,; son extraction.}} {\textsc{- Gêne
de la confession dans la famille royale.}} {\textsc{- P. de La Rue
confesseur de M\textsuperscript{me} la duchesse de Bourgogne.}}
{\textsc{- Pontchartrain se raccommode avec le maréchal de Cœuvres, et
demeure brouillé avec d'O.}} {\textsc{- Villeroy, Villars et Marsin
généraux des armées de Flandre, de la Moselle et d'Alsace.}} {\textsc{-
Laparat envoyé à Verue.}} {\textsc{- Communication de Verue avec
Crescentin coupée.}} {\textsc{- Verue rendu à discrétion.}} {\textsc{-
Prince Eugène en Italie.}} {\textsc{- Siège de Turin projeté et
publié.}} {\textsc{- Princesse des Ursins tentée de demeurer en
France.}} {\textsc{- Se résout enfin de retourner en Espagne.}}
{\textsc{- Conduite, audace et succès avortés de Maulevrier, rappelé en
France, où il arrive.}} {\textsc{- Gibraltar secouru\,; ce siège levé.}}
{\textsc{- Renault, son caractère, sa fortune.}} {\textsc{- Rochefort,
comment devenu port.}} {\textsc{- Progrès du Ragotzi.}} {\textsc{-
Princesse de Condé.}} {\textsc{- Rabutin et sa fortune en Allemagne.}}
{\textsc{- Mort de l'empereur Léopold, etc.}} {\textsc{- Deuil tardif et
abrégé pour l'empereur.}} {\textsc{- Duretés en Bavière\,; l'électrice à
Venise.}} {\textsc{- Laparat prend la Mirandole.}} {\textsc{-
Vaubecourt, lieutenant général, tué à une échauffourée en Italie\,; sa
femme\,; fatuité du maréchal de Villeroy.}}

~

Il se fit vers ces temps-ci un mariage qui causa bien du murmure dans la
maison de Lorraine. La princesse d'Harcourt avait perdu un fils en
Italie, un autre depuis deux mois dans l'empire, qui s'en allait à
Vienne servir l'empereur, dont elle fut quitte pour faire la pleureuse à
M\textsuperscript{me} de Maintenon\,; point de filles, il ne lui restait
qu'un fils qui était l'aîné. Plusieurs coups de tête reçus par accident
lui avaient fait essuyer trois ou quatre trépans, et ces trépans
l'avaient rendu fort sourd. Elle ne l'aimait point, et tant qu'elle
avait eu d'autres enfants, elle l'avait forcé tout dévotement au petit
collet, et en voulait faire un riche seigneur dans l'Église\,; elle
avait même commencé. Sa répugnance prit des forces se voyant devenu
unique\,; elle songea donc à le marier, mais son mari ni elle ne
voulaient rien donner. Elle chercha vainement\,; enfin elle se rabattit
à ce qu'elle trouva sous sa main. Elle était fort à Sceaux chez
M\textsuperscript{me} du Maine, à qui toute compagnie était bonne,
pourvu qu'on fût abandonné à ses fêtes, à ses nuits blanches\footnote{Les
  nuits \emph{blanches de Sceaux} étaient célèbres.
  M\textsuperscript{lle} de Launay (M\textsuperscript{me} de Staal), qui
  y avait joué un rôle important, en raconte ainsi l'origine dans ses
  \emph{Mémoires\,:} «\,M\textsuperscript{me} la duchesse du Maine, qui
  aimait à veiller, passait souvent toute la nuit à faire différentes
  parties de jeu. L'abbé de Vaubrun, un de ses courtisans les plus
  empressés à lui plaire, imagina qu'il fallait, pendant une des nuits
  destinées à la veille, faire paraître quelqu'un sous la forme de la
  Nuit enveloppée de ses crêpes, qui ferait un remerciement à la
  princesse de la préférence qu'elle lui accordait sur le Jour\,; que la
  déesse aurait un suivant qui chanterait un bel air sur le même sujet.
  L'abbé me confia ce secret, et m'engagea à composer et à prononcer la
  harangue, représentant la divinité nocturne. La surprise fit tout le
  mérite de ce petit divertissement\ldots. L'idée en fut applaudie\,; et
  de là vinrent les fêtes magnifiques données la nuit par différentes
  personnes à M\textsuperscript{me} la duchesse du Maine.\,»}, à ses
comédies et à toutes ses fantaisies. Il s'y était fourré, sur le pied de
petite complaisante, bien honorée d'y être comme que ce fût soufferte,
une M\textsuperscript{lle} de Montjeu, jaune, noire, laide en
perfection, de l'esprit comme un diable, du tempérament comme vingt,
dont elle usa bien dans la suite, et riche en héritière de financier.
Son père s'appelait Castille, comme un chien citron, dont le père qui
était aussi dans les finances, avait pris le nom de Jeannin pour décorer
le sien, en l'y joignant de sa mère, fille du célèbre M. Jeannin, ce
ministre d'État au dehors et au dedans, si connu sous Henri IV.

Le père de notre épousée avait pris le nom de Montjeu d'une belle terre
qu'il avait achetée. Il avait ajouté beaucoup aux richesses de son père
dans le même métier. Il avait la protection de M. Fouquet. Elle lui
valut l'agrément de la charge de greffier de l'ordre, que Novion, depuis
premier président, lui vendit en 1657, un an après l'avoir achetée. La
chute de M. Fouquet l'éreinta. Après que les ennemis du surintendant
eurent perdu l'espérance de pis que la prison perpétuelle, les
financiers de son règne furent recherchés. Celui-ci se trouva fort en
prise, on ne l'épargna pas, mais il avait su se mettre à couvert sur
bien des articles\,; cela même irrita. Le roi lui fit demander la
démission de sa charge de l'ordre\,; et sur ses refus réitérés, il eut
défense d'en porter les marques. Il avait longtemps trempé en prison\,;
on le menaça de l'y rejeter, il tint ferme. On prit un milieu, on
l'exila chez lui en Bourgogne, et Châteauneuf, secrétaire d'État, porta
l'ordre, et fit par commission la charge de greffier\,; enfin le
financier, maté de sa solitude dans son château de Montjeu, où il ne
voyait point de fin, donna sa démission. La charge fut taxée, et
Châteauneuf pourvu en titre. Montjeu eut après cela liberté de voir du
monde, et même de passer les hivers à Autun. Bussy-Rabutin, qui y était
exilé aussi, en parle assez souvent dans ses fades et pédantes lettres.
À la fin, Montjeu eut permission de revenir à Paris, où il mourut en
1688. Sa femme était Dauvet, parente du grand fauconnier.

M\textsuperscript{me} du Maine conclut le mariage et en fit la noce à
Sceaux. M. le duc de Lorraine s'en brouilla avec le prince et la
princesse d'Harcourt, et fit défendre à leur fils et à leur belle-fille
de se présenter jamais devant lui, surtout de ne mettre pas le pied dans
son État. Ce ne fut pas le seul dégoût de la princesse d'Harcourt. Elle
trouva à qui parler. Dans les commencements ce furent merveilles. Le
pied glissa, la contrainte et les exhortations suivirent. L'esprit et la
souplesse remirent tout au premier état\,; mais il arriva un malheur. La
belle-fille écrivit de Paris à sa belle-mère à Versailles avec des
tendresses et des soumissions infinies, et à une de ses amies en même
temps les plaintes d'être soumise à une mégère enragée dont la tyrannie
de belle-mère était insupportable, les caprices et les folies, et avec
qui enfants ni domestiques n'avaient jamais pu durer. Aucun terme, aucun
temps de la vie et de la conduite de la princesse d'Harcourt n'y était
ménagé, et le tout paraphrasé avec beaucoup d'esprit, de sel et de tour,
en personne qui se divertit et se soulage. L'amie reçut la lettre qui
était pour la belle-mère, et celle-ci celle qui était pour l'amie\,; on
s'était mépris au-dessus. Voilà la princesse d'Harcourt transportée de
furie, qui fut assez peu maîtresse d'elle-même pour ne s'en pouvoir
taire, en sorte que l'aventure devint publique à la cour, où elle était
crainte et abhorrée, et où on s'en divertit fort. Elle ne trouva pas
plus de consolations dans la maison de Lorraine, enragée de ce bas
mariage. Elle retomba cruellement sur sa belle-fille, qui fut
étrangement consternée, mais qui au bout de quelques mois reprit ses
esprits, et qui, voyant qu'il n'y avait plus de vraie réconciliation ni
de duperie à espérer, gagna son mari aussi impatient qu'elle de ce joug,
serrèrent tous deux leurs écus, dont ils tâchaient souvent de l'apaiser,
levèrent le masque et se moquèrent d'elle. Le prince d'Harcourt, enfoui
dans son obscurité et ses débauches, toujours absent, ne se souciait ni
d'eux ni de sa femme, et ne s'en mêla point. Ainsi la comtesse
d'Harcourt se mit en liberté et en profita avec peu de mesure.

Depuis que le P. Le Comte avait perdu sa place de confesseur de
M\textsuperscript{me} la duchesse de Bourgogne, pour aller tâcher de se
justifier à Rome de ce qu'il avait écrit sur les affaires des jésuites
de la Chine, avec tous les autres missionnaires, comme je l'ai rapporté
en son temps, elle en avait essayé plusieurs dont elle ne s'était pas
accommodée. Le roi tenait sa famille dans une cruelle gêne pour la
confession. Monseigneur n'a jamais eu un autre confesseur que celui du
roi. Il n'était pas permis à ses enfants d'en prendre ailleurs que ceux
qu'il leur donnait parmi les jésuites\,; et il fallait communier en
public au moins cinq fois par an\,: Pâques, la Pentecôte, l'Assomption,
la Toussaint et Noël, comme il faisait lui-même\,; et
M\textsuperscript{me} la duchesse de Bourgogne n'aurait pas eu bonne
grâce de ne communier pas plus souvent. À son âge, à ses goûts, la chose
avec de la religion était plus qu'embarrassante. Elle avait été fort
bien instruite à Turin par un barnabite son confesseur. Ce barnabite
n'estimait point les jésuites. M. de Savoie les tenait de fort court et
ne les aimait pas. M\textsuperscript{me} la duchesse de Bourgogne avait
sucé cet éloignement avec le lait. C'était donc pour elle un grand
surcroît de peine d'avoir sa conscience entre leurs mains. Enfin, après
plusieurs essais, on lui donna le P. de La Rue, un de leurs plus gros
bonnets, fort connu par ses sermons, par quelques ouvrages, par les
premières places qu'il avait occupées dans sa province, par son poids
parmi les siens, et par beaucoup d'usage du monde, dans lequel il était
assez répandu. Il avait trouvé le moyen de se faire une maison de
campagne à Pontoise, sous le nom des jésuites, dont la manière
d'acquérir et de s'agrandir eût perdu un homme d'une autre robe, et dont
il jouissait avec ses amis fort souvent. Ce confesseur enfin en conserva
la place\,: on verra en son temps ce qui en arriva.

Pontchartrain remis, comme on l'a vu, avec M. le comte de Toulouse par
sa femme, suivait fort à son insu le projet dont j'ai parlé. Le comte,
qui était droit et vrai, et qui comptait, après le pardon qu'il lui
avait accordé et toutes les promesses et les protestations de l'autre,
ne trouver plus de difficultés dans ce qui dépendrait de son ministère,
ne doutait pas de retourner à la mer cette année, où il espérait, étant
au large, faire mieux qu'il n'avait pu l'année précédente parmi tant de
malignes contradictions. Pontchartrain, ravi de l'endormir de cette
espérance, allait au-devant de tout ce qui pouvait l'entretenir. Pour
cela, il fallait travailler quelquefois chez l'amiral avec le maréchal
de Cœuvres, et quelquefois tous trois avec le roi. Le maréchal et
Pontchartrain étaient demeurés fort mal ensemble, et le maréchal était
outré de la compassion que le comte avait eue de M\textsuperscript{me}
de Pontchartrain. Cette situation néanmoins était gênante pour tous les
deux avec la nécessité de ce travail. Le maréchal, abandonné du comte
dans cette haine commune, s'ennuya de rester dans la nasse, et craignit
le secrétaire d'État. Celui-ci avait ses raisons pour n'être pas moins
lassé d'être brouillé avec toute une famille si appuyée\,; celle d'être
plus en état de tromper le comte et le maréchal sur la flotte qu'ils se
proposaient de commander, et qu'il avait bien résolu de leur soustraire,
fut un des plus puissants motifs qui le portèrent à ce frauduleux
accommodement. Cette division importunait le roi\,; de part et d'autre
on lui fit un sacrifice de ce que chacun désirait par des vues fort
différentes. Le duc de Noailles, toujours désireux de se mêler, prit
cette affaire en main, et finalement il conclut le raccommodement et le
consomma entre eux deux dans le cabinet du chancelier. Pour d'O, qui
n'avait point de travail à faire avec Pontchartrain, il vit d'un air
froid et méprisant tous ces manèges, et demeura si réservé sur son
raccommodement avec Pontchartrain, qu'on ne le put pas même entamer.

Vers la mi-mars, les maréchaux de Villeroy, Villars et Marsin
travaillèrent ensemble avec le roi et Chamillart chez
M\textsuperscript{me} de Maintenon, pour concerter les projets de la
campagne\,: le premier pour la Flandre, le second pour la Moselle, oui
on craignait le principal effort des ennemis, le troisième pour
l'Alsace. Villeroy partit quinze jours après pour aller à Bruxelles
donner tous les ordres nécessaires\,; Villars quelque temps après, et
Marsin le 1er mai pour Strasbourg, qui paraissait le côté le plus
retardé.

Vendôme, devant Verue depuis le 14 octobre, amusait le roi par de
fréquents courriers et par force promesses qui ne s'exécutaient point.
L'infanterie y périssait de fatigues et de misère, dans la fange
jusqu'au cou, et les officiers sans équipage, et par conséquent sans
aucun soulagement contre la rigueur de la saison et du terrain. La garde
était infinie contre une place qui n'était investie qu'à demi, et qui
communiquait par tout un grand côté avec un camp retranché dans une
entière liberté, et ce camp retranché séparé des assiégeants par la
rivière. L'inquiétude enfin prévalut à cette confiance sans bornes en M.
de Vendôme. Le roi voulut que Laparat, le premier ingénieur d'alors, et
lieutenant général, y allât, quoique mal avec M. de Vendôme, pour
accélérer ce siège, y rectifier et y régler, de concert avec ce général,
ce qui serait pour le mieux, et surtout en mander au roi son avis bien
en détail. Laparat en savait trop pour commettre sa fortune à faire un
affront à un homme si puissamment accrédité et appuyé, qui ne lui aurait
pardonné de sa vie, et qui lui aurait détaché Chamillart, M. du Maine et
M\textsuperscript{me} de Maintenon. L'affaire était trop engagée, il
trouva tout bien, et fut toujours d'avis commun avec M. de Vendôme. Lui
aussi, content de sa conduite et plus embarrassé de jour en jour qu'il
ne le montrait, se laissa enfin persuader que jamais il ne prendrait
Verue, tant que la place serait en communication avec ce camp retranché,
vidée de morts, de blessés, de malades, rafraîchie de troupes et de
munitions de guerre et de bouche, à plaisir et à volonté. On était au
dernier février, ainsi depuis quatre mois et demi devant Verue. Le parti
fut donc pris enfin de faire un effort pour rompre cette communication,
avec laquelle, quoi qu'eût soutenu M. de Vendôme avec son opiniâtreté et
son autorité ordinaire, il était visible que Verue ne se pouvait
prendre.

Il fut donc résolu de faire attaquer, la nuit du 1er au 2 mars, le fort
de l'Isle, gardé par deux bataillons de Savoie\,; il fut escaladé et
emporté. Tout y fut tué, excepté deux cents soldats et vingt-quatre
officiers qu'on prit en même temps\,; leur pont fut rompu à coups de
canon\,; huit bateaux emportés par le courant, et la communication de
Crescentin à Verue coupée. On s'établit dans le fort\,; et en même temps
deux compagnies de grenadiers, soutenues de deux bataillons, montèrent
aux brèches de la grande attaque et entrèrent jusque dans la seconde
enceinte, où ils tuèrent une cinquantaine de soldats. Les grenadiers,
qui n'avaient ordre que de reconnaître, se retirèrent et perdirent peu
en cette action, qui fut brusque et peu attendue. Aucun de leurs
fourneaux ne joua. Cette expédition faite, on commença d'espérer avec
raison une bonne et prompte issue de ce long siège, qui n'en donnait
aucune auparavant. Il dura pourtant encore tout le mois (cinq et demi en
tout). On n'en avait point vu de si long à beaucoup près de ce règne, ni
de si ruineux en tout. Enfin, le 5 avril, ils battirent la chamade. Ils
demandèrent une capitulation honorable, mais M. de Vendôme, qui les
tenait à la fin, les voulut prisonniers de guerre. Ils continuèrent donc
à se défendre jusqu'au 9, qu'eux-mêmes mirent le feu à leurs fourneaux
et renversèrent toute la place, excepté le donjon, après quoi ils se
rendirent à discrétion. Ainsi le siège dura six mois moins cinq jours.
Il ne fut plus question après que de mettre, et pour longtemps, en
quartier les troupes -rainées de ce long siège, dans le temps qu'il
fallait avoir déjà mis en campagne, à quoi on suppléa comme l'on put,
mais qui fit un grand tort aux troupes et aux opérations de la campagne
suivante.

Trois semaines après, le prince Eugène arriva en Italie avec un puissant
renfort pour profiter de l'épuisement de notre principale armée, et du
délabrement des troupes qui avaient fait ce long et pénible siège. Cela
n'empêcha pas de se proposer le siège de Turin, même de le résoudre, et,
qui pis fut, de le publier, dont on ne se trouva pas bien.

M\textsuperscript{me} des Ursins se trouvait dans son pays si fort
au-dessus de tout ce qu'elle avait pu même imaginer, qu'elle balança sur
son retour en Espagne. Les empressements de la reine ne la touchaient
plus avec le même retour, et les insinuations légères qui commençaient à
lui être faites, elle les éludait. L'âge et la santé de
M\textsuperscript{me} de Maintenon la tentaient elle eût mieux aimé
dominer ici qu'en Espagne. Elle se flattait sur toutes les distinctions
et les marques de confiance qu'elle recevait d'elle et du roi, et qui
souvent s'étendaient hors de la sphère d'Espagne, et la mettaient en
occasion de servir et de nuire aux personnes de la cour, et à celles
dont les places et la faveur semblait les mettre hors de sa portée. Elle
espérait se maintenir en cet état à l'appui des affaires d'Espagne, et
de s'en faire un petit ministère qui lui ouvrirait les moyens de
l'étendre et d'entrer dans toutes. Flattée des louanges ou plutôt des
serviles adorations de tout ce qu'il y avait de plus grand, elle compta
se les perpétuer par ce grand personnage. Le goût et l'habitude du roi
et de M\textsuperscript{me} de Maintenon pour elle, et personne
vis-à-vis d'elle par la singularité de sa situation, lui semblèrent des
avantages dont elle se pouvait tout promettre\,; et, pendant ce combat
en elle-même, sa santé et ses affaires couvraient ses retardements,
auxquels elle ne fixait point de terme.

L'archevêque d'Aix et son frère, dont je parlerai après pour ne pas
m'interrompre ici, étaient les chefs de son conseil. Elle n'osait leur
dire ses pensées là-dessus. Ils la devinèrent sur son aveu soutenu des
raisons que je viens de dire\,; ils la combattirent par l'entière
différence de ce qui n'est accordé qu'à un court passage et au besoin
qu'on se faisait d'elle en Espagne, à un état fixe et permanent. Ils lui
firent sentir qu'aveuglée du brillant prodigieux qui l'environnait,
plutôt qu'éblouie, elle ne prenait pas garde qu'il ne lui venait que de
l'intérêt de M\textsuperscript{me} de Maintenon, attisé par Harcourt
pour le sien, de régner en Espagne, que tout en passât directement par
elle au roi, et de s'emparer de nouveau, aux dépens des ministres, de
cette portion si considérable du gouvernement\,; que cela même ne se
pouvait que par le retour en Espagne de celle qui en y régnant lui
rendait un compte direct de tout, et l'y faisait régner\,; que, n'y
retournant plus, il ne restait aucun moyen à M\textsuperscript{me} de
Maintenon de rattraper cette précieuse partie des affaires, qui par leur
nature ne pourraient que retomber au canal naturel des ministres, et
l'en laisser dans l'entière privation\,; que le dépit qu'elle en aurait
ferait bientôt tomber tout ce brillant séducteur, et que plus
M\textsuperscript{me} des Ursins avait été initiée, plus elle
demeurerait bientôt écartée par la jalousie, à laquelle un court passage
ne pouvait donner lieu\,; mais que la continuité de ce qu'elle y avait
acquis exciterait dans un état fixe et de consistance en ce pays-ci\,;
que bientôt elle s'y verrait aussi délaissée qu'elle s'y trouvait
environnée et poursuivie\,; enfin que sa situation ne pouvait être
durable ni bonne qu'autant qu'elle en saurait tirer les plus utiles et
les plus avantageux partis\,; que pour ce but il n'était peut-être pas
mauvais de laisser quelque lieu à de l'inquiétude pour se procurer de
plus en plus un pont d'or, et ne la pousser pas assez loin aussi pour
gâter ses affaires, avec une bien absolue détermination de partir et de
prendre bien garde entre le trop tôt pour en tirer tout ce qu'elle
pourrait, et plus encore le trop tard pour ne pas s'en aller de mauvaise
grâce, et n'emporter pas en Espagne un pouvoir moins vaste, moins
absolu, moins connu qu'était celui qu'on lui voulait maintenant confier.

La solidité de ces raisons persuada la princesse des Ursins. Elle ne
regarda plus ce qu'elle avait balancé que comme des tentations et une
séduction dangereuse. Elle résolut donc de partir, mais de différer le
compas dans l'œil, de se faire prier, payer même, si elle pouvait, au
delà de ce qu'elle l'était, mais d'éviter surtout de rompre le fil en le
tirant par trop, et de ne plus songer à ce pays-ci que comme au
fondement de son règne en Espagne. Nous verrons bientôt qu'elle sut
mettre un si bon conseil à profit, et au profit encore de ceux qui le
lui donnèrent. À la façon dont j'étais avec elle je sentis toutes ces
époques\,: l'extrême désir en arrivant de retourner en Espagne,
l'ivresse qui le balança, enfin la dernière résolution prise. J'écumai
bien aussi quelque chose de ces détails, mais pour leur précision, telle
que je la raconte ici, je ne l'ai bien sue que depuis.

Il se passait cependant bien des choses en Espagne. Maulevrier, dans la
plus intime confiance de la reine sur ce qui regardait le retour et les
avantages de M\textsuperscript{me} des Ursins, et seul à Madrid de sa
sorte, qui y fut par l'absence de Tessé sur la frontière, profitait
merveilleusement des instructions utiles de conduite qu'il avait données
à la reine par ses connaissances si exactes de l'intérieur de notre
cour, par les entrées que la reine lui avait fait donner\,; il entrait
chez elle à toute heure par l'appartement du roi, comme je crois l'avoir
déjà dit. Il passait des heures entières entre le roi et elle, et fort
souvent tête à tête avec elle. La duchesse de Montellano n'était pas une
femme à contraindre, et de plus le roi le savait et le trouvait bon.
Maulevrier voyait les lettres qu'ils recevaient. Il en faisait et leur
en dictait les réponses, et par cette confiance entrait d'ailleurs
autant qu'il le pouvait dans la leur sur toutes les autres affaires. Son
esprit, son instruction, le succès de ses conseils sur ce qui regardait
la princesse des Ursins, avaient infiniment augmenté la croyance que le
roi et la reine avaient prise en lui. On a voulu dire qu'il avait voulu
plaire aux yeux de la reine et qu'il y avait réussi. Il est vrai que ces
particuliers, si longs, si journaliers, si continuels, donnèrent fort à
penser et même à parler. Il était temps de moissonner après avoir si
heureusement semé. Le compagnon ne songea pas à moins qu'à la grandesse,
et l'obtint. Mais il était trop vain pour n'être pas indiscret, comme on
en a vu ici des traits que j'ai rapportés.

Le duc de Grammont en eut le vent. Il n'en avait eu que des mépris,
comme un homme qu'on veut chasser et qu'un nouveau favori ne ménage
guère. Il se hâta d'avertir le roi et les ministres du bruit que
commençait à faire la conduite audacieuse de Maulevrier avec la reine,
qui offensait tous les Espagnols, et que sûrement il allait être déclaré
grand d'Espagne. La jalousie, en effet, de toute la cour et ses murmures
alarmèrent Tessé, qui les apprit sur les frontières. Il en craignit
l'effet aux deux cours, et plus encore en celle de France\,; il manda
son gendre devant Gibraltar où il était, qui fut obligé de partir
sur-le-champ de Madrid pour l'y aller trouver. En même temps, arriva un
courrier de Torcy avec des lettres du roi très fortes au roi d'Espagne
sur Maulevrier, et une de Torcy à celui-ci, qui lui mandait que le roi
lui défendait très expressément d'accepter la grandesse ni aucune autre
grâce du roi d'Espagne, et lui ordonnait d'aller sur-le-champ joindre
Tessé, avec une réprimande très sévère, non d'un cousin germain, mais
d'un ministre offensé de ses manèges, de ses intrigues et du parti qu'il
avait pris. Le courrier fit remettre au roi d'Espagne les dépêches du
roi, et courut après Maulevrier à Gibraltar lui porter les siennes. Ce
fut un étrange coup pour cet ambitieux qui, ayant si bien conduit sa
trame, et réussi pour autrui, se trouvait privé de la récompense qu'il
tenait déjà. La rage et le dépit cédèrent aux espérances qu'il se forgea
de venir à bout pour soi de Versailles par Madrid. Son beau-père ne put
le retenir au siège comme il l'aurait voulu. Ses représentations et son
autorité furent inutiles.

Maulevrier, après un court séjour devant Gibraltar, retourna à Madrid,
sous prétexte d'y aller rendre compte de l'état du siège\,; mais en
effet pour tout tenter auprès du roi et de la reine d'Espagne, pour, par
eux, forcer la main au roi et le faire consentir à sa grandesse.
Malheureusement pour lui il trouva le duc de Grammont encore à Madrid,
d'où il était prêt à partir, qui dépêcha un courrier sur ce retour d'un
homme qu'il savait avoir eu ordre d'aller au siège de Gibraltar et qu'il
ignorait avoir eu la permission d'en revenir. Cette désobéissance fut
promptement châtiée. Torcy eut ordre de dépêcher un courrier à
Maulevrier, avec commandement absolu de partir au moment qu'il le
recevrait pour revenir en France. Alors il n'y eut plus de remède ni à
différer. Il prit congé du roi et de la reine d'Espagne en homme
désespéré, et partit. Le rare est qu'en arrivant à Paris, il trouva la
cour à Marly et sa femme du voyage. Il fit demander la permission d'user
du droit des maris sur Marly, quand leurs femmes y étaient, ce que le
roi, pour éviter un éclat, voulut bien ne pas lui refuser. Sa
consolation fut d'y trouver la princesse des Ursins de plus en plus au
pinacle, par le moyen de laquelle il espéra se raccommoder, brouillé
comme il l'était pour elle, ou plutôt pour ses vues ambitieuses, avec
Torcy, et avec le duc de Beauvilliers, ses cousins germains.

Cependant les choses allaient fort mal à Gibraltar. Il y arriva un
prodigieux secours de Lisbonne, conduit par trente-cinq gros vaisseaux
de guerre. Ils entrèrent dans la baie de Gibraltar, où ils trouvèrent
Pointis avec cinq vaisseaux, qui ne s'y croyait pas en sûreté, mais qui
avait un ordre positif du roi d'Espagne d'y demeurer. Un brouillard fort
épais lui déroba la vue de cette flotte, qui tomba sur lui qu'à peine
l'avait-il aperçue. Il n'en avait eu aucun avis\,; quoiqu'il eût envoyé
deux autres vaisseaux dans l'Océan, pour découvrir et l'avertir, ce
qu'ils n'avaient pu faire. Malgré l'inégalité du nombre, le combat dura
cinq heures\,; mais à la fin le grand nombre l'emporta. Trois vaisseaux
de soixante pièces de canon chacun furent pris, deux de quatre-vingts
pièces de canon que les ennemis n'osèrent aborder s'échouèrent. Pointis,
qui montait le plus gros, sauva les deux équipages et les brûla après
pour que les ennemis n'en profitassent point, qui après cette victoire
entrèrent à Gibraltar et y jetèrent tout ce qu'ils avaient apporté. Le
roi reçut cette mauvaise nouvelle le 5 avril. Cinq jours après, le petit
Renault arriva de ce siège pour lui en rendre compte. Il y avait déjà du
temps que le roi pressait pour qu'on le levât, et que le roi d'Espagne
s'opiniâtrait à le continuer. Enfin, le 6 mai, il arriva un courrier
dépêché de Séville par le maréchal de Tessé, qui apprit la levée du
siège dont il avait retiré tout le canon, et que Villadarias était
demeuré devant cette place avec dix pièces de canon seulement et ce peu
de troupes espagnoles qui lui restaient, moins nombreuses de moitié que
la garnison de la place. Ce fut huit jours après cette nouvelle que
Maulevrier arriva. À la fin de ce même mois de mai, le petit Renault fut
renvoyé à Cadix pour y demeurer pendant toute la campagne. Il était chef
d'escadre et avait fort la confiance du roi.

On ne l'appela jamais que le petit Renault, de sa taille singulièrement
petite, mais bien proportionnée et jolie. Il était Basque, et il était
entré tout jeune à Colbert du Terron, intendant de marine à la Rochelle,
qui, ayant voulu acheter Rochefort et le seigneur s'étant opiniâtré à ne
le point vouloir vendre, de dépit y voulut être plus maître que lui. Il
persuada à la cour, où son nom alors l'appuyait fort, que c'était le
lieu du monde le meilleur pour en faire un excellent port et le plus
propre aux constructions des navires. On le crut, on y dépensa des
millions. Du Terron, par ce moyen, devint le maître et le tyran du lieu
et du seigneur qui n'avait pas voulu le lui vendre. Mais quand tout fut
fait, il se trouva une telle distance de ce lieu à la mer, un coude
entre autres si fâcheux, et la Charente si basse, que les fort gros
vaisseaux ne pouvaient y aller de la mer, ni de Rochefort à la mer, et
que les autres n'y pouvaient aller qu'avec leur lest et désarmés, encore
avec deux vents différents pour en faire le trajet. Il n'eût pas été
difficile de voir ce défaut, qui sautait aux yeux, avant de s'engager en
une telle dépense\,; mais si le sort des choses publiques est presque
toujours d'être gouvernées par des intérêts particuliers, il se peut
dire et trop continuellement vérifier qu'il est très singulièrement
attaché à la France. Du Terron trouva de l'esprit et de l'application à
ce petit Basque. Il le fit étudier, le jeta dans les mathématiques et
tout ce qui pouvait l'instruire dans la marine, et trouva qu'il passait
de bien loin les espérances qu'il en avait conçues. Il épuisa bientôt
ses maîtres et devint le sien à lui-même. Il fut bon géomètre, bon
astronome, grand philosophe et posséda parfaitement l'algèbre\,; avec
cela, particulièrement savant dans toutes les parties de la construction
et de la navigation. C'était d'ailleurs un homme doux, simple, modeste
et vertueux, fort brave et fort honnête homme. Il servit sur mer avec
réputation. M. de Seignelay établit une école de marine tenue par lui,
dont le roi n'exempta personne, et ce fut pour ne pas vouloir prendre
ses leçons publiques que Saint-Pierre et d'autres capitaines de vaisseau
furent cassés. Renault fut grand admirateur et grand ami du P.
Malebranche, connu et fort protégé des ducs de Chevreuse et de
Beauvilliers, beaucoup aussi de M. le duc d'Orléans. Tout le monde
l'aima et en fit cas. Il eut des actions heureuses à la mer, et son
désintéressement lui lit beaucoup d'honneur. Il eut beaucoup d'emplois
de confiance et de rapports immédiats avec le roi. Rien de tout cela ne
l'éleva et ne le fit sortir de son caractère. Nous le verrons monter
plus haut et toujours le même.

Ragotzi continuait ses progrès deçà et delà le Danube jusqu'en Moravie.
Il menaçait Bude\,; et le comte Forgatz, maître de la Transylvanie,
assiégeait Rabutin dans Hermanstadt. Ce Rabutin était ce page pour
lequel M\textsuperscript{me} la Princesse fut renfermée à Châteauroux,
d'où elle n'est jamais sortie, et où, après tant d'années, elle ignora
toujours la mort de M. le Prince, son mari, gardée avec autant
d'exactitude que jamais jusqu'à sa mort, par les ordres de M. le Prince
son fils. Le page se sauva de vitesse, se mit dans le service de
l'empereur, s'y distingua, épousa une princesse fort riche, et parvint
avec réputation aux premiers honneurs militaires.

Pendant ces désordres en Hongrie et dans les provinces voisines,
l'empereur Léopold mourut à Vienne le 5 mai sur le soir, d'une assez
longue maladie, sans enfants de ses deux premières femmes. Il laissa
deux fils et trois filles de la troisième, sœur de l'électeur palatin\,:
Joseph, déjà roi reconnu de Hongrie, Bohème et des Romains\,; et
Charles, qui était en Portugal, se prétendant roi d'Espagne, qui l'un
après l'autre lui succédèrent à l'empire. Ce fut un prince qui sut
régner sans être jamais sorti de Vienne que pour se sauver à Lintz,
lorsque les Turcs en firent le siège, que le fameux J. Sobieski, roi de
Pologne, leur fit si glorieusement lever. Une laideur ignoble, une mine
basse, une simplicité fort éloignée de la pompe impériale ne l'empêcha
pas d'en pousser l'autorité beaucoup plus loin qu'aucun de ses
prédécesseurs, si on en excepte Charles-Quint\,; et sa vie extérieure,
plus monacale que de prince, ne l'empêcha pas de se servir de toutes
sortes de voies pour arriver à ses fins. Témoin la mort du prince
électoral de Bavière, fils de sa fille, d'un de ses premiers lits\,;
celle de la reine d'Espagne, fille de Monsieur\,; l'étrange objet de
l'envoi du prince de Hesse-Darmstadt en Espagne du temps de la reine,
seconde femme de Charles II\,; la part si principale qu'il eut au
renversement du trône d'Angleterre et de la religion catholique en ces
royaumes pour y placer le célèbre prince d'Orange\,; ses usurpations
sans nombre dans l'empire et en Hongrie et Bohème contre le serment de
ses capitulations\,; et les vengeances sans mesure et sans oubli qu'il
tira des moindres manquements à son égard des princes et des seigneurs
d'Allemagne. Son éloignement personnel de la guerre, pour n'en rien dire
de plus, émoussa la crainte et la jalousie jusqu'à ce qu'il ne fût plus
temps de remuer contre lui. Il la fit toujours par ses généraux,
auxquels il fut singulièrement heureux. Il ne le fut pas moins en
ministres, qu'il sut si bien choisir que son conseil fut toujours le
meilleur de l'Europe. Il eut le bon esprit de le croire et il s'en
trouva toujours bien. La terreur que le roi causa par ses conquêtes et
par un ministre habile qui voulut et qui fit toujours la guerre, et le
dépit que le prince d'Orange conçut enfin de n'avoir pu amortir, par ses
longues et persévérantes soumissions, la haine étrange du roi pour sa
personne, qui bâtirent les ligues contre la France, formèrent aussi la
dictature de Léopold dans l'Europe. En un mot, il fut habile et fier,
toujours suivi dans ses plans et dans sa conduite, heureux en tout et en
famille.

La dernière impératrice était fort impérieuse\,; il la laissait
maîtresse d'une infinité de petites choses, mais elle n'entrait en
aucune des grandes, et point du tout dans les affaires. Elle lui était
tellement attachée, qu'elle ne s'en fait qu'à elle-même, dès qu'il était
malade (ce qui n'arriva presque point que pour mourir), pour faire son
pot dans sa chambre, préparer les remèdes qu'il devait prendre, les lui
donner de sa main et le servir comme une simple garde-malade. La vie
privée de ce prince fut un continuel exercice de religion, et, comme je
l'ai dit, une vie tout à fait monacale, avec un usage le plus fréquent
des sacrements. Il les reçut plusieurs fois dans sa maladie, et encore
le matin du jour qu'il mourut. Ce qui est bien étrange, c'est que
sentant sa fin approcher, après avoir mis ordre à toutes choses, il
demanda sa musique, qui avait toujours fait son unique plaisir. Il
l'entendit plusieurs heures, et mourut en l'entendant.

Le roi des Romains fut très longtemps sans en donner part au roi. Enfin
le 30 juin, le nonce, qui avait demandé audience, lui présenta les
lettres de ce prince, de la princesse son épouse et de l'impératrice
douairière, écrites, selon leur usage, en italien\,; aussi le roi ne
drapa point quoique beau-frère, prit le deuil en violet, mais le compta,
pour la durée, du jour que l'empereur était mort. Le successeur de ce
prince se montra, incontinent après, bien plus dur et plus fâcheux que
Léopold n'avait été encore sur la Bavière. Il fit entrer six mille
hommes dans Munich, contre le traité qu'il avait signé lui-même avec
l'électrice, laquelle s'était retirée à Venise, et à qui il ne voulut
pas permettre de retourner en Bavière. La reine de Pologne, sa mère, y
avait été passer quelque temps avec elle, outrée contre la cour de
Vienne de l'enlèvement de ses fils, que le roi Auguste avait fait
enlever en Silésie, et qu'il ne voulait pas rendre.

Laparat, après la prise de Verue, était allé à la Mirandole, que M. de
Vendôme faisait assiéger depuis longtemps, et encore sans investiture
entière, en sorte que la garnison était continuellement rafraîchie. Cet
ingénieur, qui était aussi lieutenant général, y commanda en chef, et
vint enfin à bout de cette place, le 11 mai, la garnison prisonnière de
guerre. Le comte de Koenigsech, qui y commandait, subit ce sort avec
soixante-dix officiers et cinq cents soldats\,; il était général major.
Il s'y trouva force artillerie et munitions de guerre et des vivres
encore pour trois mois. On sut en même temps que le prince Eugène avait
fait traverser plusieurs petites rivières et plus de trente lieues à
huit mille chevaux qui étaient tombés entre plusieurs villages près de
Lodi, où étaient les équipages de nos officiers généraux, dont ils
emmenèrent près de mille avec quelques-uns de l'artillerie. Vaubecourt,
lieutenant général, qui était là auprès, y accourut avec ce qu'il put
ramasser, et y fut tué\,: c'était un homme fort court, mais brave, fort
appliqué et très honnête homme. Sa femme, dont il n'avait point
d'enfants, avait fait bruit dans le monde. Le maréchal de Villeroy, qui
en était amoureux, eut, une de ses campagnes, la fatuité de faire faire
le tour de la place Royale, où elle logeait, à son magnifique équipage
qui partait pour l'armée. Elle était sœur d'Amelot, qui venait d'aller
ambassadeur en Espagne, et de la femme d'Estaing, qui eut le petit
gouvernement de Châlons et la lieutenance générale de Champagne qu'avait
Vaubecourt. Ce dernier s'appelait Nettancourt, et était homme de
qualité. M. de Vendôme fit raser la Mirandole, Verceil et les trois
premières enceintes de Verue, ne laissant que la quatrième\,; et
continua toujours, lui et le grand prieur, d'amuser le roi par des
courriers, des espérances d'attaquer le prince Eugène, et de différents
petits projets sans exécution\,: par-ci par-là quelque cassine enlevée
ou forcée.

\hypertarget{chapitre-ii.}{%
\chapter{CHAPITRE II.}\label{chapitre-ii.}}

1705

~

{\textsc{Goutte du roi empêche la cérémonie ordinaire de l'ordre de la
Pentecôte.}} {\textsc{- Prisonniers échappés de Pierre-Encise.}}
{\textsc{- Procès jugé devant le roi sur l'arrêt de la coadjutorerie de
Cluny rendu au grand conseil.}} {\textsc{- Mort de l'abbé
d'Hocquincourt.}} {\textsc{- Mort de M\textsuperscript{me} de
Florensac.}} {\textsc{- Mort de M\textsuperscript{me} de Grignan.}}
{\textsc{- Mariage de Sézanne avec M\textsuperscript{lle} de Nesmond.}}
{\textsc{- Nouveau brevet de retenue à Torcy.}} {\textsc{- Mort de la
duchesse de Coislin.}} {\textsc{- Mort de M\textsuperscript{me} de
Vauvineux\,; sa famille.}} {\textsc{- Duc de Grammont de retour.}}
{\textsc{- Amelot dans la junte.}} {\textsc{- Mort de l'amirante en
Portugal.}} {\textsc{- Mort à Madrid du marquis de Villafranca.}}
{\textsc{- Conspirations en Espagne\,; Legañez arrêté et conduit au
château Trompette, à Bordeaux.}} {\textsc{- Princesse des Ursins prend
congé et diffère encore son départ un mois.}} {\textsc{- Noirmoutiers
duc vérifié, et autres grâces à la princesse des Ursins.}} {\textsc{-
Vie et caractère de Noirmoutiers.}} {\textsc{- Vie et caractère de
l'abbé depuis cardinal de La Trémoille.}} {\textsc{- Prétention de la
princesse des Ursins de draper en violet de son mari, qui la brouille
pour toujours avec le cardinal de Bouillon.}} {\textsc{- Raison pour
laquelle les cardinaux ne drapent plus en France.}}

~

La goutte du roi l'empêcha de faire à la Pentecôte la cérémonie
ordinaire de l'ordre, ce qu'il n'avait jamais manqué de faire trois fois
l'année aux jours destinés. Il eut quelque dépit de l'entreprise de cinq
prisonniers d'État enfermés à Pierre-Encise, qui trouvèrent moyen de
poignarder les soldats qui les gardaient, Manville, gouverneur de ce
château, qui avait été lieutenant-colonel du régiment lyonnais, et de se
sauver si bien qu'ils n'ont jamais été repris.

Le cardinal de Bouillon dans son exil, et l'abbé d'Auvergne à Paris, qui
avaient gagné le procès de la coadjutorerie de Cluni contre les moines,
croyaient que Vertamont, premier président du grand conseil, avait fait
des changements à leur arrêt en faveur des moines en le signant\,; ils
en avaient fait grand bruit aussitôt après, et l'affaire avait été revue
par le grand conseil qui n'y changea rien, quoique fort mal de tout
temps avec leur premier président. Enfin l'affaire fut portée devant le
roi et rapportée au conseil de dépêches. L'arrêt fut maintenu, mais il
fut laissé des voies ouvertes au cardinal et à son neveu de revenir
contre les altérations dont ils se plaignaient. Cela s'appelle que pour
des gens en disgrâce on ne voulut pas réformer l'arrêt, et que la
justice empêcha pourtant la confirmation de ce dont ils criaient. Cela
ne fit pas honneur à Vertamont, qui se vanta pourtant d'avoir gagné son
procès et maintenu son honneur, puisque son arrêt avait été jugé entier
au grand conseil, et ensuite devant le roi.

En ce même temps mourut l'abbé d'Hocquincourt, petit-fils du maréchal,
et le dernier de cette maison de Mouchy, ancienne et illustre, dont
M\textsuperscript{me} de Feuquières, sa sœur, demeura héritière, mais
qui la fut du peu qui restait à une maison ruinée.

La marquise de Florensac mourut aussi, à trente-cinq ans, la plus belle
femme qui fût peut-être en France. Elle était fille de Saint-Nectaire et
d'une sœur de Longueval, lieutenant général, tué en Catalogne sans avoir
été marié. Sa mère avait été fille de la reine, avait été belle, et avec
de l'esprit, du crédit et de l'intrigue, avait fait des procès à son
beau-frère, qu'elle sut tourner en criminel et qu'elle abrutit dans les
prisons, dont il ne sortit qu'avec beaucoup de temps et de peine,
s'accommoda et ne se maria point. Ainsi M\textsuperscript{me} de
Florensac fut fort riche. Elle fit bien des passions, et fut accusée de
n'être pas toujours cruelle\,; d'ailleurs la meilleure femme du monde,
la plus douce et la plus simple dans sa beauté. Elle fut exilée pour
Monseigneur, dont l'amour commençait à faire du bruit. Son mari, frère
du duc d'Uzès, menin de Monseigneur et le plus sot homme de France, ne
s'en aperçut point, et l'aimait passionnément. Elle mourut en deux jours
de temps. Elle ne laissa qu'une fille, belle aussi, mais non comme elle,
qui se piquait de toutes sortes de savoir et d'esprit, qui est
aujourd'hui duchesse d'Aiguillon, Dieu sait comment et
M\textsuperscript{me} la princesse de Conti aussi. M\textsuperscript{me}
de Grignan, beauté vieille et précieuse, dont j'ai suffisamment parlé,
mourut à Marseille bien peu après, et, quoi qu'en ait dit
M\textsuperscript{me} de Sévigné dans ses lettres, fort peu regrettée de
son mari, de sa famille et des Provençaux.

Berwick, en Languedoc, achevait d'anéantir les fanatiques par être bien
averti et par ses promptes exécutions. Il surprit cinq ou six de leurs
chefs dans Montpellier, dont il fit fermer les portes, et les fit
pendre\,; il en fit autant à celui qui fournissait l'argent et à celui
qui les payait. Il découvrit leur cache de poudre et de munitions, et à
la fin éteignit tout à fait ces misérables, et remit le calme et la
sûreté dans cette province et dans les Cévennes.

Sézanne, frère de père du duc d'Harcourt et de mère de la duchesse sa
femme, chose tout à fait singulière, épousa la fille de Nesmond, mort
lieutenant général fort distingué des armées navales, qui était une
riche héritière.

Torcy, dont la conduite avait plu au roi à l'égard de
M\textsuperscript{me} des Ursins, eut une augmentation de brevet de
retenue de cent cinquante mille livres sur ses charges.

Bientôt après mourut la duchesse de Coislin, pauvre et retirée à la
campagne depuis la mort de son mari, sans avoir plus vu personne. Elle
était riche héritière de Bretagne et s'appelait du Halgoet. Elle était
médiocrement âgée, femme de mérite et de vertu, et mère de la duchesse
de Sully, du duc de Coislin et de l'évêque de Metz.

À peu près en même temps qu'elle, mourut à Paris M\textsuperscript{me}
de Vauvineux, qui avait été fort belle, vertueuse et dans la bonne
compagnie à Paris. Elle était fort des amies de ma mère, et sa cousine
germaine par son défunt mari, du nom de Cochefilet, fils de Vaucelas,
ambassadeur en Espagne et chevalier du Saint-Esprit, en 1619, et d'une
sœur du père de ma mère. Le prince de Guéméné avait épousé la fille
unique de M\textsuperscript{me} de Vauvineux et n'eut d'enfants que
d'elle. M\textsuperscript{me} de Vauvineux était Aubry, d'une famille de
Paris, comme la mère de la princesse des Ursins.

Cette dernière, toujours également brillante, faisait ses affaires et
tenait ses conseils secrets à Paris, avec une liberté que Marly ne
comportait pour personne, et y revenait comme il lui plaisait, reçue
avec les mêmes empressements, et sans cesse admise chez
M\textsuperscript{me} de Maintenon et aux particuliers longs entre elle
et le roi, en tiers. Le duc de Grammont était déjà arrivé à Bayonne,
d'où peu après il arriva à Paris, médiocrement reçu. Amelot et Orry
étaient à Madrid, et le premier admis dans la junte avec toutes les
grâces de la reine et l'autorité dans les affaires que, pour elle-même,
M\textsuperscript{me} des Ursins lui avait ménagée. Elle s'était bien
gardée de rien laisser soupçonner en Espagne de sa tentation de n'y plus
retourner. Elle y prétextait ses délais de sa santé, et de la nécessité
de se donner le temps de concerter ici des mesures solides sur leurs
affaires. L'amirante était mort, délaissé et méprisé en Portugal\,; et à
la cour d'Espagne, le marquis de Villafranca, majordome-major du roi et
chevalier du Saint-Esprit, duquel j'ai tant parlé à propos du testament
de Charles II. Celui-ci était demeuré dans la première considération, et
sa charge était la première de la cour. Le duc d'Albe l'avait toujours
regardée comme la récompense de sa ruineuse ambassade, et tout en lui
l'exigeait, naissance (il était Tolède comme Villafranca), dignité, âge,
emploi, fidélité, esprit, application, honneur et probité, splendeur et
capacité dans son ambassade, et il plaisait fort ici et y était fort
considéré. Le roi voulut bien s'intéresser pour lui auprès du roi et de
la reine d'Espagne, et en parler à M\textsuperscript{me} des Ursins\,;
il semblait que l'affaire dût aller tout de suite\,; il n'y avait point,
en Espagne, de compétiteurs si marqué ni si appuyé.
M\textsuperscript{me} des Ursins, à qui le duc et la duchesse d'Albe
avaient fait une cour assidue, promit tous ses bons offices, qu'elle se
garda bien de tenir. L'attachement que le duc d'Albe avait eu pour les
Estrées ne pouvait s'effacer de son cœur\,; il en coûta cette grande
charge au duc d'Albe, de laquelle le roi d'Espagne différa à disposer.

Dès avant que le duc de Grammont partît de Madrid, il s'était découvert
une conspiration à Grenade et une autre à Madrid, qui toutes deux
devaient éclater le jour de la Fête-Dieu\,: le projet était d'égorger
tous les Français dans ces deux villes, et de se saisir de la personne
du roi et de la reine. On crut trouver que le marquis de Legañez en
était le chef. C'était un homme d'esprit et de courage, qui, sous
Charles II, avait passé par les premiers emplois de la monarchie,
gouverneur des armes aux Pays-Bas, gouverneur général du Milanais, grand
maître de l'artillerie, enfin conseiller d'État, des premiers entre les
grands, et gouverneur héréditaire du palais de Buen-Retiro à Madrid. Il
avait toujours été fort attaché à la maison d'Autriche et lié avec ceux
qui passaient pour en être les partisans\,; il s'était toujours dispensé
de prêter serment de fidélité à Philippe V, sous prétexte que l'exiger
d'un homme comme lui, c'était une défiance qu'il réputait à injure, et
on avait eu la faiblesse de s'arrêter tout court pour ne pas l'offenser,
tandis que tous les autres de sa sorte le prêtaient. On crut en savoir
assez pour devoir l'arrêter. Serclaës, capitaine des gardes du corps et
capitaine général, en eut la commission\,; il l'exécuta le 10 juin dans
les jardins du Retiro, lui-même, avec vingt gardes du corps à pied. Il
le conduisit avec cette escorte à une porte qui donne dans la campagne,
où il était attendu dans un carrosse à six mules, trente gardes du corps
à cheval, et trois officiers de confiance dans le carrosse, qui le
menèrent à six lieues de Madrid, à un relais et de là très diligemment à
Pampelune, et tous ses domestiques arrêtés en même temps et ses papiers.
On fit mourir, à Grenade, plusieurs convaincus de la conspiration. Elle
s'étendait en plusieurs autres villes\,; on en arrêta à Cadix, à Malaga,
à Badajoz, même le major de la place, et on leur trouva des lettres de
l'amirante, mort fort peu après, du prince de Darmstadt et de l'archiduc
même. M. de Legañez était déjà venu à Versailles quelques années
auparavant se justifier des soupçons qu'on avait pris sur lui\,; ainsi,
quoiqu'il ne se trouvât que des présomptions et point de preuves, on ne
le laissa pas longtemps à Pampelune, on l'amena à Bordeaux, où on le mit
dans le château Trompette.

Toutes ces choses étaient des motifs de presser le départ de
M\textsuperscript{me} des Ursins\,; elle-même le sentait, et
M\textsuperscript{me} de Maintenon commençait à avoir impatience de s'en
trouver débarrassée. Ces délais lui devenaient suspects\,; elle n'en
apercevait point de raison réelle. On commença donc à la presser. C'est
où M\textsuperscript{me} des Ursins les attendait. Alors elle commença à
s'expliquer davantage sur le poids dont elle allait être chargée dans un
pays d'où elle était partie avec tous les affronts d'une criminelle\,;
qu'il était difficile qu'elle y pût reparaître avec honneur, et surtout
avec la considération qui lui était indispensablement nécessaire pour
bien servir les deux rois, si quelque chose de public n'y annonçait la
confiance qu'ils voulaient bien prendre en elle\,; que bien que comblée
ici de celle du roi et de ses bontés, c'étaient de ces choses
particulières qui s'ignoraient en Espagne, où elle avait besoin pour se
bien acquitter de ce dont elle allait s'y trouver chargée, qu'il y fût
public qu'elle n'y entreprenait rien que par mission, et que plus cette
mission était importante, plus ce besoin devenait pressant pour le
service du roi et pour la mettre en état de le faire obéir. L'éloquence,
l'adresse, le tour, les grâces, la finesse de l'expression, l'attention
à l'effet des paroles, l'air dont elles étaient reçues, tout fut bien
déployé et bien remarqué sous les voiles de la simplicité, de la
nécessité, du naturel\,; l'effet aussi en passa les espérances. Ce fut à
Marly, dans un tiers de plus de deux heures entre le roi et
M\textsuperscript{me} de Maintenon, le 15 juin. M\textsuperscript{me}
des Ursins y prit congé plus que contente. Elle crut ne devoir pas
prolonger\,; mais, en femme aussi habile qu'elle l'était, elle demanda
la permission de voir le roi encore une fois à son retour à Versailles.
C'est que, les mettant à leur aise par le congé qu'elle en prenait, elle
ne voulait pourtant pas partir que les grâces qu'elle venait d'obtenir
ne fussent, les unes expédiées et consommées, les autres acheminées
aussi certainement qu'elles le pouvaient être\,; de façon qu'elle tint
bon sous différents prétextes à ne point partir que tout cela fût fait,
à Versailles, où elle fut encore longtemps enfermée avec le roi et
M\textsuperscript{me} de Maintenon, et où elle acheva de dire tous les
adieux et de prendre ses congés. Elle obtint encore de revoir le roi une
fois à Marly, ce fut la dernière, et elle partit enfin à la mi juillet.

Les grâces qu'elle obtint furent prodigieuses\,: vingt mille livres de
pension du roi et trente mille livres pour son voyage. Son frère, bien
qu'aveugle depuis l'âge de dix-huit ou vingt ans, fut fait duc
héréditaire, et le roi consentit à la promotion du duc de Saxe-Zeitz,
évêque de Javarin, à condition qu'en même temps que lui son autre frère
fût fait cardinal, pour les deux couronnes, qui, en sa faveur, se
désistèrent du droit d'avoir chacune un cardinal en compensation de
celui de l'empereur. Pour bien entendre jusqu'à quel point ces grâces
étaient prodigieuses, il faut faire connaître quels étaient ces deux
frères, et comment leur puissante et habile sœur était avec eux.

M. de Noirmoutiers, beau, très bien fait, avec beaucoup d'esprit et
d'ambition, entra fort agréablement dans le monde, mais ce ne fut que
pour le regretter. À dix-huit ou vingt ans, allant trouver la cour à
Chambord, il tomba malade et se trouva si pressé à Saint-Laurent des
Eaux qu'il ne put aller plus loin. La petite vérole se déclara, elle fut
fâcheuse\,; mais il en était presque guéri lorsqu'une nouvelle repoussa
et lui creva les deux yeux. On peut imaginer quel fut son désespoir.
Guéri et retourné à Paris, il y passa vingt ans entiers à ne pouvoir se
résoudre de sortir de sa maison ni d'y recevoir aucune visite. Il y
passa sa vie à se faire lire. Il avait beaucoup de mémoire, il n'oublia
jamais rien de tout ce qu'il avait ouï dire ou lire\,; et comme dans
cette longue solitude son esprit, naturellement agréable et solide,
avait eu loisir de se former par ses lectures et par ses réflexions, il
devint une excellente tête, et un homme de la meilleure compagnie quand
enfin il en voulut bien recevoir. Le comte de Fiesque était son ami
intime avant son aveuglement\,; il ne voulut jamais le quitter et logea
avec lui\,; il le voyait autant que la dissipation de la jeunesse, la
guerre et la cour le lui pouvaient permettre, mais il fut longtemps sans
avoir le crédit d'obtenir de lui de souffrir aucun de ses amis qui le
venaient voir. Au bout de vingt ans, moins volage et plus souvent chez
soi, il vint à bout d'apprivoiser son ami avec quelques-uns des siens,
et de l'un à l'autre de lui amener compagnie. Noirmoutiers s'y accoutuma
peu à peu, il parut aimable à tout ce qui fut admis. Le cercle
s'élargit\,; il s'y trouva des gens avec qui il lia plus qu'avec de
simples connaissances. Quelques-uns lui parlèrent de leurs affaires soit
de cœur et de monde, soit domestiques. Ils se trouvèrent bien de ses
conseils\,; en un mot, il devint à la mode d'être en commerce avec M. de
Noirmoutiers, et tout ce qui le vit fut charmé de son esprit, de sa
conversation et de sa justesse en toutes choses. Un homme de cette sorte
et qu'on est sûr de trouver chez lui n'y est plus guère en solitude. Les
gens de la cour et du grand monde, ceux de la ville et de la
magistrature, tout y abonda\,: c'était le bel air. Parmi cette
diversité, il se forma des amis considérables en tout genre. Sa maison
devint un tribunal où il n'était pas indifférent d'être blâmé ou
approuvé. Soit conseil, soit confiance, Noirmoutiers entra et se mêla
dans une infinité d'affaires, et se trouva, sans sortir de sa chambre,
l'homme le mieux informé de tout ce qui se passait à la cour et dans le
monde, fort compté et fort accrédité pour servir ses amis.

Sa santé qui fut toujours délicate, un bien fort court, le désir de
pouvoir suppléer à ses yeux par un autre soi-même en bien des occasions
où la nécessité d'en emprunter lui devint un joug embarrassant, le
tournèrent au désir du mariage. Pauvre et aveugle, de grande naissance,
mais fils d'un duc à brevet qui ne lui avait point laissé de rang, il
était difficile de rencontrer un mariage avantageux\,; il ne songea donc
qu'à se donner une femme avec un bien médiocre, de qui il pût espérer ce
qu'il en cherchait. Il crut la trouver dans une fille de La Grange,
président d'une chambre des requêtes du palais, et il l'épousa au
commencement de 1688, mais il la perdit au bout de dix-huit mois sans
enfants. M\textsuperscript{me} des Ursins cria à la mésalliance, comme
si leur mère n'eût pas été Aubry, leur grand'mère Bouhier, fille d'un
trésorier de l'épargne, et leur {[}arrière-{]} grand'mère Beaune,
petite-fille du vertueux et malheureux Semblançay de François Ier. Ces
cris mirent du refroidissement entre le frère et la sœur, qui ne s'était
pas encore entièrement réchauffé, lorsque les mêmes raisons qui avaient
engagé M. de Noirmoutiers à ce premier mariage le firent, dix ans après,
penser à un second et de la même espèce. Il épousa donc en mai 1700 une
fille de Duret, seigneur de Chevry, président en la chambre des comptes.

Ce mariage outra la princesse des Ursins, qui était à Rome, et renouvela
leurs précédentes aigreurs. Elles n'étaient point adoucies lorsqu'elle
fut obligée de sortir si brusquement d'Espagne. Arrivée à Toulouse, elle
avait eu loisir de toutes sortes de réflexions. M. de Noirmoutiers, de
quelque façon qu'il fût avec sa sœur, fut sensible à sa chute, peut-être
plus encore à la manière qu'à la chose même. Elle se vit en besoin de ne
rien laisser en arrière de tout ce qui pouvait l'aider. Quoiqu'elle ne
pût pardonner à son frère de s'être marié comme il avait fait, il lui
savait un bon esprit, capable de conduite, de conseil et d'intrigue, et
beaucoup d'amis de toutes sortes à la pouvoir servir. Ainsi, gloire de
famille d'une part, besoin de l'autre, les rapprochèrent. M. de
Noirmoutiers eut des conférences avec l'archevêque d'Aix, et tous deux
se mirent à la tête des affaires de M\textsuperscript{me} des Ursins,
dont ils devinrent l'âme, et les directeurs de son conseil et de ses
démarches, et les moteurs de tous les ressorts qu'ils purent faire
jouer. On a vu que cet archevêque entra à la fin là-dessus dans la
confidence d'Harcourt qu'il lia secrètement avec Noirmoutiers, et le
demeurèrent toujours depuis, et dans celle de M\textsuperscript{me} de
Maintenon, mais qui n'eut point de commerce avec cet habile aveugle. Il
en était là avec sa sœur lorsqu'elle arriva à Paris\,; mais autre est
une liaison de nécessité qui ne prend que sur la raison et l'esprit,
autre celle du cœur. Le leur ne pouvait oublier les mésalliances et les
hauteurs dont elles avaient été suivies. Cela fit que
M\textsuperscript{me} des Ursins vit son frère par raison, par
bienséance, par reconnaissance de ses services, et pour ceux qu'elle
pouvait en tirer encore et pour l'utilité de ses conseils, mais
d'ailleurs peu libres ensemble. Elle ne logea point chez lui, et se mit
chez la comtesse d'Egmont, où elle était au large et à son aise pour les
raisons que j'en ai rapportées. Les grâces éclatantes qu'elle voulut,
ses frères, sur qui elles tombèrent, y eurent la moindre part. En rang,
en biens, en places, en autorité, elle avait tout, n'y pouvait donc rien
ajouter pour elle, nécessité lui fut de les faire tomber sur eux pour
réfléchir sur elle-même ce rayon de gloire qu'elle voulait faire briller
aux yeux des deux monarchies\,: c'est ce qui fit faire duc vérifié au
parlement un aveugle sans enfants, et qui n'en bougea jamais de sa
chaise. Sa femme, qui n'avait pas seulement été présentée à la cour,
alla y prendre son tabouret et participer quelques moments à la gloire
de sa belle-sœur.

L'abbé de La Trémoille était un petit bossu fort vilain, fort débauché,
qui n'avait jamais voulu rien apprendre ni rien faire de conforme à
l'état qu'il n'avait pris que pour réparer sa pauvreté par des
bénéfices. Il avait de l'esprit, un esprit plaisant et d'agréable
compagnie, mais qui n'avait aucune solidité, et tout tourné au plaisir.
Ses mœurs et sa pauvreté aidèrent au goût naturel de l'obscurité, où il
trouvait plus de liberté qu'avec des gens de son état et de sa
naissance. Cette conduite ne lui, procura pas de quoi vivre. Ennuyé d'en
attendre vainement, et incapable d'en mériter par un changement de vie,
il prit le parti de s'en aller à Rome trouver ses sœurs. Il y attrapa
l'auditorat pour la France, que le cardinal de Bouillon et d'Estrées lui
ménagèrent pour l'amour de la duchesse de Bracciano, avec un emploi qui
demandait de la science, de l'application, de la gravité\,; la première
ne lui vint pas\,; les deux autres lui étaient inconnues\,; ses mœurs
furent les mêmes\,: à Rome c'eût été un inconvénient léger pour la
fortune\,; mais l'obscurité, la bouffonnerie et le jeu où il consumait
tout ce qu'il avait et ce qu'il n'avait pas, le perdirent d'honneur et
de réputation. Pour comble, il se brouilla avec sa fameuse sœur pour
avoir pris le parti de son mari contre elle dans leurs démêlés
domestiques. Ils étaient donc en ces termes lorsqu'elle devint veuve.
Elle prétendit la distinction de draper en violet.

Le cardinal de Bouillon, qui était alors à Rome et qui jusqu'alors avait
été intimement avec elle, prit cette prétention avec une grande hauteur,
et s'en brouilla irréconciliablement avec elle. Il avait dans sa faveur
introduit cet usage en France pour les cardinaux\,; à la fin, Monsieur
se fâcha de ne voir que le roi et les cardinaux drapés en violet, tandis
que les fils de France, le Dauphin même, et la reine, quand il y en
avait une, ne l'étaient qu'en noir. Il en parla si souvent au roi, qu'à
la fin, à je ne sais plus quel deuil où il drapa, il défendit au
cardinal de Bouillon et aux autres cardinaux de draper en violet. Le
cardinal de Bouillon, outré et ne pouvant soutenir un usage si nouveau,
si peu fondé, si supérieur à celui de la reine même et des fils de
France, fit un effort de crédit pour n'en avoir pas au moins à son avis
le démenti entier, et obtint que les cardinaux ne draperaient plus ni
pour deuils de cour ni pour ceux de famille, et depuis cette époque,
aucun n'a drapé en France. Pour la livrée, celle du roi étant en noir
lorsqu'il drape, le cardinal de Bouillon avait laissé la sienne et celle
de ses confrères en noir, et lorsqu'ils devaient draper, ils continuent
d'habiller de noir toute leur livrée. Il y avait peu que le cardinal de
Bouillon avait essuyé ce dégoût, lorsque le duc de Bracciano mourut,
c'est ce qui le rendit encore plus vif sur la prétention de sa veuve.

Je ne sais si l'abbé de La Trémoille prit le parti du cardinal de
Bouillon contre sa sœur, ou celui des créanciers dans l'accommodement
des affaires de la succession contre les prétentions de la veuve\,; ce
qui est certain c'est qu'elle fut mal contente de lui sur ces deux
points, l'un desquels, je ne dirai pas lequel, mais sûrement l'un des
deux la mit dans une telle colère, qu'elle voulut perdre son frère, et
qu'elle le fit déférer à l'inquisition pour de fâcheuses débauches.
L'abbé sentit son cas si sale qu'il s'en alla à Naples, de peur d'être
arrêté. Le cardinal de Bouillon déjà fort mal à la cour, sur l'affaire
de M. de Cambrai, mais qui était encore chargé des affaires de France à
Rome, vint au secours de l'abbé de La Trémoille, persécuté par sa sœur.
Il prétexta quelques affaires à Naples, pour lesquelles, disait-il, il
l'y avait envoyé pour y travailler sous ses ordres et ceux du duc
d'Uzeda, ambassadeur d'Espagne à Rome. Cette gaze n'empêcha pas tout
Rome de voir fort clair à travers. Les affaires de Naples y durèrent
jusqu'à ce qu'on eût mis l'abbé de La Trémoille en sûreté, ce qui fut
long, parce que l'inquisition avait déjà commencé d'agir, et que la
duchesse de Bracciano qui, depuis la vente de ce duché à don Livio
Odescalchi, à condition d'en quitter le nom, avait pris celui de
princesse des Ursins, continuait à remuer tout ce quelle pouvait contre
son frère. Il fallut donc lui faire entendre raison là-dessus, ce qui ne
fut pas aisé\,: à la fin, contente de lui avoir fait la peur entière, et
de lui avoir montré ce qu'elle savait faire, elle consentit de le
recevoir à pardon. Alors il revint à Rome, et reprit, mais à son
ordinaire, les fonctions de son emploi\,; la terreur qui lui était
restée, et la vie qu'il continuait de mener la même, le rendirent souple
à l'égard de M\textsuperscript{me} des Ursins, mais avec un commerce
froid et rare de la plus simple bienséance.

Ils en étaient en ces termes depuis quatre ans, sans s'être plus
rapprochés, lorsque M\textsuperscript{me} des Ursins partit de Rome pour
aller joindre la reine d'Espagne, et la conduire au roi son époux. Ce
fut une délivrance pour l'abbé de La Trémoille. L'absence ne les avait
pas réchauffés, et ils en étaient là ensemble lors du triomphe de
M\textsuperscript{me} des Ursins qui, ne se pouvant venger des Estrées,
fut réduite pour sa propre gloire, et pour mieux consolider sa
toute-puissance par des choses de grand éclat, de les faire tomber sur
ses frères\,; haïssant l'un et en étant haïe, et se souciant très
médiocrement de l'autre. Tel était donc l'abbé de La Trémoille à Rome,
c'est-à-dire dans le dernier mépris, et perdu d'honneur et de
réputation, lorsque sa sœur entreprit de le faire cardinal. On se
souviendra de ce que j'ai rapporté en son lieu, de l'opposition formelle
et constante que le roi apportait depuis plusieurs années à la promotion
du duc de Saxe-Zeitz, évêque de Javarin, et des motifs pressants de
cette opposition. On n'aura pas oublié aussi combien fortement elle fut
renouvelée, lorsque le cardinal de Bouillon, dans l'abus de sa faveur,
tenta avec une si adroite audace de duper le pape et le roi sur cette
promotion en faveur de son neveu, et c'est cette opposition du roi si
ferme, si éclatante, si soutenue, que M\textsuperscript{me} des Ursins
entreprit de vaincre, et d'en faire l'échelon de la promotion de son
frère, à laquelle elle ne pouvait ignorer qu'elle-même n'eût mis un
empêchement dirimant, que la conduite persévérante de ce frère avait
sans cesse confirmé. Aussi n'espéra-t-elle pas réussir, que par
intéresser le pape par un motif aussi pressant qu'était pour lui de se
délivrer des prières instantes et continuelles de l'empereur, souvent
aiguisées de menaces, en lui procurant, moyennant la promotion de son
frère, la liberté de le contenter.

Elle connaissait encore trop bien le terrain de Rome pour se flatter que
ce motif-là seul pût l'emporter sur le scandale de faire cardinal un
homme dans la réputation et dans la situation où y était son frère, et
de plus noté par l'inquisition d'une manière si publique, tache qui
soulèverait toute la cour de Rome, et le sacré collège particulièrement,
contre sa promotion. Elle crut donc qu'il y en fallait joindre un autre
qui, aux dépens des deux couronnes, fît gagner un chapeau au pape, et
lui donnât un moyen de gratifier d'autant l'empereur en faisant un
cardinal pour lui, contre un seul pour les deux couronnes, au lieu d'un
pour chacune, comme elles étaient en plein droit non contesté de
l'exiger. Que de choses donc à vaincre, à aplanir à la fois\,? Priver un
Espagnol de la pourpre en pure perte, faire relâcher les deux rois pour
cette fois de leur droit, et obtenir du roi la condescendance la plus
préjudiciable en ce genre à sa gloire et à son intérêt. C'est néanmoins
ce qu'elle obtint, tant M\textsuperscript{me} de Maintenon était pressée
de se défaire d'elle, et de l'envoyer régner en Espagne, pour y régner
elle-même. Les dépêches en furent donc faites et, envoyées avant son
départ. De celles d'Espagne elle n'en était pas en peine. Elle n'eut
qu'à y écrire dès qu'elle eut obtenu ici, et aussitôt après on envoya
d'Espagne à Rome les dépêches telles qu'elle les avait prescrites. Elle
fit encore que le roi parla fortement de cette promotion à Gualterio,
nonce en France, après quoi elle n'eut plus rien à exiger de lui.
C'était à Rome où il fallait faire le reste, et ce reste n'y fut pas
facile\,; il n'y avait pas moyen d'en attendre le succès en ce pays-ci.
Contente et comblée plus que sujette le fut jamais, elle partit enfin
vers la mi-juillet, et fut près d'un mois en chemin. On peut juger
quelle fut sa réception en Espagne\,: elle trouva le roi et la reine
au-devant d'elle, à près d'une journée de Madrid. Voilà cette femme dont
le roi avait si ardemment procuré la chute, de laquelle Maréchal m'a
conté qu'il s'était applaudi avec complaisance entre lui, Fagon et
Bloin, en se félicitant de l'art qu'il avait eu de séparer de lieu, le
roi et la reine d'Espagne, pour être plus sûr alors de frapper son coup
sur elle.

\hypertarget{chapitre-iii.}{%
\chapter{CHAPITRE III.}\label{chapitre-iii.}}

1705

~

{\textsc{Belle campagne de Villars.}} {\textsc{- Roquelaure battu et
culbuté dans nos lignes.}} {\textsc{- Belle action et récompense de
Caraman.}} {\textsc{- Reste de la campagne de Flandre.}} {\textsc{-
Ambition, art et malignité de Lauzun.}} {\textsc{- Dezzedes tué.}}
{\textsc{- Haguenau pris par les Impériaux\,; Peri et Arling
récompensés.}} {\textsc{- Siège de Chivas.}} {\textsc{- Prince d'Elbœuf
tué.}} {\textsc{- Fascination du roi sur MM. de Vendôme.}} {\textsc{-
Combat du Cassano.}} {\textsc{- Mort de Praslin.}} {\textsc{- Disgrâce
du grand prieur sans retour.}} {\textsc{- La connétable Colonne près de
Paris.}} {\textsc{- Archevêque d'Arles tancé pour son commerce à Rome\,;
ma liaison avec lui et avec le nonce depuis cardinal Gualterio.}}
{\textsc{- Fantaisie des nonces sur la main, cessée depuis.}} {\textsc{-
Caractère de Gualterio.}} {\textsc{- La Feuillade achève le siège de
Chivas.}} {\textsc{- L'archiduc passe par mer devant Barcelone et
l'assiège.}} {\textsc{- Fâcheux démêlé entre Surville et La Barre\,;
leur état et leur caractère.}} {\textsc{- Affaire du banquillo.}}
{\textsc{- Connétable de Castille majordome-major.}} {\textsc{- Voyage
de Fontainebleau par Sceaux.}} {\textsc{- Mariage de Bercy à une fille
de Desmarets.}} {\textsc{- Mort, famille et caractère de Bournonville.}}
{\textsc{- Mort, caractère et famille de Virville.}} {\textsc{- Mort et
caractère d'Usson.}} {\textsc{- Comte de Toulouse et maréchal de Cœuvres
à Toulon, et reviennent tout court.}} {\textsc{- Comte de Toulouse
achète Rambouillet à Armenonville, à qui on donne la capitainerie de la
Muette et du bois de Boulogne seulement.}}

~

Villars fit cette année une campagne digne des plus grands généraux. Le
projet des ennemis était de pénétrer par le côté de la Sarre, de prendre
l'Alsace à revers, de tomber sur les Évêchés, et de là plus avant en
France, où leur bonheur les pourrait conduire. Marlborough y menait une
armée de plus de quatre-vingt mille hommes. Villars se posta à Circk, où
il l'attendit de pied ferme, et où il n'osa jamais l'attaquer, quoique
très supérieur en nombre. Le prince Louis de Bade s'approcha de son côté
et s'avança de sa personne pour conférer avec Marlborough. Là-dessus le
maréchal de Villeroy envoya d'Alègre joindre Villars avec vingt
escadrons et quinze bataillons qu'il attendit sans inquiétude dans
l'excellent poste qu'il avait pris\,: aussi n'en eut-il pas besoin.
L'impossibilité de réussir en l'attaquant et de subsister devant lui
dans un pays qui ne pouvait suffisamment fournir de fourrages obligea
Marlborough de se retirer sur Trèves, ce qui fit que Villars envoya dire
à d'Alègre de s'arrêter où son courrier le rencontrerait, parce qu'il
n'avait plus besoin du renfort qu'il lui amenait. Marlborough, enragé de
voir tous ses projets avortés par le poste que Villars avait su prendre,
lui manda par un trompette qu'il l'eût attaqué le 10 juin, comme il se
l'était proposé, sans le prince Louis de Bade, qui, au lieu d'arriver le
9 à Trèves comme il avait promis, n'était arrivé que le 15, et encore
avec ordre de ne point combattre, dont il se plaignait amèrement.
Villars, délivré de tout soupçon, envoya un détachement fort nombreux
mené par quatre lieutenants généraux au maréchal de Villeroy, sur qui
les ennemis paraissaient se proposer de retomber par les mouvements
qu'ils faisaient vers lui. Avec cette occupation qu'il leur donna, il
marcha avec le reste de son armée en Alsace, où Marsin l'attendait, où
il prit Weissembourg, chassa les Impériaux de leurs lignes sur la
Lauter, prit plusieurs petits châteaux et cinq cents prisonniers, et
s'étendit dans le pays qu'ils occupaient. Ainsi par le poste de Circk il
obligea les ennemis de changer tous les projets de leur campagne, et
profita par sa diligence de l'éloignement de l'armée du prince Louis de
Bade, pour renverser les lignes de Lauterbourg avant qu'elle pût être
revenue, qui étoient une barrière de la montagne au Rhin, qui nous
resserrait entièrement dans notre Alsace\,; mais le poste particulier de
Lauterbourg fut toujours soutenu par eux.

Les ennemis abandonnèrent Trèves précipitamment et arrivèrent le 17 juin
sous Maestricht. Le duc de Marlborough, retourné en Flandre, y fit
divers mouvements jusque vers le 20 juillet qu'ayant donné le change au
maréchal de Villeroy, il fit une marche, sur nos lignes entre Lave et
Heylesem, les força, les rasa en grande partie, et y fit un grand
désordre. Roquelaure, qui les gardait avec peu de précaution, arriva
tard au combat. D'Alègre, le comte d'Horn et deux des commandants des
gardes d'Espagne et plusieurs autres y furent pris\,; le troisième
commandant de ces gardes et Chamlin, brigadier, tués avec beaucoup
d'autres, et tout aurait été perdu sans Caraman, qui forma un bataillon
carré de son infanterie avec lequel il arrêta les ennemis et sauva notre
cavalerie\,; il avait onze bataillons. Il en eut sur-le-champ promesse
de la première grand'croix de Saint-Louis vacante et permission de la
porter en attendant, ce que le roi n'avait encore fait pour personne. Le
maréchal de Villeroy, ami de Roquelaure, le protégea en cette occasion
comme il put par son silence\,; mais les armées ne le gardèrent pas\,;
on n'ouït jamais tant crier contre personne\,; et quelque effronté qu'il
frit, il n'osait plus paraître devant les troupes. Le roi en fut très
bien informé et résolut de ne s'en servir jamais. Nous verrons bientôt
qu'il avait une femme qui toute sa vie l'a bien servi, mais qui à la
vérité y était plus que doublement obligée. Les derniers jours de
juillet, n'y ayant que la Dyle entre le maréchal de Villeroy et les
ennemis, ils tentèrent de la passer. Un gros détachement s'était déjà
emparé de deux villages en deçà, lorsque l'électeur et le maréchal s'en
aperçurent et le firent rechasser au delà fort loin et fort
heureusement. Huy, que Gacé avait pris, fut repris par les ennemis.
Artagnan prit Diest tout à la fin de la campagne, et les ennemis Lave et
Saint-Wliet, que le comte de Noyelles fit raser. Les garnisons de ces
trois places furent respectivement prisonnières de guerre\,: ainsi finit
la campagne en Flandre, et les armées se séparèrent tout à la fin
d'octobre.

Je ne puis quitter la Flandre sans rapporter un trait plaisant de la
malignité de M. de Lauzun. On a vu en son temps qu'il ne s'était marié
que pour essayer de se rapprocher de l'ancienne confiance du roi et
entrer avec lui dans ce qui regardait l'Allemagne, où M. le maréchal de
Lorges commandait les armées\,; qu'ayant trouvé tout fermé de ce côté
par un ordre secret au maréchal, il se brouilla avec lui d'une manière
éclatante\,; que la même espérance de rentrer dans quelque chose lui
avait fait presser et terminer le mariage du duc de Lorges avec la fille
de Chamillart, pour tâcher de s'introduire à l'appui de ce ministre\,; à
bout de voie là-dessus, il imagina, se portant à merveille, de faire le
dolent et de demander la permission d'aller aux eaux d'Aix-la-Chapelle.
Il ne persuada à personne qu'il en eût besoin, mais aux sots qui,
ignorant tout, veulent être pénétrants, et de ceux-là il y en a
beaucoup, que ce voyage était mystérieux. Il l'était en effet, mais non
comme ils le pensèrent. Ce n'était pas les eaux qu'il allait chercher,
mais, sous ce prétexte, d'y voir les étrangers qui y abondaient, de
discerner les plus considérables ou les plus importants, de lier avec
eux, d'en tirer ce qu'il pourrait, et, de retour ici, d'en rendre compte
au roi et de faire valoir ses découvertes, en sorte qu'il obtînt ordre
de les suivre, et par ce moyen quelque commerce direct d'affaires avec
le roi. Il fut trompé\,; la guerre occupait trop tout ce qu'il y avait
de considérable et d'important, pour qu'il pût trouver ce qu'il y
cherchait. À ces eaux, il ne vit d'un peu distingué qu'Hompesch, lors
général-major dans les troupes de Hollande, et qui y monta presque à
tout dans la suite, mais qui alors n'était pas du genre de ce que M. de
Lauzun cherchait, quoique à son retour il ne parlât que de lui, faute de
mieux.

Son séjour à Aix-la-Chapelle ne fut pas long, faute de matière. Il
revint par l'armée du maréchal de Villeroy qui le craignait, et qui lui
fit rendre tous les honneurs militaires comme à un seigneur qui avait eu
en chef le commandement de l'armée du roi en Irlande. Il le logea chez
lui pendant trois jours qu'il demeura dans l'armée\,; il lui fit voir
les troupes et il lui donna des officiers généraux pour le promener\,:
les deux armées étaient lors comme en présence, extrêmement proches, et
rien ne les séparait. On s'attendait donc à une bataille qu'on
n'ignorait pas que le roi désirait, et c'était ce qui avait donné envie
à M. de Lauzun d'aller en cette armée. Ceux à qui le maréchal de
Villeroy le remit pour lui faire les honneurs du camp le promenèrent à
vue des grandes gardes de l'armée ennemie\,; et, fatigués de ses
questions et de ses propos, auxquels ils n'étaient pas accoutumés,
l'exposèrent fort au coup de pistolet, et même à être enveloppé, folie
qu'ils eussent bien payée, puisqu'ils l'auraient été avec lui. Il était
très brave, et avec tout son feu il avait une valeur froide qui
connaissait le péril dans tous ses divers degrés, qui ne s'inquiétait
d'aucun, qui reconnaissait tout, remarquait tout, comme s'il eût été
dans sa chambre. Comme il n'avait là qu'à voir et rien à décider ni à
faire, il se divertit à redoubler ses propos et ses questions, à
s'arrêter dans les endroits les plus jaloux, dès qu'il s'aperçut de la
conduite de ces messieurs avec lui, et leur en donna tant et si bien
qu'ils le voulurent écarter plusieurs fois, sentant d'une part leur
indiscrétion, et de l'autre qu'ils avaient affaire à un homme qui les
mènerait toujours au delà de ce qu'ils voudraient.

Revenu à la cour, on s'empressa autour de lui sur la situation des
armées. Il fit le réservé, le disgracié à son ordinaire, l'homme rouillé
et l'aveugle qui ne discerne pas deux pas devant soi. Le lendemain de
son retour il alla chez M\textsuperscript{me} la princesse de Conti
faire sa cour à Monseigneur, qui ne l'aimait point, mais qu'il savait
n'aimer point aussi le maréchal de Villeroy. Monseigneur lui fit force
questions sur la situation des armées et sur ce qui les avait empêchées
de se joindre. M. de Lauzun se défendit en homme qui veut être pressé,
ne cacha pas qu'il s'était fort promené entre les deux armées et fort
près des grandes gardes de celle des ennemis, se rabattant incontinent
sur la beauté de nos troupes, sur leur gaieté de se trouver si proches
et en si beau début, et sur leur ardeur de combattre. Poussé enfin au
point où il voulait l'être\,: «\,Je vous dirai, Monseigneur, puisque
absolument vous me le commandez, lui dit-il, {[}que{]} j'ai très
exactement reconnu le front des deux armées de la droite à la gauche, et
tout le terrain entre-deux. Il est vrai qu'il n'y avait point de
ruisseau, et que je n'y ai vu ni ravins ni chemins creux, ni à monter ni
à descendre\,; mais il est vrai aussi qu'il y avait d'autres
empêchements que j'ai fort bien remarqués. --- Mais quels encore, lui
dit Monseigneur, puisqu'ils n'y avait rien entre-deux\,?» M. de Lauzun
se fit encore battre longtemps là-dessus, répétant toujours les mêmes
empêchements qui n'y étaient pas\,; enfin, poussé à bout, il tira sa
tabatière de sa poche\,: «\,Voyez-vous, dit-il à Monseigneur, il y avait
une chose qui embarrasse fort les pieds, une bruyère, à la vérité point
mêlée de rien, de sec ni d'épineux, peu pressée encore, c'est la vérité,
je ne puis pas dire autrement, mais une bruyère haute, haute, comment
vous dirai-je\,? (regardant partout pour trouver sa comparaison) haute,
je vous assure, haute comme cette tabatière.\,» L'éclat de rire prit à
Monseigneur et à toute la compagnie, et M. de Lauzun à faire la
pirouette et à s'en aller. C'était tout ce qu'il en avait voulu. Le
conte courut la cour et bientôt gagna la ville. Il fut rendu le soir
même au roi. Ce fut le grand merci de M. de Lauzun de tous les honneurs
que le maréchal de Villeroy lui avait fait faire, et sa consolation de
n'avoir rien trouvé à Aix-la-Chapelle de ce qu'il y était allé chercher.

Villars, n'ayant rien à craindre au deçà du Rhin, le passa le 6 août sur
le pont de Strasbourg avec toute sa cavalerie et deux brigades
d'infanterie dont il laissa le reste en deçà, derrière nos lignes sur la
Lauter. Il fit attaquer un poste de six cents hommes qui fut emporté, et
tout ce qui y était, tué ou pris. Il n'en coûta pas une vingtaine
d'hommes, mais on y perdit Dezzedes, officier très entendu et fort brave
homme, d'un esprit agréable et orné, et qui avait été un des six aides
de camp choisis par distinction, envoyés en Italie au roi d'Espagne lors
de la découverte de cette conspiration à son arrivée à Milan, dont j'ai
parlé en son lieu. La subsistance que Villars était allé chercher pour
sa cavalerie ne fut pas longue. Il s'oublia encore moins pour les
contributions, à son ordinaire, mais le prince Louis de Bade ne lui en
laissa pas le temps. Il passa le Rhin, obligea Villars à le repasser
aussi et à faire des marches forcées pour prévenir le mal qu'il en
pouvait recevoir. Là-dessus il amusa le roi d'une bataille avec ses
fanfaronnades accoutumées, mais dont le roi était aussi volontiers la
dupe que de celles de M. de Vendôme. Il arriva pourtant que, n'osant
prêter le collet au prince Louis, à qui il était, dit-il, arrivé du
renfort, il se retira vers Strasbourg et lui laissa toute liberté de
faire le siège de Haguenau.

Peri, très brave Italien, d'esprit et fort entendu, y commandait et s'y
défendit avec tout le courage possible huit jours durant\,; mais, la
place n'étant pas tenable, il battit la chamade au bout de ce temps.
Thungen, qui faisait ce siège, les voulut prisonniers de guerre, sur
quoi le feu recommença. Alors, Peri, qui s'était secrètement ménagé un
trou pour sortir, en fit usage à l'entrée de la nuit suivante avec la
plupart de sa garnison et ordonna à Arling, colonel d'infanterie,
d'amuser quelques heures les ennemis avec cinq cents hommes qu'il lui
laissait, puis de le venir joindre en un lieu qu'il lui marqua, où il
l'attendrait. Arling était Allemand, élevé page de Madame. Elle avait
beaucoup de bonté pour lui, et lui avait obtenu un régiment. Il exécuta
très heureusement et très adroitement les ordres de Peri. Il le joignit
et ils arrivèrent à Saverne avec quinze cents hommes, qui était toute
leur garnison, au moins ce qui en restait en état de les suivre. Cette
ruse de guerre fut extrêmement louée, Peri en fut fait lieutenant
général et Arling brigadier. C'était à la mi-octobre, après quoi les
armées de part et d'autre ne tardèrent pas à se séparer.

M. de Vendôme avait assiégé Chivas, et encore sans pouvoir l'investir,
tant il était incorrigible même par sa propre expérience. M. de Savoie,
campé à Castagnette, communiquait avec la place par un pont sur le Pô
tant qu'il voulait. Le 25 juin, le prince d'Elbœuf, posté avec cinq
cents chevaux derrière un naviglio\footnote{Petit bâtiment.} avec
défense de le passer, ne put résister à l'envie de combattre trois
escadrons des ennemis qu'il avisa de l'autre côté. Il n'avait pas tout
vu\,: ils étaient là quinze cents chevaux. Il passa donc le naviglio\,;
mais, apercevant ce grand nombre triple du sien, il voulut repasser. Il
n'en eut pas le temps. Il fut chargé brusquement\,; il soutint
vaillamment leur effort avec trois cents chevaux qui n'avaient encore pu
repasser, et fut tué d'un coup de pistolet. Ce fut grand dommage par
toute l'espérance qu'il donnait à son âge. Il était fils unique du duc
d'Elbœuf, point encore marié et brigadier. Marcillac, qui a depuis fait
un si triste personnage, mais fortune en Espagne, était avec lui comme
mestre de camp. Il sortait d'exempt des gardes du corps et avait eu
l'agrément d'un régiment. Il avait reçu là dix blessures, dont une dans
le ventre, et eut toutes les mains estropiées et mutilées. Cette triste
échauffourée se passa le 23 juin. Quinze jours après, le grand prieur,
qui par connivence de son frère conservait toujours sa petite armée à
part, prit si mal ses précautions que quatre bataillons de ses troupes
furent enveloppés et pris.

Le roi, en apprenant cette nouvelle par un billet de Chamillart, comme
il regardait jouer au mail à Marly, la dit à ce qui était autour de lui
et ajouta tout de suite que M. de Vendôme joindrait bientôt le grand
prieur, et qu'il raccommoderait tout cela. Cette fascination ne se
pouvait comprendre. De temps en temps Vendôme faisait attaquer quelques
petits postes de rien, quand ils étaient faciles à emporter, quoique ce
succès ne servît de quoi que ce pût être\,; mais pour dépêcher un
courrier, grossir l'objet, et entretenir le roi de ces exploits que lui
seul ne voulait pas voir ce qu'ils étaient. Enfin, il s'y passa, le 16
août, une affaire véritable et où l'opiniâtreté de Vendôme pensa tout
perdre.

Il était auprès de Cassano, d'où le combat prit le nom. Le prince Eugène
crut le lieu propre à l'attaquer. Il marcha à lui sans que Vendôme en
voulût jamais croire les avis très réitérés qu'il en eut, disant
toujours qu'il n'oserait seulement y penser. Enfin Eugène osa si bien,
que Vendôme en vit lui-même les premières troupes. Celles de son frère
étaient avec lui alors. Dans cette précipitation de faire ses
dispositions, il ordonna à son frère de prendre un nombre de troupes et
de les porter où il lui marqua, d'y demeurer avec elles, d'y observer
les mouvements des ennemis, et de faire, suivant l'occasion, ce qu'il
lui prescrivit. L'attaque ne tarda pas de la part du prince Eugène\,:
elle fut vive et heureuse contre des gens mal préparés et à peine
disposés. Vendôme, avec tout son mépris et son audace, crut si bien
l'affaire sans ressource, qu'il poussa à une cassine fort éloignée pour
considérer de là comment et par où il pourrait faire sa retraite avec le
débris de son armée. Pour achever de tout perdre, le grand prieur, dès
le premier commencement du combat, quitta son poste et s'enfuit à une
cassine à plus de demi-lieue de là, emmenant avec lui quelques troupes
pour l'y garder, tellement que son frère, qui comptait sur le poste où
il l'avait envoyé, et sur ce qu'il lui avait ordonné d'y faire, demeura
à découvert de ce côté-là, où le grand prieur, en s'en allant, n'avait
laissé nul ordre. Vendôme mangeait un morceau à cette autre cassine\,;
d'où il considérait quelle pourrait être sa retraite, et il faut avouer
que ce moment à prendre pour manger fut singulièrement étrange, lorsque
Chemerault, lieutenant général des meilleurs, et intimement dans sa
confiance, inquiet au dernier point de le voir si longtemps disparu du
combat, le découvrit mangeant dans la cassine, y courut, et lui apprit
que la brigade de la vieille marine avait fait des prodiges sous Le
Guerchois qui la commandait, lequel, par des efforts redoublés, avait
rétabli le combat. Vendôme eut peine à l'en croire, demanda pourtant son
cheval, poussa avec Chemerault au lieu du combat et l'acheva
glorieusement. Le champ de bataille lui demeura, et le prince Eugène se
retira avec son armée à Treviglio. Il y perdit le comte de Linange, qui
commandait l'armée avant son arrivée, le comte de Guldenstein, un prince
d'Anhalt, un frère de M. de Lorraine qui mourut après de sa blessure, et
un prince de Würtemberg eut le bras cassé et mourut aussi, et beaucoup
de leurs officiers généraux {[}furent{]} blessés. M. de Vendôme eut
dix-huit cents prisonniers et quelques drapeaux. Le combat dura plus de
quatre heures\,; mais la cavalerie n'y eut aucune part. Le Guerchois,
qui avait si bien fait, Mirebaut et quelques autres furent pris\,;
Chaumont, colonel de Soissonnais, gendre de M\textsuperscript{me} de
Jussac, de M\textsuperscript{me} la duchesse d'Orléans, Moriac,
brigadier distingué de cavalerie, qui, impatient de ne rien faire, s'y
mêla de sa personne, le chevalier de Fourbin, maréchal des logis de la
cavalerie, et Vaudray, officier général extrêmement brave et capable y
furent tués. Praslin y faisant des merveilles de soldat et de capitaine,
qui fit marcher la brigade de la marine et qui redonna une nouvelle face
au combat, reçut une blessure mortelle. Ainsi périssent dans des emplois
communs des seigneurs de marque dont le génie supérieur soutiendrait
avec gloire le faix des plus grandes affaires et de guerre et de paix,
si la naissance et le mérite n'étaient pas des exclusions certaines,
surtout quand ils sont joints à un cœur élevé, qui ne peut se frayer un
chemin par des bassesses et qui ne connaît que la vérité. J'ai eu
occasion de parler de lui assez dans ces Mémoires pour me contenter d'en
marquer ici mon extrême regret. J'eus la consolation que les trois ou
quatre mois qu'il dura après sa blessure lui ouvrirent les yeux sur ce
qu'il y a de plus important, et qu'il fit une fin aussi chrétienne et
ferme qu'il avait mené une vie honnête et courageuse. Saint-Nectaire,
chevalier de l'ordre en 1724, apporta au roi la nouvelle de Cassano.

Vendôme, à son ordinaire, manda ses triomphes avec tout ce qui les
pouvait rendre tels. Accoutumé à être cru sur sa parole et à n'être
contredit de nulle part au milieu de tant d'yeux qui voyaient clair et
de tant d'épaules qui se haussaient, il osa mander la perte des ennemis
à plus de treize mille hommes, et la nôtre à moins de trois mille. La
vérité bien reconnue fut pourtant que la perte fut du moins égale, et
que la suite de ce combat fut totalement nulle et sans en tirer le
moindre avantage, pas même de commodités de guerre. Cet exploit
néanmoins retentit à la cour et à la ville comme un avantage le plus
complet, le plus décisif, le plus dû à la prudence, à la vigilance, à la
valeur et à la capacité de Vendôme. On se garda bien de parler de
cassine, et en Italie d'en faire mention. On ne sut ce fait que par le
retour des officiers généraux et particuliers, de ceux qui eurent
permission de faire un tour à Paris ou chez eux. Les uns le contèrent,
les autres l'écrivirent à leurs amis de leur province, se croyant là en
sûreté contre la poste de l'armée d'Italie, et tous ne se pouvaient
lasser d'admirer que leur général pût avoir recueilli tant
d'applaudissements de ce qui, en tout genre, lui méritait tant de blâme.

Dès qu'après le combat il revit son frère, il ne put s'empêcher de lui
demander pourquoi il avait quitté le poste dont il l'avait chargé\,;
quoiqu'il le fît avec mesure, l'orgueilleux cadet, qui se sentait sans
excuse, ne le paya que d'emportement devant tout le monde. Vendôme, avec
qui il ne conservait presque que de l'extérieur depuis qu'il lui avait
ôté, et à l'abbé de Chaulieu, le pillage de ses affaires, et qui lui
avait causé tant d'inconvénients toute cette campagne, se trouva hors
d'état, et peut-être de volonté de l'excuser pour se délivrer d'un si
fâcheux second. La désobéissance était formelle, la poltronnerie
publique par sa fuite, et le crime complet par la licence d'emmener des
troupes pour s'en faire garder dans la cassine si éloignée où il s'était
relaissé. La brouillerie des deux frères éclata. Le grand prieur n'osant
plus se montrer redoubla de crapule obscure\,; mais peu après il reçut
un ordre de quitter l'armée et de repasser les monts. Il s'en vint droit
à Lyon, puis, par permission qu'il dut à son frère, à sa maison de
Clichy, près de Paris, d'où il prétendit être admis devant le roi à se
justifier. Il le demanda avec une hauteur et une audace qu'avait
nourries l'expérience du pouvoir de sa naissance et de tout ce qu'elle
lui avait fait pardonner. Pour cette fois il se trompa. Le roi ne voulut
ni le voir ni l'entendre, et ne le revit jamais. Plus outré du
châtiment, quelque léger qu'il fût, que honteux de ce qui l'avait
mérité, il retourna à Lyon et avec la permission du roi s'en alla à Rome
et y demeura quelque temps. Lassé d'y vivre dans le commun, sans pouvoir
parvenir, dans un pays si réglé pour le cérémonial, à aucune de ses
prétentions, il en sortit. Il s'accrocha à la marquise de Richelieu qui
courait le monde depuis quelque temps. Ils passèrent ensemble quelque
temps à Gènes, d'où il revint en France, y vit son frère à la
Ferté-Alais, et sans être entré dans Paris, s'en alla à
Châlons-sur-Saône, qui lui fut fixé pour exil, où il vécut dans l'excès
de ses débauches et de son obscurité ordinaire. D'ici à la régence on
n'en entendra plus parler.

Cette race demi-mazarine me fait souvenir de la connétable Colonne que
le roi avait eu en sa jeunesse tant d'envie d'épouser, qui ne
contraignit pas ses mœurs à Rome, ni de courir le bon bord du vivant et
surtout depuis la mort de son mari. C'était la plus folle, et toutefois
la meilleure de ces Mazarines. Pour la plus galante on aurait peine à
décider, excepté la mère de M. de Vendôme et du grand prieur, qui mourut
trop jeune dans la première innocence des mœurs. Cette connétable
s'avisa cette année de venir d'Italie débarquer en Provence. Elle y fut
plusieurs mois sans permission d'approcher plus près. Enfin elle
l'obtint à la sollicitation de sa famille pour la voir sans l'aller
chercher si loin, à condition qu'elle ne mettrait pas le pied dans
Paris, beaucoup moins à la cour. Elle vint à Passy dans une petite
maison du duc de Nevers, son frère. Hors sa famille, elle ne connaissait
plus personne. Tout était renouvelé depuis qu'elle était partie de
France pour s'aller marier avant le mariage du roi. L'ennui lui prit
d'être si mal accueillie, et d'elle-même elle s'en retourna assez
promptement.

Il arriva en ce temps-ci une aventure imprudente à un de mes amis qui me
donna de la peine, et qui serait fade à rapporter ici, sans les suites
tardives auxquelles elle donna commencement. L'abbé de Mailly était
extrêmement de mes amis\,; nos maisons, souvent alliées, avaient dans
tous les temps été unies. Son père, plus connu par l'hôtel qu'il bâtit
au bout du pont Royal que par une vie plus marquée, quoique extrêmement
longue, et sa mère que son long nez faisait appeler \emph{la bécasse},
et qui avait, à force de successions et de procès gagnés, comblé cette
maison de biens, ne bougeaient de chez mon père pendant sa vie, et
depuis de chez ma mère. L'abbé de Mailly, frère du marquis de Nesle, tué
devant Philippsbourg en 1688, et du comte de Mailly dont la dame
d'atours de M\textsuperscript{me} la duchesse de Bourgogne était femme,
avait été mis jeune à Saint-Victor avec un autre de ses frères, qui,
plus pieux et plus aisé à réduire, y avait pris l'habit, était devenu
prieur, puis évêque de Lavaur. L'abbé de Mailly, qui n'avait jamais
voulu tâter de la moinerie, n'avait pas plus d'inclination à la
profession ecclésiastique. Sa mère l'y força et lui laissa percer les
coudes dans l'extérieur de ce couvent jusqu'à ce qu'il fût prêtre. On
peut juger quel prêtre ce fut, et quelles études il fit\,; mais il avait
de l'honneur, et fit de nécessité vertu. Il eut enfin une méchante
petite abbaye, une place d'aumônier du roi et une autre abbaye ensuite
encore fort chétive. Ce n'était pas un homme de beaucoup d'esprit, mais
il n'en manquait pas, avait des vues et une vaste ambition, était suivi
dans toutes ses idées, et fort attentif à ne se barrer sur rien et à
s'aplanir les chemins à tout. Il rouit longtemps dans ce petit état
enviant celui des soldats à qui il voyait monter la garde, à ce qu'il
m'a souvent avoué. Dès lors il pensait au cardinalat, il faisait sa cour
à Saint-Germain pour s'en frayer la route à la nomination. Je me moquais
de lui, d'idées si éloignées de sa portée. Il me répondait qu'en
dirigeant toute sa conduite suer un même projet, et ne s'en lassant
point, souvent on y réussissait. Enfin il fut nommé à l'archevêché
d'Arles où je le servis fort en excitant sa belle-sœur, et par d'autres
amis. C'était un pas fort extraordinaire que celui d'être fait
archevêque sans avoir été évêque, et je ne sais que l'archevêque de
Bourges, Gesvres, à qui cela fût arrivé auparavant lui, encore par les
circonstances que j'ai rapportées en leur temps. Mon ami fut moins
touché de se voir sorti de l'état commun où il était, et d'être tout à
coup archevêque, que de l'être d'Arles. Bordeaux qui fut donné le même
jour à Besons, évêque d'Aire, mort depuis archevêque de Rouen, ne lui
aurait pas plu de même.

La position d'Arles, par rapport à l'Italie et à Avignon, le charma. Il
se proposa bien d'en tirer tout le parti possible, et il me le confia.
Dans ses vues il voulut joindre le mérite du courtisan avec celui de la
résidence. Il dit au roi, en prenant congé, qu'il ne pouvait se résoudre
à être longtemps sans le voir, et qu'il le suppliait de trouver bon
qu'il vînt passer trois semaines tous les ans à Versailles, qui serait
le seul objet de son voyage. En effet, il n'y manqua point et ne
s'arrêtait point à Paris. Il débarquait chez moi\,; je le couchais dans
un trou d'entresol qui me servait de cabinet, et le roi lui savait le
meilleur gré du monde d'une conduite qui lui manquait un attachement
dont il était jaloux, sans entamer les devoirs de l'épiscopat et de la
résidence\,; et l'archevêque en profitait pour voir par lui-même tous
les ans ce que les lettres ne lui pouvaient pas apprendre. Son premier
soin, en arrivant à Arles, fut de prévenir le vice-légat d'Avignon de
toutes sortes de civilités et de devoirs. Le vice-légat y répondit avec
empressement\,: c'était Gualterio qui mourait d'envie de venir ici
nonce. Il avait dressé ses batteries à Rome pour cela, et il faisait de
ce côté-ci tout ce qu'il croyait l'y pouvoir faire réussir. Les trois
grandes couronnes, c'est-à-dire l'empereur, le roi et le roi d'Espagne,
ont le privilège que le pape leur propose trois ou quatre sujets, et
celui qu'ils choisissent est nommé à la nonciature auprès d'eux, de
laquelle il est comme certain qu'ils ne retournent que cardinaux.

Gualterio avait infiniment d'esprit, et un esprit réglé, sensé, sage,
prudent, mais gai et souple, beaucoup d'agrément et de douceur, avec
cela beaucoup d'érudition, une grande connaissance du monde et une fort
aimable conversation, avec toute l'aisance d'un homme accoutumé aux
grandes cours, et à la meilleure compagnie\,; il la faisait lui-même, et
sa conversation était charmante et souvent instructive sur une infinité
de choses. Ce qu'il avait de plus recommandable, mais de plus singulier
pour un homme de son pays et de son état, c'était la probité, la vérité,
la fidélité et la candeur, avec tout l'art nécessaire pour les conserver
entières dans le maniement des affaires et parmi le commerce du monde.
Mieux informé de notre cour que la plupart de ceux qui la composaient,
il répondit aux avances de son voisin en homme qui connaissait ce que sa
belle-sœur était à M\textsuperscript{me} de Maintenon, tellement qu'à
force de civilités, de visites, de désir de se plaire l'un à l'autre,
ils lièrent ensemble une véritable amitié. Au bout de deux ou trois ans,
Gualterio eut la nonciature de France. L'archevêque d'Arles me le
recommanda fort. Il lui avait parlé de moi, et le prélat italien, qui
n'ignorait rien de notre cour, même avant d'y arriver, ne désirait pas
moins que l'archevêque de pouvoir lier avec un homme qu'il savait si
étroitement uni avec le duc de Beauvilliers, le chancelier et
Chamillart, et avec d'autres personnes considérables. Alors encore les
nonces conservaient la morgue de refuser chez eux la main aux ducs et
aux princes étrangers, tandis qu'ils la donnaient sans difficulté aux
secrétaires d'État. Les ducs et les princes étrangers ne les voyaient
donc jamais chez eux, et ce ne fut que depuis la nonciature de
Gualterio, que cette prétention finit, que les nonces ne firent plus
difficulté de donner la main chez eux, et que les ducs et les princes
étrangers les virent. Gualterio et moi ne nous visitâmes donc d'abord
que par des messages, et quand il venait les mardis à Versailles, nous
nous y voyions dans les appartements. Nous nous plûmes réciproquement, à
moi parce que je lui trouvai bientôt de quoi plaire, à lui parce qu'il
avait résolu de devenir de mes amis. Quand nous nous fûmes un peu plus
connus, cette gêne de lieu tiers nous fatigua. Il me proposa son
escalier secret et qu'à porte fermée il me recevrait sans façon. Ce
\emph{mezzo termine} ne m'accommoda pas, et je le lui dis franchement.
Cela lui fit prendre son parti de venir chez moi et à Paris où je
n'étais presque point, et à Versailles toutes les fois qu'il y venait.
Du commerce fréquent nous vînmes à l'amitié et à la confiance qui a duré
entre nous jusqu'à sa mort, avec un commerce réglé de lettres toutes les
semaines depuis son départ, et presque toujours en chiffre.

M. d'Arles avait profité de la facilité du commerce par mer de la
Provence avec l'Italie. Il s'était servi à Rome de moines et
d'émissaires obscurs, par le moyen desquels il était parvenu à se mettre
bien avec les principaux ministres et avec le pape même. Il parvint
jusqu'à se procurer des occasions de lui écrire, d'en recevoir des
marques d'estime et de bonté, enfin d'en recevoir des brefs, et peu à
peu de se faire considérer comme un prélat distingué par son siège et
par sa naissance, dont l'attachement méritait d'être ménagé et qui
pouvait raisonnablement aspirer à la pourpre. En ces temps-là, les
cabales de la constitution \emph{Unigenitus} n'étaient pas nées et
n'étaient pas corrompu le clergé ni la politique si sage et si constante
de la cour. Elle regardait comme un crime tout commerce direct d'un
évêque avec Rome. Ce qui regardait les bénéfices, ils le traitaient par
des banquiers\,; sur toute autre matière ils étaient obligés de passer
par la permission du roi et par le secrétaire des affaires étrangères.
Écrire directement au pape, à ses ministres ou à des personnes en place
de cette cour, ou en recevoir des lettres, sans qu'à chacune le roi et
son secrétaire d'État sût pourquoi et l'eût permis, c'était un crime
d'État qui ne se pardonnait point et qui était puni, de sorte que
l'usage s'en était entièrement aboli. M. d'Arles avait donc mené ce
commerce fort secrètement.

Le nonce et moi étions dans cette confidence. Nous l'avions souvent
averti du danger, mais le désir du cardinalat et les espérances que
cette cour fait si aisément naître et remplit si difficilement, étaient
des aiguillons auxquels il ne put résister. Le pape, dans une lettre
qu'il lui fit écrire, lui parla de saint Trophime, l'apôtre et le
premier évêque d'Arles. L'archevêque lui écrivit là-dessus pour lui en
faire désirer des reliques\,; il n'y réussit que trop. Le pape lui
écrivit lui-même et lui en demanda. L'archevêque lui en envoya avec une
belle lettre et il en reçut un bref de remerciements. Détacher des
reliques du principal corps saint qui repose à Arles et ce commerce
subséquent si près à près, ne put demeurer secret\,; l'affaire fut
éventée. Torcy, par ordre du roi, en écrivit très fortement à
l'archevêque, et en parla au nonce sur le même ton, qui vint tout
courant me le conter. Nous eûmes grand'peine à le tirer d'affaire\,; il
en fut pourtant quitte pour une dure réprimande et pour un ordre bien
exprès de prendre garde de plus avoir aucun commerce à Rome, sous peine
de l'indignation du roi. L'archevêque fit l'ignorant, le piteux, le
désespéré d'avoir déplu au roi pour une bagatelle qu'il avait crue
innocente, protesta merveilles\,; mais il ne quittait pas prise
aisément. Il se croyait avancé à Rome pour ses espérances\,; c'était les
perdre que de cesser de les cultiver. L'excès d'ambition lui fit
continuer son commerce. Il essaya de se faire un mérite à Rome de ce
qu'il venait de lui arriver, mais il prit de meilleures précautions pour
se cacher, et si bonnes qu'il ne fut plus découvert. Il eut peine
pourtant à effacer l'impression que le roi avait prise\,; le secours
quoique assez froid de sa belle-sœur en vint à bout par
M\textsuperscript{me} de Maintenon.

La Feuillade avait eu ordre de mener en Lombardie dix bataillons et
trois escadrons de dragons. Il n'avait plus rien à faire en Savoie et il
allait en pays ami. Vendôme, que son beau-père servait si bien, n'avait
garde de lui faire sentir le poids de son commandement. Il envoya
d'Estaing au-devant de lui, avec trois mille cinq cents chevaux et vingt
compagnies de grenadiers, qui chassèrent quelques troupes ennemies
postées au pont de Lens sur la Sture pour empêcher la jonction. On fit
fort valoir la marche de La Feuillade, suivi trois jours durant par
mille chevaux qui ne l'attaquèrent point. Il n'eut pas la peine d'aller
jusqu'en Lombardie. Vendôme le chargea de la continuation du siège de
Chivas. Trois semaines après, M. de Savoie abandonna Chivas, Castagnette
et toutes les hauteurs qu'il occupait entre ces places, pour se retirer
vers Turin avec le peu de troupes qu'il avait là. Quelques jours
auparavant, La Feuillade avait fait pousser quelque cavalerie entre le
Melo et la Sture, pour déposter un petit camp, qui prit la fuite dès
qu'il vit la tête de ses troupes. Il manda qu'on leur avait tué trois
cents hommes et pris cinquante officiers ou cavaliers, six étendards et
deux paires de timbales, sans y avoir perdu personne, et que c'était
cette action qui avait fait prendre à M. de Savoie le parti qu'il venait
de prendre. Lambert, conduit par Chamillart, apporta ces nouvelles au
roi à Marly, qu'on fit fort valoir. Ces merveilles précédèrent de
dix-huit jours le combat de Cassano.

L'archiduc, ennuyé d'une campagne assez stérile jusqu'alors, quoique
fort supérieur à l'armée d'Espagne sur les frontières de Portugal, où
tout s'était passé en prises et reprises de postes et de petites places,
mécontent d'ailleurs de la cour de Portugal, fut conseillé d'aller
donner vigueur à ses amis de Catalogne et d'Aragon, de s'embarquer sur
la flotte anglaise et hollandaise, et d'aller tenter Barcelone. Il y fit
mettre pied à terre, le 23 août, à quinze bataillons et plus de mille
chevaux, qui furent aussitôt joints par six mille révoltés de Vigo, et
ils envoyèrent, quinze vaisseaux devant Palamos, cinq mille autres du
royaume de Valence allèrent les grossir, et ils ouvrirent la tranchée
devant Barcelone, le 1er septembre. Le vice-roi de Catalogne mit dehors
Rose, gouverneur de la ville, et le major, fort soupçonné
d'intelligences avec l'archiduc. La garnison était nombreuse, mais de
mauvaises troupes.

Il arriva une fâcheuse affaire à l'armée de Flandre entre Surville et La
Barre. Étant à table, et Surville pris de vin, il maltraita cruellement
La Barre de paroles. La compagnie qui les vit se lever se jeta
entre-deux, chose fort ordinaire et dont ordinairement aussi elle se
repent après. Malgré cela, ils se rapprochèrent, et La Barre crut avoir
essuyé quelque mainmise dans ces moments si peu mesurés, et où tout est
pêle-mêle. Surville, ayant cuvé son vin, mit en usage tout ce qu'il put
honnêtement pour satisfaire La Barre et finir cette affaire. Ce fut en
vain. L'électeur de Bavière, de l'avis du maréchal de Villeroy, envoya
Surville à Bruxelles, et mit La Barre aux arrêts. Sur ville était frère
cadet d'Hautefort, tous deux lieutenants généraux, mais de réputation
fort différente. Rien de plus corrompu que les mœurs de Surville, rien
de plus équivoque que son courage, personne plus grossièrement borné. On
a vu en son lieu de quelle façon il épousa une fille du maréchal
d'Humières, veuve de Vassé. Malgré tant de choses exclusives, je ne sais
par quelle intrigue il avait eu le régiment d'infanterie du roi, place
qui donnait des rapports continuels immédiatement avec lui, parce que le
roi faisait sa poupée de son régiment, entrait dans tous les détails
comme un simple colonel, et le distinguait en toutes manières\,: c'était
donc une source de privances, de grâces et d'utilité\,; car Surville en
tirait fort gros, et il était de tous les Marlys.

La Barre était un simple gentilhomme pauvre et de fortune,
capitaine-lieutenant de la compagnie colonelle du régiment des gardes,
et par conséquent ayant brevet, nom et rang de capitaine aux gardes. Il
était très malvoulu dans son corps, et peu accueilli ailleurs. Sa
réputation sur le courage n'était pas meilleure que celle de Surville\,;
mais il montra depuis qu'on s'y était fort trompé. C'était un compagnon
d'esprit, de manège, de souterrains, ami de plusieurs garçons bleus les
plus intérieurs et des valets principaux du roi. Accusé de plus de lui
tout rapporter, et ce qui en fortifiait la pensée, c'était de le voir
bien traité et distingué par le roi, fort au-dessus d'un homme de son
état. Le roi qui avait de la bonté pour ces deux hommes, et qui vit la
difficulté qui se rencontrerait à les accommoder, même au tribunal
naturel des maréchaux de France, voulut bien pour la première fois de sa
vie entre des personnes comme ils étaient s'en charger lui-même. Il fit
mettre Surville en prison pour en sortir peu après, aller demander
pardon à l'électeur, dans l'armée et le voisinage duquel la querelle
était arrivée, et faire en sa présence satisfaction à La Barre. Pendant
tous ces procédés, la gloire des Hautefort s'offensa. Ils tinrent des
propos de hauteur qui gâtèrent tout. La Barre cria à la nouvelle injure,
tellement qu'Arras fut donné pour prison à Surville, jusqu'à là fin de
la campagne que La Barre acheva à l'armée, pour finir cette affaire
ensuite par le roi seul de manière à n'y laisser aucunes suites. Nous
les verrons l'année suivante telles que Surville demeura perdu. Secouru
depuis et remis à flot par la générosité du maréchal de Boufflers, il se
perdit de nouveau lui-même et sans ressource\,; mais il n'est pas temps
d'en parler.

L'affaire du banquillo fit en ce temps-ci un grand bruit en Espagne, et
donna ici de l'inquiétude. Je l'ai expliquée d'avance (t. III, p.~287)
lorsque je me suis étendu sur les grands d'Espagne\,; je n'en répéterai
donc rien ici. M\textsuperscript{me} des Ursins, qui aperçut de loin ce
petit orage se former en arrivant à Madrid, saisit la conjoncture de
disposer de la charge de majordome-major. On a vu la juste prétention du
duc d'Albe fort appuyée du roi, et la raison qui y rendait la princesse
des Ursins contraire. Elle prit donc cette occasion de la donner à un
seigneur actuellement sur les lieux, qui, par la considération qu'elle
lui donnait parmi les grands dont elle le faisait comme le chef, les pût
ramener, et que lui-même, gagné par cet honneur, se rangeât pour le roi
dans cette affaire, services qui ne se pouvaient tirer d'un absent. Le
connétable de Castille avait été peu compté depuis l'avènement de
Philippe V à la couronne d'Espagne. On l'estimait peu, on le soupçonnait
d'être un peu autrichien. Il croyait avoir reçu un grand dégoût sur sa
prétention de commander les armées par son titre de connétable. La
campagne de Portugal n'avait pas bien basté\,; on avait perdu Gibraltar,
la Catalogne et les provinces voisines étaient plus que suspectes\,;
toutes ces circonstances persuadèrent la princesse des Ursins de ramener
un aussi grand seigneur et si distingué que l'était le connétable de
Castille, et lui fit donner la charge de majordome-major, qui consentit
contre son droit et l'usage jusqu'alors observé qu'au lieu de lui porter
tous les soirs les clefs des portes du palais, elles le seraient au
capitaine des gardes du corps en quartier, charge jusqu'alors inconnue
en Espagne, et fit par cette adresse approuver au roi que sa
recommandation en faveur du duc d'Albe n'eût pas lieu.

Le roi partit le 22 septembre pour Fontainebleau par Sceaux où il alla
de Marly, et y séjourna un jour. Le roi d'Angleterre y arriva le ter
octobre, et s'en retourna à Saint-Germain le 12. La reine, qui était
fort incommodée d'un mal au sein dont on craignait de funestes suites,
qu'il n'eut pourtant pas, ne put aller à Fontainebleau cette année. En
ce même temps, Desmarets maria une de ses filles au fils de Bercy,
maître des requêtes, extrêmement riche.

Le prince de Bournonville mourut à Bruxelles. C'était un homme
d'honneur, fort brave, qui avait beaucoup de savoir, et qui ne manquait
point d'esprit\,; mais d'un esprit tout à fait désagréable. Il était
riche, fils et petit-fils de deux hommes qui avaient fort figuré sous la
maison d'Autriche. Il était veuf, avec un fils et deux filles, d'une
sœur du duc de Chevreuse du second lit\,; et la maréchale de Noailles et
lui étaient enfants des deux frères, laquelle l'aimait à cause de cette
proximité. J'en eus beaucoup dans la suite avec ses enfants, car sa
fille aînée épousa le duc de Duras, et la veuve de son fils mon fils
aîné. Avec tous ses proches, Bournonville ne parvint à rien et servit
toute sa vie. Il était sous-lieutenant des gens d'armes sous le prince
de Rohan, cousin germain de sa femme. Il n'avait aucun rang ni honneurs.

Virville mourut en même temps, du nom de Groslée, illustre en Dauphiné.
Il avait été capitaine de gendarmerie, brave et fort bon officier, mais
perdu de gouttes qui l'obligèrent à quitter et qui à la fin le tuèrent.
C'était un fort aimable homme, de beaucoup d'esprit, et fort orné, et de
très bonne compagnie, fort honnête homme aussi, et fort aimé et
considéré. Le maréchal de Tallard avait épousé sa sœur\,; et lui, qui
voulait tout laisser à son fils unique, donna pour rien sa fille à
Senozan, homme de rien, dès lors fort riche, et qui le devint énormément
depuis. Il arriva ce qu'on voit ordinairement de ces mariages\,: le fils
de Virville le survécut peu, la veuve du même Virville hérita de ses
frères et de ses oncles\,; il se forma de tout cela une succession
prodigieuse qui tomba à la femme de Senozan.

Usson, lieutenant général distingué, dont il a été mention ici plus
d'une fois, mourut aussi à Marseille\,; il commandait dans les pays de
Nice et Villefranche. C'était un petit homme, fait comme un potiron,
mais plein d'esprit, de valeur, et de talent pour la guerre. Il n'était
point marié\,; Bonrepos était son frère aîné.

Pontchartrain se tint exactement ce qu'il s'était promis. Le comte de
Toulouse et le maréchal de Cœuvres allèrent à Toulon, comptant monter
une flotte. Tantôt un retardement, tantôt une difficulté, tantôt un
manquement de quelque chose\,; bref, tous deux demeurèrent au port, et
la flotte ennemie maîtresse de la mer. L'amiral, pour charmer son ennui,
alla visiter Antibes et se promener par les ports du pays, et revint à
Fontainebleau, où le maréchal de Cœuvres aussi peu content que lui ne
tarda pas à le suivre. Pontchartrain, qui avait de longue main prévenu
le roi sur la dépense d'une puissante flotte, sur le grand nombre de
gros vaisseaux des Anglais et des Hollandais joints ensemble, sur le
danger de la personne du comte de Toulouse si sa valeur était écoutée,
s'en tira à joints pieds et se moqua d'eux tout à son aise, au grand
malheur de Barcelone et des extrémités dont cette perte fut suivie,
comme on les verra en leur temps.

Ce fut à ce retour du comte de Toulouse qu'il acheta d'Armenonville la
terre de Rambouillet, à six lieues de Versailles, près de Maintenon,
dont le comte fit un duché-pairie, érigé pour lui, et une terre
prodigieuse par les acquisitions qu'il y fit dans la suite.
Armenonville, qui ne vendait que par respect, eut en pot-de-vin, pour
lui et pour son fils après lui, l'usage du château et des jardins de la
Muette\footnote{Saint-Simon écrit toujours la \emph{Meute} en parlant du
  château qu'on appelle aujourd'hui la \emph{Muette}. Nous prévenons,
  une fois pour toutes, que nous avons conservé l'orthographe ordinaire.}
et du bois de Boulogne, que le roi détacha de la capitainerie de
Catelan, en l'en dédommageant.

\hypertarget{chapitre-iv.}{%
\chapter{CHAPITRE IV.}\label{chapitre-iv.}}

1705

~

{\textsc{Mort de la première présidente Lamoignon\,; sa famille.}}
{\textsc{- Caractère et fortune du premier président Lamoignon.}}
{\textsc{- Corruption des premiers présidents successeurs de
Bellièvre.}} {\textsc{- Catastrophe singulière de Fargues.}} {\textsc{-
Mort et singularités de Ninon, dite M\textsuperscript{lle} de
L'Enclos.}} {\textsc{- Mort de Rossignol.}} {\textsc{- Inquisition de ce
prince.}} {\textsc{- Mort du comte de Tonnerre.}} {\textsc{- La
Feuillade proposé par le roi à Chamillart pour faire en chef le siège de
Turin.}} {\textsc{- Gratitude et grandeur d'âme de Vauban.}} {\textsc{-
Vendôme grand courtisan.}} {\textsc{- Siège de Turin différé.}}
{\textsc{- Darmstadt tué devant le mont Joui.}} {\textsc{- Lerida et
Tortose saisis par les Catalans révoltés.}} {\textsc{- Siège de Badajoz
levé par les ennemis.}} {\textsc{- Barcelone rendu à l'archiduc.}}
{\textsc{- La garnison prisonnière de guerre.}} {\textsc{- Retour de
Fontainebleau par Villeroy et Sceaux.}} {\textsc{- Couronnement de
Stanislas en Pologne.}} {\textsc{- Mort du fameux Tekeli.}} {\textsc{-
Prises de mer\,; Saint-Paul tué.}} {\textsc{- Cruelle méprise de La
Feuillade.}} {\textsc{- Augmentation des compagnies.}} {\textsc{-
Nouveaux régiments.}} {\textsc{- Force milice.}} {\textsc{- Idées de nos
ministres bien différentes sur la paix.}} {\textsc{- Aguilar à Paris\,;
sa mission, son caractère, sa fortune.}} {\textsc{- Ordres d'Espagne
devenus compatibles avec ceux de la Toison et du Saint-Esprit.}}
{\textsc{- Ronquillo gouverneur du conseil de Castille.}} {\textsc{- Duc
de Noailles en Roussillon.}} {\textsc{- Mort des deux fils du duc de
Beauvilliers.}} {\textsc{- Piété du père et de la mère.}} {\textsc{-
Jésuites emportent la cure de Brest devant le roi.}} {\textsc{- Retour
de Marsin, Villars et Villeroy.}} {\textsc{- Surville à la Bastille.}}
{\textsc{- Roquelaure tâche de se justifier au roi\,; sa femme.}}
{\textsc{- Mariage du fils aîné de Tessé avec la fille de Bouchu, du duc
de Duras avec M\textsuperscript{lle} de Bournonville, de Listenais avec
une fille de la comtesse de Mailly.}} {\textsc{- Folies de la duchesse
du Maine.}} {\textsc{- Duc de Berry délivré de ses gouverneurs.}}
{\textsc{- Montmélian rendu par les ennemis.}} {\textsc{- Aventure
étrange de l'évêque de Metz.}}

~

Deux personnes fort différentes moururent en ce même temps\,: la
première présidente Lamoignon et Ninon. M\textsuperscript{me} de
Lamoignon (car ces avocats renforcés et qui, du barreau où ils gagnaient
leur vie il n'y a pas longtemps, sont devenus des magistrats
considérables, ont pris le \emph{de}), M\textsuperscript{me} de
Lamoignon, dis-je, était Potier, fille du secrétaire d'État Ocquerre,
frère de cet évêque de Beauvais qui pensa quelques jours être premier
ministre à la mort de Louis XIII, et que le cardinal Mazarin culbuta.
Elle était sœur du père du président de Novion, qui succéda à son mari à
la place de premier président, et mère de Lamoignon, président à mortier
à Paris, de Bâville, conseiller d'État, intendant ou plutôt roi de
Languedoc, de M\textsuperscript{me} de Broglio, dont le mari et le
second fils sont devenus depuis si peu maréchaux de France, et de la
défunte femme d'Harlay qui succéda à Novion son cousin germain, lorsque,
comme je l'ai rapporté, il fut chassé en 1689 de la place de premier
président. Lamoignon, beau, agréable, et sachant fort le monde et
l'intrigue, avec tous les talents extérieurs, avait brillé au conseil
dans la place de maître des requêtes. On a vu comment, par l'adresse des
ministres qui craignaient l'humeur de Novion, il refusa, à l'instigation
de sa maîtresse à qui ils donnèrent gros, la place de premier président,
vacante en 1658, par la mort de Bellièvre, et y portèrent Lamoignon. Les
grâces de sa personne, son affabilité, le soin qu'il prit de se faire
aimer du barreau et des magistrats, une table éloignée de la frugalité
de ses prédécesseurs, son attention singulière à capter les savants de
son temps, à les assembler chez lui à certains jours, à les distinguer,
quels qu'ils fussent, lui acquirent une réputation qui dure encore, et
qui n'a pas été inutile à ses enfants. Il est pourtant vrai qu'à lui
commença la corruption de cette place qui ne s'est guère interrompue
jusqu'à aujourd'hui. Pour Lamoignon j'en raconterai ici un seul trait,
parce qu'il est historique et curieux.

Il se fit à Saint-Germain une grande partie de chasse. Alors c'étaient
les chiens, et non les hommes, qui prenaient les cerfs\,; on ignorait
encore ce nombre immense de chiens, de chevaux, de piqueurs, de relais
et de routes à travers les pays. La chasse tourna du côté de Dourdan, et
se prolongea si bien que le roi s'en revint extrêmement tard et laissa
la chasse. Le comte de Guiche, le comte depuis duc du Lude, Vardes, M.
de Lauzun qui me l'a conté, je ne sais plus qui encore, s'égarèrent, et
les voilà à la nuit noire à ne savoir où ils étaient. À force d'aller
sur leurs chevaux recrus, ils avisèrent une lumière\,; ils y allèrent,
et à la fin arrivèrent à la porte d'une espèce de château. Ils
frappèrent, ils crièrent, ils se nommèrent, et demandèrent
l'hospitalité. C'était à la fin de l'automne, et il était entre dix et
onze heures du soir. On leur ouvrit. Le maître vint au-devant d'eux, les
fit débotter et chauffer, fit mettre leurs chevaux dans son écurie, et
pendant ce temps-là leur fit préparer à souper, dont ils avaient grand
besoin. Le repas ne se fit point attendre\,; il fut excellent, et le vin
de même, de plusieurs sortes. Le maître poli, respectueux, ni
cérémonieux, ni empressé, avec tout l'air et les manières du meilleur
monde. Ils surent qu'il s'appelait Fargues, et la maison Courson\,;
qu'il y était retiré\,; qu'il n'en était point sorti depuis plusieurs
années\,; qu'il y recevait quelquefois ses amis, et qu'il n'avait ni
femme ni enfants. Le domestique leur parut entendu, et la maison avoir
un air d'aisance. Après avoir bien soupé, Fargues ne leur fit point
attendre leur lit. Ils en trouvèrent chacun un parfaitement bon, ils
eurent chacun leur chambre, et les valets de Fargues les servirent très
proprement. Ils étaient fort las et dormirent longtemps. Dès qu'ils
furent habillés, ils trouvèrent un excellent déjeuner servi, et au
sortir de table, leurs chevaux prêts, aussi refaits qu'ils l'étaient
eux-mêmes. Charmés de la politesse et des manières de Fargues, et
touchés de sa bonne réception, ils lui firent beaucoup d'offres de
service, et s'en allèrent à Saint-Germain. Leur égarement y avait été la
nouvelle\,; leur retour et ce qu'ils étaient devenus toute la nuit en
fut une autre.

Ces messieurs étaient la fleur de la cour et de la galanterie, et tous
alors dans toutes les privances du roi. Ils lui racontèrent leur
aventure, les merveilles de leur réception, et se louèrent extrêmement
du maître, de sa chère et de sa maison. Le roi leur demanda son nom\,;
dès qu'il l'entendit\,: «\, Comment Fargues, dit-il, est-il si près
d'ici\,?» Ces messieurs redoublèrent de louanges, et le roi ne dit plus
rien. Passé chez la reine mère, il lui parla de cette aventure, et tous
deux trouvèrent que Fargues était bien hardi d'habiter si près de la
cour, et fort étrange qu'ils ne l'apprissent que par cette aventure de
chasse, depuis si longtemps qu'il demeurait là.

Fargues s'était fort signalé dans tous les mouvements de Paris contre la
cour et le cardinal Mazarin. S'il n'avait pas été pendu, ce n'avait pas
été faute d'envie de se venger particulièrement de lui\,; mais il avait
été protégé par son parti, et formellement compris dans l'amnistie. La
haine qu'il avait encourue, et sous laquelle il avait pensé succomber,
lui fit prendre le parti de quitter Paris pour toujours, afin d'éviter
toute noise, et de se retirer chez lui sans faire parler de lui, et
jusqu'alors il était demeuré ignoré. Le cardinal Mazarin était mort\,;
il n'était plus question pour personne des affaires passées\,; mais,
comme il avait été fort noté, il craignait qu'on lui en suscitât quelque
autre nouvelle, et pour cela vivait fort retiré et fort en paix avec
tous ses voisins, fort en repos des troubles passés, sur la foi de
l'amnistie et depuis longtemps. Le roi et la reine sa mère, qui ne lui
avaient pardonné que par force, mandèrent le premier président
Lamoignon, et le chargèrent d'éplucher secrètement la vie et la conduite
de Fargues\,; de bien examiner s'il n'y aurait point moyen de châtier
ses insolences passées, et de le faire repentir de les narguer si près
de la cour dans son opulence et sa tranquillité. Ils lui contèrent
l'aventure de la chasse qui leur avait appris sa demeure\,; et
témoignèrent à Lamoignon un extrême désir qu'il pût trouver des moyens
juridiques de le perdre.

Lamoignon, avide et bon courtisan, résolut bien de les satisfaire et d'y
trouver son profit\footnote{Voy., note à la fin du volume sur le procès,
  la condamnation et l'exécution de Fargues.}. Il fit ses recherches, en
rendit compte et fouilla tant et si bien, qu'il trouva moyen d'impliquer
Fargues dans un meurtre commis à Paris au plus fort des troubles, sur
quoi il le décréta sourdement, et un matin l'envoya saisir par des
huissiers, et mener dans les prisons de la Conciergerie. Fargues, qui
depuis l'amnistie était bien sûr de n'être tombé en quoi que ce fût de
répréhensible, se trouva bien étonné. Mais il le fut bien plus, quand
par l'interrogatoire il apprit de quoi il s'agissait. Il se défendit
très bien de ce dont on l'accusait, et, de plus, allégua que le meurtre
dont il s'agissait ayant été commis au fort des troubles et de la
révolte de Paris dans Paris même, l'amnistie qui les avait suivis
effaçait la mémoire de tout ce qui s'était passé dans ces temps de
confusion, et couvrait chacune de ces choses qu'on n'aurait pu suffire
ni exprimer à l'égard de chacun, suivant l'esprit, le droit, l'usage et
l'effet, non mis en doute aucun jusqu'à présent, des amnisties. Les
courtisans distingués qui avaient été si bien reçus chez ce malheureux
homme firent toutes sortes d'efforts auprès de ses juges et auprès du
roi\,; mais tout fut inutile. Fargues eut très promptement la tête
coupée, et sa confiscation donnée en récompense au premier président.
Elle était fort à sa bienséance, et fut le partage de son second fils.
Il n'y a guère qu'une lieue de Bâville à Courson. Ainsi le beau-père et
le gendre s'enrichirent successivement dans la même charge, l'un du sang
de l'innocent, l'autre du dépôt que son ami lui avait confié à garder,
qu'il déclara ensuite au roi qui le lui donna, et dont il sut très bien
s'accommoder. Novion, qui fut entre-deux depuis 1677 jusqu'en 1688, ne
fut chassé que pour avoir sans cesse vendu la justice, comme je l'ai
raconté en son lieu. Nous verrons en leur temps leurs successeurs\,; ce
n'est pas encore celui d'en parler. La première présidente Lamoignon
mourut dans une grande et longue piété. Avec tant d'enfants bien
pourvus, elle ne laissa pas de mourir avec plus de un million cinq cent
mille livres de bien.

Ninon, courtisane fameuse, et depuis que l'âge lui eut fait quitter le
métier, connue sous le nom de M\textsuperscript{lle} de L'Enclos, fut un
exemple nouveau du triomphe du vice conduit avec esprit, et réparé de
quelque vertu. Le bruit qu'elle fit, et plus encore le désordre qu'elle
causa parmi la plus haute et la plus brillante jeunesse, força l'extrême
indulgence que, non sans cause, la reine mère avait pour les personnes
galantes et plus que galantes, de lui envoyer un ordre de se retirer
dans un couvent. Un de ces exempts de Paris lui porta la lettre de
cachet, elle la lut, et remarquant qu'il n'y avait pas de couvent
désigné en particulier\,: «\,Monsieur, dit-elle à l'exempt sans se
déconcerter, puisque la reine a tant de bonté pour moi que me laisser le
choix du couvent où elle veut que je me retire, je vous prie de lui dire
que je choisis celui des grands cordeliers de Paris,\,» et lui rendit la
lettre de cachet avec une belle révérence. L'exempt, stupéfait de cette
effronterie sans pareille, n'eut pas un mot à répliquer, et la reine la
trouva si plaisante qu'elle la laissa en repos. Jamais Ninon n'avait
qu'un amant à la fois, mais des adorateurs en foule, et quand elle se
lassait du tenant, elle le lui disait franchement, et en prenait un
autre. Le délaissé avait beau gémir et parler, c'était un arrêt\,; et
cette créature avait usurpé un tel empire qu'il n'eût osé se prendre à
celui qui le supplantait, trop heureux encore d'être admis sur le pied
d'ami de la maison. Elle a quelquefois gardé à son tenant, quand il lui
plaisait fort, fidélité entière pendant toute une campagne.

La Châtre, sur le point de partir, prétendit être de ces heureux
distingués. Apparemment que Ninon ne lui promit pas bien nettement. Il
fut assez sot, et il l'était beaucoup et présomptueux à l'avenant, pour
lui en demander un billet. Elle le lui fit. Il l'emporta et s'en vanta
fort. Le billet fut mal tenu, et à chaque fois qu'elle y manquait\,:
«\,Oh\,! le bon billet, s'écriait-elle, qu'a La Châtre\,!» Son fortuné à
la fin lui demanda ce que cela voulait dire, elle le lui expliqua\,; il
le conta, et accabla La Châtre d'un ridicule qui gagna jusqu'à l'armée
où il était.

Ninon eut des amis illustres de toutes les sortes, et eut tant d'esprit
qu'elle se les conserva tous, et qu'elle les tint unis entre eux, ou
pour le moins sans le moindre bruit. Tout se passait chez elle avec un
respect et une décence extérieure que les plus hautes princesses
soutiennent rarement avec des faiblesses. Elle eut de la sorte pour amis
tout ce qu'il y avait de plus trayé et de plus élevé à la cour,
tellement qu'il devint à la mode d'être reçu chez elle, et qu'on avait
raison de le désirer par les liaisons qui s'y formaient. Jamais ni jeux,
ni ris élevés, ni disputes, ni propos de religion ou de gouvernement\,;
beaucoup d'esprit et fort orné, des nouvelles anciennes et modernes, des
nouvelles de galanteries, et toutefois sans ouvrir la porte à la
médisance\,; tout y était délicat, léger, mesuré, et formait les
conversations qu'elle sut soutenir par son esprit, et par tout ce
qu'elle savait de faits de tout âge. La considération, chose étrange,
qu'elle s'était acquise, le nombre et la distinction de ses amis et de
ses connaissances {[}continuèrent{]} quand les charmes cessèrent de lui
attirer du monde, quand la bienséance et la mode lui défendirent de plus
mêler le corps avec l'esprit. Elle savait toutes les intrigues de
l'ancienne et de la nouvelle cour, sérieuses et autres\,; sa
conversation était charmante\,; désintéressée, fidèle, secrète, sûre au
dernier point, et, à la faiblesse près, on pouvait dire qu'elle était
vertueuse et pleine de probité. Elle a souvent secouru ses amis d'argent
et de crédit, est entrée pour eux dans des choses importantes, a gardé
très fidèlement des dépôts d'argent et des secrets considérables qui lui
étaient confiés. Tout cela lui acquit de la réputation et une
considération tout à fait singulière.

Elle avait été amie intime de M\textsuperscript{me} de Maintenon, tout
le temps que celle-ci demeura à Paris. M\textsuperscript{me} de
Maintenon n'aimait pas qu'on lui parlât d'elle, mais elle n'osait la
désavouer. Elle lui a écrit de temps en temps jusqu'à sa mort avec
amitié. L'Enclos, car Ninon avait pris ce nom depuis qu'elle eut quitté
le métier de sa jeunesse longtemps poussée, n'y était pas si réservée
avec ses amis intimes, et quand il lui est arrivé de s'intéresser
fortement pour quelqu'un ou pour quelque chose, ce qu'elle savait rendre
rare et bien ménager, elle en écrivait à M\textsuperscript{me} de
Maintenon qui la servait efficacement et avec promptitude\,; mais,
depuis sa grandeur, elles ne se sont vues que deux ou trois fois, et
bien en secret. L'Enclos avait des reparties admirables. Il y en a deux
entre autres au dernier maréchal de Choiseul, qui ne s'oublient point\,:
l'une est une correction excellente, l'autre un tableau vif d'après
nature. Choiseul, qui était de ses anciens amis, avait été galant et
bien fait. Il était mal avec M. de Louvois, et il déplorait sa fortune
lorsque le roi le mit, malgré le ministre, de la promotion de l'ordre de
1688. Il ne s'y attendait en façon du monde, quoique de la première
naissance et des plus anciens et meilleurs lieutenants généraux. Il fut
donc ravi de joie, et se regardait avec plus que de la complaisance paré
de son cordon bleu. L'Enclos l'y surprit deux ou trois fois. À la fin
impatientée\,: «\,Monsieur le comte, lui dit-elle devant toute la
compagnie, si je vous y prends encore, je vous nommerai vos
camarades.\,» Il y en avait eu en effet plusieurs à faire pleurer, mais
quels et combien en comparaison de ceux de 1724, et de quelques autres
encore depuis\,! Le bon maréchal était toutes les vertus mêmes, mais peu
réjouissantes et avec peu d'esprit. Après une longue visite, L'Enclos
baille, le regarde, puis s'écrie\,:

«\,Seigneur, que de vertus vous me faites haïr\,!»

qui est un vers de je ne sais plus quelle pièce de théâtre. On peut
juger de la risée et du scandale. Cette saillie pourtant ne les brouilla
point. L'Enclos passa de beaucoup quatre-vingts ans, toujours saine,
visitée, considérée. Elle donna à Dieu ses dernières années, et sa mort
fit une nouvelle. La singularité unique de ce personnage m'a fait
étendre sur elle.

Rossignol, président aux requêtes du palais, mourut en ce même temps.
Son père avait été le plus habile déchiffreur de l'Europe. Je ne sais
comment il s'avisa de s'appliquer à une connaissance jusqu'à lui si
cachée, ni comment M. de Louvois le connut et l'employa à ce talent.
Aucun chiffre ne lui échappait, il y en avait qu'il lisait tout de
suite. Cela lui donna beaucoup de particuliers avec le roi et en fit un
homme important. Il instruisit son fils dans cette science, il y devint
habile, mais non pas au point de son père. C'étaient d'honnêtes gens et
modestes, qui l'un et l'autre tirèrent gros du roi, qui même laissa cinq
mille livres de pension à sa famille qui n'était pas d'âge à déchiffrer.

Peu de temps après qu'on fut à Fontainebleau, il arriva à Courtenvaux
une aventure terrible. Il était fils aîné de M. de Louvois, qui lui
avait fait donner puis ôter la survivance de sa charge dont il le trouva
tout à fait incapable. Il l'avait fait passer à Barbezieux son troisième
fils, et il avait consolé l'aîné par la survivance de son cousin
Tilladet, à qui il avait acheté les Cent-Suisses, qui, après les grandes
charges de la maison du roi, en est sans contredit la première et la
plus belle. Courtenvaux était un fort petit homme obscurément débauché,
avec une voix ridicule, qui avait peu et mal servi, méprisé et compté
pour rien dans sa famille, et à la cour où il ne fréquentait personne\,;
avare et taquin, et quoique modeste et respectueux, fort colère, et peu
maître de soi quand il se capriçait\,: en tout un fort sot homme, et
traité comme tel, jusque chez la duchesse de Villeroy et la maréchale de
Cœuvres, sa sœur et sa belle-sœur\,; on ne l'y rencontrait jamais.

Le roi, plus avide de savoir tout ce qui se passait, et plus curieux de
rapports qu'on ne le pouvait croire (quoiqu'on le crût beaucoup), avait
autorisé Bontems, puis Bloin, gouverneur de Versailles, à prendre
quantité de Suisses outre ceux des portes, des parcs et des jardins, et
ceux de la galerie et du grand appartement de Versailles, et des salons
de Marly et de Trianon, qui, avec une livrée du roi, ne dépendaient que
d'eux. Ces derniers étaient secrètement chargés de rôder, les soirs, les
nuits et les matins dans tous les degrés, les corridors, les passages,
les privés, et quand il faisait beau, dans les cours et les jardins, de
patrouiller, se cacher, s'embusquer, remarquer les gens, les suivre, les
voir entrer et sortir des lieux où ils allaient, de savoir qui y était,
d'écouter tout ce qu'ils pouvaient entendre, de n'oublier pas combien de
temps les gens étaient restés où ils étaient entrés, et de rendre compte
de leurs découvertes. Ce manège, dont d'autres subalternes et quelques
valets se mêlaient aussi, se faisait assidûment à Versailles, à Marly, à
Trianon, à Fontainebleau et dans tous les lieux où le roi était. Ces
Suisses déplaisaient fort à Courtenvaux, parce qu'ils ne le
reconnaissaient en rien, et qu'ils enlevaient à ses Cent-Suisses des
postes et des récompenses qu'il leur aurait bien vendus, tellement qu'il
les tracassait souvent. Entre la grande pièce des Suisses et la salle
des gardes du roi à Fontainebleau, il y a un passage étroit entre le
degré et le logement occupé lors par M\textsuperscript{me} de Maintenon,
puis une pièce carrée où est la porte de ce logement qui, en la
traversant droit, donne dans la salle des gardes, et qui a une autre
porte sur le balcon qui environne la cour en ovale, lequel communique
aux degrés et en beaucoup d'endroits. Cette pièce carrée est un passage
public de communication indispensable à tout le château, pour qui ne va
point par les cours, et par conséquent fort propre à observer les
allants et venants, et par elle-même et par ses communications. Jusqu'à
cette année, il y avait toujours couché quelques gardes du corps, et
quelques Cent-Suisses, qui, lorsque le roi entrait et sortait de chez
M\textsuperscript{me} de Maintenon, s'y mettaient mêlés sous les armes,
de sorte que cette pièce passait pour une extension de salle des gardes
et des Cent-Suisses. Le roi s'avisa cette année d'y faire coucher des
Suisses de Bloin au lieu des Cent-Suisses et de gardes.

Courtenvaux, sans en parler au capitaine des gardes en quartier,
puisqu'on en avait ôté les gardes aussi bien que les Suisses, eut la
sottise de prendre ce changement pour une nouvelle entreprise de ces
Suisses sur les siens, et s'en mit en telle colère qu'il n'y eut menaces
qu'il ne leur fît, ni pouilles qu'il ne leur chantât. Ils le laissèrent
aboyer sans s'émouvoir\,; ils avaient leurs ordres et furent assez sages
pour ne rien répondre. Le roi, qui n'en fut averti que sur le soir, au
sortir de son souper, entré à son ordinaire dans son grand cabinet ovale
avec ce qui avait accoutumé de l'y suivre, de sa famille, et des dames
des princesses, qui, à Fontainebleau, faute d'autres cabinets, se
tenaient toutes dans celui-là autour du roi, envoya chercher
Courtenvaux. Dès qu'il parut dans ce cabinet, le roi lui parla d'un bout
à l'autre sans lui donner loisir d'approcher, mais dans une colère si
terrible, et pour lui si nouvelle et si extraordinaire, qu'il fit
trembler non seulement Courtenvaux, mais princes, princesses, dames, et
tout ce qui était dans le cabinet. On l'entendait de sa chambre. Les
menaces de lui ôter sa charge, les termes les plus durs et les plus
inusités dans sa bouche, plurent sur Courtenvaux, qui, pâmé d'effroi et
prêt à tomber par terre, n'eut ni le temps ni le moyen de proférer un
mot. La réprimande finit par lui dire avec impétuosité\,: «\,Sortez
d'ici\,!» À peine en eut-il la force et de se traîner chez lui.

Quelque peu de cas que sa famille fît de lui elle fut étrangement
alarmée\,; chacun eut recours à quelque protection.
M\textsuperscript{me} la duchesse de Bourgogne, qui aimait fort la
duchesse de Villeroy et la maréchale de Cœuvres, parla de son mieux à
M\textsuperscript{me} de Maintenon, et même au roi. À la fin, il
s'apaisa, mais avec avis qu'il chasserait Courtenvaux à la première de
ses sottises et lui ôterait sa charge. Après cela il osa en reprendre
les fonctions. La cause d'une scène si étrange était que Courtenvaux
avait mis le doigt sur la lettre à toute la cour, par le vacarme qu'il
avait fait d'un changement dont le motif sautait aux yeux dès qu'on y
prenait garde\,; et le roi, qui cachait avec le plus grand soin ses
espionnages, avait compté que ce changement ne s'apercevrait pas, et
était outré de colère du bruit qu'il avait fait et qui l'avait appris et
fait sentir à tout le monde. Quoique déjà sans considération, sans
agrément, sans familiarité la moindre, il en demeura plus mal avec le
roi et ne s'en releva de sa vie\,; sans sa famille, il était chassé et
sa charge perdue.

Il mourut en même temps un autre homme encore plus méprisé, qui fut le
comte de Tonnerre\,; ce n'est pas que la naissance ou l'esprit lui
manquassent\,; mais tout le reste entièrement. Avec une poltronnerie qui
lui faisait tout souffrir, il s'attirait cent affaires par son
escroquerie et ses bons mots, et il était tombé enfin à un tel point
d'abjection qu'on avait honte de l'insulter quand il disait quelque
sottise. Il avait été longtemps premier gentilhomme de la chambre de
Monsieur, et il était fils du frère aîné de cet évêque de Noyon dont il
a été parlé ici plus d'une fois, et frère de l'évêque de Langres dont il
le sera encore.

Quoique le combat de Cassano eût été sans aucun fruit, le siège de
Turin, si mal à propos annoncé dès l'entrée du printemps, et peut-être
aussi peu à propos conçu, n'en demeurait pas moins résolu. Le roi, si
différent sur La Feuillade de ce qu'on le vit lorsque Chamillart lui en
proposa le mariage avec sa fille, ou plutôt occupé de plaire à son
ministre par l'endroit qui lui était le plus sensible, lui proposa
lui-même de charger son gendre de ce grand siège en chef. Chamillart,
surpris et comblé, s'en excusa faiblement. Le roi lui fit des amitiés,
lui dit du bien de La Feuillade et qu'il voulait essayer des jeunes gens
qui montraient des talents et de l'application. Ce choix arrêté, La
Feuillade eut ordre de s'approcher de Turin, après le siège de Chivas
achevé, et de se préparer pour en faire le siège\,; il y arriva le 6
septembre. On peut juger que rien ne lui manqua\,: il y eut soixante
bataillons, soixante-dix escadrons, onze cent milliers de poudre,
quarante mortiers, quatre-vingts pièces de canon de batterie et
vingt-six autres pièces pour tirer à ricochet, de disposés à ses ordres.
Mais il se trouva des difficultés à résoudre pour lesquelles La
Feuillade envoya Dreux, son beau-frère, qui, le jour même que le roi
arriva à Fontainebleau, fut mené par Chamillart lui rendre compte de ce
qui l'amenait, chez M\textsuperscript{me} de Maintenon. Le lendemain ils
y retournèrent, et le maréchal de Vauban avec eux, et le surlendemain,
Dreux s'en retourna trouver La Feuillade.

Vauban fit là une grande action, il s'offrit au roi et le pressa de
l'envoyer à Turin pour y donner ses conseils et se tenir, dans les
intervalles, à deux lieues de l'armée, sans s'y mêler de rien quand il y
serait. Il ajouta qu'il mettrait son bâton derrière la porte, qu'il
n'était pas juste que l'honneur auquel le roi l'avait élevé le rendît
inutile à son service, et que, plutôt que cela fût, il aimerait mieux le
lui rendre. Cette offre romaine ne fut point acceptée\,; le contraste de
Vauban et de La Feuillade eût été trop grand et l'obscurcissement de ce
dernier trop accablant. La Feuillade, contre l'avis de Vauban, voulait
attaquer par la citadelle et ne point faire de circonvallation de
l'autre côté du Pô.

M. de Vendôme manda par un courrier, arrivé en cadence, qu'il était du
même avis\,; que, pour les difficultés extérieures, il ne fallait point
s'en embarrasser\,; qu'il n'y avait rien à craindre du prince Eugène\,;
qu'il était de la dernière importance de faire alors le siège de Turin,
sans quoi les conquêtes faites sur le duc de Savoie demeureraient
inutiles\,; et il offrit d'envoyer de ses troupes si on n'en avait pas
assez pour le siège. Il fit sa cour au roi, plut au ministre, ce fut
tout. Dreux était parti avec l'ordre de ne point faire ce siège. La
Feuillade, opiniâtre, dépêcha Marignane, qui ne vit point le roi, et que
Chamillart, qui gardait sa chambre pour un torticolis, renvoya
sur-le-champ. À son retour, La Feuillade contremanda tout ce qui lui
devait arriver, retira ce qui l'était déjà, quitta la Vénerie, où il
s'était établi, et envoya un gros détachement à Vendôme.

Le siège de Barcelone était mieux concerté\,; mais l'archiduc y fit une
grande perte. Ils emportèrent, le 16 septembre, des ouvrages
nouvellement augmentés au mont Joui. La résistance fut grande, ils y
perdirent huit cents hommes, et le prince de Darmstadt dont il a été
tant parlé y fut tué\,; mais ces ouvrages coupant toute communication
avec la ville, et la garnison du mont Joui manquant de tout, elle
s'ouvrit un passage l'épée à la main, et rentra dans Barcelone, n'ayant
perdu à cette belle action que douze ou quinze hommes. Ce fut un grand
point pour l'archiduc que d'être maître du mont Joui. Ce malheur fut
incontinent suivi d'un autre. Les Catalans révoltés se saisirent de
Lérida et de Tortose. D'autre part, vers le Portugal, les ennemis
levèrent le siège de Badajoz aux approches de Tessé. Ruvigny, qui
portait le nom de milord Galloway, y commandait les Anglais et y eut un
bras emporté. C'était un très bon officier parmi eux, qui se retira en
Angleterre et n'a pas servi depuis. Ils furent plus heureux devant
Barcelone, qui se rendit le 4 octobre, la garnison prisonnière de
guerre, excepté le vice-roi, le duc de Popoli et quelques officiers
distingués. On voulut longtemps douter de cette nouvelle, et {[}de{]}
beaucoup de cruautés exercées par les Allemands.

Le roi partit le 26 octobre de Fontainebleau, s'en retournant par
Villeroy et par Sceaux, où il séjourna. Il apprit en même temps le
couronnement du roi Stanislas Lesczinski. Il ne prévoyait pas alors
assurément, et s'il se peut beaucoup moins auparavant, que dans sa chute
la plus profonde, sans pain et sans un pouce de terre, il deviendrait
beau-père de son héritier, et aussi peu encore de qui serait cet
ouvrage. Il apprit aussi en même temps la mort du fameux Tekeli, arrivée
à Constantinople, jeune encore, mais perdu de goutte et depuis longtemps
ne pouvant plus se remuer. Il était sur un grand pied de considération
et de rang, à peu près comme un grand souverain en asile\,; et y
touchait fort gros, et très exactement payé.

La mer aurait été plus heureuse par la quantité de riches et grosses
prises et de combats particuliers de nos vaisseaux et de nos armateurs
sans la mort de Saint-Paul, qui s'y était le plus signalé, et qui fut
fort regretté. Il mourut en se rendant maître de onze vaisseaux
marchands venant de la mer Baltique par la prise de trois gros vaisseaux
anglais qui les convoyaient. Cette action se passa le dernier octobre.
Saint-Paul ne laissa que trois neveux fort jeunes\,; le roi donna des
pensions à tous les trois.

La Feuillade, ou son secrétaire, fit une méprise qui coûta bon. Il manda
au gouverneur d'Acqui de le venir joindre avec sa garnison. Au lieu
d'Acqui, il mit d'Asti\,; et le gouverneur de cette dernière place
obéit. M. de Savoie, incontinent averti d'une évacuation si peu
attendue, se saisit d'Asti tout aussitôt, et mit tout le Montferrat à
contribution. La Feuillade marcha pour la reprendre\,; il fallut
emporter des postes sur le chemin. En arrivant sur Asti, il trouva
toutes les troupes du duc de Savoie et du comte de Staremberg, qui
étaient derrière la place, dans laquelle ils firent passer beaucoup de
cavalerie et d'infanterie, qui tomba rudement sur la tête de la petite
armée que La Feuillade amenait. On fit fort valoir qu'il mit pied à
terre à la tête des grenadiers, qu'il rétablit le combat, qu'il poussa
les ennemis jusque sur la contrescarpe, qu'il prit deux étendards. On ne
se vanta point de la perte, et on mit sur le compte des pluies et du
débordement des rivières la retraite qu'il fit d'Asti, où il était
arrivé pour en faire le siège, mais où il avait trouvé ce combat à
soutenir, à Casal, où son dessein n'avait pas été d'aller. On perdit à
ce combat d'Asti Imécourt et force gens, et Asti demeura au duc de
Savoie.

Les pertes d'hommes en Allemagne et en Italie, plus grandes par les
hôpitaux que par les actions, firent prendre le parti d'une augmentation
de cinq hommes par compagnie, et d'une levée de vingt-cinq mille hommes
de milice, laquelle fut une grande ruine et une grande désolation dans
les provinces. On berçait le roi de l'ardeur des peuples à y entrer\,;
on lui en montrait quelques échantillons de deux, de quatre, de cinq à
Marly, en allant à la messe, gens bien trayés, et on lui faisait des
contes de leur joie et de leur empressement. J'ai entendu cela plusieurs
fois, et le roi les rendre après en s'applaudissant, tandis que moi par
mes terres et par tout ce qui s'en disait, je savais le désespoir que
causait cette milice, jusque-là que quantité se mutilaient eux-mêmes
pour s'en exempter. Ils criaient et pleuraient qu'on les menait périr\,;
et il était vrai qu'on les envoyait presque tous en Italie, dont il n'en
était jamais revenu un seul. Personne ne l'ignorait à la cour. On
baissait les yeux en écoutant ces mensonges et la crédulité du roi, et
après on s'en disait tout bas ce qu'on pensait d'une flatterie si
ruineuse. On donna aussi quantité de régiments à lever, ce qui fit une
foule étrange de colonels et d'états-majors à payer, qui fut d'un grand
préjudice\,; au lieu de donner un bataillon et un escadron de plus aux
régiments déjà faits qui en auraient bientôt pris l'esprit, et
n'auraient point eu l'inconvénient de nouvelles troupes et de petits
régiments, qui par leur peu de nombre se détruisent promptement.

Je voyais souvent Callières\,; il avait pris de l'amitié pour moi, et je
trouvais une grande instruction avec lui. Hochstedt, Gibraltar,
Barcelone, la triste campagne de Tessé, la révolte de la Catalogne et
des pays voisins, les misérables succès de l'Italie, l'épuisement de
l'Espagne, celui de la France qui se faisait fort sentir d'hommes et
d'argent, l'incapacité de nos généraux que l'art de la cour protégeait
contre leurs fautes, toutes ces choses me firent faire des réflexions.
Je pensai qu'il était temps, avant de courir les risques de tomber plus
bas, de finir la guerre, et qu'elle se pouvait terminer en donnant à
l'archiduc ce que nous pourrions difficilement soutenir, et faisant un
partage qui n'aurait pas l'inconvénient de ne pouvoir soutenir le nôtre
comme celui du traité de partage fait d'abord en Angleterre et accepté
jusqu'au testament de Charles II\,; et un partage qui laisserait
Philippe V un grand roi en lui donnant toute l'Italie, excepté ce qu'y
tenaient le grand-duc et les républiques de Venise et de Gênes, l'État
ecclésiastique de Naples et Sicile, trop éloignés et coupés du reste par
l'État du pape\,; avoir pour le roi la Lorraine et quelques autres
arrondissements et placer ailleurs les ducs de Savoie, de Lorraine, de
Parme et de Modène. J'en fis le plan dans ma tête sans l'écrire, et je
le dis à Callières, plutôt pour m'instruire que par croire avoir rien
imaginé de fort bon et de praticable\,; je fus surpris de le lui voir
goûter. Il m'exhorta à le mettre sur du papier, et à le montrer comme un
projet aux trois ministres avec qui j'étais dans une liaison intime. Je
résistai plusieurs jours\,; enfin, pressé par Callières, je lui promis
d'en parler à ces messieurs, mais je ne pus me résoudre de rien mettre
par écrit. M. de Beauvilliers, à qui j'en parlai le premier, trouva ce
plan fort bon et fort raisonnable\,; M. de Chevreuse aussi. Ils
voulurent que j'en parlasse aux deux autres. Le contraste de leur
réponse perdrait trop, si la modestie m'empêchait de rapporter leur
réponse, qui les peint tous deux au naturel. Le chancelier me répondit,
après m'avoir écouté fort attentivement, qu'il voudrait me baiser au cul
et que cela fût exécuté, et Chamillart, avec gravité, que le roi ne
céderait pas un moulin de toute la succession d'Espagne. Dès lors je
compris l'étourdissement où nous étions, et combien les suites en
étaient à craindre.

Vers la fin de novembre arriva le comte d'Aguilar à Paris, qui fut
présenté au roi par le duc d'Albe. Le roi d'Espagne l'envoyait au roi
pour lui persuader le siège de Barcelone, et de trouver bon qu'il le fît
en personne, avec le secours des vaisseaux et des troupes du roi.
Aguilar ne réussit que trop dans sa commission, au malheur des deux
couronnes, et qui mit celle du roi d'Espagne dans le plus extrême péril.
Il était ou prétendait être Manrique de Lara, grand d'Espagne par sa
mère et fils unique de ce comte de Frigilliane dont il a été parlé à
l'occasion du testament de Charles II, et qui en apprit publiquement les
dispositions à l'ambassadeur de l'empereur d'une manière si cruelle et
si plaisante, comme je l'ai raconté alors. Il y aurait bien des choses
curieuses et singulières à raconter de ce comte de Frigilliane, qui
disait de soi-même qu'il serait le plus méchant homme d'Espagne et le
plus laid, s'il n'avait pas un fils. Ce dernier était jeune, plein
d'ambition, de ruse, de fausseté, de noirceur. Je ne sais si la
similitude avait fait cette union, mais le duc de Noailles et lui
avaient lié une amitié étroite en Espagne, qui a toujours duré intime et
avec une confiance entière. En sus de son ami, le premier homme
d'Espagne en capacité, et le premier aussi en esprit et à être dangereux
dans une cour\,; grand poltron, grand pillard, et ne put pourtant
s'enrichir. Les premières places lui passèrent successivement par les
mains\,: jamais content d'aucune, et pas une aussi ne lui demeura. Il
était lors l'un des quatre capitaines des gardes du corps, et fut
successivement colonel du régiment des gardes espagnoles, chef des
finances, et plus longtemps de la guerre avec tout pouvoir\,; capitaine
général et commandant en chef, gentilhomme de la chambre et favori,
enfin conseiller d'État, c'est-à-dire ministre, et tout cela rapidement.
Toujours craint et généralement haï, il a passé les vingt dernières
années de sa vie en disgrâce, presque toujours exilé à sa commanderie de
Saint-Jacques, à plus de quarante lieues de Madrid, et de lieues
d'Espagne, et d'ailleurs éloignée de tout. Il y aura plus d'une fois
occasion de parler de lui. Cette commanderie était de plus de trente
mille livres de rente, affectée au chancelier de l'ordre.

Aguilar, qui avait la Toison, brigua cette place de chancelier, l'obtint
et quitta la Toison, alors incompatible. Le duc de Frias, qu'on connaît
mieux sous le nom de connétable de Castille, le même dont j'ai parlé,
fut si indigné de cette action, que par rodomontade il remit sa croix de
Saint-Jacques avec une commanderie de vingt mille livres de rente qu'il
avait, et demanda et eut la Toison qu'Aguilar avait quittée. Ces grosses
commanderies, assez communes dans les trois ordres d'Espagne, faisaient
négliger la Toison aux seigneurs espagnols, qui était répandue aux
grands seigneurs sujets ou affectionnés à l'Espagne, en Italie et aux
Pays-Bas, qui en étaient fort avides, outre quelques-unes que l'empereur
demandait pour des seigneurs principaux qui le servaient. Mais douze ou
quinze ans depuis l'avènement de Philippe V à la couronne, ils ont
trouvé moyen de s'accommoder avec Rome, qui a rendu ces trois ordres
compatibles en payant tous les cinq ans une modique annate sur leurs
commanderies quand ils ont d'autres ordres, dont ils obtiennent encore
de fortes remises. Depuis cette invention, les plus grands seigneurs
d'Espagne sont devenus fort empressés pour la Toison, et peut-être plus
encore pour l'ordre du Saint-Esprit. En ce même temps Ronquillo, dont
j'ai parlé, fut fait gouverneur du conseil de Castille.

Tout étant réglé avec Aguilar pour le siège de Barcelone, le duc de
Noailles, qui n'avait pu faire les deux dernières campagnes, et qui se
portait mieux, aiguillonné par l'exemple de La Feuillade et par celui de
son père, voulut se servir du même chausse-pied pour arriver rapidement
au commandement des armées. Il demanda d'aller commander dans son
gouvernement de Roussillon, l'obtint et se hâta de s'y rendre, pour
l'exercer quelque temps avant d'être effacé en servant au siège de
Barcelone.

Je partageai en même temps, avec la plus sensible amertume, le malheur
de M. et de M\textsuperscript{me} de Beauvilliers\,; ils avaient deux
fils de seize, et dix-sept ans, bien faits et qui promettaient toutes
choses. L'aîné venait d'avoir un régiment sans avoir eu d'autre emploi,
et le cadet en allait avoir un autre. Le cadet mourut de la petite
vérole à Versailles, le 25 novembre. La même maladie commençait à
prendre à l'aîné, qui en mourut aussi le 2 décembre. Le père et la mère
pénétrés de douleur à la mort du premier, allèrent sur-le-champ en faire
un sacrifice à la messe, et y communièrent l'un et l'autre\,; à la mort
de l'autre ils eurent la même foi, le même courage, la même piété. Leur
affliction fut extrême et ce ver rongeur dura le reste de leur vie\,:
l'extérieur n'en changea point. M. de Beauvilliers continua ses
fonctions ordinaires. Pour chez lui, il se donna relâche, et pendant
quelques jours ne vit que sa plus étroite famille et ses plus intimes
amis. Je ne connais point de sermon si touchant que la douleur et la
résignation profonde de l'un et de l'autre. Leur sensibilité entière,
sans rien prendre sur leur soumission et leur abandon à Dieu\,; un
silence, un extérieur doux, apparemment tranquille, mais concentré et
toujours quelques paroles de vie qui sanctifiaient leurs larmes. Après
les premiers temps, je détournais doucement la conversation quand M. de
Beauvilliers me parlait de ses enfants\,; il s'en aperçut et me dit que
je croyais bien faire pour détourner l'objet de la douleur, qu'il m'en
remerciait, mais qu'il y avait un si petit nombre de personnes à qui il
se permît d'en parler, qu'il me priait d'en continuer les discours quand
il m'en parlerait, parce que cela le soulageait, et qu'il ne le faisait
que quand il s'en sentait pressé\,; je lui obéis, et très souvent tête à
tête il m'en parlait, et je vis en effet que de continuer avec lui
là-dessus le soulageait. Son gendre n'était pas tourné à lui donner de
la consolation, il tenait toujours sa femme à Paris, et toutes les
autres filles de M. de Beauvilliers étaient religieuses. Je n'aurai que
trop occasion de parler du duc de Mortemart.

Les jésuites cherchaient depuis longtemps à s'emparer de la cure de
Brest, et d'en faire un bon bénéfice. Ils en trouvèrent la jointure, et
ils ne la manquèrent pas\,; mais ils y trouvèrent aussi tous les
habitants si opposés, qu'ils ne les purent gagner avec toutes leurs
douces et fines industries. Ils se gardèrent bien de commettre leur
affaire à aucun tribunal. Ils obtinrent une évocation pour être jugés
devant le roi. Quel que fût leur crédit et le désir du roi de leur
accorder toutes leurs demandes, il fut impossible de briser toute règle
et toute équité devant eux. Le roi pourtant de son autorité leur accorda
la cure, mais avec des modifications qui ne leur plurent pas, et qui ne
consolèrent pas les habitants d'avoir de tels pasteurs malgré eux.

Les armées de Flandre et d'Allemagne étant séparées, Marsin et peu après
Villars arrivèrent. Le maréchal de Villeroy fut le dernier\,; il prit
son temps de paraître la nuit de Noël pendant matines. Le roi lui fit
une réception dont il fut d'autant plus content qu'elle fut plus
publique, et qu'il avait fait bien du brouhaha en entrant. Il s'occupa
le reste de l'office à galantiser les dames, à recevoir les compliments
de ce qu'il y avait là de principal, les respects des autres, et à
battre la mesure de la meilleure grâce du monde, avec une justesse que
lui-même admirait.

Surville, dont l'affaire en vieillissant ne devenait pas meilleure, fut
amené d'Arras à la Bastille, La Barre demeurant en pleine liberté.

Roquelaure eut peu après son retour une petite audience du roi pour se
justifier de sa négligence à garder les lignes, de sa fuite et de tout
le désordre qui s'en était suivi. Le roi épris de M\textsuperscript{lle}
de Laval, fille d'honneur de M\textsuperscript{me} la Dauphine, la maria
à Biran, fils de Roquelaure, duc à brevet, moyennant un autre brevet de
duc pour lui. On n'oubliera guère le bon mot qui lui échappa en
nombreuse compagnie à la naissance de sa fille aînée. «\, Mademoiselle,
dit-il, soyez la bienvenue, je ne vous attendais pas sitôt.\,» En effet,
elle ne s'était pas fait attendre. C'était un plaisant de profession,
qui, avec force bas comique, en disait quelquefois d'assez bonnes et
jusque sur soi-même, comme on le voit ici. Le roi eut toujours de la
considération et de la distinction pour M\textsuperscript{me} de
Roquelaure, née aussi plus que personne que j'aie connu pour cheminer
dans une cour. Il ne put enfin résister à ses peines sur la situation de
son mari. On verra bientôt de quelle façon il fut tiré du service pour
toujours. Elle n'apporta pas un écu en mariage dans une maison fort
obérée. Son art et son crédit la rendirent une des plus solidement
riches\,; mais la beauté heureuse était sous Louis XIV la dot des dots,
dont M\textsuperscript{me} de Soubise est bien un autre exemple.

Vers la fin de l'année Tessé maria son fils aîné à la fille de Bouchu,
conseiller d'État, duquel j'ai parlé il n'y a pas longtemps. Ce fut le
contraire de celui de M\textsuperscript{me} de Roquelaure, ni esprit, ni
art, ni naissance, ni beauté, mais des écus sans nombre, et c'est ce
qu'il fallait à Tessé.

Le duc de Duras en fit un plus assorti. Il épousa M\textsuperscript{lle}
de Bournonville, dont tout le bien, qui était fort grand, était acquis
par la mort de son père et de sa mère. Elle était à Paris dans un
couvent\,; la maréchale de Noailles l'avait souvent chez elle à la cour
pour les bals, où elle dansait à ravir. Jamais personne ne représenta
mieux la déesse de la Jeunesse. Elle en avait tous les agréments et
toute la gaieté. La maréchale en fit tellement comme de sa fille qu'elle
la maria chez elle et y logea et nourrit les mariés. Qui l'aurait dit au
maréchal de Duras qui haïssait le maréchal de Noailles et qui le
ménageait si peu\,?

Listenais épousa aussi vers le même temps une fille de la comtesse de
Mailly\,; ces deux mariages signés et déclarés les derniers jours de
cette année ne furent célébrés que les premiers jours de la suivante.
M\textsuperscript{me} du Maine depuis longtemps avait secoué le joug de
l'assiduité, de la complaisance et de tout ce qu'elle appelait
contrainte\,; elle ne se souciait ni du roi ni de M. le Prince qui
n'aurait pas {[}été{]} bien reçu à contrarier où le roi ne pouvait plus
rien, qui était entré dans les raisons de M. du Maine. À la plus légère
représentation il essuyait toutes les hauteurs de l'inégalité du
mariage, et souvent pour des riens, des humeurs et des vacarmes qui avec
raison lui firent tout craindre pour sa tête. Il prit donc le parti de
la laisser faire, et de se laisser ruiner en fêtes, en feux d'artifice,
en bals et en comédies qu'elle se mit à jouer elle-même en plein public,
et en habit de comédienne, presque tous les jours à Clagny, maison près
Versailles et comme dedans, superbement bâtie pour M\textsuperscript{me}
de Montespan qui l'avait donnée à M. du Maine depuis qu'elle
n'approchait plus de la cour.

À la fin de l'année M. le duc de Berry fut délivré de ses gouverneurs.
Jamais jeune homme ne fut si aise.

Enfin Montmélian, bloqué depuis si longtemps, se rendit le 12 décembre.
On prit le bon parti aussitôt après de le faire sauter.

L'année finit et la suivante commença par un cruel fracas sur l'évêque
de Metz. Jamais aventure si éclatante ni plus ridicule. Un enfant de
chœur, qu'on dit après être chanoine de l'église de Metz, fils d'un
chevau-léger de la garde, sortit fuyant et pleurant de l'appartement de
M. de Metz où il était seul pendant que ses domestiques dînaient, et
s'alla plaindre à sa mère d'avoir été fouetté cruellement par M. de
Metz. De ce fouet fort indiscret et, s'il fut vrai, fort peu du métier
d'un évêque, des gens charitables voulurent faire entendre pis, et le
chapitre de la cathédrale à s'émouvoir et à instrumenter. Le
chevau-léger accourut en poste à Versailles où il se jeta aux pieds du
roi avec un placet, demandant justice et réparation. La maréchale de
Rochefort m'envoya chercher partout, m'apprit l'aventure, et me pria de
prévenir Chamillart, qui avait Metz dans son département, et de ne rien
oublier pour l'engager à servir efficacement M. de Metz dans une affaire
si cruelle que ses ennemis lui suscitaient, et qui intéressait l'honneur
de toute sa famille. Je m'en acquittai sur-le-champ, et Chamillart,
naturellement obligeant, s'y porta le mieux du monde. Il se fit donc
ordonner par le roi d'écrire à l'intendant de Metz d'assoupir cette
affaire, et de faire en sorte qu'il n'en fût plus parlé. Mais le
cardinal de Coislin, averti à Orléans de ce fracas, qui était l'honneur,
la piété et la pureté même, accourut dans l'instant qu'il l'apprit, et
supplia le roi pour lui et pour son neveu que l'affaire fût éclaircie,
qu'on punît ceux qui méritaient de l'être\,; que, si c'était son neveu,
il perdît son évêché et sa charge dont il était indigne\,; mais qu'il
était juste aussi, s'il était innocent, que la réparation de la calomnie
fût publique, et proportionnée à la méchanceté qu'on lui avait voulu
faire. L'affaire dura depuis Noël, que le cardinal de Coislin arriva,
jusqu'au 18 janvier, que le roi ordonna que le chevau-léger avec toute
sa famille irait demander pardon en public à M. de Metz chez lui, dans
l'évêché, et que les registres du chapitre de la cathédrale seraient
visités, et tout ce qui pouvait y avoir été mis et qui pouvait blesser
M. de Metz entièrement tiré et ôté, tellement que ce vacarme,
épouvantable d'abord, s'en alla bientôt en fumée.

Le rare est que M. de Metz s'était fait prêtre de concert avec son
oncle, malgré et à l'insu de son père qui le voulait marier, voyant le
marquis de Coislin, son fils aîné (et il n'avait que ces deux-là),
impuissant plus que reconnu depuis son mariage. On crut donc que l'abbé
de Coislin, qui avait une petite abbaye et la survivance de son oncle,
se sentant impuissant comme son frère, n'avait pas voulu, comme lui,
s'exposer au mariage, et que cette raison l'en avait plus éloigné que la
peur de mourir de faim, encore plus que son frère. La vérité est qu'il
n'avait que si peu de barbe, qu'on pouvait dire qu'il n'en avait point,
et qu'encore que sa vie n'eût jamais été ni dévote ni bien mesurée, on
n'avait jamais pu attaquer ses mœurs. La suite de sa vie toujours
singulière, parce qu'il l'était beaucoup, et qui a été infiniment
réglée, appliquée à son diocèse jusqu'à sa mort arrivée en 1733, et tout
éclatante des plus grandes et des meilleures couvres en tout genre, et
cachées et publiques, a magnifiquement démenti ou l'imprudence ou le
guet-apens dont son oncle et lui pensèrent mourir de douleur, et dont la
santé du premier ne s'est jamais bien rétablie.

\hypertarget{chapitre-v.}{%
\chapter{CHAPITRE V.}\label{chapitre-v.}}

1705

~

{\textsc{Mon procès de Brissac.}} {\textsc{- Deux fortes difficultés à
succéder à la dignité de Brissac.}} {\textsc{- Cossé reçu duc et pair de
Brissac.}} {\textsc{- État et reprise de mon procès de Brissac.}}
{\textsc{- Voyage à Rouen.}} {\textsc{- Singulière attention du roi.}}
{\textsc{- Intimité de tout temps à jamais interrompue entre le duc
d'Humières et moi.}} {\textsc{- Ingratitude de Brissac.}} {\textsc{-
Course à Marly.}} {\textsc{- Service de La Vrillière.}} {\textsc{- Je
gagne mon procès.}} {\textsc{- M. et M\textsuperscript{me}
d'Hocqueville.}} {\textsc{- Fortunes nées de ce procès.}} {\textsc{-
Anecdote sur l'abbé depuis cardinal de Polignac.}}

~

Je n'ai pas cru devoir interrompre le fil des événements de cette année
par le récit d'un événement particulier à moi, qui pourrait même ne
tenir ici aucune place, sans le rapport qui se trouvera des semences qui
s'y jetèrent fort naturellement à des affaires plus importantes qui se
développeront dans la suite. On a vu ci-devant (t. II, p.~231) les
difficultés que le comte de Cossé rencontra à succéder à la dignité du
duc de Brissac, son cousin germain et mon beau-frère\footnote{Ce
  passage, jusqu'à \emph{d'un beau-frère qui avait été le fléau de ma
  sœur}, est omis dans les précédentes éditions.}\,; combien peu j'avais
de raisons de famille de m'intéresser pour lui, avec qui, d'ailleurs, je
n'avais aucune liaison, et que néanmoins l'intérêt de la continuation de
nos dignités dans nos maisons et que leur durée ne dépendît pas du
mauvais état d'une succession, de l'humeur des créanciers et de la
fantaisie des hommes, me fit prendre l'intérêt de Cossé jusqu'à faire ma
partie pour lui avec plusieurs des principaux pairs que j'excitai et que
j'entraînai, contre un nombre d'autres, qui très mal à propos touchés de
gagner un rang d'ancienneté (et Brissac est antérieur à moi) s'étaient
unis pour l'extinction de cette pairie et m'avaient fait parler pour
m'unir à eux, et qui furent arrêtés tout court par l'union contraire que
j'avais faite aussitôt. Maintenant il faut dire qu'outre toutes les
raisons de mécontentement que j'avais d'un beau-frère qui avait été le
fléau de ma sœur, au point que leur séparation ne put se faire que par
l'intervention de M. le Prince le héros, qui se chargea des pièces pour
les représenter si jamais M. de Brissac voulait revenir contre cette
séparation, et qui l'auraient mené personnellement bien loin, laquelle
fut homologuée au parlement et constamment tenue, j'avais un procès
contre mon beau-frère depuis la mort de ma sœur, et depuis la sienne
avec ses représentants, où il s'agissait de cinq cent mille livres. Ma
sœur, morte en 1683, m'avait fait son légataire universel. MM. de La
Reynie et Fieubet, deux conseillers d'État si connus, exécuteurs de son
testament, et M. Bignon, autre conseiller d'État aussi fort considéré,
élu en justice mon tuteur pour cette succession pendant ma minorité,
sans que pas un des trois eussent avec nous la moindre parenté. M. de
Brissac, et après lui ses représentants, me demandaient cent mille écus.
Je prétendais n'en rien devoir, et je leur demandais au contraire deux
cent mille francs restant des six cent mille de la dot de ma sœur. Cette
créance si privilégiée, si elle était déclarée bonne, était antérieure à
tous les créanciers personnels de mon beau-frère, et faisait porter à
faux pour autant de leurs créances par la multitude qu'il y en avait. M.
de Cossé, qui ne pouvait être duc qu'en vertu de son héritage, était
donc obligé de les payer tous. Il me proposa de passer un acte par
lequel il s'engageait pour mes cinq cent mille livres, en son propre et
privé nom, et sa femme avec lui, afin de me mettre hors d'intérêt
quelque succès qu'eût mon procès. Je ne le voulus point quelque presse
qu'il m'en fît, et ceux qui se mêlaient de mes affaires.

Je considérai\footnote{Nouveau passage omis, jusqu'à \emph{Cossé se
  trouva comblé}.} que je le ruinais, non seulement par un engagement si
fort, au cas que je perdisse mon procès, mais que c'était un éveil que
je donnerais si la chose venait à être connue, comme il était difficile
qu'elle ne le fût pas, et que beaucoup de créanciers périclitants
forceraient Cossé à faire pour eux la même chose et l'épuiseraient
entièrement. J'aimai donc mieux hasarder cinq cent mille livres au
jugement qui interviendrait, que me les laisser assurer, quelque
certaine qu'en fût l'assurance que Cossé m'en offrait, et par la force
de l'acte, et par l'ancienneté de cette créance et son privilège. Cossé
se trouva comblé d'une générosité si peu attendue\,; les maréchales de
La Meilleraye et de Villeroy ne le furent pas moins. Je devins le chef
de son conseil pour toutes ses démarches. Il était tous les matins chez
moi, et mes gens d'affaires conduisaient les siens pas à pas. Ce ne fut
pas sans peines et sans obstacles. Le maréchal de Villeroy lui en
aplanit un qui eût ruiné tous nos soins\,: il lui rendit favorable le
premier président Harlay, esclave de la faveur. Le maréchal en brillait
alors, et Harlay, de plus, se trouvait flatté de sa parenté proche\,; la
mère du premier maréchal de Villeroy, grand'mère de celui-ci, était
Harlay, fille du célèbre Sancy.

Deux difficultés capitales étaient en ses mains, gouvernant comme il
faisait le parlement à baguette. La maréchale de Villeroy, sœur de mon
beau-frère, et son héritière naturelle et nécessaire, avait renoncé à sa
succession en faveur de Cossé, leur cousin germain. Le maréchal de
Villeroy l'y avait autorisée, et fait renoncer aussi ses enfants. Mais
il ne dépendait pas de la faveur d'une héritière de faire un duc et
pair. En acceptant la succession, la dignité demeurait éteinte, parce
qu'elle n'était pas pour les femelles\,; en y renonçant, Cossé qui était
mâle, issu de l'impétrant, recueillait la dignité avec la succession.
Ainsi, la succession ne lui arrivant qu'au refus d'une femelle, on lui
pouvait objecter qu'il ne pouvait recevoir que ce que la femelle aurait
recueilli, en qui la dignité se serait éteinte, par quoi il n'était
recevable qu'aux biens non à la dignité, et c'est ce à quoi Cossé n'eût
jamais pu parer si cette objection lui avait été faite par gens qui
eussent eu qualité pour la pouvoir faire, tels qu'étaient les pairs,
surtout les postérieurs à l'érection de Brissac.

L'autre difficulté, dont le premier président fut le maître, avait une
autre épine plus fâcheuse encore, et qui, relevée par des pairs
opposants, eût suffi seule pour éteindre la pairie\,; c'est que
l'enregistrement fait par le parlement de la pairie de Brissac en
exceptait formellement les collatéraux exprimés dans les lettres\,; et
Cossé, bien qu'issu de mâle en mâle de l'impétrant, son
arrière-grand-père, était cadet, et partant collatéral. Harlay, partie
adresse, partie autorité, glissa sur l'une et sur l'autre, et quand tout
fut ajusté avec les créanciers, ce qui dura assez longtemps, prépara
tout polir la réception au parlement de Cossé, comme duc et pair de
Brissac, qui y prêta serment et prit séance sans aucune difficulté
alors, 6 mai 1700. Ce ne fut pas sans de nouveaux remerciements de sa
part et de toute sa famille, pleins de protestations publiques qu'il nie
devait entièrement, et plus d'une fois, la dignité dont il venait
d'entrer en possession. Le roi n'avait point voulu s'en mêler et avait
renvoyé cette affaire au parlement.

Cette grande affaire consommée, je ne craignis plus de lui causer
d'embarras en reprenant mon procès que je n'avais interrompu que pour
lui. Je l'avais gagné deux fois de suite au parlement de Rouen contre
mon beau-frère, qui, remarié à la sœur de Vertamont, premier président
au grand conseil, en avait toute la parenté nombreuse au parlement de
Paris\,; c'est ce qui avait fait évoquer cette affaire en celui de
Rouen. Il ne s'agissait de rien de nouveau. La duchesse d'Aumont, qui,
dans les dernières années de la vie de mon beau-frère, lui avait prêté
de l'argent, et dont la dette périclitait, prétendait, avec quelques
autres créanciers aussi nouveaux, remettre ce même procès au jugement du
parlement de Paris, comme chose à son égard toute neuve, n'étant pas
encore créancière lors de mes arrêts, quoiqu'elle n'eût rien à alléguer
qui n'eût été dit par mon beau-frère lors du premier arrêt que j'avais
obtenu, et par ses créanciers avec lui lors du second. Il en fallut
venir à un règlement de juges au conseil\footnote{Voy., sur le conseil
  des parties et ses attributions, t. Ier, note II.}. La duchesse
d'Aumont, abusant de l'abattement des derniers temps de la vie du
chancelier Boucherat, retarda tant qu'elle put, et vint à bout de faire
nommer vingt-deux rapporteurs l'un après l'autre, qu'elle récusa tous
vingt-deux, et que j'acceptai tous. Ce chancelier enfin nomma Méliant,
fils de ce Méliant, parent et serviteur si particulier de M. de
Luxembourg, et qui s'intrigua tant et si publiquement pour lui dans son
procès de préséance contre nous. Ce rapporteur me déplut fort par cette
raison\,; mais c'était le vingt-troisième, et il ne fallait pas donner
lieu à M\textsuperscript{me} d'Aumont de chicaner sans fin. Nous sûmes,
à n'en pas douter, qu'elle était sûre du succès au fond, en demeurant à
la chambre des enquêtes, où ses causes étaient commises au parlement de
Paris, et Menguy, rapporteur de toutes, et qui l'eût été de celle-ci,
n'avait pas été honteux de s'en expliquer tout haut. Aloi aussi,
j'espérais trouver une troisième fois la même justice au parlement de
Rouen, que j'y avais rencontrée les deux premières. Ainsi de part et
d'autre, nous fûmes en grand mouvement, et nous en étions là lorsque je
commençai à presser ce jugement que la duchesse d'Aumont avait tant
éloigné, et qu'elle aurait laissé dormir toute sa vie.

Nous voilà donc aux sollicitations. Ma surprise, pour ne rien dire de
plus, fut grande de trouver le nouveau duc de Brissac en mon chemin,
après tout ce que j'avais fait pour lui et toutes ses protestations. Je
m'en plaignis à la maréchale de Villeroy. Elle le blâma, mais\,; dans la
suite, un si grand intérêt pour lui la séduisit à le servir de son
crédit par cet amour démesuré qu'elle avait pour sa maison, en me
conservant toutefois la même amitié et cette même familiarité et liberté
de commerce. Quoique je fusse peu ébloui d'autre chose que du mérite des
maréchaux de Brissac, des exploits et des services du premier, de
l'adresse, de la science de cour, des tortuosités, de la valeur et des
actions du second, des changements de parti faits avec justesse du
troisième, et nullement de rien qui les eût précédés, où en effet il n'y
a pas à se prendre, l'amitié et la connaissance que j'avais de cette
folie de maison de la maréchale me fit le lui pardonner et vivre avec
elle à l'ordinaire. Ce qui me sembla le plus étrange fut la découverte
que nous fîmes que ce que j'avais refusé M\textsuperscript{me} d'Aumont
l'avait exigé pour s'ôter du chemin de M. de Brissac sur sa dignité. Lui
et sa femme s'étaient obligés à la dette de M\textsuperscript{me}
d'Aumont, si elle venait à la perdre, tellement que ce procès était
moins le sien que celui de M. de Brissac.

Méliant, sollicité contre moi par toute sa famille, que j'avais peu
ménagée lors du procès de M. de Luxembourg, examina le nôtre. Il était
prévenu contre moi, il souhaitait de plus que j'eusse tort et de pouvoir
s'affermir dans l'opinion qu'il avait prise d'avance. Le travail qu'il
fit le désabusa, et l'équité l'emporta sur la volonté. Il fut même si
indigné des chicanes qu'il y vit et de celles que M\textsuperscript{me}
d'Aumont, le comptant à elle, ne lui dissimula pas qu'elle préparait,
qu'il se hâta de rapporter l'affaire, et cacha pour cela à sa famille la
mort d'une sœur qu'il aimait fort.

L'intérêt, qui amène la bassesse, avait introduit depuis plusieurs
années la coutume de se faire accompagner aux jugements des grands
procès. Nous parûmes donc, de part et d'autre, à l'entrée des juges au
conseil avec une nombreuse parenté. Je causais dans la pièce du conseil
avec quelques juges, tandis que M. de Brissac était à la porte à les
voir entrer. Il lui échappa quelque bêtise sur M\textsuperscript{me} de
Mailly, la dame d'atours, et tous les Bouillon entre autres qui étaient
avec nous, et bavardait avec les juges qui entraient, avec affectation,
pour empêcher M\textsuperscript{me} de Saint-Simon de leur parler.
Quelque douce et modeste qu'elle fût, ce procédé lui déplut. Elle ne put
s'empêcher de lui dire qu'elle était étonnée de le voir si vif contre
moi. Il répondit avec quelque politesse que cinq cent mille livres de
différence pour lui, lui en faisaient une si grande qu'il ne fallait pas
s'étonner s'il y était sensible. «\,Mais, monsieur, lui répliqua
M\textsuperscript{me} de Saint-Simon d'une voix mesurée, mais avec
hauteur, c'en était une bien plus grande d'être M. de Cossé, ou de vous
trouver duc de Brissac.\,» Il fit la pirouette et disparut. Il traversa
la cour et s'en alla chez Livry, où il y avait toujours grand monde et
grand jeu tout le jour. Il se mit à parler de son procès, qui était la
nouvelle du jour. La Cour, qui jouait, et qui avait été capitaine des
gardes de M. le maréchal de Lorges, lui demanda s'il n'avait pas ouï
dire que je l'avais fait duc et pair. La force de la vérité le lui fit
avouer formellement. Là-dessus chacun lui, tomba sur le corps. Pour fin,
lui et M\textsuperscript{me} d'Aumont perdirent leur procès avec
ignominie, c'est-à-dire avec amende et dépens, et l'affaire renvoyée à
Rouen. On veut bien être ingrat, mais on ne veut pas en être soupçonné.
La cour, qui en est pleine, cria fort contre Brissac et contre les
chicanes de M\textsuperscript{me} d'Aumont, que nous n'avions pas laissé
ignorer, et, depuis la maison royale, tous nous firent des
félicitations.

Il y avait déjà des années que tout était prêt à juger sans y avoir pu
parvenir. M. d'Aumont allait passer sept ou huit mois tous les ans à
Boulogne, et tous les ans c'étaient des lettres d'État. Après sa mort,
M\textsuperscript{me} d'Aumont, qui avait fait en sorte d'y mettre son
beau-fils en quelque intérêt, voulut user de même de ses lettres d'État.
Il était extrêmement de ma connaissance, et n'avait jamais eu lieu
d'aimer ni d'estimer sa belle-mère. Il me donna sa parole qu'elle
n'aurait point ses lettres d'État, et sur cette parole nous nous mîmes
en état cette année-ci de faire juger ce procès à Rouen. J'y avais déjà
été une fois qu'il fut appointé. Le Guerchois, avec qui ce procès
m'avait lié de jeunesse, y était venu avec moi. Son père y était mort
procureur général en première réputation, et sa famille la plus proche y
occupait les premières places de la magistrature. M. de Bouillon, et
tous les Bouillon qui se souvenaient de ce que j'avais fait dans leur
procès de la coadjutorerie de Cluni, n'oublièrent rien pour me le
rendre, et ils avaient grand crédit à Rouen. L'affaire, ce nous
semblait, allait toute seule, nous ne songeâmes point à faire le voyage
de Rouen. Tandis qu'on y travaillait à notre affaire, nous allâmes à la
Ferté avec M. et M\textsuperscript{me} de Lauzun et bonne compagnie pour
une quinzaine. Il n'y avait pas huit jours que nous y étions, qu'on nous
manda de Rouen que MM. de Brissac et d'Humières y étaient, et que tous
nos amis nous conseillaient fort d'y aller. Nous partîmes donc
sur-le-champ pour nous y rendre, et nous allâmes loger dans la belle
maison d'Hocqueville, premier président de la cour des aides, qui avait
un frère président à mortier. La mère de Guerchois était leur sœur\,;
j'avais eu occasion de faire des plaisirs considérables à plusieurs des
principaux de ce parlement\,; ce fut donc, dans toute la ville, à qui
nous festinerait le plus. Il fallut capituler pour dîner chez nous,
parce que nous en voulions donner tous les jours à grand monde, et
allions les soirs où nous étions retenus, et nous l'étions toujours et
de huit jours d'avance. C'étaient des fêtes plutôt que des soupers. Chez
moi, on s'y portait. Je ne vis jamais gens si polis, si aimables, ni
plus magnifiques et de meilleure compagnie. Le mal était que nous n'y
dormions point, parce qu'il fallait courir la matinée de bonne heure
pour notre affaire. MM. de Brissac et d'Humières s'étaient mis dans une
hôtellerie et furent peu accueillis. Ils étaient venus en poste et sans
équipage\,; notre représentation plaisait davantage.

Au bout de huit ou dix jours que nous fûmes là, je reçus une lettre de
Pontchartrain, qui me mandait que le roi avait appris avec surprise que
j'étais à Rouen, et l'avait chargé de me demander de sa part pourquoi et
pour combien j'y étais, tant il était attentif à ce que devenaient les
gens marqués et qu'il avait accoutumé de voir autour de lui, quoique
sans aucune privance. Ma réponse ne fut pas difficile.

J'étais d'enfance ami intime du duc d'Humières à nous voir tous les
jours. Ce procès ne fit pas la plus légère altération dans notre amitié
et dans notre conduite. Nous nous cherchâmes dès que je fus à Rouen. Il
venait dîner chez moi, et comme j'eus fait entendre cette liaison, on le
priait à souper avec nous. Pour le Brissac, j'affichai son ingratitude,
et je déclarai que je ne voulais ni le voir ni le rencontrer. Il en fut
si accablé de honte et d'embarras, qu'il nous évita si bien qu'en effet
nous ne le vîmes nulle part. Il m'en fit parler avec douleur, mais je
tins ferme dans cette conduite avec lui, et il me revint qu'il convenait
partout de tout ce que j'avais fait pour lui. Au palais, qui fut le seul
lieu où je le vis à l'entrée des juges, son air embarrassé avec moi, et,
si je l'osais dire, respectueux, d'un homme qui ne me devait que par ce
que je l'avais fait, montrait à tout le monde le poids du personnage
qu'il faisait, et ce contraste de lui et de M. d'Humières avec moi était
un spectacle pour la ville.

Ils étaient presque seuls au palais. Avec nous étaient une foule de gens
et toutes les principales femmes, même celles de plusieurs de nos juges,
presque toutes celles des présidents à mortier, ce qui nous surprit fort
des femmes de nos juges. Le parlement eut la considération, c'est-à-dire
la grand'chambre, de suspendre toute autre affaire pour juger la nôtre.
Le rapport était déjà avancé, lorsqu'il fut suspendu par l'obstacle de
tous le moins possible à prévoir. J'avais passé une partie de
l'après-dînée à la promenade avec M. d'Humières. Il m'avait semblé peiné
et embarrassé avec moi. Il y avait du monde avec nous, qui m'empêcha de
lui demander ce qu'il avait, et lui aussi, à ce qu'il m'a dit depuis,
eut plusieurs fois la bouche ouverte pour me parler. Je revins chez
M\textsuperscript{me} de Saint-Simon, et nous nous disposions à nous en
aller souper chez le président de Motteville, lorsque nous fûmes,
avertis qu'il y avait des lettres d'État qui nous seraient signifiées le
lendemain matin. Mon dessein n'est pas d'ennuyer par le récit de ce qui
n'intéresse que moi\,; mais il faut expliquer ce qui a trait à des
choses plus importantes qui se retrouveront. C'était le lundi au soir.
Le parlement de Rouen, dont les vacances ne sont pas réglées aux mêmes
temps que Paris, finissait le samedi suivant. La tournelle et le
changement des présidents, tous là à mortier, et qui président tantôt à
la grand'chambre, tantôt en celle des enquêtes, nous donnait, au
parlement suivant, tous juges nouveaux, ni instruits ni au fait de cette
affaire, qu'il aurait fallu recommencer comme toute neuve devant eux,
sans savoir encore quand les chicanes auraient fini. D'un autre côté, le
roi était à Marly, où il n'y avait point d'exemple qu'il eût ouï parler
d'aucune affaire de particuliers, qu'elles se rapportassent ailleurs
devant lui qu'aux conseils de dépêches qui se tenaient de quinzaine en
quinzaine, et souvent plus rarement, ni que des lettres d'État et de
gens de cette considération fussent cassées sans communication, ce qui
emportait encore d'autres longueurs.

M. d'Hocqueville et M\textsuperscript{me} de Saint-Simon me
conseillèrent d'aller à Marly, au lieu d'y envoyer un courrier et des
lettres, comme je voulais faire, et de tenir ce voyage caché. Je les
crus. J'y arrivai à huit heures du matin le mardi 8 août. Le chancelier,
et Chamillart me plaignirent, mais jugèrent le remède impossible.

La Vrillière, qui avait Boulogne dans son département, et qui était
celui par qui mon affaire devait passer, s'offrit à tout, au hasard
d'être mal reçu du roi. Conseil pris, il me donna à dîner, dressa
lui-même ma requête avec moi, et se proposa de demander le lendemain
matin permission au roi de la rapporter à l'entrée du conseil d'État.
Les deux ministres l'approuvèrent sans oser espérer de succès. J'allai
instruire le duc de Beauvilliers de mon aventure et de mes mesures, qui
envoya prier Torcy de venir chez lui pour que je l'instruisisse aussi
sans me montrer, après quoi j'allai coucher à Versailles, et le
lendemain matin y attendre La Vrillière chez lui. Il arriva sur le midi
et m'apprit que les lettres d'État avaient été cassées de toutes les
voix. Il dressa l'arrêt devant moi, me donna à dîner pendant lequel il
fut mis au net. Il le signa. Je le portai au chancelier, qui était aussi
venu dîner à Versailles, allant à Pontchartrain, et c'était merveille
comme il avait couché à Marly. Il me scella sur-le-champ mon arrêt, et
je partis pour retourner à Rouen, où j'arrivai le jeudi à deux heures du
matin, trois heures après un courrier par lequel j'y avais envoyé cette
nouvelle peu espérée.

M. de Brissac s'en était allé, faisant confidence de sa joie de m'avoir
remis à longs jours à tous les maîtres de poste de la route, qui, de
surprise de me voir repasser sitôt, me le contèrent. J'eus encore un
ordre du chancelier au parlement de passer outre au jugement, quoi qu'il
pût arriver. Pontcarré, premier président, était de nos amis. Il n'avait
eu aucune opinion de mon voyage, qui lui avait été confié, et fut fort
aise d'en apprendre le succès. Il fit avertir les juges de s'assembler
le samedi 11 août, dernier jour du parlement, de grand matin. Nous
eûmes, dès quatre heures, un nombre infini d'hommes et de femmes chez
nous pour nous accompagner au palais. Ce ne fut qu'alors que la
cassation des lettres d'État fut signifiée. Le parlement était fort
irrité de ces lettres d'État, après avoir tout suspendu pour notre
affaire. Nous la gagnâmes tout d'une voix avec amende et dépens, et une
acclamation qui fit retentir le palais et qui nous suivit par les rues.
Le premier président, extrêmement pressé d'affaires domestiques, avait
bien voulu attendre le succès de mon voyage, quoiqu'il n'en espérât
rien. Nous le fûmes remercier et notre ancien et nouveau rapporteur.
Nous ne pûmes aborder notre rue, tant elle était pleine, et la foule
était dans la maison. Le feu prit à la cuisine, et ce fut merveille
qu'il fut éteint sans dommage, après avoir étrangement menacé et nous
avoir converti notre joie en amertume. Il n'y eut que le maître de la
maison qui ne s'en émut point, avec une fermeté admirable. Nous dînâmes
pourtant en grande compagnie\,; et, nos remerciements faits pendant
trois ou quatre jours, ma mère s'en retourna à la Ferté, et nous
allâmes, M\textsuperscript{me} de Saint-Simon et moi, voir la mer à
Dieppe, puis à Cani, belle maison et belle terre de notre hôte, qui
avait fort désiré de nous y voir.

C'était de ces magistrats simples, droits, modestes, des anciens temps,
généreux, capables d'amitié et de services, mais justes avant tout. Il
était fort riche et sans enfants. Sa femme ne sortait jamais de ce
château. Elle était sœur de l'abbé Le Boults, mort aumônier du roi,
grande, bien faite et avait été longtemps extrêmement du monde. Comme
elle avait beaucoup d'esprit et un esprit aimable, aisé, gai, elle en
avait conservé toutes les grâces, les manières et la liberté, dans la
plus haute dévotion et la vie la plus austère qu'elle menait depuis
plusieurs années, dans une solitude et une oraison presque continuelle,
et toujours occupée de bonnes œuvres, et les plus pénibles et les plus
pénitentes\,; mais tout cela n'était que pour elle, on ne s'en
apercevait pas. Tous deux donnaient beaucoup aux pauvres et vivaient
dans une grande intelligence. Ils étaient l'admiration de leur pays.
Nous les quittâmes à regret pour nous en retourner nous reposer trois
semaines à la Ferté, et de là à la cour.

M\textsuperscript{me} d'Aumont ne pouvait comprendre le succès de son
affaire, dont elle devint furieuse. Elle avait escamoté d'autorité les
lettres d'État à l'intendant de son beau-fils, qui de Boulogne où il
était les désavoua, et me le manda dès qu'il le sut, mais l'affaire déjà
finie. M\textsuperscript{me} de Brissac, passant devant notre logis à
Paris, y vit un feu que les domestiques que nous y avions laissés
s'avisèrent d'allumer. Elle en fit demander la cause, et apprit par là
l'événement de son procès. Son mari eut une telle honte, qu'il fut
longtemps à m'éviter partout.

Cette affaire fit des fortunes que je dus à l'amitié de Chamillart. Il
envoya Méliant intendant à Pau et de là à l'armée d'Espagne, où, par
M\textsuperscript{me} des Ursins et par M. le duc d'Orléans, je lui
procurai beaucoup d'agréments, et pendant la régence je lui obtins, et à
Guerchois, à chacun une place de conseiller d'État. J'avais fait donner
à ce dernier l'intendance d'Alençon, d'où il passa à celle de
Franche-Comté. Son frère était capitaine aux gardes, et mourait d'envie
de se tirer d'une situation où on ne chemine point. Le roi s'était fait
une règle de ne jamais laisser passer ceux de ce corps à des régiments.
Chamillart voulut bien en parler au roi, et fut repoussé par deux
différentes fois. Il m'en vit si affligé que, sans que je lui en
parlasse plus, ni lui à moi, il hasarda une troisième tentative, et
emporta le régiment de la vieille marine. Le Guerchois fit merveilles à
la tête de ce corps. Il fut bientôt maréchal de camp, puis lieutenant
général, très distingué par sa capacité et fort employé. On a su par
toute l'armée d'Italie que c'est à lui à qui fut dû le gain de la
bataille de Parme, par la justesse de son coup d'œil, et la hardiesse
avec laquelle, étant de jour, il prit sur lui de faire occuper des
cassines et de changer la disposition déjà faite, qui fut le salut de
cette action. Mais il y reçut une blessure dont il mourut quelque temps
après, avec les regrets de toutes les troupes, de tous les généraux, de
tout le pays, par la netteté de ses mains et son exacte discipline, et
avec les miens très sensibles.

La Vrillière, qui avait la Guyenne dans son département, avait eu des
occasions de me faire des plaisirs sensibles sur mon gouvernement de
Blaye. Son grand-père et son père étaient fort amis du mien. Ce dernier
service couronna les autres, et lui valut la figure, unique dans le
naufrage des secrétaires d'État, que celui-ci fit dans la régence. Cela
se retrouvera en son lieu.

Avant que finir cette année, il faut ébaucher une anecdote dont la suite
se retrouvera en son temps. L'abbé de Polignac, après ses aventures de
Pologne et l'exil dont elles furent suivies, était enfin revenu sur
l'eau. C'était un grand homme très bien fait avec un beau visage,
beaucoup d'esprit, surtout de grâces et de manières, toute sorte de
savoir, avec le débit le plus agréable, la voix touchante, une éloquence
douce, insinuante, mâle, des termes justes, des tours charmants, une
expression particulière\,; tout coulait de source, tout persuadait.
Personne n'avait plus de belles-lettres\,; ravissant à mettre les choses
les plus abstraites à la portée commune, amusant en récits, et possédant
l'écorce de tous les arts, de toutes les fabriques, de tous les métiers.
Ce qui appartenait au sien, au savoir et à la profession ecclésiastique,
c'était où il était le moins versé. Il voulait plaire au valet, à la
servante, comme au maître et à la maîtresse. Il butait toujours à
toucher le cœur, l'esprit et les yeux. On se croyait aisément de
l'esprit et des connaissances dans sa conversation\,; elle était en la
proportion des personnes avec qui il s'entretenait, et sa douceur et sa
complaisance faisaient aimer sa personne et admirer ses talents\,;
d'ailleurs tout occupé de son ambition, sans amitié, sans
reconnaissance, sans aucun sentiment que pour soi\,; faux, dissipateur,
sans choix sur les moyens d'arriver, sans retenue ni pour Dieu ni pour
les hommes, mais avec des voiles et de la délicatesse qui lui faisaient
des dupes\,; galant surtout, plus par facilité, par coquetterie, par
ambition que par débauche\,; et si le cœur était faux et l'âme peu
correcte, le jugement était nul, les mesures erronées et nulle justesse
dans l'esprit, ce qui, avec les dehors les plus gracieux et les plus
trompeurs, a toujours fait périr entre ses mains toutes les affaires qui
lui ont été commises.

Avec une figure et des talents si propres à imposer, il était aidé par
une naissance à laquelle les biens ne répondaient pas, ce qui écartait
l'envie et lui conciliait la faveur et les désirs. Les dames de la cour
les plus aimables, celles d'un âge supérieur les plus considérables, les
hommes les plus distingués par leurs places ou par leur considération,
les personnes des deux sexes qui donnaient le plus le ton, il les avait
tous gagnés. Le cardinalat était de tout temps son grand point de vue.
Deux fois il avait entrepris une licence, deux fois il l'avait
abandonnée. Les bancs, le séminaire, l'apprentissage de l'épiscopat,
toutes ces choses lui puaient, il n'avait pu s'y captiver. Il lui
fallait du grand, du vaste, des affaires, de l'intrigue. Celles du
cardinal de Bouillon, auquel il s'était attaché, l'avaient fort écarté,
et plus d'une fois, avaient pensé le perdre. Torcy, que pour ses vues il
avait toujours particulièrement cultivé, l'avait sauvé plusieurs fois,
et était toujours son ami intime, et depuis ce dernier retour, toute la
fleur de la cour l'environnait sans cesse, il y brillait avec éclat, il
en faisait les délices. Le roi même s'était rendu à lui par M. du Maine,
à la femme duquel il s'était livré. Il était de tous les voyages de
Marly, et c'était à qui jouirait de ses charmes. Il en avait pour toutes
sortes d'états, de personnes, d'esprit.

Avec tout le sien, il lui échappa une flatterie dont la misère fut
relevée, et dont le mot est demeuré dans le souvenir et le mépris du
courtisan. Il suivait le roi dans ses jardins de Marly, la pluie vint\,;
le roi lui fit une honnêteté sur son habit peu propre à la parer. «\,Ce
n'est rien, sire, répondit-il\,; la pluie de Marly ne mouille point.\,»
On en rit fort, et ce mot lui fut fort reproché.

Dans une situation si agréable, celle de Nangis qui était permanente,
celle où il avait vu Maulevrier un temps, excita son envie. Il chercha à
participer au même bonheur\,; il prit les mêmes routes.
M\textsuperscript{me} d'O, la maréchale de Cœuvres, devinrent ses amies,
il chercha à se faire entendre et il fut entendu. Bientôt il affronta le
danger des Suisses, les belles nuits, dans les jardins de Marly. Nangis
en pâlit. Maulevrier, bien que hors de gamme, à son retour en augmenta
de rage. L'abbé eut leur sort\,: tout fut aperçu\,; on s'en parla tout
bas, le silence d'ailleurs fort observé. Triompher de son âge ne lui
suffit pas, il voulait du plus solide. Les arts, les lettres, le savoir,
les affaires qu'il avait maniées, le faisaient aspirer à être reçu dans
le cabinet de Mgr le duc de Bourgogne, dont il se promettait tout s'il
pouvait y être admis.

Pour y aborder, il fallut gagner ceux qui en avaient la clef. C'était le
duc de Beauvilliers qui, après l'éducation achevée, avait conservé toute
la confiance du jeune prince. Son ministère et sa charge occupaient tout
son temps. Il n'était ni savant, ni homme de beaucoup de lettres, l'abbé
n'était lié avec personne qui le fût avec lui, il ne put donc frapper là
directement. Mais le duc de Chevreuse, en apparence moins occupé (et cet
en apparence j'aurai bientôt lieu de l'expliquer), Chevreuse, dis-je,
parut à l'abbé plus accessible. Il l'était par les lettres et les
sciences, et une fois entamé, il était facile\,; ce fut par là qu'il fut
attaqué. Tourné d'abord dans le peu de moments qu'il paraissait chez le
roi en public, tenté par l'hameçon de quelque problème, ou de quelque
question curieuse à approfondir, arrêté après aisément et longtemps dans
la galerie, l'abbé de Polignac s'ouvrit la porte de son appartement si
ordinairement fermée. En peu de temps, il charma M. de Chevreuse, il eut
d'heureux hasards d'y voir arriver M. de Beauvilliers, il parut discret,
retenu, fugitif. Peu à peu il se fit retenir en des moments de loisir.
Chevreuse le vanta à son beau-frère\,; l'abbé épiait tous les moments\,:
les deux ducs n'étaient qu'un cœur et qu'une âme\,; plaisant à l'un il
plut à l'autre, et reçu chez le duc de Chevreuse, il le fut bientôt chez
le duc de Beauvilliers.

C'étaient deux hommes uniquement occupés, n'osant dire noyés, dans leurs
devoirs, et qui, au milieu de la cour où leurs places et leur faveur les
rendait des personnages, y vivaient comme dans un ermitage, dans la plus
volontaire ignorance de ce qui se passait autour d'eux. Charmés de
l'abbé de Polignac, et n'en connaissant rien de plus, tous deux crurent
faire un grand bien d'approcher un homme si agréablement instruit de Mgr
le duc de Bourgogne, qui l'était tant lui-même, et si capable de
s'amuser et de profiter encore dans des conversations telles que
Polignac saurait avoir avec lui. Le résoudre, le vouloir, l'exécuter,
fut pour eux une même chose\,; et voilà l'abbé au comble de ses
souhaits. Nous verrons dans quelque temps jusqu'où il se poussa avec le
jeune prince\,; ce n'est pas encore le temps d'en parler, mais celui de
revenir un peu sur nos pas.

Je vis tout le manège de Polignac autour de Chevreuse. Malheureusement
pour moi, la charité ne me tenait pas renfermé dans une bouteille comme
les deux ducs. J'allai un soir à Marly, comme je faisais presque tous
les jours, causer chez le duc de Beauvilliers tête à tête. Dès lors sa
confiance dépassait mon âge de bien loin, et j'étais à portée et même
{[}dans{]} l'usage de lui parler de tout, et sur lui-même. Je lui dis
donc ce que je remarquais depuis un temps de l'abbé de Polignac et du
duc de Chevreuse\,; j'ajoutai qu'il n'y avait pas deux autres hommes à
la cour, qui se convinssent moins que ces deux-là\,; que, excepté Torcy,
tous les gens avec qui cet abbé avait les plus grandes liaisons étaient
pour eux de contrebande\,; qu'aussi n'était-ce que depuis peu que je
voyais former et tout aussi naître cette liaison nouvelle\,; que M. de
Chevreuse était la dupe de l'abbé, et qu'il n'était que le pont par
lequel il se proposait d'aller jusqu'à lui, de le charmer par son
langage comme il faisait Chevreuse par les choses savantes\,; que le but
de tout cela n'était que de s'ouvrir par eux le cabinet de Mgr le duc de
Bourgogne. Je m'y prenais trop tard\,; Beauvilliers était déjà séduit,
mais il n'était pas encore en commerce bien direct, et par conséquent
encore il n'était pas question dans son esprit de l'approcher du jeune
prince. «\,Eh bien\,! me dit-il, où va ce raisonnement, et qu'en
concluez-vous\,? --- Ce que j'en conclus, lui dis-je, c'est que vous ne
connaissez ni l'un ni l'autre ce que c'est que l'abbé de Polignac\,;
vous serez tous deux ses dupes, vous l'introduirez auprès de Mgr le duc
de Bourgogne, c'est tout ce qu'il veut de vous. --- Mais quelle duperie
y a-t-il à cela\,? me dit-il en m'interrompant, et si en effet ses
conversations peuvent être utiles à Mgr le duc de Bourgogne, que peut-on
mieux faire que de le mettre à portée d'en profiter\,? --- Fort bien,
lui dis-je, vous m'interrompez et suivez votre idée, et moi je vous
prédis, qui le connais bien, que vous êtes les deux hommes de la cour
qui lui convenez le moins, qui l'entraveriez le plus, et qu'une fois
établi par vous auprès de Mgr le duc de Bourgogne, il le charmera comme
une sirène enchanteresse, et vous même à qui je parle, qui, avec tant de
raison, vous croyez si avant dans le cœur et dans l'esprit de votre
pupille, il vous expulsera de l'un et de l'autre, et s'y établira sur
vos ruines.\,» À ce mot, toute la physionomie du duc changea, il prit un
air chagrin et me dit avec austérité\,: qu'il n'y avait plus moyen de
m'entendre, que je passais le but démesurément, que j'avais trop
mauvaise opinion de tout le monde, que ce que je prétendais lui prédire
n'était ni dans l'idée de l'abbé, ni dans la possibilité des choses, et
que, sans pousser la conversation plus loin, il me priait de ne lui en
plus parler. «\,Monsieur, lui répondis-je fâché aussi, vous serez obéi,
mais vous éprouverez la vérité de ma prophétie, je vous promets de ne
vous en dire jamais un mot.\,» Il demeura quelques moments froid et
concentré\,; je parlai d'autre chose, il y prit et revint avec moi à son
ordinaire. C'est ici qu'il faut s'arrêter jusqu'à un autre temps, et
cependant commencer à voir les cruelles révolutions de l'année en
laquelle nous allons entrer.

\hypertarget{chapitre-vi.}{%
\chapter{CHAPITRE VI.}\label{chapitre-vi.}}

1706

~

{\textsc{Année 1706.}} {\textsc{- Force bals à Marly tout l'hiver, et à
Versailles.}} {\textsc{- Surville perd le régiment du roi, donné à du
Barail.}} {\textsc{- Révolte de Valence et sédition à Saragosse.}}
{\textsc{- Berwick prend Nice et retourne à Montpellier.}} {\textsc{-
Bozelli décapité.}} {\textsc{- Mort de la princesse d'Isenghien Mort de
Bellegarde\,; histoire singulière.}} {\textsc{- Mort de Ximénès.}}
{\textsc{- Je suis choisi, sans y penser, pour l'ambassade de Rome, qui,
par l'événement, n'eut point lieu.}} {\textsc{- Mort de la comtesse de
La Marck.}} {\textsc{- Ma situation à la cour après ce choix pour
Rome.}} {\textsc{- La Trémoille cardinal avec dix-neuf autres.}}
{\textsc{- Abbé de Polignac auditeur de rote.}}

~

Je ne sais si les malheurs de l'année qui vient de finir, et les grandes
choses qu'on méditait pour celle-ci, persuadèrent au roi les plaisirs de
l'hiver comme une politique qui donnerait courage à son royaume, et qui
montrerait à ses ennemis le peu d'inquiétude que lui donnaient leurs
prospérités. Quoi qu'il en soit, on fut surpris de lui voir déclarer,
dès les premiers jours de cette année, qu'il y aurait des bals à Marly
tous les voyages, et dès le premier de l'année jusqu'au carême, d'en
nommer les hommes et les femmes pour y danser, et dire qu'il serait bien
aise qu'on en donnât sans préparatifs à Versailles à
M\textsuperscript{me} la duchesse de Bourgogne. Aussi lui en donna-t-on
beaucoup, et à Marly il y eut de temps en temps des mascarades. Un jour
même le roi voulut que tout ce qui était à Marly de plus grave et de
plus âgé se trouvât au bal, et masqué, hommes et femmes\,; et lui-même,
pour ôter toute exception et tout embarras, y vint et y demeura toujours
avec une robe de gaze par-dessus son habit\,; mais cette légèreté de
mascarade ne fut que pour lui seul, le déguisement entier n'eut
d'exception pour personne. M. et M\textsuperscript{me} de Beauvilliers
l'étaient parfaitement. Qui dit ceux-là, à qui a connu la cour, dit plus
que tout. J'eus le plaisir de les y voir et d'en rire tout bas avec eux.
La cour de Saint-Germain fut toujours de ces bals, et le roi y fit
danser des gens qui en avaient de beaucoup dépassé l'âge, comme le duc
de Villeroy, M. de Monaco, et plusieurs autres. Pour le comte de Brionne
et le chevalier de Sully, leur danse était si parfaite, qu'il n'y avait
point d'âge pour eux.

L'affaire de Surville avait, comme je l'ai dit, changé de face par
l'indiscrétion des siens. Le roi ne voulut plus juger cette affaire. Il
la renvoya au tribunal naturel des maréchaux de France. Ils condamnèrent
Surville à une année de prison, à compter du jour qu'il avait été
conduit à Arras, c'est-à-dire encore à huit mois de Bastille, et La
Barre à rien. Le roi trouva le jugement trop doux, il cassa Surville et
donna son régiment à du Barail, qui en était lieutenant-colonel, dès le
lendemain de ce jugement, qui fut les premiers jours de cette année.

Le royaume de Valence et sa ville capitale se révoltèrent, entraînés par
l'exemple des Catalans leurs voisins. Las Torrès y fut envoyé avec
quinze escadrons et trois bataillons, qui était tout ce qu'il y avait en
Aragon, que Tessé remplaça par nos troupes venant d'Estrémadure. Las
Torrès fit tout ce qu'il put\,: il prit de petits lieux l'épée à la
main\,; il défit deux mille révoltés qui le poursuivirent quelque temps,
parce qu'il était plus faible qu'eux, et ne fit quartier à aucun\,; mais
cela n'arrêta pas la révolte. Le maréchal de Tessé venait de courir
fortune à Saragosse, qui se souleva, courut aux armes et l'assiégea dans
sa maison, pour trois paysans que le régiment de Sillery, qui passait
par la ville, emmenait pour avoir assassiné un soldat où ils avaient
couché. Le bagage fut pillé, les paysans sauvés, quarante grenadiers et
trois de leurs officiers tués ou blessés. Tossé et ce qu'il avait
d'officiers principaux eurent peine à se sauver chez le vice-roi, et
plus encore à pacifier cette affaire. Le pont de Saragosse était
nécessaire pour les convois. Il fit revenir quelques troupes qui
marchaient en Catalogne, et quitta promptement cette ville, où il ne se
trouvait pas en sûreté. Le vice-roi y était considéré\,: c'était le duc
d'Arcos, le même qui vint en France pour avoir présenté un mémoire
contre l'égalité réciproque des ducs et des grands. C'était un savant de
mérite et de beaucoup d'esprit, mais comme tous ces seigneurs espagnols,
à l'exception de cinq ou six, d'une ignorance à la guerre jusqu'à n'en
avoir pas la moindre notion. Avec cela il voulut la faire et la
gouverner en Aragon. Las Torrès, ne pouvant tenir à ses ordres étranges,
ni lui faire rien comprendre, prit le parti de s'en aller à Madrid, où
on prit celui d'y rappeler le duc d'Arcos, en lui laissant son titre de
vice-roi, et le consolant des fonctions en le faisant conseiller d'État,
c'est-à-dire ministre, médiocre emploi pour lors, mais jusqu'à
l'avènement de Philippe V, le \emph{non plus ultra} en Espagne. Je ne
sais pourquoi ils avaient rappelé peu de temps auparavant Serclaës
d'Aragon pour y envoyer Las Torrès en sa place.

Berwick, parti depuis quelque temps de Languedoc, faisait le siège du
château de Nice, et le prit en ce même temps, et tout de suite s'en
retourna à Montpellier. Cette petite conquête fut un léger contrepoids
aux affaires de Valence et d'Aragon.

Vaudemont s'était fort servi à maints usages d'un Milanais de condition,
qui s'appelait le comte Bozelli. Il était entré au service de France, et
y avait été quelque temps. C'était un homme de beaucoup d'esprit et de
valeur, mais homme à tout faire, et un franc bandit. Les assassinats et
toutes sortes de crimes ne lui coûtaient rien\,; il se tirait d'affaires
à force d'intrigues. Je ne sais s'il était entré en quelqu'une qui pût
embarrasser Vaudemont. Il avait quitté le service de France, et faisait
des siennes dans ses terres et dans tout le pays. Vaudemont le fit
avertir de prendre garde à lui, parce qu'il ne lui pardonnerait plus.
Bozelli n'en tint compte et commit un assassinat. Vaudemont le fit
traquer et prendre, et couper la tête fort peu de jours après. Il laissa
un fils au service de France, aussi brave que lui, mais aussi honnête
homme et aussi modeste et retenu que le père l'était peu. Il est
lieutenant général et connu sous le nom du comte Scipion\,; il omet
volontiers son nom de Bozelli.

M. d'Isenghien perdit sa femme de la petite vérole, dans ce mois de
janvier. Elle était fille du prince de Fürstemberg, et ne laissa point
d'enfants.

En même temps mourut le vieux Bellegarde, à quatre-vingt-dix ans, qui
avait longtemps servi avec grande distinction. Il était officier général
et commandeur de Saint-Louis\,; il avait été très bien fait et très
galant\,; il avait été longtemps entretenu par la femme d'un des
premiers magistrats du parlement par ses places et par sa réputation,
qui s'en doutait pour le moins, mais qui avait ses raisons pour ne pas
faire de bruit (on disait qu'il était impuissant). Un beau matin sa
femme, qui était une maîtresse commère, entre dans son cabinet suivie
d'un petit garçon en jaquette. «\,Hé\,! ma femme, lui dit-il, qu'est-ce
que ce petit enfant\,? --- C'est votre fils, répond-elle résolument, que
je vous amène, et qui est bien joli. --- Comment, mon fils\,!
répliqua-t-il, vous savez bien que nous n'en avons point. --- Et moi,
reprit-elle, je sais fort bien que j'ai celui-là, et vous aussi.\,» Le
pauvre homme, la voyant si résolue, se gratte la tête, fait ses
réflexions assez courtes\,: «\,Bien, ma femme, lui dit-il, point de
bruit, patience pour celui-là, mais sur parole que vous ne m'en ferez
plus.\,» Elle le lui promit, et a tenu parole\,; mais toujours
Bellegarde assidu dans le logis.

Voilà donc le petit garçon élevé dans la maison, la mère l'aimait fort,
le père point du tout\,; mais il était sage. Jamais ni lui ni elle ne
l'ont appelé qu'Ibrahim. Ils avaient accoutumé leurs amis à ce nom de
guerre. J'ai vu tout cela de fort près dans ma jeunesse. Ce magistrat
était extrêmement des amis de mon père, et je voyais Ibrahim fort
souvent, mais je n'en ai su l'histoire que depuis. Il voulut être de la
profession de son véritable père\,; l'autre ne s'y opposa point du tout.
Il est mort en Italie\,; je ne dirai ni où ni en quel grade, car il a
laissé un fils très honnête homme, et qui a rattrapé au parlement la
même magistrature dans laquelle son prétendu grand-père était mort. Je
n'ai pu m'empêcher de rapporter une si singulière histoire, dont tous
les personnages m'ont été si connus.

Ximénès mourut aussi en ce même temps. C'était un Catalan qui n'avait ni
ne prétendait aucune parenté avec les Ximénès du fameux cardinal, mais
un homme d'un grand mérite, lieutenant général très ancien et très
distingué, qui avait le gouvernement de Maubeuge. Le roi lui avait
permis de faire passer à son fils le régiment Royal-Roussillon
infanterie, qui était sur le pied étranger, et qui valait beaucoup.

Il y avait cinq ans que le cardinal de Janson était à Rome chargé des
affaires du roi. Il les y avait faites avec dignité, et beaucoup plus en
digne Français qu'en cardinal, cela ne plaisait ni au pape ni à sa cour.
Il était désagréablement avec l'un et point bien avec l'autre, qui veut
tout voir ployer devant elle. Il avait été considérablement malade, il
pressait depuis longtemps la liberté de revenir. À la fin, il
l'obtint\,; mais nul cardinal qui pût le remplacer, et l'abbé de La
Trémoille destiné, faute de tout autre, à être chargé des affaires à son
départ. Cela força à penser à envoyer promptement un ambassadeur à Rome,
dont il n'y en avait point eu depuis le court et troisième voyage que le
duc de Chaulnes y avait si subitement fait à la mort d'Innocent XI pour
l'élection de son successeur.

Dangeau et d'Antin, deux hommes d'espèce si différente, mais dont
l'ambition avait le même but, y pensèrent tous deux dans l'espérance que
ce grand emploi les élèverait au duché-pairie\,: l'un porté par ses
charges qui pour son argent en avaient fait non pas un seigneur, mais,
comme a si plaisamment dit La Bruyère sur ses manières, un homme d'après
un seigneur, par ses fades privances d'ancienneté avec le roi, le mérite
d'une assiduité infatigable et d'une éternelle louange, celles de sa
femme avec M\textsuperscript{me} de Maintenon qui l'aimait\,; l'autre
par sa naissance, parce qu'il était aux enfants du roi et de sa mère,
par son esprit et sa capacité, par son manège et son intrigue. Dangeau y
avait pensé de plus loin, il s'était avisé de saisir des occasions de se
faire connaître à quelques cardinaux. Il avait été jusqu'à faire des
présents au cardinal Ottoboni, et quelquefois à en recevoir des lettres
et à s'en vanter avec complaisance. Tous deux étaient bien avec Torcy,
qui ménageait extrêmement M\textsuperscript{me} de Dangeau, devenue fort
son amie. M\textsuperscript{me} de Bouzols, sa sœur, passait sa vie avec
M\textsuperscript{me} la Duchesse dans l'intimité de tout avec elle.
Elle pouvait beaucoup sur son frère. D'Antin, tout tourné à
M\textsuperscript{me} la Duchesse, faisait agir ce ressort auprès du
ministre des affaires étrangères, et ne négligeait rien d'ailleurs pour
réussir.

Gualterio me parla de cette ambassade\,; il était tout français, et il
ne lui était pas indifférent de pouvoir compter sur l'amitié d'un
ambassadeur de France à Rome. À trente ans que j'avais pour lors, je
regardai cette idée comme une chimère, avec l'éloignement qu'avait le
roi des jeunes gens, surtout pour les employer dans les affaires.
Callières aussi m'en parla après, je lui répondis dans la même pensée,
et j'ajoutai les difficultés de réussir à Rome et de ne m'y pas ruiner,
et celles, établi comme je l'étais, de parvenir à rien de plus par cette
ambassade. Huit jours après que le nonce m'en eut parlé, je le vis
entrer dans ma chambre un mardi, vers une heure après midi, les bras
ouverts, la joie peinte sur son visage, qui m'embrasse, me serre, me
prie de fermer ma porte, et même celle de mon antichambre, pour que
personne n'y pût voir de sa livrée, puis me dit qu'il était au comble de
sa joie, et que j'allais ambassadeur à Rome. Je le lui fis répéter par
deux fois. Je n'en crus rien et lui dis que son désir lui faisait
prendre son idée pour réelle, et que cela était impossible. De joie et
d'impatience, il me demande le secret, et m'apprend que Torcy, de chez
qui il venait, lui avait confié qu'au conseil dont il sortait la chose
avait été résolue, et arrêté qu'il ne me le dirait de la part du roi
qu'après un autre conseil. Celui d'État s'était tenu ce jour-là
extraordinairement, car c'était le jour de celui des finances, et ce
même jour extraordinairement aussi le roi allait à Marly. Si un des
portraits de ma chambre m'eût parlé, ma surprise n'aurait pas été plus
grande\,; Gualterio m'exhorta tant qu'il put à accepter\,; l'heure du
dîner où il était prié nous sépara bientôt. M\textsuperscript{me} de
Saint-Simon, à qui je le dis incontinent, n'en fut pas moins étonnée.

Nous envoyâmes prier Callières et Louville de venir sur-le-champ\,; nous
nous consultâmes tous quatre. Ils furent d'avis que cela ne se pouvait
refuser. De là je fus trouver Chamillart, à qui je reprochai fort de ne
m'avoir pas averti. Il sourit de ma colère et me dit que le roi avait
demandé le secret, et au reste me conseilla de toutes ses forces
d'accepter. Il s'en allait à l'Étang et nous à Marly, où il me dit que
nous nous verrions le lendemain. J'allai de là faire la même sortie au
chancelier, qui se moqua de moi, et me fit la même réponse que
l'autre\,; pour de conseil, je n'en pus jamais tirer. Il s'en allait à
Pontchartrain, et me dit que nous nous verrions au retour. M. de
Beauvilliers s'en était allé à Vaucresson au sortir du conseil, je le
vis un moment à Marly, quand il y vint pour le conseil. Il me fit la
même excuse que les autres. La question était de prendre mon parti avant
que la proposition me fût faite, et je craignais à tout instant la
visite de Torcy.

J'avoue que je fus flatté du choix pour une ambassade si considérable à
mon âge, sans y avoir pensé et sans y avoir été porté par personne. Je
n'avais pas la moindre liaison, pas même la plus légère connaissance
avec Torcy\,; M. de Beauvilliers était trop mesuré pour m'avoir proposé
sans savoir auparavant si l'emploi était compatible avec l'état de mes
affaires\,; le chancelier n'en était pas à portée\,; Chamillart n'aurait
pas fait cette démarche à mon insu, et d'ailleurs assez de travers avec
Torcy, comme je le dirai dans la suite, il n'aurait pas hasardé de faire
au roi une proposition du ministère d'autrui.

Depuis la mort du roi, Torcy et moi nous nous rapprochâmes, et l'amitié,
comme je le rapporterai en son temps, se mit véritablement entre nous
deux et a toujours depuis duré telle. Je lui demandai alors par quelle
aventure j'avais été choisi pour Rome. Il me protesta qu'il n'en savait
autre chose, sinon qu'au conseil où je fus désigné, et au sortir duquel
il le dit au nonce qui vint aussitôt m'en avertir, le roi, déjà résolu
d'envoyer un ambassadeur à Rome, sur le retour accordé au cardinal de
Janson et la répugnance extrême du pape de faire La Trémoille cardinal,
le roi, dis-je, arrêta Torcy comme il allait commencer la lecture des
dépêches de Rome, et, fatigué des demandeurs qu'il voyait tendre au
duché et qu'il ne voulait pas faire, dit aux ministres qu'il fallait
choisir un ambassadeur pour Rome\,; qu'il voulait un duc, et qu'il n'y
avait qu'à voir dans la liste sur qui il pourrait s'arrêter. Il prit un
petit almanach et se mit à lire les noms, commençant par M. d'Uzès. Mon
ancienneté le conduisit bientôt jusqu'à moi sans s'être arrêté
entredeux. À mon nom, il fit une pause, puis dit\,: «\,Mais que vous
semble de celui-là\,? il est jeune, mais il est bon,\,» etc.
Monseigneur, qui voulait d'Antin, ne dit mot. Mgr le duc de Bourgogne
appuya. Le chancelier et M. de Beauvilliers pareillement. Torcy loua
leur avis, mais proposa de continuer à parcourir la liste. Chamillart
opina qu'on n'y pouvait trouver mieux. Le roi ferma son almanach, et
conclut que ce n'était pas la peine d'aller plus loin\,; qu'il
s'arrêtait à mon choix\,; qu'il en ordonnait le secret jusqu'à quelques
jours qu'il me le ferait dire. La chose ne balança pas plus que cela, et
ne dura pas au delà. Torcy lut ses dépêches, il n'en fut pas question
davantage. Voilà tout ce que j'en ai su plus de dix ans après d'un homme
vrai, et qui ne pouvait plus avoir d'intérêt ni de raison de m'en rien
déguiser.

Beauvilliers et Chamillart, chacun séparément, examinèrent mes dettes,
mes revenus, la dépense de l'ambassade et ses appointements, les
premiers sur des états que M\textsuperscript{me} de Saint-Simon leur fit
apporter et qu'elle examina avec eux, les autres par estime. Tous deux
conclurent à accepter\,: le duc, parce qu'après un sérieux examen, il se
trouvait que je pouvais suffire à cette ambassade sans me ruiner\,; que,
si je la refusais, jamais le roi ne me le pardonnerait, surtout ayant
quitté le service\,; ne me regarderait plus que comme un paresseux qui
ne voudrait rien faire\,; s'attacherait à me faire sentir son
mécontentement par toutes sortes de dégoûts et par toutes sortes de
refus en choses où j'aurais besoin de lui\,; gâterait plus mes affaires
par là, et ma situation présente et future que ne pourrait faire quelque
fâcheux succès que je pusse avoir dans l'ambassade. À ces raisons il
ajoutait ma liaison intime avec trois des quatre ministres d'État, qui
de silence ou d'excuse protégeraient mes fautes et m'avertiraient, et
qui le feraient hardiment parce qu'étant tous trois mes amis, ils ne
craindraient pas d'être relevés par aucun d'eux, comme cela leur
arrivait et les retenait souvent\,; que pour le quatrième, avec qui je
n'avais aucune liaison, celle qui était entre ce ministre et lui était
suffisante pour m'en pouvoir répondre, outre son caractère doux et rien
moins que malfaisant\,; enfin que ce choix s'était fait sans que j'eusse
jamais pensé à cette ambassade, qui était une excuse générale pour moi
et une raison particulière pour Torcy de ne me savoir nul mauvais gré de
l'avoir eue. Toutes ces raisons étaient sans prévention et solides. Le
chancelier fut du même avis, et ajouta qu'il n'y avait point de milieu
entre accepter ou me perdre. Chamillart allégua à peu près les mêmes
raisons, après quoi il s'ouvrit franchement à M\textsuperscript{me} de
Saint-Simon et à moi des siennes. Moins ébloui de l'éclat de ses places
qu'attentif à l'établissement durable de sa famille, il songeait à lui
procurer de solides appuis. Elle ne lui offrait que le seul La
Feuillade, que dans cette vue il tâchait assidûment d'agrandir\,; mais
il ne s'en contentait pas. La jeunesse de son fils, à peine hors du
collège, le poids de son double travail, l'incertitude des affaires,
tout cela l'inquiétait, et il ne pensait qu'à trouver des sujets
également capables d'élévation et de reconnaissance. Je lui avais paru
de ceux-là, et, pour son intérêt propre, il nie désirait ambassadeur à
Rome, pour me faire de ce grand emploi un échelon à d'autres dans
lesquels je fusse en état de rendre à son fils, et peut-être à lui-même,
si les choses changeaient, les plaisirs et les services que j'en aurais
reçus, par une protection sûre et solide à mon tour. Il nous offrit sa
bourse et son crédit sans mesure, et tout ce qui pouvait dépendre de lui
et de ses places.

Vaincu enfin, j'acceptai, c'est-à-dire j'en pris la résolution, et
j'avoue que ce fut avec plaisir. M\textsuperscript{me} de Saint-Simon,
plus sage et plus prudente, peinée aussi de quitter sa famille, demeura
persuadée, mais peinée. Je ne puis me refuser au plaisir de raconter ici
ce que ces trois ministres, et tous trois séparément, et tous trois sans
que je leur en parlasse, me dirent sur une femme de vingt-sept ans,
qu'elle avait alors, mais qu'une longue habitude, et souvent d'affaires
de cour et de famille (car c'étaient nos conseils pour tout), et en
dernier lieu celle-ci, leur avait bien fait connaître. Ils me
conseillèrent tous trois, et tous trois avec force, de n'avoir rien de
secret pour elle dans toutes les affaires de l'ambassade, de l'avoir au
bout de ma table quand je lirais et ferais mes dépêches, et de la
consulter sur tout avec déférence. J'ai rarement goûté aucun conseil
avec tant de douceur, et je tiens le mérite égal de l'avoir mérité, et
d'avoir toujours vécu depuis comme si elle l'eût ignoré\,; car elle le
sut, et par moi, et après d'eux-mêmes.

Je n'eus pas lieu de le suivre à Rome, où je ne fus point, mais je
l'avais exécuté d'avance depuis longtemps, et je continuai toute ma vie
à ne lui rien cacher. Il faut encore me passer ce mot. Je ne trouvai
jamais de conseil si sage, si judicieux, si utile, et j'avoue avec
plaisir qu'elle m'a paré beaucoup de petits et de grands inconvénients.
Je m'en suis aidé en tout sans réserve, et le secours que j'y ai trouvé
a été infini pour ma conduite et pour les affaires, qui ne furent pas
médiocres dans les derniers temps de la vie du roi et pendant toute la
régence. C'est un bien doux et bien rare contraste de ces femmes
inutiles ou qui gâtent tout, qu'on détourne les ambassadeurs de mener
avec eux, et à qui on défend toujours de rien communiquer à leurs
femmes, dont l'occupation est de faire la dépense et les honneurs,
contraste encore plus grand de ces rares capables qui font sentir leur
poids, d'avec la perfection d'un sens exquis et juste en tout, mais doux
et tranquille, et qui, loin de faire apercevoir ce qu'il vaut, semble
toujours l'ignorer soi-même avec une uniformité de toute la vie de
modestie, d'agrément et de vertu.

Cependant mon choix pénétra et se dit peu à peu à l'oreille. Torcy ne me
parlait point, je ne savais que répondre à nies amis\,; on me traînait
d'un conseil à l'autre\,; à la fin il devint public. Nous retournâmes à
Versailles, nous revînmes à Marly, on ne s'en contraignait plus. M. de
Monaco m'offrit au bal de m'accommoder pour ce qui était resté à Rome
des meubles et des équipages de son père\,; et quand nous dansions,
M\textsuperscript{me} de Saint-Simon ou moi, nous entendions dire\,:
«\,Voilà M. l'ambassadeur ou M\textsuperscript{me} l'ambassadrice qui
danse.\,» Ce malaise me fit presser Torcy par Callières de finir de
façon ou d'autre. Il sentait l'indécence de la chose en elle-même et
tout mon embarras, mais il n'osait presser le roi. La raison de ces
prolongations vint de quelque espérance de fléchir le pape sur l'abbé de
La Trémoille, de presser la promotion de dix-neuf chapeaux vacants qui
mettait tout Rome en mouvement, et qui, par ce grand nombre, ne pouvait
plus guère se différer. Elle se différa pourtant, et il arriva que, sans
avoir été déclaré, mon choix n'en fut pas moins public à Paris et à
Rome. Mgr le duc de Bourgogne m'en lit un jour des honnêtetés à Marly, à
la dérobée, quoique alors je ne fusse en aucune privance avec lui. Il
trouvait ces délais trop poussés, et sur ce que je lui répondis sur cet
emploi avec modestie, il m'encouragea et me dit que je ne pouvais mieux
commencer pour me former aux affaires et aux grandes places. Il ajouta
qu'il était fort aise pour cela que je me fusse résolu de l'accepter, et
par ce encore que le roi ne m'eût jamais pardonné le refus.

Tandis que j'étais ainsi en spectacle, la comtesse de La Marck mourut à
Paris de la petite vérole. Elle était fille du duc de Rohan\,; comme je
l'ai dit lors de son mariage. Elle était amie intime de
M\textsuperscript{me} de Saint-Simon, et fort aussi de
M\textsuperscript{me} de Lauzun, anciennes compagnes de couvent. C'était
une grande femme très bien faite, mais laide, avec un air noble et
d'esprit qui accoutumait à son visage. Elle avait infiniment d'esprit,
et elle l'avait vaste, mâle, plein de vues, beaucoup de discernement, de
justesse, de précision, un air simple et naturel, et une conversation
charmante\,; fort sûre, un peu sèche, et un cœur excellent, qui lui
coûta la vie par les extravagants contrastes de sa plus proche famille.
C'était une personne que les vues, l'ambition, le courage et la
dextérité auraient menée loin\,; aussi était-elle la bonne nièce de
M\textsuperscript{me} de Soubise, qui l'aimait passionnément. Son mérite
la fit fort regretter. M\textsuperscript{me} de Saint-Simon la pleura
amèrement, et j'en fus fort touché. Cinq ou six heures après avoir
appris cette mort, il fallut aller danser, M\textsuperscript{me} de
Saint-Simon et sa sœur, avec les yeux gros et rouges, sans qu'aucune
raison pût en excuser. Le roi connaissait peu les lois de la nature et
les mouvements du cœur. Il étendait les siennes sur les choses d'État,
et sur les amusements les plus frivoles, avec la même jalousie. Il fit
venir et danser à Marly la duchesse de Duras, dans le premier deuil du
maréchal de Duras. On a vu sur Madame, à la mort de Monsieur, combien
les bienséances les plus respectées trouvèrent en lui peu de
considération et de ménagement.

J'ai envie d'achever tout de suite cette trop longue histoire de mon
ambassade de Rome, aussi bien la promotion des cardinaux vint-elle dans
un temps trop vif et trop intéressant, pour faire scrupule de l'en
déplacer. Je fus traîné de la sorte jusque vers la mi-avril\,; enfin je
sus que mon sort serait décidé au premier conseil. Nous étions à Marly
et logés avec Chamillart dans le même pavillon, je le priai, en rentrant
de ce conseil, d'entrer chez moi avant de monter chez lui, pour
apprendre en particulier ce que j'allais devenir. Il vint donc dans la
chambre de M\textsuperscript{me} de Saint-Simon, où nous l'attendions
avec inquiétude. «\,Vous allez être bien aise, lui dit-il, et moi bien
fâché\,; le roi n'envoie plus d'ambassadeur à Rome. Le pape à la fin
s'est rendu à faire l'abbé de La Trémoille cardinal, il s'est en même
temps résolu à faire la promotion que sa répugnance à l'y comprendre a
tant retardée, et le nouveau cardinal sera chargé des affaires du roi
sans ambassadeur.\,» M\textsuperscript{me} de Saint-Simon, en effet, fut
ravie\,; il semblait qu'elle pressentait l'étrange discrédit où les
affaires du roi allaient tomber en Italie, l'embarras et le désordre que
les malheurs allaient mettre dans les finances, et la situation cruelle
où toutes ces choses nous auraient réduits à Rome.

Les réflexions que j'avais eu un si long loisir de faire me consolèrent
aisément d'un emploi qui m'avait flatté\,; mais je ne me doutais pas du
mal qu'il me ferait. D'Antin et Dangeau avaient été enragés de la
préférence, et le maréchal d'Huxelles encore, qui avait voulu se faire
prier, pour demander comme condition à être fait due, et qui avait été
laissé là fort brusquement. Ne pouvant faire pis pour couper chemin à un
jeune homme qu'ils voyaient pointer à leurs dépens, et connaissant
combien le roi était en garde contre l'esprit et l'instruction, ils
s'étoient mis à me louer là-dessus outre mesure, en applaudissant au
choix du roi, devenu public à force de longueurs et de temps. M. et
M\textsuperscript{me} du Maine ne m'avaient point pardonné de n'avoir pu
m'attirer à Sceaux, et de m'avoir trouvé inébranlable à toutes les
avances qu'ils m'avaient prodiguées, comme je l'ai marqué en leur temps.
Je ne m'étais pas caché de ce que je sentais du rang que les bâtards
avaient usurpé. Me voir pointer leur donna de la crainte et du dépit, et
je n'ai pu attribuer qu'à M. du Maine, si naturellement timide et
malfaisant, l'aversion étrange de M\textsuperscript{me} de Maintenon
pour moi, dont je ne me doutai que dans les suites. Chamillart ne me
l'avoua qu'après la mort du roi, et en même temps qu'elle était telle,
qu'il en avait eu des prises avec elle, et qu'elle avait été l'obstacle
qui l'avait empêché de me raccommoder plus tôt avec le roi, ce qui est
bien antérieur à ceci\,; que poussée par lui, elle n'avait pu rien
alléguer de particulier sur elle ni sur les siens, mais vaguement que
j'étais glorieux, frondeur, et plein de vues, sans avoir pu jamais la
ramener, non pas même l'émousser\,; et qu'elle m'avait rendu auprès du
roi beaucoup de mauvais offices. Ce bruit d'esprit et de lecture, de
capacité et d'application, d'homme enfin très propre aux affaires, fut
aisément porté au roi par ces mêmes canaux de M. du Maine, en louanges
empoisonnées, et de M\textsuperscript{me} de Maintenon plus à découvert.
M. du Maine, lié alors avec M\textsuperscript{me} la Duchesse qui
l'était étroitement avec d'Antin, avait porté ce dernier. Il était piqué
de n'avoir pas réussi, il l'était d'ailleurs contre moi comme je viens
de le dire\,; il n'en fallut pas davantage. Ils mirent le roi si bien en
garde sur moi, qu'ils le conduisirent jusqu'à la crainte, pour
l'éloigner davantage et plus sûrement, et bientôt après je m'aperçus
d'un changement en lui, qui comme les langueurs ne put finir que par une
dangereuse maladie, c'est-à-dire par une sorte de disgrâce dont je
parvins à me relever, mais dont il ne s'agit pas encore.

La même impression sur moi fut donnée à Monseigneur. D'Antin pour cela
n'eut que faire de personne, mais il trouva là-dessus
M\textsuperscript{lle} de Lislebonne et M\textsuperscript{me} d'Espinoy
à son point. Elles n'ignoraient pas mes sentiments ni ma conduite à
l'égard du rang et des usurpations de leur maison. C'était leur endroit
sensible. Elles menaient ce bon Monseigneur, qui prit sur moi toutes les
opinions qu'il leur convint de lui donner, et M\textsuperscript{me} la
Duchesse dès lors, et encore plus bientôt après, comme je le dirai en
son lieu, y travailla avec la même affection. La Choin se laissa
persuader et par elles ses meilleures amies, et par le maréchal
d'Huxelles, qui la courtisait fort, et par qui ce pauvre Monseigneur se
persuada qu'il était la meilleure tête du royaume. Telle devint ma
situation à la cour, de laquelle je ne tardai pas à m'apercevoir. Mais
achevons ce qui regarde Rome, afin de n'avoir pas à y revenir, ni à
couper des choses trop intéressantes, si je remettais à parler de la
promotion des cardinaux au temps où elle fut faite, qui fut le 17 mai.

Elle fut de dix-neuf sujets. Le savant Casoni en fut porté par son
érudition profonde et l'intégrité de sa vie\,; Corsini qui a depuis été
pape\,; ce duc de Saxe-Zeitz dont il a été tant parlé\,; notre nonce
Gualterio\,; l'abbé de La Trémoille\,; Fabroni, pour le malheur de
l'Église\,; et Filipucci qui donna un rare exemple de modestie et de
piété, en refusant le chapeau. C'était un savant jurisconsulte. En vain,
le pape l'exhorta et lui donna du temps à réfléchir, il demeura constant
dans son refus. Un autre eut son chapeau, et le vingtième demeura
\emph{in petto}, Conti, nonce en Portugal, et depuis pape, eut le
chapeau que Filipucci avait si constamment refusé.

Pendant ces longs délais du pape, Torcy avait eu loisir de faire ses
réflexions sur le brillant mais dangereux personnage que faisait à la
cour son ami l'abbé de Polignac. C'était merveilles que le roi l'ignorât
encore. M. de Beauvilliers avait plus d'une raison de le désirer hors
d'ici. Torcy crut donc rendre un grand service à son ami de l'en tirer
promptement, et tout d'un temps au roi et à bien d'autres. Il le proposa
pour l'auditorat de rote\footnote{Voy., sur le tribunal romain appelé
  \emph{la rote}, t. II, p.~833, note.}. Il y fut nommé et il recul cet
emploi comme un honnête exil, dont à la fin Torcy lui fit comprendre la
nécessité et les avantages, vers lequel néanmoins il s'achemina tout le
plus tard qu'il put.

\hypertarget{chapitre-vii.}{%
\chapter{CHAPITRE VII.}\label{chapitre-vii.}}

1706

~

{\textsc{Mort du cardinal de Coislin et sa dépouille.}} {\textsc{- Trois
cent mille livres sur Lyon au maréchal de Villeroy\,; sa puissance à
Lyon.}} {\textsc{- Trois cent mille livres de brevet de retenue au grand
prévôt\,; chanson facétieuse.}} {\textsc{- Quatre cent mille livres de
brevet de retenue au premier écuyer.}} {\textsc{- Grâces pécuniaires
chez M\textsuperscript{me} de Maintenon.}} {\textsc{- Exil de du Charmel
et ses singuliers ressorts.}} {\textsc{- Piété de du Charmel.}}

~

Il se peut dire que l'affaire de M. de Metz mit son oncle au tombeau.
Elle l'avait fait arriver d'Orléans, contre sa coutume, à Noël, et cette
triste affaire s'était terminée avec toutes sortes d'avantages pour M.
de Metz\,; mais le cœur du cardinal de Coislin en avait été flétri, et
ne put reprendre son ressort. Il ne dura que six semaines depuis. Tout à
la fin de janvier, il fut arrêté au lit, et il mourut la nuit du 3 au 4
février. C'était un assez petit homme, fort gros, qui ressemblait assez
à un curé de village, et dont l'habit ne promettait pas mieux, même
depuis qu'il fut cardinal. On a vu en différents endroits la pureté de
mœurs et de vertu qu'il avait inviolablement conservée depuis son
enfance, quoique élevé à la cour et ayant passé sa vie au milieu du plus
grand monde\,; combien il en fut toujours aimé, honoré, recherché dans
tous les âges\,; son amour pour la résidence, sa continuelle sollicitude
pastorale, et ses grandes aumônes. Il fut heureux en choix pour lui
aider à gouverner et à instruire son diocèse, dont il était sans cesse
occupé. Il y fit, entre autres, deux actions qui méritent de n'être pas
oubliées.

Lorsque après la révocation {[}de l'édit{]} de Nantes on mit en tête au
roi de convertir les huguenots à force de dragons et de tourments, on en
envoya un régiment à Orléans, pour y être répandu dans le diocèse. M.
d'Orléans, dès qu'il fut arrivé, en fit mettre tous les chevaux dans ses
écuries, manda les officiers et leur dit qu'il ne voulait pas qu'ils
eussent d'autre table que la sienne\,; qu'il les priait qu'aucun dragon
ne sortît de la gille, qu'aucun ne fît le moindre désordre, et que,
s'ils n'avaient pas assez de subsistance, il se chargeait de la leur
fournir\,; surtout qu'ils ne dissent pas un mot aux huguenots, et qu'ils
ne logeassent chez pas un d'eux. Il voulait être obéi et il le fut. Le
séjour dura un mois et lui coûta bon, au bout duquel il fit en sorte que
ce régiment sortît de son diocèse et qu'on n'y renvoyât plus de dragons.
Cette conduite pleine de charité, si opposée à celle de presque tous les
autres diocèses et des voisins de celui d'Orléans, gagna presque autant
de huguenots que la barbarie qu'ils souffraient ailleurs. Ceux qui se
convertirent le voulurent et l'exécutèrent de bonne foi, sans contrainte
et sans espérance. Ils furent préalablement bien instruits, rien ne fut
précipité, et aucun d'eux ne retourna à l'erreur. Outre la charité, la
dépense et le crédit sur cette troupe, il fallait aussi du courage pour
blâmer, quoique en silence, tout ce qui se passait alors et que le roi
affectionnait si fort, par une conduite si opposée. La même bénédiction
qui la suivit s'étendit encore jusqu'à empêcher le mauvais gré et pis
qui en devait naturellement résulter.

L'autre action, toute de charité aussi, fut moins publique et moins
dangereuse, mais ne fut pas moins belle. Outre les aumônes publiques,
qui de règle consumaient tout le revenu de l'évêché tous les ans, M.
d'Orléans en faisait quantité d'autres qu'il cachait avec grand soin.
Entre celles-là, il donnait quatre cents livres de pension à un pauvre
gentilhomme ruiné qui n'avait ni femme ni enfants, et ce gentilhomme
était presque toujours à sa table tant qu'il était à Orléans. Un matin
les gens de M. d'Orléans trouvèrent deux fortes pièces d'argenterie de
sa chambre disparues, et un d'eux s'était aperçu que ce gentilhomme
avait beaucoup tourné là autour. Ils dirent leur soupçon à leur maître
qui ne le put croire, mais qui s'en douta sur ce que ce gentilhomme ne
parut plus. Au bout de quelques jours il l'envoya quérir, et tête à tête
il lui fit avouer qu'il était le coupable. Alors M. d'Orléans lui dit
qu'il fallait qu'il se fût trouvé étrangement pressé pour commettre une
action de cette nature, et qu'il avait grand sujet de se plaindre de son
peu de confiance de ne lui avoir pas découvert son besoin. Il tira vingt
louis de sa poche qu'il lui donna, le pria de venir manger chez lui à
son ordinaire, et surtout d'oublier, comme il le faisait, ce qu'il ne
devait jamais répéter. Il défendit bien à ses gens de parler de leur
soupçon, et on n'a su ce trait que par le gentilhomme même, pénétré de
confusion et de reconnaissance.

M. d'Orléans fut souvent et vivement pressé par ses amis de remettre son
évêché, surtout depuis qu'il fut cardinal. Ils lui représentaient que,
n'en ayant jamais rien touché, il ne s'apercevrait pas de cette perte du
côté de l'intérêt, que de celui du travail ce lui serait un grand
soulagement, et que cela le délivrerait des disputes continuelles qu'il
avait avec le roi, et qui le fâchaient quelquefois sur la résidence. En
effet, lorsque M\textsuperscript{me} la duchesse de Bourgogne approcha
du terme d'accoucher du prince qui ne vécut qu'un an, et qui fut le
premier enfant qu'elle eut, le roi envoya un courrier à M. d'Orléans
avec une injonction très expresse de sa main de venir sur-le-champ, et
de demeurer à la cour jusqu'après les couches\,; à quoi il fallut obéir.
Le roi, outre l'amitié, avait pour lui un respect qui allait à la
dévotion. Il eut celle que l'enfant qui naîtrait ne fût pas ondoyé d'une
autre main que la sienne\,; et le pauvre homme, qui était fort gras et
grand sœur, ruisselait dans l'antichambre, en camail et en rochet, avec
une telle abondance que le parquet en était mouillé tout autour de lui.

Jamais il ne voulut entendre à remettre son évêché. Il convenait de
toutes les raisons qui lui étaient alléguées\,; mais il y objectait
qu'après tant d'années de travail dont il voyait les fruits, il ne
voulait pas s'exposer de son vivant à voir ruiner une moisson si
précieuse, des écoles si utiles, des curés si pieux, si appliqués, si
instruits, ecclésiastiques excellents qui gouvernaient avec lui le
diocèse, et d'autres, qui le conduisaient par différentes parties, qu'on
chasserait et qu'on tourmenterait, et pour cela seul il demeura
fermement évêque. On verra bientôt que ce fut une prophétie.

Toute la cour s'affligea de sa mort\,; le roi plus que personne, qui fit
son éloge. Il manda le curé de Versailles, lui ordonna d'accompagner le
corps jusque dans Orléans, et voulut qu'à Versailles et sur la route on
lui rendît tous les honneurs possibles. Celui de l'accompagnement du
curé n'avait jamais été fait à personne.

On sut de ses valets de chambre, après sa mort, qu'il se macérait
habituellement par des instruments de pénitence, et qu'il se relevait
toutes les nuits et passait à genoux une heure en oraison. Il reçut les
sacrements avec une grande piété, et mourut comme il avait vécu, la nuit
suivante.

Dès le lendemain le roi manda par un courrier au cardinal de Janson
qu'il lui donnait sa charge. Ce fut pour lui art nouveau sujet
d'empressement de -retour, et au cardinal de Bouillon un nouveau coup de
massue. M. de Metz, qui arriva pour l'extrémité de son oncle à qui il
devait tout, en parut le moins touché, et scandalisa fort toute la cour.
Orléans fut donné à l'évêque d'Angers. Pelletier, son père, écrivit au
roi, de sa retraite, pour le supplier de dispenser son fils de cette
translation. Le roi, excité par M\textsuperscript{me} de Maintenon et
par M. de Chartres, le voulut absolument\,; et Saint-Sulpice, qui avec
sa grossièreté ordinaire regardait ce diocèse comme fort infecté, mais
qui n'osait encore le dire, fit accepter M. d'Angers, dont son père fut
très affligé. Il parut que Dieu n'approuva pas ce choix, par la mort du
translaté qui ne dura pas deux ans. La persécution était réservée à
l'évêque d'Aire, frère d'Armenonville, qu'un coup de soleil avait achevé
d'hébéter, et qui n'en revint jamais bien dans le long temps qu'il
vécut.

Le roi avait donné au maréchal de Villeroy trois cent mille livres à
prendre sur les octrois de Lyon, payables cinquante mille livres par an,
en six années. Elles venaient de finir. Le même don lui fut renouvelé.
On se repent quelquefois après d'avoir payé d'avance de méchants
ouvriers. Alincourt, son grand-père, avait eu la survivance du
gouvernement de Lyon, Lyonnais, etc., de Mandelot, en épousant sa fille,
sous Henri III. La Ligue avait fait ce mariage entre Mandelot et le
secrétaire d'État Villeroy, plus ardents ligueurs l'un que l'autre. De
père en fils ce gouvernement était demeuré aux Villeroy. Alincourt, par
son père et par la surprenante alliance que ce gouvernement lui fit
faire avec le connétable de Lesdiguières et le maréchal de Créqui,
s'était rendu le maître à Lyon. La faveur et la souplesse de son fils,
le premier maréchal de Villeroy, l'y maintint, et plus encore le
commandement en chef qu'y eut toute sa vie l'archevêque de Lyon, frère
du maréchal qui s'y rendit le maître despotique de tout. La faveur de ce
maréchal-ci, son neveu, n'eut qu'à maintenir ce qui était établi. Il
disposait donc seul de toutes les charges municipales de la ville\,; il
nommait le prévôt des marchands. L'intendant de Lyon n'a nulle
inspection sur les revenus de la ville, qui sont immenses et peu connus
dans leur étendue, parce qu'ils dépendent en partie du commerce qui s'y
fait, qui est toujours un des plus grands du royaume. Le prévôt des
marchands l'administre seul et n'en rend compte qu'au gouverneur tête à
tête, lequel lui-même n'en rend compte à personne. Il est donc aisé de
comprendre qu'avec une telle autorité c'est un Pérou, outre celle qui
s'étend sur tout le reste, et qui rend la protection du gouverneur si
continuellement nécessaire à tous ces gros négociants de Lyon, comme à
tous les autres bourgeois de la ville, où tout depuis un si long temps
{[}dépend{]} de la même autorité, tout est créature des gouverneurs, et
rien ne se peut que par eux, qui influent jusque dans les affaires
particulières de toutes les familles.

Aussi dînant un jour chez Dangeau avec le maréchal de Villeroy et
beaucoup d'ambassadeurs et d'autres gens, car Dangeau aimait à faire les
honneurs de la cour et les faisait fort bien et magnifiquement, il lui
échappa une fatuité pour faire le grand seigneur, mais fort véritable.
«\,Messieurs, dit-il à la compagnie, de tous nous autres gouverneurs de
province, il n'y a que M. le maréchal qui ait conservé l'autorité dans
la sienne.\,» Le rire me surprit. M\textsuperscript{me} de Dangeau, qui
me regarda et qui plaisantait la première des sottises de son mari,
quoique vivant à merveille ensemble, ne put s'empêcher de sourire. Il
avait acheté le gouvernement de Touraine, et il ne voulait pas que ces
étrangers ignorassent qu'il était aussi gouverneur de province.

Le grand prévôt obtint trois cent mille livres de brevet de retenue sur
sa charge pour son fils, qui épousa une M\textsuperscript{lle} du Hamel
de Picardie, fort riche, et qui ne fut pas heureuse. Heudicourt, le
fils, qui était une espèce de satyre fort méchant et fort mêlé dans les
hautes intrigues galantes, fit dans la suite sur tous ces
Montsoreau\footnote{Nom de famille du grand prévôt.} une chanson si
naïve, si fort d'après nature et si plaisante, que quelqu'un l'ayant
dite à l'oreille au maréchal de Boufflers pendant la messe du roi où il
avait le bâton, il ne put s'empêcher d'éclater de rire. C'était l'homme
de France le plus grave, le plus sérieux, le plus esclave de toute
bienséance. Le roi se retourna de surprise, qui augmenta fort voyant le
maréchal pâmé, à qui les larmes en tombaient des yeux. Rentré dans son
cabinet, il l'appela et lui demanda ce qui l'avait pu mettre en cet
état, à la messe. Le maréchal lui dit la chanson. Voilà le roi plus pâmé
que n'avait été le maréchal, et qui fut plus de quinze jours sans
pouvoir s'empêcher de rire de toute sa force sitôt que le grand prévôt
ou un de ses enfants lui tombait sous les yeux. La chanson courut fort
et divertit extrêmement la cour et la ville.

Le premier écuyer obtint, quelques jours après, aussi un brevet de
quatre cent mille livres sur sa charge. En même temps le roi répandit
quelques grâces pécuniaires dans le domestique de M\textsuperscript{me}
de Maintenon.

Je reçus en ce temps une véritable affliction par l'exil de M. du
Charmel, avec qui depuis longtemps j'avais lié une vraie amitié, et que
je voyais le plus souvent qu'il m'était possible dans sa retraite de
l'Institution. Les ressorts de cet exil méritent de trouver place ici,
et c'est une histoire qui demande des connaissances et des souvenirs
pour être bien entendue. Il faut d'abord connaître le Charmel, se
souvenir de ce que j'ai dit de lui sur sa vie à la cour, du grand monde,
de gros jeu, et de la manière dont il se retira, de la bonté avec
laquelle le roi lui parla alors, et de la dureté avec laquelle il lui
répondit qu'il ne le verrait jamais. Il faut maintenant expliquer quel
il fut dans sa retraite. Ce fut un homme à cilice, à pointes de fer, à
toutes sortes d'instruments de continuelle pénitence. Jeûneur extrême et
sobre d'ailleurs à l'excès, quoique naturellement grand mangeur, et
d'une dureté générale sur lui-même impitoyable. Il passait les carêmes à
la Trappe, au réfectoire soir et matin à la portion des religieux, et
sans manquer aucun de leurs offices du jour et de la nuit. Outre cela,
longtemps en prière en quelque lieu qu'il fût\,; et le vendredi saint, à
la Trappe, il passait à genoux à terre, sans appui, sans livre, sans
changer de posture, sans branler, depuis la fin des matines jusqu'à,
l'office, c'est-à-dire depuis quatre heures du matin jusqu'à dix\,; avec
cela toujours gai et toujours libre et aisé. Il avait une fidélité
inflexible sur tout ce qu'il se proposait. On ne saurait moins d'esprit
que couvrait un grand usage du monde et de la meilleure compagnie, mais
que sa retraite avait rouillé. Il s'était livré à Paris à beaucoup de
bonnes couvres, qui le faisaient un peu courir et se mêler de trop de
choses. Au latin près qu'il avait retenu du collège, il ne savait rien
du tout que ce que les lectures de piété lui avaient appris\,; et comme
il était naturellement tourné à la dureté de l'austérité âpre, il le fut
aisément du côté janséniste, et lia étroitement avec ce qu'il trouva de
gens les plus marqués à ce coin. Il fut ami intime de M. Nicole, jusqu'à
être un des exécuteurs de son testament. Il le fut peut-être plus encore
de M. Boileau, élève de Port-Royal, que M. de Luynes avait mis auprès du
comte d'Albert et du chevalier de Luynes dans leur jeunesse, qui
retinrent mal ses leçons.

C'est ce même Boileau que M. de Paris, depuis cardinal de Noailles, prit
à l'archevêché et à sa table quand il devint archevêque de Paris, et qui
fit contre lui, dans sa propre maison et vivant de son pain, cet étrange
\emph{Problème} dont j'ai parlé (t. II, p.~248), dont le prélat se prit
aux jésuites, mais dont les brouillons originaux et plusieurs lettres à
ce sujet, de la main de ce Boileau, furent trouvés dans l'abbaye
d'Auville, avec ces autres, qui firent à l'archevêque de Reims une
affaire si cruelle avec le roi que j'ai racontée (t. IV, p.~127). Ces
originaux du \emph{Problème}, trouvés par ce hasard, de la main de
Boileau, furent envoyés au cardinal de Noailles. Les jésuites en
triomphèrent, Boileau ne les put ni osa méconnaître. On à vu (t. II,
p.~249) avec quelle bonté le cardinal de Noailles se défit de ce
pernicieux hôte (qui n'avait de pain que celui qu'il lui donnait de sa
propre table) en lui donnant un canonicat de Saint-Honoré qui lui
fournit une très honnête subsistance et un logement. Cette noire
ingratitude ne se pouvait excuser, non plus que la noirceur d'avoir si
naturellement fait retomber ce cruel trait sur les jésuites, avec qui le
cardinal de Noailles, évêque, archevêque et cardinal sans eux, et
pensant fort différemment d'eux, ne fut jamais bien.

Le Charmel, qui voyait souvent le cardinal de Noailles, et que le
cardinal aimait et distinguait fort, cessa dans cet éclat de le voir, et
continua avec Boileau le commerce et l'amitié la plus étroite. Le
cardinal (je l'appelle ainsi sans distinction des temps où il ne l'était
pas encore) en fut moins blessé que touché par amitié. Il fit parler au
Charmel, le fit prier de le venir voir, l'obtint avec peine, lui parla
lui-même. Tant d'avances furent inutiles\,; le Charmel s'aigrit de plus
en plus. Les jansénistes, fâchés que le cardinal n'épousât pas toutes
leurs idées, et qui de dépit s'étaient portés à cette étrange extrémité,
avaient infatué leur prosélyte, qui ne put jamais apercevoir
d'ingratitude, de crime, de trahison, de noirceur où ils étaient si
évidents\,; et voilà où son peu d'esprit et de lumières, et un fol
abandon de ce qu'il croyait des saints, conduisirent un homme d'ailleurs
si droit et si saint lui-même. Il faudrait prétendre porter les hommes
au-dessus de toute humanité, pour se persuader que le cardinal de
Noailles ne dût pas être très sensible à la conduite du Charmel à son
égard, surtout après celle qu'il avait eue et avec Boileau et avec
lui-même. Telle fut la faute inexcusable du Charmel à l'égard du
cardinal de Noailles. Venons maintenant à celle qu'il fit dans la suite
à l'égard du roi.

On a vu (t. IV, p.~282) sur Troisvilles, que le roi empêcha d'être de
l'Académie, son dépit contre les gens retirés qui ne le voyaient point.
J'ai réservé pour ce lieu-ci à dire que le même jour qu'il refusa
Troisvilles, il s'alla promener à Marly, où il s'étendit amèrement sur
cette matière. Il loua les solitaires de la campagne\,; il s'étendit sur
M. de Saint-Louis, sur ses actions sous ses yeux en la guerre de
Hollande et ailleurs, sur la vie qu'il menait à la Trappe, et dit qu'il
ne trouvait point mauvais que ceux-là ne vinssent pas de loin pour le
voir\,; retombant de là sur les gens retirés à Paris et aux environs, il
loua Pelletier, Fieubet, le chevalier de Gesvres, qui le venaient voir
une ou deux fois l'année, et qui valaient bien Troisvilles et le
Charmel, sur qui il tomba fort, et répéta souvent qu'ils avaient plus de
commerce d'intrigues et d'affaires qu'avant leur retraité, et que toute
leur dévotion ils la mettaient à ne le point voir. Le duc de Tresmes,
fort ami du Charmel, ricanait jaune, et se mettait tantôt sur un pied,
tantôt sur un autre. Cavoye, autre ami du Charmel, se mit dans la
conversation, et avec sa réputation et sa morgue, bavarda force sottes
flatteries, et tomba sur son ami pour faire le bon valet. On ne
devinerait jamais qui le défendit\,: un homme qui à peine l'avait connu,
un homme d'ailleurs fort courtisan, mais courtisan en homme qui se sent,
qui a de la hauteur et de la dignité, qui connaissait Cavoye pour ami
particulier du Charmel, et qui fut indigné de ce qu'il entendait. Ce fut
Harcourt qui prit sa défense, si honnêtement et avec tant d'esprit que
le roi cessa ce propos et se mit sur autre chose.

Cavoye pourtant fit apparemment ses réflexions. Harcourt l'avait fait
rentrer en lui-même. Il écrivit donc au Charmel ce qui s'était passé à
Marly, mais non le personnage qu'il y avait fait, et lui conseilla de
lui écrire de manière qu'il pût dire au roi qu'il désirait l'honneur de
se présenter devant lui après tant d'années, sans oser le faire qu'il ne
sût qu'il le trouverait bon\,; moyennant quoi, accordé, il né lui en
coûterait qu'une course à Versailles d'une matinée, ou refusé, le roi
n'aurait plus ce dépit contre lui. Le Charmel me montra cette lettre, si
résolu de n'en faire aucun usage que je ne pus le persuader.

À quinze jours delà, en une autre promenade à Marly, le roi reprit, mais
plus légèrement, la même matière des gens retirés qui ne le voyaient
point, et tout de suite demanda à Cavoye ce que faisait le Charmel et
s'il y avait longtemps qu'il n'avait eu de ses nouvelles. Cavoye le
manda dès le lendemain au Charmel, le pressa de suivre le conseil qu'il
lui avait donné la première fois, et lui fit sentir que cette récidive
si marquée sur lui montrait évidemment qu'il s'était attendu à ouïr
parler de lui sur son premier discours, et qu'il serait fort blessé si
ce second demeurait inutile. Le Charmel me montra la lettre. Je lui dis
qu'il n'y avait ni à balancer ni un moment à perdre\,; qu'il l'avait
beau sur ce que le roi avait dit sur lui à Cavoye de lui récrire qu'il
s'en était cru oublié, que, puisqu'il était si heureux que le roi
daignât encore se souvenir de lui, il priait Cavoye de lui demander la
permission qu'il pût aller lui embrasser les genoux, dans le vif
souvenir de ses bontés passées, que c'était un désir auquel il ne
pouvait résister, etc. Je le pris par la religion, par le devoir et le
respect d'un sujet à son roi, qui doit chercher à lui plaire et non pas
à l'irriter\,; que c'était un devoir étroit d'une part, et une sage
précaution de l'autre, de saisir l'occasion de détourner l'orage auquel
ses volontaires indiscrétions sur le jansénisme ne donnaient que trop
d'ouverture, et de se faire de l'aigreur du roi si suivie un
contrepoison et un bouclier par une conduite qui sûrement lui serait
agréable, et qu'il était visible qu'il demandait de lui\,; qu'une seule
matinée, aller et venir, y serait non seulement sagement et utilement
employée, mais saintement, et qu'après tant d'années de retraite il ne
devait pas craindre une dissipation d'un moment qu'il n'avait pas
recherchée et qui devenait si nécessaire. Jamais je ne pus l'y engager.

Il se contenta d'une lettre ostensible et d'une autre pour le roi. Tout
cela fut très médiocrement reçu.

La vérité est qu'il se craignit trop lui-même\,; il redouta une trop
favorable réception. Après tant d'années de pénitence, il ne se sentit
pas assez dépouillé d'un reste de complaisance de sa faveur et de ses
agréments passés qui l'avaient tant dominé autrefois. Il avait refusé
M\textsuperscript{me} de Maintenon, il avait peu d'années, d'un commerce
de bonnes œuvres qu'elle avait voulu lier avec lui. Il appréhenda tout
autre commerce qu'avec Dieu, pour qui il voulut réserver sa liberté
entière, et peut-être y fut-il conduit par son esprit pour le purifier
par une plus dure pénitence et qui ne serait pas de son choix.

Revenons au cardinal de Noailles. L'année précédente, 1705, avait été
celle de la grande assemblée du clergé. Le cardinal de Noailles, qui y
présida, crut en devoir profiter pour y faire régler divers points de
morale et de discipline, quoique ces assemblées ne soient destinées
qu'aux affaires temporelles du clergé\,; que ceux qui y sont députés
n'aient point d'autres matières dans les procurations qu'ils y apportent
de leurs commettants\,; et que la cour même soit ordinairement en garde
contre tout ce qui s'y pourrait proposer qui ne concernerait pas l'objet
temporel de ces assemblées. Ce projet du cardinal n'était pas de lui
seul. De plus, il avait fallu le concerter d'avance avec quelques
prélats principaux qui devaient être de l'assemblée, et convenir de la
manière de le proposer par articles, et le faire passer peu à peu. Les
jésuites, toujours à l'affût sur le cardinal de Noailles et sur tout ce
qui pouvait intéresser leur doctrine et leur morale, pénétrèrent ce
projet, dans le secret duquel il se trouva quelque faux frère qui le
leur donna tel qu'il devait être proposé à l'assemblée. Le P. de La
Chaise en parla au roi, qui, en ce tentes-là aimait fort le cardinal de
Noailles, et qui s'éleva tellement contre cet avis de son confesseur,
que La Chaise, homme sage et prudent, se tut tout court, sûr de n'y
revenir que mieux dans la suite.

En effet, l'assemblée ouverte, il fut averti de point en point. Il
annonça d'avance au roi la proposition qui s'allait faire, et qui fut
faite au jour qu'il l'avait dit au roi. Il en fut de même de toutes les
autres. Le roi en parla au cardinal de Noailles, qui ne s'arrêta point
pour cela, résolu à faire ce qu'il crut être le bien, à quelque prix que
ce fût. Les jésuites, outrés du peu de fruit qu'ils retiraient de la
trahison qui avait été faite au cardinal de Noailles, qui allait
toujours en avant dans l'assemblée sur la morale et la discipline,
échauffèrent le roi par le P. de La Chaise, et procurèrent au cardinal
toutes sortes de dégoûts. J'en étais informé par l'archevêque d'Arles,
qui, député du second ordre dans une autre assemblée, s'était piqué sur
ce qu'il ne trouva pas que le cardinal de Noailles lui marquât assez de
considération, et qui, député du premier ordre en celle-ci, lui fut
opposé en tout, et servit de tout son pouvoir sa haine, sa fortune et
les jésuites tout à la fois, auxquels il n'avait garde de n'être pas
obséquieux en tout avec les vues et l'ambition qui le dévorait.

Le cardinal de Noailles sortit donc de cette assemblée fort mal avec le
roi, qui prit contre lui les plus forts soupçons du jansénisme, et qui,
profondément ignorant sur ces matières, élevé dans le préjugé le plus
extrême là-dessus, ne consulta jamais personne qui pût l'éclairer, et ne
permit même jamais à personne d'ouvrir la bouche devant lui, qui pût lui
donner la moindre lumière. Ainsi on avait beau jeu à lui faire passer
pour erreur et pour jansénisme tout ce qu'il était utile à ceux qui
profitaient de ses ténèbres de lui faire passer pour tel, soit choses,
soit gens\,; et ils avaient de plus usurpé cet incomparable avantage,
que, choses et gens, donnés pour tels, demeuraient proscrits, sans
examen, sans information et sans ressource.

Le cardinal de Noailles trempait donc dans un état de disgrâce
intérieure qui, pour ne paraître pas au dehors et ne changer rien à ses
audiences du roi de toutes les semaines, n'en était pas moins douloureux
et embarrassant. Sa famille, à qui son crédit et sa place donnaient tant
de lustre et de moyens, en était affligée. M\textsuperscript{me} de
Maintenon, sur qui les jésuites n'avaient aucune prise, ne l'était pas
moins. Nulle issue que quelque coup d'éclat contre les jansénistes qui
ramenât le roi. Mais où le prendre\,? Le cardinal voulait, avant tout,
conserver la bonne morale et la discipline, il ne voulait pas sacrifier
ses amis. Cependant il était sans cesse pressé par M\textsuperscript{me}
de Maintenon et par sa famille de chercher quelque chose à faire
là-dessus, et lui-même en sentait la nécessité, même pour l'utilité
spirituelle, à laquelle on l'avait rendu une pierre d'achoppement.

Vers le commencement de cette année, le P. Quesnel était fort pourchassé
dans les Pays-Bas espagnols, où le roi avait tout pouvoir. Ce fut
merveilles qu'il put échapper de Bruxelles et se retirer en Hollande. Il
alla et vint des gens de sa part à Paris. On en fut informé\,; on
avertit le cardinal de Noailles que ces gens-là étaient en commerce avec
le Charmel. Il les crut occupés à quelque ouvrage contre lui\,; la pique
du \emph{Problème} se renouvela. Il fut excité contre le Charmel par des
gens qui s'en aperçurent et qui en espérèrent du mal pour l'un et de
l'obscurcissement à la réputation de l'autre. Ils lui persuadèrent que
le Charmel recelait chez lui ces messagers\,; on mit des espions en
campagne qui le certifièrent, et ces rapports aigrirent tout à fait le
cardinal. Il faut avouer que, sur le jansénisme, jamais homme ne fut si
indiscret que le Charmel. Il s'en faisait une religion. On ne put jamais
lui faire entendre raison là-dessus. Il n'y avait guère de j ours où sa
conduite à cet égard ne fît trembler ses amis.

Nous étions à Marly. Pontchartrain m'apprit un matin que le roi lui
venait d'ordonner d'expédier une lettre de cachet pour exiler le Charmel
en sa maison du Charmel, près Château-Thierry, avec défense d'en
sortir\,; et que, l'ayant rappelé un peu après, il lui avait commandé de
la lui envoyer par un officier de la maréchaussée qui le fit et le vit
partir dans les vingt-quatre heures, qui se tînt cependant auprès de
lui, et qui rendît compte de tout ce qu'il aurait vu et entendu aussitôt
après son départ. Pontchartrain, qui me savait fort de ses amis, me
demanda le secret jusqu'à ce que la chose fût répandue, et avait voulu
m'en avertir d'avance pour prévenir ce que la surprise et la colère
eussent pu tirer de moi en l'apprenant par le monde. Le soir, à la
musique, la comtesse de Mailly se vint mettre auprès de moi un peu après
qu'elle fut commencée. Nos deux sièges se trouvèrent un peu écartés des
autres. Elle me fit la même confidence, et dans la même vue, que m'avait
fait Pontchartrain. Je fis le surpris à cause du secret qu'il m'avait
demandé\,; mais je le devins tout de bon lorsqu'elle ajouta que c'était
un coup du cardinal de Noailles, qui, le matin même, avait dit au roi
que le Charmel était un janséniste et un brouillon qui allait tête levée
par les maisons, exhortant les gens au jansénisme, qui avait dit au P.
de La Tour, général de l'Oratoire, que, maintenant qu'il était à la tète
du parti, tout était perdu s'il mollissait\,; qu'en un mot, c'était un
homme qu'il fallait chasser de Paris, ce qui avait été ordonné dans le
moment\,; que ce qu'elle me disait là, elle le savait de bon lieu,
puisque c'était de chez M\textsuperscript{me} de Maintenon. Elle était
sa nièce, sa protégée et dame d'atours de M\textsuperscript{me} la
duchesse de Bourgogne. Nous ne prolongeâmes point notre conversation
pour qu'on ne remarquât point que nous parlions de quelque chose
d'intéressant. C'était un mercredi 10 février, jour de l'audience réglée
du cardinal de Noailles, et jour encore où Chamillart s'en allait
d'ordinaire à l'Étang jusqu'au samedi.

Le lendemain matin que je projetais d'y aller, le maréchal de Noailles
me prit dans la ruelle du roi, comme nous l'attendions à sortir de son
cabinet pour la promenade, me dit l'exil du Charmel, qu'il en avait reçu
une lettre sur laquelle il avait essayé d'obtenir qu'il pût demeurer aux
Camaldules de Gros-Bois\footnote{Les camaldules, ordre monastique
  originaire d'Italie, tiraient leur nom de Camaldoli, solitude située
  au milieu des Apennins. Ils vinrent s'établir en France en 1626, et y
  fondèrent six maisons dont la plus considérable était près de
  Gros-Bois (Seine-et-Oise).}, où il allait un jour ou deux tous les
mois, qu'il en avait été refusé avec aigreur\,; s'étonna et se lamenta
fort de ce coup imprévu, et me pressa d'en découvrir la cause par
Pontchartrain qui avait expédié la lettre de cachet. Je fus doublement
piqué, sachant si sûrement ce que je savais, de la feinte du maréchal,
et du panneau où était tombé mon pauvre ami en s'adressant à lui. Je
répondis brusquement au maréchal qu'il était plus à portée que moi d'en
être informé, puisque à la vie que menait le Charmel, il ne pou voit
être question que de doctrine, laquelle était de la compétence de son
frère, qui avait longtemps vu le roi seul la veille, au matin, jour que
cet ordre avait été donné à ce qu'il m'apprenait. Là-dessus le roi
sortit de son cabinet. Nous nous quittâmes, et jamais depuis nous ne
nous en sommes parlé.

Au partir de là j'allai dîner à l'Étang, et comme j'étais en toute
intimité avec Chamillart, je lui contai avec dépit le malheur du Charmel
qui venait de devenir public. Il me dit qu'il le savait. J'ajoutai qu'au
moins je lui en apprendrais ce qu'il ne savait pas\,; et je lui contai,
sans nommer personne, ce que M\textsuperscript{me} de Mailly m'avait
dit, et la fausseté avec laquelle le maréchal de Noailles venait de m'en
parler. Je n'eus pas achevé que Chamillart si doux, si modéré, si
tranquille, entra tout à coup en fureur. Nous étions dans son cabinet
tête à tête. Il pesta, il frappa des pieds, il ne se possédait pas. Je
lui demandai à qui il en avait. «\,Ce que j'ai\,? me répondit-il en
frappant du poing sur la table, c'est qu'il n'y a plus de secret chez le
roi. Ce que vous me contez là, le roi me le dit hier chez
M\textsuperscript{me} de Maintenon mot pour mot, dans le même
arrangement que vous me le dites, cinq ou six heures après avoir vu le
cardinal de Noailles, et me défendit d'en parler à qui que ce soit. Je
vois cependant que vous en êtes de point en point instruit\,; que
puisque vous l'êtes, d'autres le peuvent être de même\,; et qu'il est
bien douloureux à un honnête homme accoutumé aux plus importants
secrets, d'être chargé de ceux qui se communiquent à d'autres, et de
pouvoir ainsi être confondu avec ceux qui ne les gardent pas.\,»
Là-dessus il me raconta que, la même chose lui étant arrivée une autre
fois, il s'en fut aussitôt le dire au roi, et le supplier de ne le pas
rendre responsable de ce dont il s'ouvrirait à d'autres qu'à lui, sur
quoi le roi lui avait avoué qu'il en avait aussi fait part à une autre
personne. J'approuvai sa colère, mais je le priai de ne se pas servir du
même remède.

Plus certain encore, si faire se pouvait, par le récit de Chamillart,
d'où le coup était parti, j'en fis avertir le Charmel. Il était déjà
parti. Il est difficile de comprendre avec combien d'humilité et de
douceur cet homme, naturellement impétueux, reçut sa lettre de cachet et
ce garde à vue, et avec quelle ponctualité il obéit. J'essayai divers
moyens de le faire revenir, mais l'aigreur était trop grande. Le Charmel
eût été bien aise de recouvrer sa liberté, mais il ne voulut pas y
contribuer en rien, persuadé qu'il devait se tenir fidèlement sous la
main de Dieu dans une pénitence qu'il n'avait pas choisie, dans un
pardon effectif de ceux qui l'y avaient confiné, et dans une paix
profonde. Beauvau, fils de sa sœur et son héritier, marié en Lorraine,
et qui sous le nom de M. de Craon y a fait, lui et sa femme, une si
énorme fortune, pointait déjà dans cette faveur qui lui a valu tant de
millions et de titres. Le duc de Lorraine s'offrit de s'intéresser pour
le Charmel auprès du roi. Il l'en remercia et le supplia de le laisser
dans l'état où Dieu l'avait mis, et où il demeura le reste de sa vie qui
dura encore longtemps. Nous verrons à sa fin combien tout adoucissement
était impossible, et quel fut l'excès de la dureté que le roi exerça sur
lui, et qui put être cause de sa mort.

\hypertarget{chapitre-viii.}{%
\chapter{CHAPITRE VIII.}\label{chapitre-viii.}}

1706

~

{\textsc{Duc de Vendôme\,; ses mœurs\,; son caractère\,; sa conduite.}}
{\textsc{- Albéroni\,; commencement de sa fortune.}} {\textsc{- Voyage
triomphant de Vendôme à la cour.}} {\textsc{- Patente de maréchal
général offerte, et refusée par Vendôme.}} {\textsc{- Grand prieur\,;
son caractère.}} {\textsc{- Berwick, fait maréchal de France à
trente-cinq ans, retourne en Espagne.}} {\textsc{- Roquelaure va
commander en Languedoc.}} {\textsc{- Le comte de Toulouse et le maréchal
de Cœuvres à Toulon.}} {\textsc{- Petits exploits du duc de Noailles.}}
{\textsc{- Tessé fait asseoir sa belle-fille en dupant les deux rois.}}
{\textsc{- Mort de la reine douairière d'Angleterre.}} {\textsc{- Comte
de Fervesham.}} {\textsc{- Mort de Belesbat.}} {\textsc{- Mort de
Polastron.}} {\textsc{- Catastrophe de Saint-Adon.}} {\textsc{- Querelle
qui jette M\textsuperscript{me} de Barbezieux dans un couvent.}}
{\textsc{- Mariage du comte de Rochechouart avec M\textsuperscript{lle}
de Blainville.}} {\textsc{- Mariage du duc d'Uzès avec une fille de
Bullion.}} {\textsc{- Mariage du prince de Tarente avec
M\textsuperscript{lle} de La Fayette.}} {\textsc{- Origine des
distinctions de M. de La Trémoille.}} {\textsc{- Ducs de Bouillon et
d'Albret raccommodés.}} {\textsc{- Vingt mille livres de pension pendant
la guerre au comte d'Évreux.}} {\textsc{- Victoire des Suédois.}}

~

La cour et Paris virent en ce temps-ci un spectacle vraiment prodigieux.
M. de Vendôme n'était point parti d'Italie, depuis qu'il y avait succédé
au maréchal de Villeroy après l'affaire de Crémone, Ses combats tels
quels, les places qu'il avait prises, l'autorité qu'il avait saisie, la
réputation qu'il avait usurpée, ses succès incompréhensibles dans
l'esprit et dans la volonté du roi, la certitude de ses appuis, tout
cela lui donna le désir de venir jouir à la cour d'une situation si
brillante, et qui surpassait de si loin tout ce qu'il avait pu espérer.
Mais avant de voir arriver un homme qui va prendre un ascendant si
incroyable, et dont jusqu'ici je n'ai parlé qu'en passant, il est bon de
le faire connaître davantage, et d'entrer même dans des détails qui ont
de quoi surprendre, et qui le peindront d'après nature.

Il était d'une taille ordinaire pour la hauteur, un peu gros, mais
vigoureux, fort et alerte\,; un visage fort noble et l'air haut\,; de la
grâce naturelle dans le maintien et dans la parole\,; beaucoup d'esprit
naturel qu'il n'avait jamais cultivé, une énonciation facile, soutenue
d'une hardiesse naturelle, qui se tourna depuis en audace la plus
effrénée\,; beaucoup de connaissance du monde, de la cour, des
personnages successifs, et sous une apparente incurie un soin et une
adresse continuelle à en profiter en tout genre, surtout admirable
courtisan, et qui sut tirer avantage jusque de ses plus grands vices, à
l'abri du faible du roi pour sa naissance\,; poli par art, mais avec un
choix et une mesure avare\,; insolent à l'excès dès qu'il crut le
pouvoir oser impunément, et en même temps familier et populaire avec le
commun, par une affectation qui voilait sa vanité et le faisait aimer du
vulgaire\,; au fond, l'orgueil même, et un orgueil qui voulait tout, qui
dévorait tout. À mesure que son rang s'éleva et que sa faveur augmenta,
sa hauteur, son peu de ménagement, son opiniâtreté jusqu'à l'entêtement,
tout cela crût à proportion, jusqu'à se rendre inutile toute espèce
d'avis, et se rendre inaccessible qu'à un nombre très petit de familiers
et à ses valets. La louange, puis l'admiration, enfin l'adoration furent
le canal unique par lequel on put approcher ce demi-dieu, qui soutenait
des thèses ineptes sans que personne osât, non pas contredire, mais ne
pas approuver.

Il connut et abusa plus que personne de la bassesse du Français. Peu à
peu il accoutuma les subalternes, puis de l'un à l'autre toute son
armée, à ne l'appeler plus que Monseigneur et Votre Altesse. En moins de
rien cette gangrène gagna jusqu'aux lieutenants généraux et aux gens les
plus distingués, dont pas un, comme des moutons à l'exemple les uns des
autres, n'osa plus lui parler autrement, et qui d'usage ayant passé en
droit, y auraient hasardé l'insulte si quelqu'un d'eux se fût avisé de
lui parler autrement.

Ce qui est prodigieux à qui a connu le roi, galant aux dames une si
longue partie de sa vie, dévot l'autre, souvent avec importunité pour
autrui, et dans toutes ces deux parties de sa vie plein d'une juste,
mais d'une singulière horreur pour tous les habitants de Sodome, et
jusqu'au moindre soupçon de ce vice, M. de Vendôme y fut plus salement
plongé toute sa vie que personne, et si publiquement, que lui-même n'en
faisait pas plus de façon que de la plus légère et de la plus ordinaire
galanterie, sans que le roi, qui l'avait toujours su, l'eût jamais
trouvé mauvais, ni qu'il en eût été moins bien avec lui. Ce scandale le
suivit toute sa vie à la cour, à Anet, aux armées. Ses valets et des
officiers subalternes satisfirent toujours cet horrible goût, étaient
connus pour tels, et comme tels étaient courtisés des familiers de M. de
Vendôme et de ce qui voulait s'avancer auprès de lui. On a vu avec
quelle audacieuse effronterie il fit publiquement le grand remède, par
deux fois prit congé pour l'aller faire, qu'il fut le premier qui l'eût
osé, et que sa santé devint la nouvelle de la cour, et avec quelle
bassesse elle y entra, à l'exemple du roi, qui n'aurait pas pardonné à
un fils de France ce qu'il ménagea avec une faiblesse si étrange et si
marquée pour Vendôme.

Sa paresse était à un point qui ne se peut concevoir. Il a pensé être
enlevé plus d'une fois pour s'être opiniâtré dans un logement plus
commode, mais trop éloigné, et risqué les succès de ses campagnes, donné
même des avantages considérables à l'ennemi, pour ne se pouvoir résoudre
à quitter un camp où il se trouvait logé à son aise. Il voyait peu à
l'armée par lui-même, il s'en fiait à ses familiers que très souvent
encore il n'en croyait pas. Sa journée, dont il ne pouvait troubler
l'ordre ordinaire, ne lui permettait guère de faire autrement. Sa saleté
était extrême, il en tirait vanité\,; les sots le trouvaient un homme
simple. Il était plein de chiens et de chiennes dans son lit qui y
faisaient leurs petits à ses côtés. Lui-même ne s'y contraignait de
rien. Une de ses thèses était que tout le monde en usait de même, mais
n'avait pas la bonne foi d'en convenir comme lui. Il le soutint un jour
à M\textsuperscript{me} la princesse de Conti, la plus propre personne
du monde et la plus recherchée dans sa propreté.

Il se levait assez tard à l'armée, se mettait sur sa chaise percée, y
faisait ses lettres, et y donnait ses ordres du matin. Qui avait affaire
à lui, c'est-à-dire pour les officiers généraux et les gens distingués,
c'était le temps de lui parler. Il avait accoutumé l'armée à cette
infamie. Là, il déjeunait à fond, et souvent avec deux ou trois
familiers, rendait d'autant, soit en mangeant, soit en écoutant ou en
donnant ses ordres, et toujours force spectateurs debout. (Il faut
passer ces honteux détails pour le bien connaître.) Il rendait
beaucoup\,; quand le bassin était plein à répandre, on le tirait et on
le passait sous le nez de toute la compagnie pour l'aller vider, et
souvent plus d'une fois. Les jours de barbe, le même bassin dans lequel
il venait de se soulager servait à lui faire la barbe. C'était une
simplicité de mœurs, selon lui, digne des premiers Romains, et qui
condamnait tout le faste et le superflu des autres. Tout cela fini, il
s'habillait, puis jouait gros jeu au piquet ou à l'hombre, ou s'il
fallait absolument monter à cheval pour quelque chose, c'en était le
temps. L'ordre donné au retour, tout était fini chez lui. Il soupait
avec ses familiers largement\,; il était grand mangeur, d'une
gourmandise extraordinaire, ne se connaissait à aucun mets, aimait fort
le poisson, et mieux le passé et souvent le puant que le bon. La table
se prolongeait en thèses, en disputes, et par-dessus tout, louanges,
éloges, hommages toute la journée et de toutes parts.

Il n'aurait pardonné le moindre blâme à personne. Il voulait passer pour
le premier capitaine de son siècle, et parlait indécemment du prince
Eugène et de tous les autres. La moindre contradiction eût été un crime.
Le soldat et le bas officier l'adoraient pour sa familiarité avec eux,
et la licence qu'il tolérait pour s'en gagner les cœurs, dont il se
dédommageait par une hauteur sans mesure avec tout ce qui était élevé en
grade ou en naissance. Il traitait à peu près de même ce qu'il y avait
de plus grand en Italie, qui avait si souvent affaire à lui. C'est ce
qui fit la fortune du fameux Albéroni.

Le duc de Parme eut à traiter avec M. de Vendôme\,; il lui envoya
l'évêque de Parme, qui se trouva bien surpris d'être reçu par M. de
Vendôme sur sa chaise percée, et plus encore de le voir se lever au
milieu de la conférence et se torcher le cul devant lui. Il en fut si
indigné que, toutefois sans mot dire, il s'en retourna à Parme sans
finir ce qui l'avait amené, et déclara à son maître qu'il n'y
retournerait de sa vie après ce qui lui était arrivé. Albéroni était
fils d'un jardinier, qui, se sentant de l'esprit, avait pris un petit
collet pour, sous une figure d'abbé, aborder où son sarrau de toile eût
été sans accès. Il était bouffon\,; il plut à M. de Parme comme un bas
valet dont on s'amuse\,; en s'en amusant il lui trouva de l'esprit, et
qu'il pouvait n'être pas incapable d'affaires. Il ne crut pas que la
chaise percée de M. de Vendôme demandât un autre envoyé, il le chargea
d'aller continuer et finir ce que l'évêque de Parme avait laissé à
achever.

Albéroni, qui n'avait point de morgue à garder et qui savait très bien
quel était Vendôme, résolut de lui plaire à quelque prix que ce fût,
pour venir à bout de sa commission au gré de son maître et de s'avancer,
par là auprès de lui. Il traita donc avec M. de Vendôme sur sa chaise
percée, égaya son affaire par des plaisanteries qui firent d'autant
mieux rire le général qu'il l'avait préparé par force louanges et
hommages. Vendôme en usa avec lui comme il avait fait avec l'évêque, il
se torcha le cul, devant lui. À cette vue Albéroni s'écrie\,: \emph{O
culo di angelo\,!} et courut le baiser. Rien n'avança plus ses affaires
que cette infâme bouffonnerie. M. de Parme qui dans sa position avait
plus d'une chose à traiter avec M. de Vendôme, voyant combien Albéroni y
avait heureusement commencé, se servit toujours de lui\,; et lui, prit à
tâche de plaire aux principaux valets, de se familiariser avec tous, de
prolonger ses voyages. Il fit à M. de Vendôme, qui aimait les mets
extraordinaires, des soupes au fromage et d'autres ragoûts étranges
qu'il trouva excellents. Il voulut qu'Albéroni en mangeât avec lui, et
de cette sorte, il se mit si bien avec lui, qu'espérant plus de fortune
dans une maison de Bohèmes et de fantaisies qu'à la cour de son maître,
où il se trouvait de trop bas aloi, il fit en sorte de se faire
débaucher d'avec lui, et de faire accroire à M. de Vendôme que
l'admiration et l'attachement qu'il avait conçu pour lui lui faisait
sacrifier tout ce qu'il pouvait espérer de fortune à Parme. Ainsi il
changea de maître\,; et bientôt après, sans cesser son métier de bouffon
et de faiseur de potages et de ragoûts bizarres, il mit le nez dans les
lettres de M. de Vendôme, réussit à son gré, devint son principal
secrétaire, et celui à qui il confiait tout ce qu'il avait de plus
particulier et de plus secret. Cela déplut fort aux autres. La jalousie
s'y mit au point que, s'étant querellés dans une marche,
\ldots{}\footnote{Nom en blanc dans le manuscrit.} le courut plus de
mille pas à coups de bâton à la vue de toute l'armée. M. de Vendôme le
trouva mauvais, mais ce fut tout\,; et Albéroni, qui n'était pas homme à
quitter prise pour si peu de chose et en si beau chemin, s'en fit un
mérite auprès de son maître, qui, le goûtant de plus en plus et lui
confiant tout, le mit de toutes ses parties et sur le pied d'un ami de
confiance plutôt que d'un domestique, à qui ses familiers, même les plus
haut huppés de son armée, firent la cour.

On a vu ce que put sur le roi la naissance de M. de Vendôme\,; le parti
qu'il en sut tirer par M. du Maine, et de là par M\textsuperscript{me}
de Maintenon, toujours en montant\,; comment par là il se dévoua
Chamillart\,; et l'intérêt que Vaudemont et ses habiles nièces
trouvèrent à se lier avec lui. Bien de tout temps avec Monseigneur par
la chasse et par d'autres endroits de jeunesse ancienne, jusqu'à être
dans l'intérieur de cette cour l'émule du prince de Conti\,; cette
émulation plut au roi qui haïssait le prince, et qui, dès avant tout ce
que nous venons de voir, avait pris du goût et de la distinction pour
Vendôme, qui l'avait flatté par son goût pour la chasse, pour la
campagne, par son assiduité près de lui, et par l'aversion de Paris
surtout, où il n'allait comme jamais. On a vu son art et son audace
d'entretenir le roi de projets, d'entreprises, de petits combats de rien
grossis, de vrais combats très douteux, donnés comme décisifs, avec une
hardiesse à l'épreuve du plus prompt démenti, en un mot, de courriers
continuels dont le roi voulait bien être la dupe et se persuader tout ce
que voulait Vendôme, appuyé et prôné si solidement dans le plus
intérieur des cabinets et contredit de personne, avec la précaution
qu'on a vu qu'il avait prises sur les lettres d'Italie, et le silence
profond, excepté pour l'exalter, que son poids et sa faveur mit imprimé
à son armée.

La situation où il la trouvait et l'absence du prince Eugène, qui était
à Vienne, lui parut une jointure favorable pour aller recueillir le
fruit de ses travaux. Il eut permission de faire un tour à la cour et
laisser son armée sous les ordres de Médavy, le plus ancien lieutenant
général, parce que la politique de Vaudemont, ou l'orgueil de ne
commander pas par l'absence d'un autre, lui en fit faire l'honnêteté à
Médavy.

Vendôme arriva droit à Marly, où nous étions, le 12 février. Ce fut une
rumeur épouvantable\,: les galopins, les porteurs de chaises, tous les
valets de la cour quittèrent tout pour environner sa chaise de poste. À
peine monté dans sa chambre tout y courut. Les princes du sang, si
piqués de sa préférence sur eux à servir et de bien d'autres choses, y
arrivèrent tout les premiers. On peut juger si les deux bâtards s'y
firent attendre. Les ministres y accoururent, et tellement tout le
courtisan, qu'il ne resta dans le salon que les dames\,: M. de
Beauvilliers était à Vaucresson\,; et pour moi, je demeurai spectateur
et n'allai point adorer l'idole.

Le roi, Monseigneur, l'envoyèrent chercher. Dès qu'il put être habillé
parmi cette foule, il alla au salon, porté par elle plutôt qu'environné.
Monseigneur fit cesser la musique où il était pour l'embrasser. Le roi,
qui était chez M\textsuperscript{me} de Maintenon, travaillant avec
Chamillart, l'envoya chercher encore, et sortit de la petite chambre où
il travaillait dans le grand cabinet au-devant de lui, l'embrassa à
diverses reprises, y resta quelque temps avec lui, puis lui dit qu'il le
verrait le lendemain à loisir, il l'entretint en effet chez
M\textsuperscript{me} de Maintenon plus de deux heures.

Chamillart, sous prétexte de travailler avec lui plus en repos à
l'Étang, lui donna deux jours durant une fête superbe. À son exemple,
Pontchartrain, Torcy, puis les seigneurs les plus distingués de la cour,
crurent faire la leur d'en user de même. Chacun voulut s'y signaler\,;
Vendôme retenu et couru de toutes parts n'y put suffire. On briguait à
lui donner des fêtes, on briguait d'y être invité avec lui. Jamais
triomphe n'égala le sien\,; chaque pas qu'il faisait lui en procurait un
nouveau. Ce n'est point trop dire que tout disparut devant lui, princes
du sang, ministres et les plus grands seigneurs, ou ne parut que pour le
faire éclater bien loin au-dessus d'eux, et que le roi ne sembla
demeurer roi que pour l'élever davantage.

Le peuple s'y joignit à Versailles et à Paris, où il voulut jouir d'un
enthousiasme si étrange, sous prétexte d'aller à l'Opéra. Il y fut couru
par les rues avec des acclamations\,; il fut affiché\,; tout fut retenu
à l'Opéra d'avance\,; on s'y étouffait partout, et les places y furent
doublées comme aux premières représentations.

Vendôme, qui recevait tous ces hommages avec une aisance extrême, était
pourtant intérieurement surpris d'une folie si universelle. Quelque
court qu'il eût résolu de rendre son séjour, il craignit que cette
fougue ne pût durer. Pour se rendre plus rare, il pria le roi de trouver
bon qu'il allât à Anet d'un Marly à l'autre, et ne fut que deux jours à
Versailles, qu'il coupa encore d'une nuit à Meudon, dont il voulut bien
gratifier Monseigneur. Vendôme ne fut pas plutôt à Anet avec fort peu de
gens choisis, que de l'un à l'autre la cour devint déserte, et le
château et le village d'Anet remplis jusqu'aux toits. Monseigneur y fut
chasser, les princes du sang, les ministres\,; ce fut une mode dont
chacun se piqua. Enflé d'une réception si prodigieuse et si soutenue, il
traita à Anet toute cette foule de courtisans, et la bassesse fut telle
qu'on le souffrit sans s'en plaindre comme une liberté de campagne, et
qu'on ne cessa d'y courir. Le roi, si offensé d'être délaissé pour
quelque occasion que ce fût, prenait plaisir à la solitude de Versailles
pour Anet, et demandait aux uns s'ils y avaient été, aux autres quand
ils iraient.

Tout montrait que de propos délibéré on avait résolu d'élever Vendôme au
rang des héros\,; il le sentit, il voulut en profiter. Il renouvela ses
prétentions de commander aux maréchaux de France\,; on l'érigeait en
dieu Mars, comment l'en refuser\,? La patente de maréchal général lui
fut donc sourdement accordée, et dressée pareille à celle de M. de
Turenne, depuis lequel on n'en avait point vu. Ce n'était ni le compte
de M. de Vendôme ni celui de M. du Maine. La patente n'avait été offerte
que pour sauver ce que le roi n'avait jamais voulu\,; elle n'avait été
acceptée qu'à faute de mieux et pour en faire un chausse-pied à la
naissance. Vendôme proposa donc que ce motif y fût inséré de plus qu'en
la patente de M. de Turenne. Je ne sais par où le maréchal de Villeroy
en eut le vent, mais il le sut à temps d'en faire ses représentations au
roi. Elles étaient pour lors encore conformes à son goût\,; le maréchal
était en grande faveur, il l'emporta et il fut déclaré à M. de Vendôme
qu'il ne serait rien ajouté à sa patente, conforme en tout à celle de M.
de Turenne. Il se piqua et n'en voulut plus. Le refus était
singulièrement hardi\,; mais il connaissait à qui il avait affaire, et
la force de ses appuis. Il avait été opiniâtrement refusé de commander
ceux d'entre les maréchaux de France qui ne l'étaient que depuis qu'il
commandait les armées\,; il n'avait pas tenu aux ordres réitérés du roi
que Tessé ne le lui eût fait éprouver, qui ne l'évita que par une
volontaire adresse\,; de là à la patente qu'on lui offrit pour les
commander tous, il y avait plus loin qu'à parvenir de cette offre à ce
qu'il prétendait. On verra dans cette année même qu'il ne se trompa pas.

Son frère, quoique médiocrement bien avec lui, le fut trouver à Anet
pour se remettre par lui en selle. Vendôme lui offrit de le présenter au
roi, et de lui faire donner une pension de dix mille écus\,; mais
l'insolent grand prieur ne voulut rien moins que de retourner commander
une armée en Italie, acheva pourtant le voyage d'Anet fort mécontent et
refusa tout, et quand son frère retourna à la cour s'en revint rager à
Clichy.

Il avait tous les vices de son frère. Sur la débauche il avait de plus
que lui d'être au poil et à la plume, et d'avoir l'avantage de ne s'être
jamais couché le soir depuis trente ans que porté dans son lit ivre
mort, coutume à laquelle il fut fidèle le reste de sa vie. Il n'avait
aucune partie de général\,; sa poltronnerie reconnue était soutenue
d'une audace qui révoltait\,; plus glorieux encore que son frère, il
allait à l'insolence, et pour cela même ne voyait que des subalternes
obscurs\,; menteur, escroc, fripon, voleur, comme on l'a vu sur les
affaires de son frère, malhonnête homme jusque dans la moelle des os
qu'il avait perdus de vérole, suprêmement avantageux et singulièrement
bas et flatteur aux gens dont il avait besoin, et prêt à tout faire et à
tout souffrir pour un écu, avec cela le plus désordonné et le plus grand
dissipateur du monde. Il avait beaucoup d'esprit et une figure parfaite
en sa jeunesse, avec un visage autrefois singulièrement beau. En tout,
la plus vile, la plus méprisable et en même temps la plus dangereuse
créature qu'il fût possible.

Le projet de Barcelone occupait fort alors. Tessé ne parut pas pouvoir
suffire à tout. Il fallait une armée en Galice, et contenir, si on
pouvait, les Portugais pour vaquer plus à son aise à la partie de la
Catalogne. Le triomphe de M\textsuperscript{me} des Ursins lui avait
fait passer le dépit qu'elle avait eu contre le duc de Berwick de tout
ce qu'il avait mandé d'Orry, qui en triomphait avec elle. Il fallait un
chef contre le Portugal, Berwick en connaissait exactement toute la
frontière\,; cela les détermina à Madrid à le redemander avec des
troupes de France pour ce côté-là. Le roi, en l'accordant, en prit
occasion de combler sa fortune en faveur d'une naissance qu'il aimait,
de quelque pays qu'elle fût. Quoique Berwick n'eût pas encore trente-six
ans, il lui envoya à Montpellier le bâton de maréchal de France avec
l'ordre de s'en aller de là droit en Espagne.

En même temps, le roi, touché de la douleur des beaux yeux de
M\textsuperscript{me} de Roquelaure, envoya son mari commander en
Languedoc à la place de Berwick, au scandale de toute la France. Tout en
même temps aussi le comte de Toulouse et le maréchal de Cœuvres s'en
allèrent à Toulon préparer tout ce qui était nécessaire pour aller
eux-mêmes favoriser par mer l'entreprise de Barcelone. Son importance
leur fit espérer que Pontchartrain n'en userait pas comme on a vu qu'il
avait fait l'année précédente. L'expérience leur apprit que la
persévérance dans la résolution qu'il avait prise lui avait paru plus
importante pour lui que de les laisser réussir à Barcelone.

Le duc de Noailles fit de petits exploits. Il pourchassa des miquelets,
s'empara de Figuères que l'ennemi avait abandonné, mit quelques troupes
dans Roses dès que le blocus en fut levé, et nettoya fort aisément le
Lampourdan. Il empêcha les ennemis de prendre Bascara, et leur prit et
tua quelque monde, s'avança vers le Ter, et se rendit maître depuis
Girone jusqu'à la mer. Ces faciles exécutions furent fort célébrées. Il
était pressé d'agir en chef, et il avait beau jeu contre quelque peu de
milices, avant que les troupes destinées au siège de Barcelone
arrivassent et Legal avec elles, auquel il devait obéir, et servir après
de maréchal de camp au siège.

Tessé n'était pas tellement occupé en Espagne qu'il ne songeât à ses
affaires. Il fit un tour de son pays et dupa bel et bien le roi et le
roi d'Espagne. Sans dire mot au dernier, il demanda au premier la
permission de céder sa grandesse à son fils, chose sans aucun exemple en
Espagne. Le roi, qui n'entretint jamais personne que pour ses affaires
et par nécessité, ignorait tout et ne s'en cachait pas. Sur la demande
de Tessé, et faite d'Espagne, il ne douta pas un moment que les
grandesses ne se cédassent comme ici les duchés, et le permit. Quand
Tessé eut ce qu'il voulait du roi par la surprise qu'il lui avait faite,
il surprit de même le roi d'Espagne, en lui faisant accroire que le roi
son grand-père s'était engagé de manière à ne pouvoir être dédit.
M\textsuperscript{me} des Ursins tout à lui, comme on a vu avec étendue,
le servit puissamment, et détermina le roi d'Espagne à ne pas chicaner
et blesser, pour une bagatelle qui n'aurait point d'effet en Espagne, le
roi son grand-père, dont il avait tant de besoin. Il se rendit avec bien
de la peine, mais par un décret qui la sentit et qui expliqua bien que
c'était sans nulle conséquence, et qui exclut l'Espagne de l'effet,
tellement que, si le comte de Tessé y eût été du vivant de son père, il
n'y eût pas été traité autrement que tous les fils aînés des grands.

En ce même temps, c'est-à-dire vers la mi-février, la reine douairière
d'Angleterre mourut en Portugal, où veuve sans enfants elle s'était
retirée auprès du roi son frère, qui l'aimait et la considérait fort.
Elle l'avait toujours aussi été beaucoup en Angleterre, où on s'affligea
fort de son départ. C'est celle avec qui le comte de Feversham, frère
des maréchaux de Duras et de Lorges, était si bien qu'on ne douta pas
qu'il ne l'eût épousée dans l'intervalle de la mort de Charles II et de
son départ. Sa religion l'avait établi en Angleterre, où il est mort
sans enfants, mais riche par le mariage qu'il avait fait. Il avait été
capitaine des gardes jusqu'à la révolution, grand chambellan de la reine
jusqu'à son départ, général d'armée, et eut, en 1685, la jarretière du
duc de Monmouth qu'il avait défait et pris, et qui fut décapité. On
donna part au roi de la mort de cette reine, et il en prit le deuil.

Belesbat mourut aussi. Son nom était Hurault. Sa mère était sœur de
Brégy et belle-sœur de M\textsuperscript{me} de Brégy, dont j'ai fait
une assez plaisante mention. La sœur de son père était cette
M\textsuperscript{me} de Choisy, mère de l'abbé de Choisy, si avant dans
le monde et si instruite de toutes les intrigues de la cour. Ces deux
femmes avaient mis Belesbat à la cour et dans le monde. C'était une
manière d'éléphant pour la figure, une espèce de bœuf pour l'esprit, qui
s'était accoutumé à se croire courtisan, à suivre le roi dans tous ses
voyages de guerre et de frontières, et à n'en être pas plus avancé pour
cela. Ses pères étaient de robe\,; il ne fut ni robe ni épée, se fit
assez moquer de lui, et ne laissait pas quelquefois de lâcher des
brutalités assez plaisantes. Il avait fort accommodé le jardin de
Belesbat, près de Fontainebleau, où les eaux et les bois sont
admirables, et s'y était fort incommodé. Il mourut vieux, sans avoir été
marié. Sa sœur était mère de Canillac, dont j'aurai maintes occasions de
parler.

Polastron, ancien lieutenant général, mourut aussi. Il avait un
gouvernement et la grand'croix de Saint-Louis. Son frère était au duc
Mazarin et avait été gouverneur de son fils, gendre du maréchal de
Duras. Cette famille est féconde en gouverneurs. Le fils de celui-là a
été sous-gouverneur de Mgr le Dauphin, puis lieutenant général.

Saint-Adon, d'une famille de Paris, galant, fort dans le grand monde et
dans le grand jeu, et capitaine aux gardes à force de lessives, avait
vendu sa compagnie, et n'osant plus se montrer, s'était retiré en
Flandre, où l'électeur de Bavière, qui ramassait tout, lui avait donné
une réforme de colonel de dragons. Il ne put s'empêcher de jouer\,; il
ne fut pas plus heureux qu'il l'avait été en ce pays-ci. Il se tua un
matin dans son lit. Tout le monde le plaignit\,: il était brave, de bon
commerce, et fait, quoique de peu, pour la bonne compagnie.

Deux hommes fort querelleurs, quoique assez peu propres à quereller,
eurent une violente prise au bal du Palais-Royal. M. le duc d'Orléans,
qui survint au bruit, leur imposa et les accommoda sur-le-champ. Ils ne
demandaient, pas mieux l'un et l'autre. C'était le chevalier de Bouillon
et d'Entragues, plus connu par son jeu et par être cousin germain de
M\textsuperscript{me} la princesse de Conti que par ailleurs, neveu de
cet abbé d'Entragues si extraordinaire, dont je crois avoir parlé. Tous
deux prétendaient épouser M\textsuperscript{me} de Barbezieux. Encore le
chevalier de Bouillon avait un rang et une belle figure\,; l'autre, de
l'intrigue et de l'audace. L'éclat de cette affaire fit entrer la
prétendue dans un couvent.

La duchesse douairière de Mortemart fit un mariage hardi dans sa
famille. Elle prit pour le comte de Maure, son second fils, qui prit le
nom de comte de Rochechouart, la fille unique de son frère Blainville,
tué à Hochstedt. Elle était extrêmement riche\,; mais sa mère était
enfermée depuis longtemps folle à lier, et cette folie venait de race et
s'était plus ou moins manifestée dans toutes les générations. Sa
grand'mère était sœur de Châteauneuf. Leur frère aîné avait couru les
champs et les rues toute sa vie à Angoulême. L'archevêque de Bourges,
leur autre frère, n'avait jamais été bien sage\,; elle l'était encore
moins. Elle avait épousé un Rochechouart, qui s'appelait M. de
Tonnay-Charente, et le mal venait de la mère, qui était Particelli,
fille d'Émery, surintendant des finances, qui était femme du bonhomme La
Vrillière, secrétaire d'État.

M. d'Uzès en fit un pareil. Il n'avait plus d'enfants de sa première
femme, fille de M. de Monaco. Il s'était ruiné dans l'obscurité de la
crapule\,; il épousa une fille de Bouillon. Qui aurait pu imaginer alors
que le frère de sa femme eût été chevalier de l'ordre avec lui en
1724\,?

Fort peu après, M. de La Trémoille maria son fils unique plus
honnêtement avec M\textsuperscript{lle} de La Fayette du nom de Mottier,
fort riche héritière. Elle avait perdu père et mère qui était fille, et
par l'événement, héritière de Marillac, doyen du conseil. Ce mariage
était fait avec le fils aîné du duc de Beauvilliers lorsqu'il le perdit.
La Fayette était mort maréchal de camp. Il était fils de cette
M\textsuperscript{me} de La Fayette, célèbre par son esprit, si amie de
M. le Prince le héros, de M\textsuperscript{me} de Longueville, de M. de
La Rochefoucauld, et de toutes les personnes d'esprit et principales de
son temps, et jusqu'à la fin de sa vie distinguée par son esprit. Lors
du désordre des tabourets donnés dans la régence de la reine
mère\footnote{Ce fut en octobre 1649 que la noblesse se réunit pour
  s'opposer aux honneurs récemment accordés à plusieurs familles. Voy.,
  pour les détails, notes à la fin du volume.}, puis ôtés, après rendus
de façon ou d'autre, M\textsuperscript{me} de La Trémoille, qui voyait
MM. de Bouillon et de Turenne, ses frères, devenus princes par les
troubles, essaya de faire prince aussi son mari. Ils avaient fait un
grand mariage en 1648 par ces mêmes troubles, et par leur religion, du
prince de Tarente leur fils, avec Amélie de Hesse, dont une sœur fut
électrice palatine, mère de Madame\,; l'autre, reine de Danemark, filles
de Guillaume V, landgrave de Hesse-Cassel, et d'une Hanau, cette
guerrière illustre qui servit si utilement et si constamment la France.
La considération d'une belle-fille si distinguée lui fit accorder le
tabouret, et encore à M\textsuperscript{lle} de La Trémoille, qui épousa
depuis un duc de Saxe-Weimar. On donna aussi le \emph{pour}\footnote{Voy.,
  t. II, p.~186.} à M. de La Trémoille. J'ai expliqué ailleurs ce que
c'est. De cette manière on contenta M\textsuperscript{me} de La
Trémoille et ses frères, qui ne voulaient point multiplier la princerie
qu'ils avaient obtenue, et on accorda à M. de La Trémoille une
distinction fort grande, qui donne le tabouret à la femme de son fils
aîné, et à sa fille aînée, sans aller au delà à aucun des cadets. On
verra dans la suite la subtile escroquerie du prince de Talmont, et où
elle en est demeurée.

Parlant des Bouillon, il faut dire ici qu'en ce même temps, le duc
d'Albret, voyant la cour et la ville contre lui, et le roi contre sa
coutume ayant pris parti, envoya son blanc signé à M. de Bouillon pour
terminer leur procès tout comme il lui plairait. M. de Bouillon avait
pris congé du roi pour aller à Dijon, où ce procès avait été renvoyé et
allait commencer\,; cela remit la paix dans la famille, et raccommoda
parfaitement le père avec le fils, mais non avec le roi, auprès duquel
le père fit inutilement tout ce qu'il put pour raccommoder ce qu'il
avait gâté dans sa colère. Le roi, qui savait gré au comte d'Évreux de
s'être attaché au comte de Toulouse, lui donna vingt mille livres de
pension pour tant que la guerre durerait. Ce sont de ces grâces qu'un
terme facilite, mais qui n'y demeurent guère bornées.

Rinschild, à la tête de douze mille Suédois, sans aucune artillerie,
défit entièrement, le 12 février, Schulembourg, qui avait vingt mille
Saxons ou Moscovites et beaucoup de canon. La cavalerie de ce dernier
lâcha pied d'abord, et abandonna vingt-deux pièces de canon, dont les
Suédois se servirent. Schulembourg se mit à la tète des quinze mille
hommes d'infanterie, qui fut enfoncée de façon qu'il n'en resta pas
mille. Schulembourg se sauva seul et blessé, tous les Moscovites tués,
six mille prisonniers, dont cent cinquante officiers, le canon, le
bagage, cent drapeaux ou étendards pris. Une si complète victoire ne
coûta pas plus de mille hommes aux Suédois, et presque point
d'officiers. Quel personnage eût fait en Europe ce jeune roi de Suède
s'il eût pu se préserver des perfides conseils de son ministre Piper, et
n'aller pas se détruire follement dans les déserts de Moscovie\,!

\hypertarget{chapitre-ix.}{%
\chapter{CHAPITRE IX.}\label{chapitre-ix.}}

1706

~

{\textsc{Généraux des armées.}} {\textsc{- Du Bourg attaqué à
Versailles.}} {\textsc{- Joyeux\,; son être\,; sa mort.}} {\textsc{- Du
Mont\,; sa famille\,; son caractère.}} {\textsc{- Catastrophe curieuse
de Maulevrier.}} {\textsc{- Départ de l'abbé de Polignac, etc.}}
{\textsc{- Prince Emmanuel d'Elbœuf passe aux Impériaux et est pendu en
effigie.}} {\textsc{- Langallerie, lieutenant général, puis Bonneval,
brigadier, passent aux ennemis et sont pendus en effigie.}} {\textsc{-
Vastes projets pour la campagne\,; réflexions.}} {\textsc{- Billet signé
du roi à M. de Vendôme, qui s'engage à faire recevoir l'ordre de lui et
obéir par un maréchal de France, en Italie seulement.}} {\textsc{-
Cardinal de Médicis veut se marier de la main du roi\,;
M\textsuperscript{lle} d'Armagnac le refuse.}} {\textsc{- Villars,
maître de la Mutter et de la Lauter, prend Haguenau et délivre le fort
Louis.}} {\textsc{- Le roi d'Espagne et Tessé devant Barcelone.}}
{\textsc{- Berwick faible contre les Portugais.}} {\textsc{- Chavagnac
ravage les Anglais aux îles de l'Amérique.}}

~

Le roi régla ses armées à peu près comme les années précédentes\,: M. de
Vendôme en Italie, Tessé pour la Catalogne, alors en Espagne, Berwick
pour la frontière de Portugal. Le maréchal de Villars en Alsace, Marsin
sur la Moselle, et le maréchal de Villeroy en Flandre, avec chacun leurs
officiers généraux.

Du Bourg, lieutenant général, destiné pour l'Alsace où il était
directeur de la cavalerie, et depuis maréchal de France, était alors à
Versailles. Il avait fait casser un capitaine de cavalerie du régiment
de Bourgogne. Cet officier l'attendit le 4 mars, au soir, à Versailles,
comme il se retirait chez lui, l'attaqua, le blessa légèrement de deux
coups. Saint-Sernin qui passait par là, se retirant aussi, les sépara.
Le capitaine y laissa son chapeau, sa perruque et son épée, et s'enfuit
tant qu'il put. Il s'appelait Boile. Il fut rattrapé près de
Fontainebleau. Du Bourg se jeta aux pieds du roi pour lui demander la
grâce de cet officier sans la pouvoir obtenir, avec raison. Il fut
condamné à un bannissement perpétuel que le roi commua en une prison de
dix ans.

Le vieux Joyeux, premier valet de chambre de Monseigneur et gouverneur
de Meudon, mourut bientôt après à Versailles dans une extrême
vieillesse, sans avoir jamais été marié, et donna tout son bien, qui
était considérable, aux enfants du feu bonhomme Bontems, son ancien ami
et camarade. Ce Joyeux était une espèce toute singulière et très
dangereuse, avec qui Monseigneur se mesurait fort, et avec qui sa cour
intérieure était en grand ménagement et fort en contrainte. Il avait été
à la reine mère, puis au roi, et dans toutes les intrigues serviles de
ses amours. Bel homme et fort bien fait, dansant mieux qu'homme de
France, et avait été de tous les ballets du roi avec les meilleurs
danseurs. Le dos lui était resté fort plat, mais il s'était comme rompu
par le bas\,; il faisait une pointe, et Joyeux marchait presque ployé en
deux. Son vêtement était rare et toujours le même\,: grande perruque et
grand rabat, habit brun fort ample, culottes très larges, d'ailleurs
bien chaussé. Il avait de l'esprit beaucoup, et de cet esprit de cour et
de remarque, de l'emportement, de la malignité, de l'entêtement,
quelquefois serviable et bon homme par fantaisie. Le roi l'avait mis
auprès de Monseigneur comme un homme de confiance. Il ne faisait pas bon
lui déplaire. Monseigneur n'avait osé lui refuser le gouvernement de
Choisy, quand il l'eut, puis de Meudon, où il ordonnait de tout comme
d'abord Bontems faisait à Marly. Il le traitait bien et le ménageait\,;
il s'en consola encore mieux. Joyeux avait une bonne abbaye et je crois
quelques prieurés.

Du Mont eut le gouvernement de Meudon. C'était un gentilhomme de bon
lieu. Mon père, étant premier gentilhomme de la chambre et premier
écuyer de Louis XIII, fit la petite fortune de son père, qui se trouva
un homme de mérite et qui l'acheva. Il fut sous-gouverneur du roi, et
mourut dans cet emploi fort estimé. La Bourlie, père de Guiscard, fut
mis en sa place. Le roi prit son fils tout enfant encore, et en chargea
le vieux Beringhen, premier écuyer, et dans la suite l'attacha à
Monseigneur, duquel il commandait toute l'écurie particulière, sous le
premier écuyer du roi. C'était un grand homme, bien fait et de bonne
mine, extrêmement court d'esprit, mais qui, né et élevé à la cour où il
avait passé sa vie, en savait la routine et le manège, fort homme
d'honneur et bienfaisant, mais avec des fantaisies et des manières comme
les gens de fort peu d'esprit et gâtés par la faveur. Il posséda
toujours toute celle de Monseigneur, sa plus intime confiance sur tous
les chapitres\,; gouvernait sa bourse particulière et ordonnait ses
plaisirs\,; fort honnête homme pourtant, et qui eut le sens de se
maintenir toujours fort bien avec le roi. Avec toute cette enflure, il
n'a jamais oublié ce que son père devait au mien\,; il le publiait, il
lui rendait toutes sortes de respects, et est toujours venu au-devant de
moi pour tout et en tout, avec respect et amitié, et se piquant et
s'honorant de l'une et de l'autre à mon égard, ce qui se trouvera
curieusement dans la suite. Il fut malheureux en famille. Le comte de
Brionne en usa avec un éclat qui l'obligea à confiner sa femme à la
campagne pour toujours. Sa fille unique lui donna plus de consolation.
Elle avait du mérite, et avait épousé un homme fort riche et qu'on ne
voyait jamais, presque toujours en Normandie. Il s'appelait M. de Flers,
du séditieux nom de Pellevé. Avec Monseigneur, du Mont perdit tout ce
qu'on peut perdre, et toutefois il conserva toujours de la considération
par estime, et fut toujours bien traité du roi. Il obtint dans la
régence la survivance de Meudon pour Pellevé, son petit-fils, qui avait
une compagnie de gendarmerie, et qui avait de la valeur et de l'estime
dans le monde. Il avait épousé la fille de La Chaise, capitaine de la
porte, neveu du P. de La Chaise. Du Mont n'eut pas la douleur de voir sa
catastrophe. Il devint fou par intervalles\,; on ne put lui laisser
Meudon où il se conduisait avec toutes sortes d'extravagances. Cela
acheva de lui tourner la tête\,; il finit enfin par s'aller noyer dans
la Seine, vers le moulin de Javelle.

Une folie me conduit à une autre, pour ne pas interrompre des matières
importantes et liées, en remettant de la rapporter au temps où elle
arriva. Maulevrier, de retour d'Espagne, et débarquant à Marly où
j'étais, et comme je l'ai dit, parce que sa femme était du voyage, y
trouva la princesse des Ursins au plus brillant de son triomphe, et
M\textsuperscript{me} de Maintenon également entêtée d'elle et
impatiente de la renvoyer à Madrid. Le compagnon saisit la conjoncture.
Il était chargé de mémoires de la reine d'Espagne et de Tessé. Il
profita des premiers temps de la reconnaissance de M\textsuperscript{me}
des Ursins qu'il avait si bien servie, il la cultiva, il eut soin de la
laisser apercevoir des privances qu'il surprit avec
M\textsuperscript{me} la duchesse de Bourgogne, et qu'il s'était
ménagées avant son voyage avec Mgr le duc de Bourgogne, qui lui avait
trouvé de l'esprit. Il ne négligea pas de les grossir aux yeux de son
importante amie\,; à qui il avait appris à Toulouse tant de choses
secrètes et importantes qu'elle n'eut pas peine à croire sur sa parole
plus encore qu'elle n'en voyait. Quelque nombre d'amis qu'elle laissât
en ce pays-ci, elle ne fut pas indifférente à se bien assurer de
celui-ci, qu'elle vit, et crut encore plus qu'il n'était, tenir par les
liens les plus intimes.

Elle avait plus d'une fois éprouvé la force de ceux-là, qui si souvent
gouvernent les cours, les affaires et les succès. Les secrets
réciproques qu'ils s'étaient confiés à Toulouse, ceux qu'il rapportait
d'Espagne les lièrent étroitement. Maulevrier s'en fit une clef de la
chambre de M\textsuperscript{me} de Maintenon, si curieuse de
l'intérieur de la cour d'Espagne, qu'elle allait, comptait-elle,
gouverner plus que jamais par M\textsuperscript{me} des Ursins, à qui
elle ne put refuser d'entretenir Maulevrier. Il fut donc admis chez elle
tête à tête. Ces conversations se multiplièrent et se prolongèrent
quelquefois plus de trois heures. Il eut soin de les nourrir par des
lettres et par des mémoires. M\textsuperscript{me} de Maintenon,
toujours éprise des nouvelles connaissances, avec un épanchement fort
singulier, admira tout de Maulevrier, et fit goûter au roi ce qu'il lui
envoyait.

Maulevrier, revenu perdu, et subitement relevé de la sorte, commença à
perdre terre, à mépriser les ministres, à faire peu de compte de ce que
son beau-père lui mandait. Les affaires qui lui passaient par les mains,
des commerces secrets qu'il entretenait en Espagne, lui donnèrent des
occasions continuelles de particuliers avec Mgr et M\textsuperscript{me}
la duchesse de Bourgogne, chacun séparément, à celle-ci de le ménager et
à lui de tout, prétendre. Nangis le désespérait, l'abbé de Polignac
aussi. Il ne prétendait à rien moins qu'à toutes sortes de sacrifices,
et il n'en pouvait obtenir aucun. Sa femme, piquée contre lui, se mit à
faire des avances à Nangis\,; celui-ci, pour se couvrir mieux, à y
répondre. Maulevrier s'en aperçut. C'était trop lui en vouloir. Il
connaissait sa femme assez méchante pour la craindre. Tant de vifs
mouvements du cœur et de l'esprit le transportèrent.

Un jour qu'il était chez lui, et qu'il y avait apparemment quelque chose
à raccommoder, la maréchale de Cœuvres le vint voir. Il lui ferma la
porte de sa chambre, la barricada au dedans\,; et à travers la porte la
querella jusqu'à lui chanter pouille une grosse heure entière qu'elle
eut la patience d'y demeurer, sans avoir pu parvenir à le voir. De cette
époque il se rendit rare à la cour et se tint fort à Paris. Il sortait
souvent seul à des heures bizarres, prenait un fiacre loin de chez lui,
se faisait mener derrière les Chartreux et en d'autres lieux écartés. Là
il mettait pied à terre, s'avançait seul, sifflait\,; tantôt un grison,
sortant d'un coin, lui remettait des paquets, tantôt ils lui étaient
jetés d'une fenêtre, une autre fois il ramassait une boîte, auprès d'une
borne, qui se trouvait remplie de dépêches. J'ai su dans le temps même
ces mystérieux manèges par des gens qu'il eut quelquefois l'indiscrète
vanité d'en rendre témoins. Il écrivait après à M\textsuperscript{me} de
Maintenon et à M\textsuperscript{me} la duchesse de Bourgogne, mais sur
les fins presque uniquement à la dernière par l'entremise de
M\textsuperscript{me} Cantin. Je sais gens, et M. de Lorges entre
autres, à qui Maulevrier a extérieurement montré des bottes de ses
lettres et des réponses, et lut entre autres une que
M\textsuperscript{me} Cantin lui écrivait, par laquelle elle tâchait de
l'apaiser sur M\textsuperscript{me} la duchesse de Bourgogne, et lui
mandait, de sa part, en termes les plus exprès et les plus forts, qu'il
devait toujours compter sur elle.

Il fit un dernier voyage à Versailles où il la vit en particulier et la
querella cruellement. Il dîna ce jour-là chez Torcy, avec qui il était
resté en mesures extérieures, et eut la folie de conter sa rage et sa
conversation à l'abbé de Caumartin qu'il y trouva, qui était ami intime
de Tessé et d'eux tous, et qui me la redit mot pour mot ensuite, et de
là s'en alla à Paris. Là, déchiré de mille sortes de rages d'amour qui
était venu à force de le faire, de jalousie, d'ambition, sa tête se
troubla au point qu'il fallut appeler des médecins, et ne le laisser
voir qu'aux personnes indispensables, et encore aux heures où il était
le moins mal. Cent visions lui passaient par la tête. Tantôt, comme
enragé, il ne parlait que d'Espagne, que de M\textsuperscript{me} la
duchesse de Bourgogne, que de Nangis qu'il voulait tuer, d'autres fois
le faire assassiner. Tantôt plein de remords sur l'amitié de Mgr le duc
de Bourgogne, à laquelle il manquait si essentiellement, il faisait des
réflexions si curieuses à entendre qu'on n'osait demeurer avec lui et
qu'on le laissait seul. D'autres fois doux, détaché du monde, plein des
idées qui lui étaient restées de sa première éducation ecclésiastique,
ce n'étaient que désirs de retraite et de pénitence. Alors il lui
fallait un confesseur pour le remettre sur ses désespoirs de la
miséricorde de Dieu. Souvent encore il se croyait bien malade et prêt à
mourir.

Le monde cependant, et jusqu'à ses plus proches, se persuadaient que
tout cela n'était qu'un jeu\,; et dans l'espérance d'y mettre fin, ils
lui déclarèrent qu'il passait pour fou dans le monde, et qu'il lui
importait infiniment de sortir d'un état si bizarre et de se montrer. Ce
fut le dernier coup qui l'accabla. Outré de fureur de sentir que cette
opinion ruinait sans ressource tous les desseins de son ambition, sa
passion dominante, il se livra au désespoir. Quoique veillé avec un
extrême soin par sa femme, par quelques amis très particuliers et par
ses domestiques, il fit si bien que le vendredi saint de cette année, il
se déroba un moment d'eux tous sur les huit heures du matin, entra dans
un passage derrière son appartement, ouvrit la fenêtre, se jeta dans la
cour et s'y écrasa la tête contre le pavé. Telle fut la catastrophe d'un
ambitieux à qui les plus folles et les plus dangereuses passions
parvenues au comble renversèrent la tête et lui ôtèrent la vie, tragique
victime de soi-même.

M\textsuperscript{me} la duchesse de Bourgogne apprit la nouvelle le
même jour, à ténèbres, avec le roi et toute la cour. En public, elle ne
témoigna pas s'en soucier\,; en particulier, elle donna quelque cours
aux larmes. Ces larmes pouvaient être de pitié, mais ne furent pas si
charitablement interprétées. On remarqua fort que, dès le samedi saint,
M\textsuperscript{me} Cantin alla à Paris chez ce malheureux, où dès
auparavant elle avait fait divers voyages. Elle était tout à Tessé, le
prétexte fut de lime de Maulevrier, mais personne n'y prit, et on crut
qu'il y avait eu des raisons importantes pour ce voyage.

La douleur de la veuve ne lui ôta aucune liberté d'esprit. On ne douta
pas qu'elle ne se fût saisie de tous les papiers avant de se jeter dans
le couvent où elle passa sa première année. Elle y reçut une lettre de
M\textsuperscript{me} la duchesse de Bourgogne, dont elle se para fort,
et la visite des dames les plus avant auprès de cette princesse. Elle
les reçut froidement, et M\textsuperscript{me} de La Vallière si mal,
que d'amies intimes qu'elles étaient elles s'en brouillèrent.

Incontinent après Pâques nous fûmes à Marly, M\textsuperscript{me} de
Maintenon y parut triste, embarrassée, sévère contre son ordinaire avec
M\textsuperscript{me} la duchesse de Bourgogne. Elle la tint souvent et
longtemps tête à tête, la princesse en sortait toujours en larmes. On ne
douta plus que M\textsuperscript{me} de Maintenon n'en eût appris enfin
ce que chacun voyait depuis longtemps. On soupçonna Maulevrier de s'être
vengé par des papiers qu'il lui avait envoyés sur les fins. On imagina
même que Desmarets, cousin germain de Maulevrier, et qui s'était
toujours mêlé de ses affaires domestiques, avait été saisi de papiers
importants, que, par le canal de Chamillart, il avait fait passer à
M\textsuperscript{me} de Maintenon et au roi même. J'étais ami
particulier de toute ma vie de Desmarets, après mon père, comme je l'ai
rapporté en son lieu, et à portée de tout avec lui. Je le pris un jour
de conseil de finances que nous avions dîné ensemble chez Chamillart, et
en nous promenant dans les jardins de Marly tête à tête je lui en
demandai la vérité. Il m'avoua que Maulevrier l'avait souvent entretenu
de ses visions et de ses amours, et lui en avait tant conté de toutes
les sortes que, désespérant de l'en pouvoir déprendre, et ne doutant pas
que la fin n'en fût fâcheuse, il lui avait depuis fermé la bouche toutes
les fois qu'il avait voulu lui en parler. Il me dit que c'était lui qui
avait ordonné du scellé, qu'il ne doutait pas qu'il n'y eût là bien des
lettres et bien des papiers fort curieux\,; qu'il savait que, peu avant
sa mort, Maulevrier en avait brûlé beaucoup et mis d'autres en dépôt
dont il n'avait pas voulu se charger\,; qu'il ne doutait pas que
M\textsuperscript{me} de Maulevrier n'eût mis la main sur tout ce qui
s'en était pu trouver\,; mais il me jura qu'il n'avait eu à cet égard ni
ordre ni rien de semblable, et qu'aussi il n'avait rien trouvé.

Je fus bien aise d'être éclairci d'un fait si important. Comme il n'y
avait donc plus rien qui le fût là-dessus à l'égard de Desmarets, je
contai cette conversation à la duchesse de Villeroy, à
M\textsuperscript{me} de Lévi, à M\textsuperscript{me} de Nogaret, à
M\textsuperscript{me} du Châtelet auprès desquelles nous étions logés,
M\textsuperscript{me} de Saint-Simon et moi, lesquelles nous disaient
aussi tout ce qu'elles découvraient. À l'empressement avec lequel
M\textsuperscript{me} de Nogaret m'avait pressé de confesser Desmarets,
et sa joie de ce que je lui en rapportai, j'eus beaucoup de soupçon
qu'elle ne l'avait pas fait d'elle-même, et de l'inquiétude de
M\textsuperscript{me} la duchesse de Bourgogne là-dessus. Cependant
cette tristesse profonde, et ces yeux si souvent rouges de
M\textsuperscript{me} la duchesse de Bourgogne, commencèrent à inquiéter
Mgr le duc de Bourgogne. Peu s'en fallut qu'il n'aperçût plus qu'il
n'était besoin. Mais l'amour est crédule\,; il prit aisément aux raisons
qui lui en furent données. Les romancines s'épuisèrent ou du moins se
ralentirent, la princesse comprit la nécessité de se montrer plus gaie.
Nous ne laissâmes pas de douter longtemps si le roi n'avait pas été
instruit. Je me licenciai de traiter avec le duc de Beauvilliers cette
matière en plein. Il n'en ignorait pas le fond\,; il souffrait
cruellement pour Mgr le duc de Bourgogne, et il tremblait sans cesse de
le voir tomber dans l'horrible désespoir d'apprendre ce qui à la fin se
sait presque toujours. M. de Beauvilliers n'avait jamais estimé
Maulevrier\,; il plaignit en bon chrétien sa fin funeste, mais il se
sentit fort soulagé. Tessé, par d'autres raisons, ne le fut pas moins
quand il apprit en Espagne qu'il était délivré d'un gendre si
embarrassant. Il ne s'en cacha même pas assez.

Achevons tout d'un temps cette délicate matière. L'abbé de Polignac
était pressé par Torcy de partir et ne s'y pouvait résoudre, quoique
cette aventure qui tenait les yeux si ouverts sur lui le dût persuader,
et une autre encore fort désagréable qu'il venait d'avoir avec l'abbé de
Caumartin, à propos du procès de M. de Bouillon avec son fils. À la fin
pourtant il fallut prendre congé. On remarqua beaucoup que
M\textsuperscript{me} la duchesse de Bourgogne lui souhaita un heureux
voyage tout d'une autre façon qu'elle n'avait accoutumé de congédier
ceux qui prenaient congé d'elle. Peu de gens eurent foi à une migraine
qui la tint tout ce même jour sur un lit de repos chez
M\textsuperscript{me} de Maintenon, les fenêtres entièrement fermées, et
qui ne finit que par beaucoup de larmes. Ce fut la première fois qu'elle
ne fut pas épargnée. Madame, se promenant peu de jours après dans les
jardins de Versailles, trouva, sur une balustrade et sur quelques
piédestaux, deux vers aussi insolents qu'ils furent intelligibles, et
Madame n'eut ni la bonté ni la discrétion de s'en taire. Tout le monde
aimait M\textsuperscript{me} la duchesse de Bourgogne\,; ces vers firent
moins de bruit, parce que chacun l'étouffa.

Le prince Emmanuel, frère du duc d'Elbœuf, après avoir fait bien des
personnages différents et la plupart fort honteux, et tiré souvent du
roi de l'argent et de la protection, était allé à Milan trouver sa sœur
et Vaudemont son beau-frère. Il fit là son marché et passa à l'armée de
l'empereur, où il eut un régiment. Le roi, qui en fut piqué, lui fit
faire son procès comme on l'avait fait au prince d'Auvergne, et comme
lui, par arrêt du parlement, il fut pendu à la Grève en effigie.

Langallerie passa aussi au service de l'empereur. Son père fut tué à
Fleurus, lieutenant général fort estimé. Le fils était brave et réglé,
il était appliqué et bon officier, il était parvenu assez vite à être
lieutenant général, il avait toujours paru sage et modeste. Il servait
en Italie. Je ne sais ce qui lui tourna la tête\,; l'ambition le saisit.
Il se piqua de quelque pillage qui lui fait reproché de la cour, tandis
qu'il en voyait faire sans cesse de bien plus considérables à d'autres à
qui on ne disait mot, parce qu'ils étaient plus appuyés. Il avait épousé
une vieille femme avec qui il ne vivait point, dont il n'avait point
d'enfants, et qui avait été gouvernante des filles d'honneur de Madame
tant qu'elle en avait eu. C'était pour le plus un très simple
gentilhomme et fort court d'esprit. Il s'en alla à Venise pendant
l'inaction de l'hiver\,; il y fit son traité et en partit pour Vienne,
avec le même grade militaire chez l'empereur qu'il avait ici.

Ces deux passèrent aux ennemis en mars. Quinze jours après Langallerie,
le chevalier de Bonneval, qui était aussi allé à Venise, en fit autant.
C'était un cadet de fort bonne maison, avec beaucoup de talents pour la
guerre, et beaucoup d'esprit fort orné de lecture, bien disant, éloquent
avec du tour et de la grâce, fort gueux, fort dépensier, extrêmement
débauché, grand escroc et qui se peut dire sans honneur ni conscience,
fort pillard. Il avait rudement vexé ces petits princes d'Italie que
nous ménagions assez mal à propos, comme il y a bien paru depuis. Il
avait pris aussi assez d'argent des contributions\,; les plaintes des
princes et des trésoriers lui attirèrent des lettres de Chamillart, qui
lui voulut faire rendre gorge. Il avait un régiment d'infanterie. Il y
eut ordre de lui retenir tout ce qu'il pouvait toucher, en attendant
qu'on pût lui faire payer le reste. La misère et le dépit lui firent
faire son traité\,; et, comme Langallerie, il partit de Venise pour
Vienne, où le prince Eugène en fit son favori, et le fit avancer fort
vite aux premiers grades, dont nous verrons qu'il eut tout lieu de se
repentir. Fort peu après les avoir présentés à l'empereur et à sa
cour\,; le prince Eugène partit de Vienne pour venir commander en
Italie. Il les y mena tous deux avec lui, et ils y servirent sous ses
ordres. Le roi leur fit aussi faire leur procès comme il venait de le
faire faire au prince d'Elbœuf, et tous deux, comme lui, représentèrent
à la Grève en effigie. On verra en son temps leur diverse, mais
incroyable catastrophe.

Les projets pour la campagne qui allait commencer étaient dignes des
années de la prospérité du roi et de ces temps heureux d'abondance
d'hommes et d'argent, de ces ministres et de ces généraux qui par leur
capacité donnaient la loi à l'Europe. Le roi voulut débuter par deux
batailles, l'une en Italie, l'autre en Flandre\,; devancer l'assemblée
de l'armée impériale sur le Rhin et renverser les lignes des ennemis\,;
enfin, faire le siège de Barcelone et celui de Turin. L'épuisement de
l'Espagne, celui où la France tombait, répondait peu à de si vastes
idées. Chamillart, accablé sous le double ministère de Colbert et de
Louvois, ressemblait peu à ces deux grands ministres, les généraux des
armées aussi peu à M. le Prince, à M. de Turenne, et aux élèves de ces
héros qui n'étaient plus. C'étaient des généraux de goût, de fantaisie,
de faveur, de cabinet, à qui le roi croyait donner, comme à ses
ministres, la capacité avec la patente. Louvois, outré d'avoir eu à
compter avec ces premiers généraux, se garda bien d'en former d'autres.
Il n'en voulut que de souples et dont l'incapacité eût un continuel
besoin de sa protection. Pour y parvenir, il éloigna le mérite et les
talents, au lieu qu'on les recherchait avant le comble de sa puissance.
On tâchait de les démêler de bonne heure dans les sujets\,; on les
éprouvait par des commandements à part pour sonder leurs forces\,; et,
s'ils répondaient à ce qu'on en espérait, on les poussait. On leur
faisait faire des projets pour les former\,; quand ils étaient bons, on
les chargeait de leur exécution. On s'appliquait à démêler la nature de
leurs fautes. Il y en avait qui ne se pardonnaient point, parce qu'elles
venaient de manque de fond\,; pour les autres qui partaient de trop
d'ardeur ou de surprise, on se souvenait du grand mot de M. de
Turenne\,: qu'il fallait avoir été battu pour devenir bon, et avoir fait
des fautes pour se mieux instruire. Mais c'était des corps séparés ou
des détachements, non des armées, qu'on hasardait sous ceux qu'on
essayait de la sorte, qu'on grossissait après, et qui devenaient enfin
des armées, suivant qu'on les voyait réussir. Par là une émulation,
conséquemment une application générale, une formation continuelle de
généraux et d'officiers généraux encore, qui, n'ayant pas assez de fond
pour conduire une armée, en avaient assez pour y briller utilement en
second et en troisième, et en sous-ordre quantité d'officiers
particuliers sur qui roulaient souvent de moindres choses, mais avec
lumière et succès. On les récompensait à mesure par quelque grâce ou par
un avancement. Personne n'y trouvait à redire\,; et, dans l'espérance
d'une occasion à se distinguer aussi, chacun se faisait justice, et
chacun ne cherchait et ne songeait qu'à s'appliquer, à apprendre et à
bien faire. C'est ainsi qu'on formait toujours des sujets, et qu'un
commandant de bataillon d'alors en savait plus que nos lieutenants
généraux modernes. C'est ce que j'ai ouï souvent raconter et discuter à
M. le maréchal de Lorges, déplorer la conduite substituée à celle-là, et
prédire les malheurs qui en sont arrivés.

M. de Louvois, pour être pleinement le maître, mit dans la tête du roi
l'ordre du tableau et les promotions, ce qui égala tout le monde, rendit
l'application et le travail inutiles à tout avancement, qui ne fut dû
qu'à l'ancienneté et aux années, avec toujours de rares exceptions pour
ceux que M. de Louvois eut des raisons particulières de pousser. Il
persuada encore au roi que c'était à lui-même à diriger ses armées de
son cabinet. Cette flatterie ne servit qu'à le tromper pour les diriger,
lui Louvois, à son gré, sous le nom du roi au détriment des affaires,
dont les généraux en brassières n'eurent plus la disposition, ni la
liberté de profiter d'aucune conjoncture qui se trouvait échappée avant
le retour du courrier dépêché pour en rendre compte et recevoir les
ordres\,; tellement que le général, toujours arrêté, toujours en
brassières, toujours dans la crainte, dans l'incertitude, dans l'attente
des ordres de la cour à chaque pas, ne trouvait encore nul soulagement
dans ses officiers généraux, parvenus là par leur ancienneté sans avoir
jamais été proprement que des subalternes, ni que rien eût roulé sur
eux, et qui aussi, certains de ne monter qu'en leur rang d'ancienneté,
ne s'étaient, pour le très grand nombre, jamais donné la peine de
chercher à rien apprendre. Aussi l'ignorance était telle dans presque
tous, que le maréchal de camp venu de l'infanterie n'avait pas la
première notion de l'assiette ni de la disposition d'un fourrage\,; que
celui venu de la cavalerie ne savait ce que c'était qu'une tranchée ni
rien qui eût rapport à une attaque de place, ni à une défense\,; que
presque aucun ne savait faire un camp, ni placer les gardes, ni conduire
un convoi, ni mener un détachement\,; et les lieutenants généraux n'en
savaient guère davantage, sinon quelque routine forcément apprise
pendant qu'ils étaient maréchaux de camp.

Le luxe qui avait inondé les armées, où on voulait vivre aussi
délicatement qu'à Paris, empêchait les officiers généraux de vivre avec
les officiers, de les connaître, d'en être connus\,; par conséquent, de
savoir choisir et discerner pour des commandements qui demandent de la
confiance en la capacité des gens. Nuls propos de guerre comme autrefois
où on s'instruisait par les récits et les dissertations réciproques, où
il eût été honteux de parler et de se remplir d'autre chose, où les
jeunes écoutaient les anciens, et où ceux-ci s'entretenaient de ce
qu'ils avaient vu bien et mal faire, avec des raisons et des réflexions.
Ceux d'aujourd'hui de tout âge ne pouvant parler de ce qu'ils ignorent,
ne parlent que jeu, que femmes, les vieux que fourrages et qu'équipages,
les officiers généraux épargnent ou vivent ensemble, le général ne voit
que foule, en particulier ne fait qu'écrire, ce qui consume tout son
temps en courriers, la plupart très chers et encore plus inutiles\,; le
soir il est abandonné à trois ou quatre hommes du détail, qui souvent ne
savent pas le faire.

Le 11 mars M. de Vendôme eut à Versailles une fort longue audience du
roi dans son cabinet, où il prit congé pour aller passer deux jours dans
la maison de Crosat à Clichy, et partir de là pour l'Italie. Il avait
{[}su{]} se retourner par degrés. Porté par l'intérêt de M. du Maine et
par tout le crédit de M\textsuperscript{me} de Maintenon, il avait
représenté au roi l'extrême dégoût qu'il avait eu en Italie de la
présence de Tessé\,; que, puisqu'il avait bien voulu lui donner la
patente de maréchal général, telle que l'avait eue M. de Turenne pour
commander tous les maréchaux de France, il lui demandait au moins la
grâce de commander en Italie ceux qu'il y pourrait envoyer. Le roi,
combattu dans son plus intérieur, épris comme il l'était de M. de
Vendôme, voulant qu'il donnât bataille en arrivant, comptant sur lui
pour protéger le siège de Turin qui était résolu, ne voulut pas le
renvoyer mécontent. Il se tint quitte à bon marché de la restriction que
lui-même proposait à la grâce qu'il demandait, et mis au large sur ce
qu'il ne parlait plus du motif de sa naissance. Chamillart eut donc
ordre d'écrire de sa main un simple billet à Vendôme que le roi signa de
la sienne, par lequel le roi lui promettait qu'en cas que le bien de ses
affaires l'obligeât d'envoyer un maréchal de France en Italie, il
ordonnerait à ce maréchal de France de lui obéir et de prendre l'ordre
de lui, en Italie seulement, en considération des grands services qu'il
lui avait rendus en ce pays-là. Vendôme en fut content, l'emporta avec
lui, s'en vanta fort au point précis de son départ, bien résolu à s'en
faire un échelon à monter à sa prétention de commander à tous les
maréchaux de France à la fin, sans patente, et par naissance. Cette
première écorne les mortifia fort, et le maréchal, de Villeroy sur tous
qui avait paré le grand coup, dont celui-ci lui fit avec raison prévoir
et craindre le retour. Le roi ne recommanda rien davantage à Vendôme que
de chercher les ennemis partout en arrivant et les combattre. M. de
Vendôme le lui promit, et on va voir qu'il tint parole.

Il s'alla embarquer à Antibes avec son frère sur deux galères du roi qui
le portèrent à Gènes\,; d'où le grand prieur s'en alla à Rome, dans le
dessein de se retirer, malgré l'épreuve qu'il en avait déjà faite une
fois qu'il n'avait pu supporter, et M. de Vendôme joindre son armée.

Il y trouva tout en bon état, et ne laissa pas de faire courir le bruit
qu'elle était si affaiblie et si en désordre, qu'il ne pouvait rien
entreprendre. L'absence du prince Eugène ne le pressait pas moins que
les ordres du roi. Revenclaw, en l'attendant, commandait son armée.
Vendôme assembla diligemment cinquante-huit bataillons et six mille
chevaux à son quartier général, qui était Castiglione delle Stivere, et,
le 19 avril, marcha de grand matin à Montechiaro, où les ennemis
s'étaient fortifiés tout l'hiver, qu'ils abandonnèrent pourtant à son
approche. Ils se retirèrent à Calcinato, où tous leurs quartiers
s'étaient rassemblés. Vendôme, qui les suivit de fort près, les trouva
en bataille sur la hauteur de Calcinato, les attaqua vivement et
brusquement, et comme la partie n'était pas égale, car les ennemis
n'étaient pas là plus de dix ou de onze mille hommes, il les battit et
les défit en fort peu de temps, leur tua trois mille hommes, prit vingt
drapeaux, dix pièces de canon, huit mille prisonniers, et parmi eux un
colonel.

Le chevalier de Maulevrier apporta cette nouvelle avec un billet de huit
lignes au roi, de sur le champ de bataille à midi. Deux jours après
arriva Conches, aide de camp de M. de Vendôme, avec une longue dépêche
du 20. L'après-midi du 19, Vendôme poursuivit sa victoire. De deux mille
cinq cents hommes qui se retiraient, onze cents furent tués et le reste
pris\,; et avec ce reste, le comte de Falkenstein, officier général,
trois colonels et plusieurs officiers moindres. Le nombre des
prisonniers était, selon le rapport de Conches, de plus de deux mille
cinq cents, outre cinq cents déserteurs. Il apporta vingt-quatre
drapeaux et douze étendards. Nos troupes s'accommodèrent de douze cents
habits neufs trouvés dans Calcinato\,; il ne s'y rencontra rien autre
chose. Les ennemis jetèrent six mille fusils que Vendôme fit rechercher
en donnant un écu de la pièce. Le chevalier du Héron y fut tué, et ce
fut une perte\,; il était brigadier de dragons. Vendôme perdit peu de
monde\,; ce fut une déroute plutôt qu'un combat. Il marcha le 22 pour
achever sa victoire, mais les ennemis se retirèrent le soir qu'il arriva
sur eux, lui dérobèrent leur marche, et y surent si bien pourvoir que
leur dernière arrière-garde ne put être entamée. Le prince Eugène était
arrivé le lendemain du combat. Il rétablit si promptement les affaires
que nous ne pûmes tirer aucun fruit de ce succès. On ne laissa pas
d'abord d'en espérer tout, et d'élever M. de Vendôme aux nues. Ce qui
avait retardé le prince Eugène, c'est qu'il n'avait jamais voulu partir
qu'il n'eût vu ses recrues, ses renforts, et l'argent qu'il avait
demandé fort avancé vers l'Italie. Ces secours le joignirent peu après
son arrivée, il s'en sut trop bien servir\,; et M. de Vendôme, loin
d'attaquer, ne fut occupé qu'à parer le reste du temps qu'il demeura en
Italie.

Avant que de sortir d'Italie, il faut dire un mot de la démarche que le
cardinal de Médicis fit auprès du roi. On a vu lors du séjour du roi
d'Espagne à Naples combien ce cardinal avait le cœur français. Il
n'avait aucun ordre, il avait été cardinal fort jeune, il était
protecteur des affaires de France et d'Espagne, il voyait le grand-duc
son frère avançant en âge, brouillé avec la grande-duchesse, qui, depuis
grand nombre d'années, s'était retirée en France pour toujours. De ce
mariage, il n'y avait eu que deux fils\,: l'aîné, Ferdinand, était mort
sans avoir laissé d'enfants de la sœur de feu M\textsuperscript{me} la
Dauphine\,; Gaston, le cadet, était brouillé depuis longues années avec
sa femme dont il n'avait point d'enfants. C'était une sœur de la
princesse de Bade, mère de la feue duchesse d'Orléans, les deux seuls
restes de la maison de Saxe-Lauenbourg. La princesse de Toscane vivait
chez elle en Allemagne, et il n'était plus question de retour avec son
mari. Il n'y avait aucune autre postérité des grands-ducs. La branche de
Médicis-Ottaïano établie dans le royaume était aînée de celle des
grands-ducs, laquelle en était séparée longtemps avant d'avoir usurpé la
souveraineté. Éloignement, aversion même de tout temps entre ces deux
branches. E n'en subsistait plus d'autre des Médicis.

Le cardinal, quoique vieux, songea à rendre son chapeau, à continuer sa
maison, s'il pouvait, et à se marier. Il le voulut être de la main du
roi et à une Française. Il lui en écrivit. Le roi, comme on l'a souvent
vu, aimait M. le Grand. Il n'avait pas sur la Toscane les mêmes raisons,
à l'égard de la maison de Lorraine, qu'il avait eues pour Mantoue, à
cause du Montferrat. Il se souvenait toujours qu'il avait empêché le
comte de Toulouse d'épouser M\textsuperscript{lle} d'Armagnac, chassé
Longepierre, qu'il avait mis auprès de lui, pour avoir brassé cette
affaire, et fait longuement sentir son indignation à
M\textsuperscript{lle} d'Armagnac pour l'avoir poussée aussi loin
qu'elle avait pu. Il songea donc à dédommager M. le Grand par un mariage
qui pouvait faire sa fille grande-duchesse de Toscane. Il en parla à M.
le Grand qui en fut comblé, mais le supplia de trouver bon qu'il
consultât sa fille. M\textsuperscript{lle} d'Armagnac vivait à la cour
depuis son enfance, adorée de sa mère qui était la maîtresse de la
famille et de son mari. Elle était dans la maison de la plus grande et
de la plus brillante représentation de la cour\,; elle aimait le jeu
passionnément, on y jouait jour et nuit le plus gros jeu du monde. Elle
était encore belle comme le jour\,; elle était en maison libre et du
plus grand abord, où on ne le lui avait pas laissé ignorer. Elle ne put
consentir à changer une vie si agréable et si aisée contre un pays
étranger, austère, jaloux, avare, avec un mari vieux, qui lui laisserait
peu de liberté dans un pays où elle n'était guère en, usage et où elle
ne verrait personne que par audiences. Sa mère, qui ne s'en pouvait
passer, n'eut garde de la vouloir contraindre, et, dès qu'elle ne le
voulut pas, le père fut du même avis. Il en fit sa cour, il dit au roi
que sa fille préférait l'honneur d'être sa sujette, et de vivre dans sa
cour, aux plus grandes fortunes étrangères. Le roi lui en sut le
meilleur gré du monde. Il ne trouva point d'autres partis français à
proposer au cardinal de Médicis, qui, à la fin, épousa une Guastalla,
c'est-à-dire une Gonzague de branche cadette des ducs de Mantoue, qu'il
rendit fort heureuse, mais dont il ne laissa point d'enfants.

Marsin avait fait un projet pour forcer les lignes des ennemis avant que
les Impériaux eussent assemblé leur armée sur le Rhin. Il fut
approuvé\,; il partit secrètement de Marly le 18 avril, sans avoir pris
congé de personne. En même temps, tous les officiers généraux et
particuliers destinés sur le Rhin eurent ordre de partir et de n'en rien
dire, et le 21 avril, Villars partit aussi secrètement de Marly. Ces
deux maréchaux s'abouchèrent à Phalsbourg et marchèrent chacun de leur
côté. À leur approche, les ennemis abandonnèrent leurs lignes de la
Mutter qu'on voulait attaquer, et on ne vit de leurs troupes que sept ou
huit cents chevaux que le fils du comte du Bourg poussa vigoureusement
et qui prirent la fuite. Ils y perdirent une centaine d'hommes, et du
Bourg fils deux ou trois seulement. Leur gros repassa le Rhin après
avoir jeté quelque monde dans Haguenau. Cette expédition si heureuse et
si facile délivra le fort Louis, dont la garnison fut relevée, et la
place renouvelée de tout en munitions de guerre et de bouche, et les
postes d'alentour qui la bloquaient pris.

Le comte de Frise, gouverneur de Landau, se retira très précipitamment
de Bischweiller, où il laissa de grands magasins et même sa vaisselle
d'argent, abandonna Lauterbourg où Villars mit des troupes, et fut
maître par là de la Lauter comme il venait de l'être de la Butter. Peri
prit Haguenau et deux mille hommes qui étaient dedans prisonniers de
guerre, soixante pièces de canon, cinq cents milliers de poudre, et
grande quantité de farine et d'avoine. Tout ce dépôt était destiné à
faire le siège de Phalsbourg. Villars s'étendit tout à son aise, et
n'oublia pas les contributions jusque dans la plaine de Mayence.

Le roi d'Espagne était parti à la fin de février dans le dessein de
réduire le royaume de Valence\,; mais sur les ordres du roi, pour ne
différer pas le siège de Barcelone, il changea sa marche et arriva le 3
devant Barcelone, où il trouva Legal arrivé de la veille avec toutes les
troupes françaises, et tous nos bâtiments qui débarquaient tout ce qu'il
fallait pour le siège\,; d'autres bâtiments portèrent toute la garnison
de Girone dans Barcelone avec toutes sortes de rafraîchissements, où
plus de dix mille hommes animés de la présence de l'archiduc prirent les
armes et se joignirent à la garnison. La tranchée fut ouverte la nuit du
5 au 6, par le marquis d'Ayetone, mais le canon ne tira que le 12,
encore fort faiblement. Le duc de Noailles, qui devait y servir de
maréchal de camp, tomba malade de la petite vérole qui fut très
heureuse, et qui acheva de le guérir de tous ses maux. Laparat,
ingénieur principal, et le chef des autres depuis l'élévation de Vauban
au bâton, était chargé de ce siège, et y fut tué le 15 avril en allant
reconnaître des ouvrages qu'il voulait faire attaquer.

On prétendit qu'on fit une grande faute d'avoir attaqué par le mont
Joui\,; que cette fortification séparée de celle de la ville serait
tombée avec la ville, au lieu que sa prise n'influait point sur celle de
la place. Quoi qu'il en soit, ce mont Joui dura le double de ce qu'on
avait cru, consuma beaucoup de nos munitions et coûta bien d'honnêtes
gens, et Laparat même, qui y fut tué et qui fut mal remplacé. Les
troupes qui faisaient le siège étaient peu nombreuses\,; leur fatigue
était continuelle\,; il n'y avait de repos que de trois nuits l'une, et
fort souvent beaucoup moins. Les petits combats y étaient continuels
avec les miquelets qui troublaient les convois, et qui assiégeaient
tellement les assiégeants qu'il n'y avait pas de sûreté à cent pas du
camp, qui était exposé à des alarmes continuelles. Nuls
rafraîchissements de France ni d'Espagne, tout à l'étroit pour tout. Les
sorties étaient très fortes. Les habitants y secondaient la garnison,
les moines étaient armés, et combattaient comme contre des Turcs et des
hérétiques. Pendant ces sorties, le camp était attaqué par dehors, et
c'était tout ce que les assiégeants pouvaient faire que de soutenir ces
doubles attaques à la fois, par la vigueur des assiégés et le nombre et
l'importunité des miquelets.

Tessé envoya son fils porter la nouvelle que les ennemis avaient le 25
avril abandonné le mont Joui, lequel en fut fait maréchal de camp. La
garnison sortit ensemble en plein jour, et entra dans Barcelone sans
presque aucune perte. Cifuentès, qui avait quantité de barques à la
côte, en faisait toujours entrer quelques-unes dans la place aux dépens
de quelques autres qu'il perdait, et les avenues de l'armée du roi
d'Espagne furent bientôt si resserrées par les miquelets qu'on rie vécut
plus au siège que par la mer. Le comte de Toulouse et le maréchal de
Cœuvres sous lui y commandaient une médiocre flotte arrivée assez tard,
et mettaient rarement pied à terre sans découcher de dessus leurs bords
et Tessé avait sous le roi d'Espagne le commandement de tout ce qui
regardait la terre.

Berwick était arrivé tout au commencement d'avril en Estrémadure, où il
avait vingt-six bataillons et quarante escadrons. Les Portugais, et ce
que l'archiduc leur avait laissé, étaient bien plus nombreux, et firent
contenance d'assiéger Badajoz avec quarante-cinq bataillons et
cinquante-trois escadrons, où le marquis de Richebourg commandait avec
douze bataillons. Ils tirèrent du côté d'Alcantara, et se présentèrent
en chemin au duc de Berwick, qui, avec quarante escadrons qu'il avait,
n'osa leur prêter le collet. Ils continuèrent leur chemin et prirent
Alcantara, après une courte et molle, défense (très mauvaise place à la
vérité), et dix bataillons espagnols qui étaient dedans prisonniers de
guerre.

Chavagnac, avec quatre vaisseaux du roi, ravagea cependant toute l'île
de Saint-Christophe en Amérique, dont les Anglais étaient les maîtres, y
ruina tout, en emmena huit cents nègres, puis avec Iberville, qui le
joignit au rendez-vous qu'il lui avait donné, prit aux Anglais toute la
petite île de Nièves, en détruisit les forts, les habitations, les
sucreries, firent le dégât partout, emmenèrent les principaux habitants
pour otages, prirent trente vaisseaux marchands, dont quelques-uns
percés pour trente-six pièces de canon, emmenèrent sept mille nègres et
firent un grand butin. Le gouverneur et le major de l'île furent tués.
Il n'en conta à nos deux capitaines que quelques soldats et un enseigne
de vaisseau. Ils n'avaient pour cette expédition que douze cents soldats
et treize cents flibustiers. Le chevalier de Nangis apporta cette
nouvelle.

\hypertarget{chapitre-x.}{%
\chapter{CHAPITRE X.}\label{chapitre-x.}}

1706

~

{\textsc{Électeurs de Cologne et de Bavière au ban de l'empire.}}
{\textsc{- Siège de Turin résolu, et La Feuillade, singulièrement
confirmé à le faire, arrive devant la place.}} {\textsc{- Villeroy part
avec ordre de combattre, non avant, mais dès que Marsin l'aura joint.}}
{\textsc{- Pique de Villeroy, qui n'attend point Marsin et choisit mal
son terrain.}} {\textsc{- Dispositions de Villeroy.}} {\textsc{-
Bataille de Ramillies.}} {\textsc{- Course de Chamillart en Flandre.}}
{\textsc{- Bonté du roi pour Villeroy excessive.}} {\textsc{- Folie plus
excessive du Villeroy.}} {\textsc{- Villeroy rappelé\,; Vendôme choisi à
sa place.}} {\textsc{- M. le duc d'Orléans en Italie.}} {\textsc{-
Disgrâce du maréchal de Villeroy.}}

~

L'empereur mit enfin au commencement de mai les électeurs de Cologne et
de Bavière au ban de l'empire avec autant de solennité que de violence
et d'injustice, pour une guerre qui ne regardait uniquement que la
maison d'Autriche, et point du tout l'empire. Mais l'Allemagne était
subjuguée depuis Charles-Quint, et quoique ses successeurs à l'empire
n'eussent pas la moitié des États et de la puissance qu'il possédait,
ils surent bien soutenir l'autorité qu'il leur avait acquise. La
proscription du palatin en fut un exemple éclatant. Cet empereur-ci,
soutenu de toute l'Europe et maître de la Bavière, n'eut garde de faire
moins. Parmi ces hauteurs, il venait de voir sa maison de plaisance de
Luxembourg, à deux lieues de Vienne, brûlée par les mécontents, et des
Alleurs que le roi tenait auprès de Ragotzi l'assurait de leurs forces
et de leur éloignement pour tout accommodement avec l'empereur.
Quoiqu'on eût lieu de s'attendre depuis longtemps à ce ban de l'empire,
il ne laissa pas d'étonner et de porter un grand coup pour l'autorité de
l'empereur, et pour l'embarras de sortir ces princes d'affaires à la
paix.

Tout ce qui s'était fait l'année précédente pour former le siège de
Turin, qui, prêt à se faire, n'eut pas lieu, rendit pour cette année
tous les préparatifs fort prompts. Le dépit si juste contre le duc de
Savoie, le succès de Calcinato tout récent et tout grossi, les
espérances qu'on concevait de ses suites l'extrême désir de dépouiller
M. de Savoie, et de le réduire en l'état du feu duc Charles IV de
Lorraine, affectionnaient le roi à ce projet. Chamillart, plus sage que
le monde ne l'a cru, en sentit le poids et en fut effrayé pour son
gendre auquel il était destiné. Il voulut encore tout bien examiner avec
Vauban en présence du roi. Puisqu'il avait fait la faute autrefois de le
prêter à M. de Savoie pour fortifier, ou plutôt pour perfectionner
Turin, il était bien naturel de le choisir pour en faire le siège.
Vauban, toujours le même, proposa son projet d'attaque, et les raisons
de ce projet\,; il détailla ce qu'il croyait nécessaire pour réussir\,;
il offrait, en lui fournissant ce qu'il demandait, de se charger du
siège, mais du siège uniquement, pourvu qu'il y fût le maître, et de
rien au delà, parce qu'il déclara avec franchise qu'il ne s'entendait
point à la guerre de campagne, ni à commander une armée. Ce qu'il
demanda se trouva monter en toutes sortes de choses à bien plus qu'il ne
fut possible de lui fournir, Là-dessus, il avertit le roi bien
fermement, devant son ministre, chez M\textsuperscript{me} de Maintenon,
que Turin ne se prendrait pas à moins\,; et (ce qui est incroyable, avec
la juste confiance du roi en Vauban, fondée sur une si longue
expérience, avec le silence et l'embarras de Chamillart), sur ce refus
de Vauban comme n'y pouvant réussir, la commission en fut sur-le-champ
donnée ou plutôt confirmée à La Feuillade. Quel parallèle entre ces deux
hommes\,! et quel champ aux réflexions\,! Et peut-on s'empêcher de
reconnaître que, lorsque Dieu veut châtier, il commence par aveugler\,?
C'est ce qui se retrouve sans cesse dans le cours de cette guerre, mais
c'est aussi ce qui ne saute nulle part aux yeux si fortement qu'ici.

Voilà donc La Feuillade non plus général par accidents amenés, non plus
général en peinture, mais général d'une armée sur laquelle toute
l'Europe fixa les yeux et trouva son sort attaché. Troupes d'élite
autant que la possibilité les put grossir, officiers choisis, munitions
en abondance, artillerie formidable, trésors d'argent, désir et
exécution, identité de choses, en un mot le gendre bien-aimé d'un
tout-puissant ministre des finances et de la guerre, qui mettait en lui
toutes ses complaisances, toutes ses espérances, l'appui et le salut de
sa famille, on peut juger qu'on fut jusqu'à l'impossible de toutes parts
pour le mettre en état de faire une conquête si capitale pour l'État, et
si importante à leur fortune particulière. Tout fut donc très
promptement disposé. La Feuillade arriva devant Turin le 13 mai, et se
mit à faire ses lignes et ses ponts. Tardif, à faute de mieux, fut son
premier ingénieur. Il n'avait fait que de petits sièges en Bavière.
Ainsi cette forte besogne roula tout entière sur deux novices fort
ignorants, et par cela même fort entêtés. Laissons-les s'établir.

Le roi n'avait rien tant recommandé au maréchal de Villeroy que de ne
rien oublier pour ouvrir la campagne par une bataille. Il commençait à
sentir le poids de la guerre\,; il avait dès lors envie de la terminer,
mais il voulait donner la paix et non la recevoir. Il espérait tout de
ses généraux et de ses troupes\,; les succès d'Italie et du Rhin
semblaient lui répondre de ceux de ses autres entreprises\,: il aimait
assez Villeroy pour vouloir qu'il cueillît des lauriers. Il partit à la
mi-avril pour retourner en Flandre, et depuis son départ jusqu'à
l'assemblée de son armée, le roi le pressa sans cesse d'exécuter ce
qu'il lui avait si expressément ordonné.

Le génie court et superbe de Villeroy se piqua de ces ordres si
réitérés. Il se figura que le roi doutait de son courage puisqu'il
jugeait nécessaire de l'aiguillonner si fort\,; il résolut de tout
hasarder pour le satisfaire, et lui montrer qu'il ne méritait pas de si
durs soupçons. En même temps que le roi voulait une bataille en Flandre,
il se voulait mettre en état de la gagner. Dès que les lignes du Rhin
furent prises et le fort Louis dégagé, le roi envoya ordre à Marsin de
prendre dix-huit bataillons et vingt escadrons de son armée, laissant le
reste à Villars, et de venir sur la Moselle où il trouverait vingt
autres escadrons et de marcher avec le tout en Flandre joindre le
maréchal de Villeroy\,; et à celui-ci de ne rien entreprendre avant
cette jonction faite. Cette défense fut réitérée au maréchal de Villeroy
par quatre courriers de suite coup sur coup, sur ce que ses réponses
montraient que, piqué de toutes les instances qui lui avaient été
redoublées pour donner promptement une bataille, il la voulait brusquer
sans attendre ce secours. J'insiste ici sur ce point, parce qu'il fut
celui de la division mortelle d'entre le maréchal et Chamillart, et que
ce dernier me montra les lettres originales du roi et de lui au
maréchal, et les réponses de ce dernier depuis l'ouverture de la
campagne, et quelques-unes même dès auparavant. Mais il ne s'agit pas
encore de cette querelle.

Villeroy donc poussa sa pointe malgré les ordres d'attendre Marsin.
Marlborough avait passé la mer de bonne heure, toutes ses troupes ne
l'avaient pas joint. Villeroy en avait plus que lui. Cette raison lui
donna de la confiance, il ne douta point du succès\,; il n'en voulut
partager l'honneur avec personne, non seulement avec Marsin et les
troupes qu'il lui amenait, mais avec l'électeur même, qui pourtant
commandait l'armée et que le maréchal avait laissé à Bruxelles sans lui
faire part de son dessein. Il s'avança donc, le 21 mai, vers l'endroit
où l'année précédente Roquelaure avait laissé percer nos lignes. Sur
l'avis de la marche et de l'approche de Marlborough, il fit un mouvement
pour l'attendre, puis, le 24 au matin, jour de la Pentecôte, un second
pour se poster dans un terrain où feu M. de Luxembourg n'avait jamais
voulu s'exposer à combattre. Lui-même en avait été témoin, mais son sort
et celui de la France était qu'il l'oubliât. Il le manda par un courrier
avant de prendre ce poste. M. le duc d'Orléans prédit à qui le voulut
entendre qu'il y serait battu s'il y tentait ou y souffrait une
action\,; que M. de Luxembourg n'avait jamais voulu s'y commettre\,; et
que sur le lieu même il lui en avait expliqué et montré les raisons que
ce prince rendit fort bien. Il ne fut que trop bon prophète.

Villeroy mit donc la maison du roi et deux brigades de cavalerie de
suite entre les villages de Taviers et de Ramillies. Taviers couvrait le
flanc de la maison du roi. Sa situation était sur un penchant près de la
Méhaigne qui formait un marais derrière, et dans ce village il mit le
comte de La Mothe avec six bataillons de l'électeur et trois régiments
de dragons. Il établit dans celui de Ramillies vingt-quatre pièces de
canon soutenues de vingt bataillons, qui le furent ensuite d'un plus
grand corps d'infanterie. Il en prit le surplus pour occuper le terrain
qui s'étendait vers le village de Neuféglise, laissa la droite de sa
seconde ligne dans son ordre naturel, et porta son aile gauche devant un
marais très difficile qui s'étendait au delà de cette aile, laquelle se
trouvait à peu près en ligne avec la droite. Comme il achevait ses
dispositions, l'électeur à peine averti arriva au grand galop de
Bruxelles. Il avait grand lieu de se plaindre, et peut-être encore de
blâmer ce qui se faisait\,; mais il n'était pas temps. Il n'y avait que
celui d'achever ce qui était commencé\,; à quoi il se prêta sans humeur
et de bonne grâce en attendant un autre loisir.

Il était deux heures après midi quand l'armée ennemie, arrivée en bel
ordre en présence, commença à essuyer le canon de Ramillies. Il obligea
leurs troupes à faire halte pour attendre le leur qui, fort promptement
après, se trouva en batterie. La canonnade dura bien une heure. Ils
marchèrent ensuite à Taviers avec du canon. Ils y trouvèrent moins de
résistance qu'à leur droite, ils s'en rendirent maîtres. Dès ce moment,
ils firent marcher leur cavalerie. Ils s'étaient aperçus fort à temps
que le marais qui couvrait notre gauche empêcherait les deux ailes des
deux armées de se pouvoir joindre. Ils avaient fait couler toute la leur
derrière leur centre, en avaient formé plusieurs lignes les unes sur les
autres, mais sans confusion, derrière leur gauche, eurent ainsi toute la
cavalerie de leur armée vis-à-vis notre droite et en état de s'en
servir, tandis que toute la moitié de la nôtre demeura inutile dans un
poste où elle ne pouvait rien faire. Elle avait vu toute celle des
ennemis disparaître de devant elle entièrement\,; ce mouvement, qui
devait lui servir d'exemple, ne l'ébranla point. Gassion qui la
commandait, comme l'ancien lieutenant général de notre gauche, s'en
tourmenta fort, mais sans succès. Il lui était ordonné de ne bouger de
là sans ordre\,; il eut beau envoyer des aides de camp, nul ordre ne lui
parvint.

Guiscard, l'ancien lieutenant général de la droite, la fit ébranler au
mouvement des ennemis. La maison du roi et la première ligne de la
cavalerie de cette aile fit une charge vigoureuse. Les escadrons rouges
de la maison du roi percèrent trois lignes de cavalerie qui s'ouvrirent,
tandis que leur droite emporta, la première ligne. Les rouges gagnèrent
plus de cinq cents pas de terrain. Ils chargèrent encore tout de suite
avec succès des escadrons qui les voulaient prendre en flanc. Ils se
rallièrent après en faisant demi-tour à droite, et en chargèrent encore
six autres. Ils trouvèrent après une quatrième ligne devant eux, et
furent en même temps pris par derrière. Cette aventure était arrivée
plus tôt à eux qu'à leur droite, qui ne put ainsi leur donner de
secours. Le même malheur était arrivé à leur gauche. Les ennemis qui
avaient là ligne sur ligne ne firent partout que s'ouvrir pour laisser
engager la nôtre bien avant, et se refermer ensuite et la prendre par
devant et par derrière. Plus de protection du village de Taviers, dont
les ennemis, comme je l'ai dit, s'étaient rendus maîtres, et se
servaient au contraire de notre canon sur nous, et le village de
Ramillies trop éloigné. Ce fut donc pour nos troupes à repasser, qui
put, un petit marais dont le milieu était difficile, et dont chacun ne
se serait tiré sans un peloton d'infanterie qui, de soi-même et sans
ordre, se détacha, se posta sur le bord, et protégea de son feu ceux qui
purent repasser.

Le désordre et l'inégalité de cette charge donna lieu à de grands
inconvénients et à diverses plaintes fâcheuses. Ce qui demeura ensemble
ou se rallia de la maison du roi demeura en bataille derrière le village
de Ramillies. Le feu y fut prodigieux. Nos troupes pénétrèrent jusqu'au
centre des ennemis\,; mais leur grand nombre les rechassa bien vite\,;
et, dans ce désordre, ils emportèrent le village de Ramillies, et eurent
tout le canon que nous y avions\,: mis. Le duc de Guiche, à la tête du
régiment des gardes, s'y défendit quatre heures durant, et y fit des
prodiges. La seconde ligne de cavalerie de la droite, presque toute
bavaroise ou wallonne, avait refusé tout net au duc de Villeroy et à
Sousternon, lieutenants généraux, de soutenir la première, et demeura
sans rien faire. Toute notre gauche resta inutile, le nez dans ce
marais, et personne vis-à-vis d'elle, sans branler de ce poste\,; notre
droite, tout à fait rompue, le centre enfoncé, et l'infanterie qui avait
presque toute combattu, rebutée. L'électeur se porta partout avec une
grande valeur. Le maréchal de Villeroy courait éperdu et ne savait
remédier à ce qui coup sur coup arrivait de sinistre. Il montra de la
valeur, mais ce fut tout. On n'en doutait pas, ni qu'il fût en lui d'y
mettre autre chose. Il ne fut donc plus question que de se retirer.

La retraite commença dans un grand ordre\,; mais bientôt la nuit survint
qui mit la confusion. La cavalerie de la gauche rompit l'infanterie, en
pressant trop sa marche qui dura toute la nuit. Le défilé de Judoigne se
trouva tellement engorgé des gros bagages et de quelques menus, et de ce
qu'on avait pu retirer d'artillerie, que tout y fut pris. Enfin l'armée
arriva à Louvain\,; mais on ne se crut en sûreté qu'après avoir passé le
canal de Wilworde, sans néanmoins que les ennemis eussent suivi de trop
près.

Bruxelles, dont Bagnols et Bergheyck étaient sortis à temps avec le
trésor et les blessés qu'on avait pu transporter, fut le premier fruit
de la victoire. Plusieurs personnes considérables en sortirent en même
temps\,; beaucoup davantage y demeurèrent. Anvers, Malines et Louvain ne
tardèrent pas à prêter, comme Bruxelles, serment à l'archiduc. Ce ne fut
que le commencement du retour des Pays-Bas espagnols à la maison
d'Autriche.

Une action qui eut de si grandes et de si rapides suites ne coûta pas
quatre mille hommes, mais une grande dispersion qui revint presque toute
et en fort peu de temps rejoindre chacun son corps. M. de Soubise y
perdit un de ses fils cadets qui était dans les gens d'armes, et
Gouffier. D'Aubigny, colonel de dragons, Bernière, major du régiment des
gardes et major général de l'armée, milord Clare, maréchal de camp, Bar,
brigadier de cavalerie, homme d'un singulier mérite et fort de nies
amis, furent tués\,; quelques blessés et beaucoup de prisonniers de
marque que Marlborough traita avec une politesse infinie, et permit à
beaucoup de revenir sur-le-champ pour trois mois sur leur parole.

Le roi n'apprit ce désastre que le mercredi, 26 mai, à son réveil. On
admira la platitude du maréchal de Villeroy, qui, par le même courrier,
écrivit à Dangeau merveilles de son fils, et que sa blessure à la tête
d'un coup de sabre ne serait rien. Il oublia tout le reste. J'étais à
Versailles\,; jamais on ne vit un tel trouble ni une pareille
consternation. Ce qui y mit le comble fut que, ne sachant rien qu'en
gros, on fut six jours sans courrier. La poste même fut arrêtée. Les
jours semblaient des années dans l'ignorance du détail et des suites
d'une si malheureuse bataille, et dans l'inquiétude de chacun pour ses
proches et pour ses amis. Le roi fut réduit à demander des nouvelles aux
uns et aux autres sans que personne lui en pût apprendre. Poussé à bout
d'un silence si opiniâtre, il prit le parti d'envoyer Chamillart en
Flandre, pour avoir par lui au moins sûrement des nouvelles, et pour
qu'il lui rapportât l'état de l'armée, des progrès des ennemis, et le
résultat des délibérations qui seraient prises entre l'électeur, le
maréchal de Villeroy et lui. Le dimanche, 30 mai, Chamillart sortant de
travailler avec le roi, sur les cinq heures, qui allait après se
promener à Trianon, monta en chaise de poste, disant qu'il s'en allait à
l'Étang, où j'avais dîné avec sa femme et ses filles, et s'en alla tout
de suite à Lille. Ce fut un autre étonnement fort grand à la cour que la
disparition d'un homme chargé tout à la fois des finances et de la
guerre, et de tous les ordres divers, continuels et prompts à donner
dans une si fâcheuse conjoncture.

Chamillart ne surprit pas moins l'armée. Il la trouva autour de
Courtrai, où le maréchal de Villeroy l'alla trouver dès qu'il l'y sut
arrivé\,; et dès lors on s'aperçut de quelque refroidissement entre eux.
Le ministre fut le lendemain voir l'électeur, qui le reçut en prince
malheureux et qui sentait ses besoins. Villeroy fut peu en tiers. Le
tête-à-tête dura trois heures, d'où Chamillart retourna à Courtrai. Le
lendemain, il revit encore l'électeur seul, mais moins longtemps.
Retournant de là à Courtrai, Villeroy fit peu de chemin avec lui, puis
tourna bride à son quartier. Chamillart entretint force officiers
généraux et particuliers.

Chamillart, qui de Flandre avait presque tous les jours dépêché des
courriers au roi, arriva à Versailles sur les huit heures du soir du
vendredi, 4 juin, et alla tout droit trouver le roi chez
M\textsuperscript{me} de Maintenon, où il lui rendit compte de son
voyage jusqu'à son souper. On sut donc enfin qu'après quelques marches
précipitées l'armée se trouvant sous Gand, l'électeur avait insisté à
l'y faire demeurer et à garder le grand Escaut\,; que le maréchal de
Villeroy s'y était fort opposé\,; qu'il avait consenti avec grand'peine
à un conseil de guerre, où le comte de La Mothe avait librement appuyé
l'avis de l'électeur, quoique le maréchal, en proposant d'abord le fait,
eût opiné hautement en général qui voulait contraindre les voix, qui
toutes aussi, par la crainte qu'ils en conçurent, s'étaient rangées à
son avis. L'électeur en fit, en public et en particulier, des plaintes
amères, cria contre un si grand découragement, protesta sur un si
mauvais parti à prendre et sur ses funestes suites, mais il ne voulut
pas, user du pouvoir qu'il avait de s'en faire croire, dans
l'appréhension des retours d'une cour dont les malheurs communs le
rendaient encore plus dépendant.

Gand fut donc abandonné. On revint sous Menin, on abandonna la campagne,
on sépara toute l'infanterie et beaucoup de cavalerie dans les places
avec des officiers généraux, on distribua le reste dans la châtellenie
de Lille et des environs. De cette manière, à l'exception de Namur, Mons
et fort peu d'autres places, tous les Pays-Bas espagnols furent perdus,
et une partie des nôtres même. Jamais rapidité ne fut comparable à
celle-là. Les ennemis en furent aussi étonnés que nous. La douleur s'en
augmenta chaque jour par le retour de tout ce qui rejoignait et qu'on
croyait perdu.

Mais ce qui le fut entièrement et qui perdit tout le reste, ce fut la
tête du maréchal de Villeroy. Rien ne la put remettre, personne ne le
put rassurer. Il ne voyait et n'entendait plus, il ne voyait qu'ennemis,
que périls, que défaites, de sûreté nulle part. Son fils et Sousternon,
qui avait fort sa confiance, mais à qui il s'était bien gardé de confier
son projet, l'avaient pénétré la surveille de la bataille. Ils l'avaient
conjuré de ne s'y pas commettre, ils se portèrent jusqu'à se mettre à
genoux et embrasser les siens\,; il demeura inflexible. Outré du
sinistre succès d'un projet conçu par lui seul et qu'il avait exécuté
contre l'avis de ce peu qui l'avait éventé, désespéré du remords de
n'avoir pas attendu Marsin et ses troupes, nonobstant les ordres si
réitérés qu'il en avait, la tête lui tourna tout à fait. Il fut
incapable d'écouter personne, également entêté devant et après\,; et fit
de son autorité, de la crainte de sa faveur, une plaie à l'État, qui,
très large et très funeste dès lors, le mit bientôt après à deux doigts
de sa perte. Jamais de bataille où la perte ait été plus légère, jamais
aucune dont les rapides suites aient été plus prodigieuses.

Quelque tranquillement au dehors que le roi soutînt ce malheur, il le
sentit en entier dans toutes ses parties. Il fut sensible à tout le mal
qui se débita de ses gardes du corps, et se plaignit d'eux assez
aigrement, touché de leur honneur, peut-être encore de sa sûreté. Il
manda de l'armée Darignon, leur aide-major, homme de rien et vendu à la
fortune. Des guerriers de cour rendirent de bons témoignages d'eux, qui
ne persuadèrent personne. Cela ne veut pas dire qu'on eût raison de mal
parler des gardes du corps\,; mais, bien que ces témoignages eurent peu
d'autorité, le roi les saisit avec tant de joie qu'il fit mander aux
gardes, et qu'il envoya par les salles, les assurer qu'il était éclairci
et fort content d'eux. Le monde le fut peu de cette espèce de
réparation. Quoi qu'il en ait été dans une action si mal conduite, ils
s'étaient auparavant distingués si fort, et ont toujours depuis si
constamment fait des prodiges de valeur dans toutes les actions où ils
se sont trouvés, qu'ils se sont acquis un nom qui a donné de l'émulation
à toutes les troupes, et à celles des ennemis, de leur propre aveu, une
jalousie et une crainte qui les a couverts de gloire.

Ce triste revers portait sur le seul maréchal de Villeroy, à plomb. Le
projet peu sensé et moins digéré, communiqué à personne et caché même à
l'électeur quoique généralissime, l'exécution déplorable et un terrain
proscrit en sa présence par M. de Luxembourg, les suites immenses
uniquement dues au renversement de sa tête et à son opiniâtreté, sa
précipitation et sa formelle désobéissance de n'attendre pas la jonction
si prochaine des troupes que lui amenait Marsin, le cri public de
l'armée qui avait perdu tout respect et toute mesure à son égard, le
juste mécontentement de l'électeur sur tant de points si capitaux,
firent enfin comprendre au roi qu'il était temps que la faveur cédât à
la fortune. Un général d'armée de l'empereur en eût bien sûrement perdu
la tête par le conseil aulique de guerre\,; il ne tint qu'à celui-ci
d'être mieux que jamais. Le roi le plaignit, le défendit, lui écrivit de
sa main qu'il était trop malheureux à la guerre\,; qu'il lui conseillait
et lui demandait, comme à son ami, de lui mander sa démission du
commandement de l'armée\,; qu'il voulait qu'il parut que ce n'était que
sur ses instances qu'il l'en déchargeait\,; qu'il le verrait auprès de
lui avec plus d'amitié que jamais\,; et qu'il pouvait s'assurer du gré
et du compte qu'il lui tiendrait d'un sacrifice qui lui coûtait autant
ou plus qu'à lui-même, mais que la situation présente rendait
nécessaire, et qui ne serait connu que de lui\,; tandis qu'il lui
promettait qu'il n'y aurait personne qui ne demeurât persuadé, à la
manière dont cela se passerait et dont il le traiterait, que c'était
lui, maréchal, qui l'avait forcé de lui mander la permission de quitter
le commandement de l'armée et de revenir à sa cour.

À qui n'a pas vu ces faits ils peuvent paraître incroyables. Mais outre
les minutes que Chamillart m'a fait voir des lettres signées du roi,
envoyées au maréchal, toutes plus pressantes et plus tendres les unes
que les autres, de ce même style, pour vaincre sa résistance, c'est que
je l'ai su encore de gens à qui le roi, à la fin outré, s'en est
amèrement plaint.

Villeroy, par cette première lettre de la main du roi, ne sentit qu'une
faveur étonnante dans la situation où il se trouvait, et cette faveur
l'aveugla. Il crut se maintenir en tenant ferme, et qu'avec une amitié
si singulière et si particulièrement témoignée, telle que le roi n'en
aurait pu user mieux avec son propre frère, jamais il ne se résoudrait à
l'arracher de son emploi malgré lui. Il répondit donc au roi, après
force propos de courtisan comblé, qu'il n'était point faux, qu'il
n'était ni blessé ni malade, qu'il était malheureux, mais qu'il croyait
n'avoir point failli, qu'il ne pouvait demander sa démission sous aucun
prétexte véritable, ni se déshonorer en se déclarant soi-même, par cette
démarche, incapable et indigne du commandement de ses armées dont il
l'avait honoré, et faire en même temps la plus grande injure à son
choix.

Cette première réponse fâcha le roi sans l'irriter. Il condescendit,
avec sa première amitié, à l'état douloureux d'un homme à qui on demande
la démission d'un si grand emploi, dans les circonstances fâcheuses où
il se trouvait. Il redoubla, tripla, quadrupla toujours en même style,
et ne reçut que les mêmes réponses. Par la dernière, toujours comptant
sur ce qui l'avait séduit d'abord, il manda arrogamment au roi qu'il
était maître de lui ôter le commandement de l'armée et de faire de lui
tout ce qu'il lui plairait, qu'il obéirait avec soumission et sans se
plaindre, mais qu'il n'attendît pas de lui qu'il en fût jamais de
moitié. La résolution était prise, dès la première lettre, de le faire
revenir, mais en couvrant ce retour de sa demande instante. À cette
dernière, le roi se piqua et perdit patience et espérance de ramener un
homme si fort égaré.

Pendant cette espèce de négociation de bonté avec lui, le roi avait
dépêché à M. de Vendôme pour lui proposer de venir commander l'armée de
Flandre. Il lui était fatal de réparer les malheurs du maréchal de
Villeroy, au moins d'être choisi pour cela. C'est ce qui, après
l'affaire de Crémone, l'avait mis à la tête de l'armée d'Italie.
Vendôme, avec toutes ses thèses étranges, ses entêtements et ses appuis,
sentait alors toute la difficulté de réussir à Turin et de soutenir les
affaires en Italie. Le prince Eugène et ses renforts de troupes arrivés
aussitôt après le combat de Calcinato y avaient entièrement changé la
face et le théâtre de la guerre. Vendôme, de victorieux et
d'entreprenant, était réduit à la défensive\,; et au milieu de tous ses
tons avantageux s'en trouvait fort embarrassé. Il regarda donc comme une
délivrance la proposition qui lui était faite de quitter l'Italie. Il y
laissait, non pas à l'égard du pays ni des Impériaux, mais à l'égard de
la cour et de ce qui s'appelle en France le monde, une réputation non
entamée, qui lui avait fait goûter, presque comme aux héros de
l'ancienne Rome, tous les honneurs du triomphe au voyage qu'il venait de
faire à la cour et à Paris. Il fut comblé de joie de n'avoir point à la
commettre, et de se tirer de la presse du beau-père et du gendre sur
tout ce qu'il prévoyait de Turin. Il se trouva flatté d'être regardé
comme le réparateur, et à son aise en même temps sur l'emploi auquel il
était appelé. Tout était regardé comme perdu en Flandre\,; ce qu'il n'y
pourrait soutenir ni réparer tomberait sur celui qui y avait tout perdu,
et pour peu qu'il y pût faire serait relevé comme des prodiges. En même
temps il sut donner comme un sacrifice ce qu'il considérait comme son
salut\,; et goûté et soutenu comme il l'était, ce prétendu sacrifice fut
reçu comme un sacrifice très réel, dont le roi lui sut le plus grand gré
du monde.

Tandis que toutes ces résolutions s'acheminaient dans le plus profond
secret, il en fallut prendre une en même temps sur le choix d'un général
en Italie. Chamillart, extrêmement en peine des malheurs accablants qui
accompagnaient son ministère, sentit ce que pouvait la présence d'un
prince -du sang dans une armée de François. Il avait déjà proposé le
prince de Conti pour l'envoyer en Flandre. Il se voulait concilier ces
princes, et avec eux le public, en lui montrant que, uniquement touché
du bien des affaires, il proposait lui-même ce que ses prédécesseurs
avaient le plus craint et éloigné. Il trouva l'opposition du roi si
grande pour le prince de Conti, à qui il avait peut-être encore moins
pardonné son mérite et l'amour et l'estime universelle, par jalousie
pour M. du Maine, que son voyage de Hongrie, que, le choix du roi fait
de M. de Vendôme, il n'osa plus parler du prince de Conti pour l'Italie.
Il craignit, avec raison, les fougues impétueuses de l'humeur farouche
et continuelle de M. le Duc. Il proposa donc M. le duc d'Orléans comme
celui dont le rang et l'aînesse ôtaient aux princes du sang tout sujet
de se plaindre de la préférence. Le roi, jusqu'alors si éloigné de
donner ses armées à commander à ceux de son sang, pour ne les pas trop
agrandir, et plus encore par rapport à M. du Maine qu'il ne sentait que
trop douloureusement n'y être pas propre, mais pressé par la nécessité
et par le poids accablant des conjonctures, se laissa vaincre à son
ministre favori qui avait eu soin de mettre M\textsuperscript{me} de
Maintenon de son côté.

M. le duc d'Orléans, ni aucun des princes du sang, ne songeait à servir.
Ils en avaient perdu toute espérance depuis longtemps, et personne même
ne pensait à eux. Tout le monde était imbu de l'extrême répugnance du
roi là-dessus, lorsque, le mardi 22 juin, à Marly, le roi, ayant donné
le bonsoir à tout ce qui était dans son cabinet tous les soirs après son
souper, rappela M. le duc d'Orléans qui sortait avec les autres, et le
retint seul un gros quart d'heure. Je m'étais, ce soir-là, amusé dans le
salon, où la rumeur fut tout à coup grande de la nouveauté qui se
passait. On ne fut pas longtemps dans l'ignorance. M. le duc d'Orléans,
sortant d'avec le roi, passa dans le salon pour aller chez Madame, y
revint un moment après, et y apprit qu'il allait commander l'armée
d'Italie, que M. de Vendôme l'y attendrait et reviendrait incontinent
après prendre le commandement de celle de Flandre, dont le maréchal de
Villeroy était rappelé.

Le même soir, le roi à son coucher, où depuis sa longue goutte il n'y
avait plus que les entrées grandes et secondes, tout piqué qu'il était
contre l'inflexibilité du maréchal de Villeroy, eut la bonté de dire
qu'il lui avait si instamment demandé son retour, qu'il n'avait pu le
lui refuser. C'était une dernière planche que le reste de son amitié lui
tendait encore après le naufrage. Il eut la folie de la repousser. C'est
ce qui enfin fit sa disgrâce, comme je le dirai en un autre temps pour
ne pas interrompre des choses plus intéressantes. Il eut ordre de
revenir sur-le-champ. Puis le roi changea sa lettre et lui ordonna
d'attendre M. de Vendôme en Flandre, où les ennemis prirent Ostende et
Nieuport fort promptement\,; sur quoi le maréchal de Vauban fut envoyé à
Dunkerque commander à tout ce côté-là de la Flandre maritime.

\hypertarget{chapitre-xi.}{%
\chapter{CHAPITRE XI.}\label{chapitre-xi.}}

1706

~

{\textsc{Comte de Toulouse de retour à Versailles, et sa flotte à
Toulon.}} {\textsc{- Levée du siège de Barcelone.}} {\textsc{- Le roi
d'Espagne gagne Pampelune par le pays de Foix, puis Madrid.}} {\textsc{-
Tessé revient à la cour.}} {\textsc{- Duc de Noailles fait lieutenant
général seul, et commande en chef en Roussillon.}} {\textsc{- La reine
d'Espagne, etc., à Burgos.}} {\textsc{- Le roi d'Espagne joint Berwick
de sa personne.}} {\textsc{- Dispersion de sa cour.}} {\textsc{- Ses
ennemis maîtres de Madrid.}} {\textsc{- Tessé salue le roi.}} {\textsc{-
Vaset remet au roi les pierreries du roi et de la reine d'Espagne.}}
{\textsc{- Zèle des évêques d'Espagne et des peuples.}} {\textsc{-
Évêque de Murcie.}} {\textsc{- Madrid au pouvoir du roi d'Espagne, qui y
rentre, et la reine.}} {\textsc{- Les ennemis chassés des Castilles.}}
{\textsc{- Comte d'Oropesa passe à l'archiduc.}} {\textsc{- Patriarche
des Indes arrêté y passant avec le comte et la comtesse de Lémos.}}
{\textsc{- Soulagement du palais.}} {\textsc{- Contades fait major du
régiment des gardes\,; son extraction\,; son caractère.}} {\textsc{-
Cent cinquante mille livres à M. de Soubise, et la nomination de son
fils au cardinalat déclarée.}} {\textsc{- Mort du chevalier de
Courcelles et sa parenté.}} {\textsc{- Mort de Montchevreuil.}}
{\textsc{- Mort de Bourlemont.}} {\textsc{- Mort de
M\textsuperscript{lle} de Foix.}} {\textsc{- Mort de Brou, évêque
d'Amiens\,; son caractère.}} {\textsc{- Mort de l'abbé Testu\,; son
caractère\,; personnage singulier.}} {\textsc{- Mort de Rhodes\,; son
caractère.}} {\textsc{- Mort de la mère du maréchal de Villars\,; son
caractère.}} {\textsc{- Mort de M\textsuperscript{me} de Gacé.}}
{\textsc{- Mort de la princesse de Tingry.}} {\textsc{- Mort de la
duchesse Max. de Bavière.}} {\textsc{- Mort de Congis et sa dépouille.}}
{\textsc{- Mort de Laubanie et sa dépouille.}} {\textsc{- Mort de la
duchesse de Montbazon\,; son extraction\,; son caractère.}} {\textsc{-
Mort de M\textsuperscript{me} Polignac\,; son caractère\,; ses
aventures.}} {\textsc{- Trait étrange du Bordage.}}

~

Le soir même du jour que le roi avait appris à son réveil la cruelle
nouvelle de la bataille de Ramillies, M. le comte de Toulouse arriva à
Versailles, et fut trouver le roi chez M\textsuperscript{me} de
Maintenon, où il demeura fort longtemps avec lui, ayant laissé le
maréchal de Cœuvres pour quelques jours encore à Toulon. Il s'était tenu
mouillé devant Barcelone jusqu'au 8 mai. Les frégates d'avis qu'il avait
envoyées aux nouvelles de la flotte ennemie lui rapportèrent qu'elle
approchait, forte au moins de quarante-cinq vaisseaux de guerre. Notre
amiral, grâce aux bons soins de Pontchartrain, n'en avait pas une
bastante pour les attendre. Lui et le maréchal de Cœuvres eurent, avant
partir, une longue conférence avec le maréchal de Tessé et Puységur, et
tout au soir levèrent les ancres. Ils rentrèrent le 11 mai à Toulon.

Le départ de notre flotte et l'arrivée de celle des ennemis à Barcelone
y changea fort la face de toutes les choses. Les assiégés reprirent une
vigueur nouvelle, les assiégeants rencontrèrent toutes sortes de
nouveaux obstacles. Tessé, voyant l'impossibilité de continuer le siège
et toute la difficulté de la retraite en le levant, persuada au roi
d'Espagne de faire entrer le duc de Noailles dans toutes les
délibérations qu'il avait à prendre là-dessus. Noailles était tout
nouveau maréchal de camp. Il n'avait jamais fait quatre campagnes\,; sa
longue maladie l'avait retenu les étés à la cour, et la petite vérole
dont il avait été attaqué en arrivant devant Barcelone, et de laquelle
il ne faisait que sortir, l'avait empêché de servir de maréchal de camp
à ce siège, et assez longtemps même de savoir ce qu'il s'y passait, mais
il était neveu de M\textsuperscript{me} de Maintenon, et comme tel bon
garant pour Tessé. Tous les embarras où l'on était furent donc discutés
en sa présence. Il se trouva que les ingénieurs étaient si lents et si
ignorants, qu'il n'y avait aucun fond à faire sur eux, et que par la
vénalité que le roi avait mise dans l'artillerie depuis quelque temps,
comme je l'ai dit en son lieu, non seulement ces officiers vénaux n'y
entendaient rien du tout, mais avaient perdu sans cesse en ce siège, et
perdaient encore tout leur temps à remuer inutilement leur artillerie,
et à placer mal leurs batteries, pour se mettre dans la nécessité de les
changer, parce que de ces mouvements de canon résultait un droit
pécuniaire qu'ils étaient bien aises de multiplier. L'armée assiégée par
dehors, et depuis longtemps uniquement nourrie par la mer, n'avait plus
cette ressource depuis la retraite de notre flotte et l'arrivée de celle
des Anglais, et nulle autre d'ailleurs pour la subsistance journalière.
Toutes ces raisons persuadèrent enfin le roi d'Espagne de la nécessité
de lever le siège, quelque résistance qu'il y eût apportée jusqu'alors.

Après cela il fallut délibérer de la manière de l'exécuter et du lieu où
l'armée se tournerait. On convint encore qu'il n'y avait nul moyen de se
retirer par la Catalogne, pleine de révoltés qui tenaient la campagne,
soutenus de tous ceux du royaume de Valence qui tenaient les places, et
à travers cette cruelle multitude de miquelets qui les assiégeaient. Il
fut donc résolu qu'on prendrait le chemin de la frontière de France, et
que là, on délibérerait de nouveau, quand on serait en sûreté vers le
Roussillon, de ce qu'on deviendrait.

On leva donc le siège la nuit du 10 au 11 mai, après quatorze jours de
tranchée ouverte, et on abandonna cent pièces d'artillerie, cent
cinquante milliers de poudre, trente mille sacs de farine, vingt mille
de sevade\footnote{Espèce d'avoine.}, quinze mille de grain, et un grand
nombre de bombes, de boulets et d'outils. L'armée fut huit jours durant
harcelée par les miquelets en queue et en flanc de montagne en montagne.
Le duc de Noailles, dont l'équipage avait été constamment respecté par
eux pendant le siège et dans cette retraite, parce qu'ils aimaient son
père pour les avoir bien traités et avoir sauvé la vie à un de leurs
principaux chefs, s'avisa de les appeler pour leur parler. À son nom,
les principaux descendirent des montagnes et vinrent à lui. Il en obtint
qu'ils n'inquiéteraient plus l'armée, qu'ils ne tireraient plus sur les
troupes, à condition qu'on ne les brûlerait point. Cela fut exécuté
fidèlement de part et d'autre, et de ce moment l'armée acheva sa marche
en tranquillité, qui fut encore de trois jours, où elle aurait beaucoup
souffert de ces cruelles guêpes.

L'armée n'en pouvait plus\,; elle perdit presque tous ses traîneurs et
tous les maraudeurs dans cette retraite, en sorte qu'avec le siège il en
coûta bien quatre mille hommes. Sa volonté néanmoins fut toujours si
grande, que, malgré tant d'obstacles, elle aurait pris Barcelone, sans
ceux de notre artillerie et de nos ingénieurs.

Arrivés à la tour de Montgris, il fut question de ce que deviendrait le
roi d'Espagne. Quelques-uns voulaient qu'il attendît en France le
dénouement d'une si fâcheuse affaire, et d'autres que, se trouvant dans
cette nécessité, il poussât jusqu'à Versailles. Le duc de Noailles, à ce
qu'il m'a dit, et que je ne garantis pas, ouvrit un avis tout contraire,
et qui fut le salut du roi d'Espagne\,: il soutint que cette retraite en
France, ou ce voyage à la cour perdrait un temps précieux, et serait
sinistrement interprété\,; que les ennemis des deux couronnes le
prendraient pour une abdication, et ce qui en Espagne restait
affectionné, pour un manque de courage et pour un abandon d'eux et de
soi-même\,: que, quelque peu de suite, de moyens, de ressources qu'il
restât au roi d'Espagne, il devait percer par les montagnes du pays de
Foix droit à Fontarabie, de là joindre à tous risques la reine et son
parti, se présenter à ses peuples, tenter cette voie unique pour
réchauffer leur courage, leur fidélité, leur zèle, faire des troupes de
tout, pénétrer en Espagne, et jusque dans Madrid, sans quoi il n'y avait
plus d'espérance par les efforts que les ennemis allaient faire pour
s'établir par toute l'Espagne et dans la capitale même.

La résolution en fut heureusement prise. L'armée s'arrêta en
Roussillon\,; et tandis que le roi d'Espagne s'en alla à Toulouse et par
le pays de Foix gagner Pau, puis Fontarabie, avec deux régiments de
dragons pour son escorte, quelques grands d'Espagne qu'il avait avec
lui, et le duc de Noailles qui voulut l'accompagner jusqu'à Fontarabie,
le marquis de Brancas fut dépêché au roi pour lui rendre compte de tout,
recevoir ses ordres, et les porter à Pau au roi d'Espagne. Brancas
arriva le 28 mai à Versailles, sur le soir, et vit en arrivant le roi
chez M\textsuperscript{me} de Maintenon, où Chamillart le mena.

Il y avait longtemps que le roi s'attendait à cette triste nouvelle\,;
il approuva le parti qui avait été pris, donna au roi d'Espagne trente
bataillons et vingt escadrons qu'il avait ramenés du siège en
Roussillon, et tous les officiers généraux qui y servaient, donna
permission à Tessé de revenir, fit le duc de Noailles lieutenant général
seul, et le destina à commander en chef en Roussillon, à son retour
d'avec le roi d'Espagne. C'est ainsi que le duc de Noailles, au quart de
sa troisième ou quatrième campagne pour le plus, escalada rapidement
tous les grades en neveu favori de M\textsuperscript{me} de Maintenon.
On en avait bien fait autant pour le gendre bien-aimé de Chamillart\,;
mais La Feuillade était l'ancien du duc de Noailles de près de vingt
ans. Tessé eut l'honneur d'avoir prêté l'épaule à tous les deux. On a vu
en son temps ce qu'il fit pour La Feuillade\,; ici il ne voulait point
retourner en Espagne, où il voyait tout perdu. Il aimait mieux en
laisser tout le poids à Berwick, qui était sur les lieux, et il en
savait trop pour ne pas faire place au duc de Noailles en Roussillon. Il
fit le malade comme il l'avait su faire en Savoie et en Italie, s'amusa,
prit quelques jours des eaux à Balaruc, et regagna la cour.

En même temps que Brancas, longtemps depuis maréchal de France, fut
dépêché à Versailles, le roi d'Espagne envoya le duc d'Havré à la reine
d'Espagne, que ce seigneur trouva encore à Madrid, où elle avait été
laissée régente, et de Pau le roi d'Espagne s'en alla en poste à cheval
à Pampelune, et non à Fontarabie, suivi du connétable de Castille son
majordome-major, duc de Medina-Sidonia, âgé lors de plus de soixante
ans, son grand écuyer, du duc d'Ossone, capitaine de ses gardes, et de
peu de valets, et y arriva le 1er juin aux acclamations du peuple. Il en
partit le 2 vers Madrid. Le roi apprit le 14 juin par un courrier du duc
de Noailles que le roi d'Espagne y était arrivé aux plus grandes
acclamations de joie, et le duc de Noailles à sa suite, qui s'en revint
aussitôt après droit en Roussillon.

Berwick était cependant dans une étrange presse à la tête d'une poignée
de troupes mal en ordre vis-à-vis l'armée portugaise devant laquelle il
ne pouvait se présenter, qui prenait tout ce qu'il lui plaisait, allait
librement où elle voulait, et le faisait reculer et se retirer partout.
Il se tenait néanmoins toujours à portée d'elle, faisant mine de lui
disputer les gorges et les rivières, et ralentissant ses mouvements et
ses progrès autant que la capacité pouvait suppléer aux forces. Tout son
art et ses chicanes ne purent empêcher les Portugais de tourner sur
Madrid et de s'en approcher. La reine en sortit avec ses enfants et sa
suite, le 18 juin, pour aller à Burgos, sur le chemin de Pampelune. Le
roi en partit, le 21, pour s'aller mettre à la tête de la petite armée
de Berwick. Amelot le suivit, et les conseils suivirent la reine.
Quantité de grands s'en allèrent sur leurs terres, le cardinal
Portocarrero à Tolède, laissant la plus grande consternation dans
Madrid, dont, incontinent après, les Portugais se rendirent les maîtres.
Ils n'y trouvèrent aucun grand ni aucun membre des conseils. Le roi
d'Espagne et Berwick tournèrent vers Burgos, où les vingt escadrons et
les trente bataillons français du siège de Barcelone les devaient
rejoindre. Quelques grands le joignirent d'autres allèrent trouver la
reine à Burgos. Six semaines et plus se passèrent dans ces extrémités,
pendant lesquelles la reine confia toutes les pierreries du roi son mari
et les siennes à Vaset, ce valet français dont j'ai parlé, et l'envoya
les porter en France. Il arriva à Versailles en même temps que le
maréchal de Tessé. Vaset les remit au roi, et parmi elles cette fameuse
perle en poire appelée la \emph{Pérégrine}, qui, pour sa forme, son
poids, son eau parfaite et sa grosseur, est sans prix et sans
comparaison avec aucune qu'on ait jamais vue.

Enfin les troupes françaises arrivèrent en Espagne et joignirent le roi
et Berwick tout à la fin de juillet. L'archiduc se tenait cependant à
Saragosse, et laissait faire ses armées.

Les évêques d'Espagne s'étaient signalés entre tous à lever des troupes
à leurs dépens, et à donner au roi des sommes très considérables.
L'évêque de Murcie fit plus qu'aucun, qui avait été simple curé de
village avec tant de réputation et de vertu, que le roi d'Espagne
l'avait élevé à cet épiscopat, d'où il donna l'exemple à tous les
autres. Le cardinal Portocarrero, quoique si justement mécontent, donna
beaucoup et continua toujours de signaler son attachement. Celui des
prélats fut très important au roi. Ils s'appliquèrent à envoyer des
prédicateurs choisis dans tous les lieux de leurs diocèses affermir les
peuples dans leur fidélité et leur zèle, qui aussi en donnèrent les plus
grandes marques et les plus utiles.

Berwick, renforcé de vingt escadrons et de trente bataillons français,
changea toute la face de cette guerre. Il se présenta à l'armée ennemie
avec le roi d'Espagne\,: il chercha partout à la combattre. À son tour,
elle se tint sur la défensive et recula partout. Partout elle fut
poussée et perdit les lieux qu'elle avait pris ou occupés. Les peuples
armés par toute la Castille reprirent vigueur, et, sans troupes avec
eux, firent rebrousser l'archiduc qui venait joindre son armée. Ils
reprirent Ségovie, où les Portugais avaient laissé cinq cents hommes en
garnison, qui sortit du château à condition de se retirer en Portugal
par le chemin qui lui fut prescrit, et de ne servir de six mois contre
le roi d'Espagne. Ce prince, alors au large, envoya Mejorada avec cinq
cents chevaux à Madrid, d'où les Portugais s'étaient éloignés. Il y fut
reçu avec les plus grandes acclamations, et peu à peu les ennemis se
trouvèrent chassés de toute la Castille. Le roi d'Espagne rentra dans
Madrid à la fin de septembre, la reine incontinent, avec les plus
grandes marques de joie.

Pendant ce temps-là Berwick poursuivait l'armée de l'archiduc qui se
retirait devant lui de lieu en lieu. Il prit Cuença, mais Malaga et
l'île de Majorque demeurèrent encore à l'archiduc, à qui ils s'étaient
donnés dans cette prospérité de ses affaires. Le comte d'Oropesa,
président du conseil de Castille, que le roi d'Espagne avait trouvé
exilé depuis deux ans à son arrivée en Espagne, et qu'il y avait
toujours laissé, alla, en ce même temps de prospérité, trouver
l'archiduc avec toute sa famille. Le patriarche des Indes fut arrête
avec le comte et la comtesse de Lémos qui y allaient aussi ensemble.
M\textsuperscript{me} des Ursins, retournée avec la reine à Madrid,
profita de l'occasion de soulager le palais de trois cents femmes qui
avaient ou refusé de la suivre, ou dont les parents avaient montré leur
attachement pour l'archiduc. Tel fut l'étrange succès du siège mal
entrepris de Barcelone, et la rapidité avec laquelle il pensa renverser
Philippe V de son trône, qui avec la même célérité y fut reporté par son
courage, l'affection de la Castille, la sagesse et la capacité de
Berwick et les secours si prompts du roi son grand-père. Il ne fallait
pas couper ce grand événement par des choses moins intéressantes,
auxquelles il faut retourner présentement.

Le roi disposa assez promptement des emplois que la bataille de
Ramillies avait fait vaquer. Contades, dont il sera mention dans la
suite, fut fait major du régiment des gardes. C'était un gentilhomme
d'Anjou dont le père était connu du roi par plusieurs présents de
chiennes couchantes fort belles et fort bien dressées. Le fils, assez
bien fait, d'un visage agréable, eut le langage de la cour et celui des
dames, auxquelles il plut beaucoup. Il fut galant, mais souvent pour sa
fortune\,; il s'attacha extrêmement au duc de Guiche qui lui valut cet
emploi qu'il fit très bien et fort noblement. Il sut se tenir en sa
place avec tout le monde, plaire aux courtisans, aux généraux, ne se
mettre mal avec personne, cultiver les maris dont il l'était par leurs
femmes, et toutefois cheminer honnêtement et vivre recherché à Paris, à
la cour, aux armées, de la meilleure, de la plus utile et de la plus
brillante compagnie, se soutenir encore en toutes sortes de temps et de
changements dans la même situation, être dans la confiance de ceux qui
gouvernaient et qui commandaient\,; et le miracle de tout cela, c'est
qu'il avait fort peu d'esprit, et qu'il ne sut jamais faire une lettre.

M. de Soubise eut cinquante mille écus pour lui sur ce qui vaqua dans
les gens d'armes, y compris la charge du fils qu'il y avait perdu, et
déclara à Marly, le 12 juin, la nomination de son fils au cardinalat
dont les beaux yeux de M\textsuperscript{me} de Soubise avaient tiré
parole du roi il y avait déjà quelque temps.

Plusieurs personnes moururent en ce même temps\,:

Le chevalier de Courcelles, lieutenant général, qui servait à Luxembourg
et qui s'était distingué à la guerre\,; il s'appelait Champlais, d'une
noblesse fort commune\,; sa grand'mère était sœur du premier maréchal de
Villeroy\,; elle avait épousé en premières noces le vicomte de Tallard,
du nom de Bonne, du feu connétable de Lesdiguières\,; la fille unique de
ce mariage fut mère du maréchal de Tallard. En secondes noces elle
épousa Courcelles, lieutenant général d'artillerie, et fit fort parler
d'elle par des galanteries éclatantes auxquelles on n'était pas
accoutumé en ce temps-là, et qui la brouillèrent avec toute sa famille.
Elle mourut en 1688, dans une grande vieillesse, et avait beaucoup
d'esprit.

Montchevreuil, dont j'ai parlé si souvent qu'il ne me reste plus rien à
en dire\,; il mourut à Saint-Germain. Mornay son fils avait la
survivance de ce gouvernement et de la capitainerie.

Bourlemont, du nom d'Anglure\,; il était lieutenant général, avait fort
servi autrefois, et s'était brouillé avec M. de Louvois qui lui rasa, de
pique, Stenay dont il était gouverneur. C'était un très galant homme,
ami de mon père, qui avait, je ne sais comment tonnelé, marié sa fille
unique à Chamarande, qui était à la vérité très laide, mais avec
beaucoup de mérite et de vertu. Il était fort vieux. Son frère était
mort archevêque de Bordeaux.

Une vieille M\textsuperscript{lle} de Foix, tante paternelle du duc de
Foix, fort riche et de beaucoup d'esprit, à ce que j'ai ouï dire à M. de
Lauzun, qui en hérita en partie\,; elle n'avait jamais voulu sortir de
ses terres, où elle vivait en grande dame et avec des hauteurs qu'on
passait à l'âge et à la coutume, et qui ne seraient de mise aujourd'hui.

L'évêque d'Amiens, qui était Brou, d'une famille de Paris, et fort
distingué dans le clergé par ses mœurs, sa piété, le gouvernement de son
diocèse, sa science, sa capacité en affaires du clergé, son attachement
aux maximes du royaume et à la bonne morale, avec beaucoup de sagesse et
de discernement\,; il avait été aumônier du roi, et avait toujours
conservé les grâces du monde. Il était fort considéré de la bonne
compagnie et recherché de ce qu'il y avait de meilleur. Ami intime du
grand évêque de Meaux et de ce qu'il y avait de plus réglé et de plus
éclairé dans l'épiscopat. Il était oncle paternel de la femme du
président de Mesmes, depuis premier président. Son évêché y perdit tout
et fut donné à une barbe sale de Saint-Sulpice.

L'abbé Testu, qui était un homme fort singulier, mêlé toute sa vie dans
la meilleure compagnie de la ville et de la cour, et de fort bonne
compagnie lui-même\,; il ne bougeait autrefois de l'hôtel d'Albret, où
il s'était lié intimement avec M\textsuperscript{me} de Montespan, qu'il
voyait tant qu'il voulait dans sa plus grande faveur, et à qui il disait
tout ce qu'il lui plaisait\,; il s'y lia de même avec
M\textsuperscript{me} Scarron\,; il la voyait dans ses ténèbres avec les
enfants du roi et de M\textsuperscript{me} de Montespan qu'elle
élevait\,; il la vit toujours et toutes les fois qu'il voulut depuis le
prodige de sa fortune\,; ils s'écrivirent toute leur vie souvent, et il
avait un vrai crédit auprès d'elle\,; il était ami de tout ce qui
l'approchait le plus, et en grand commerce surtout avec M. de Richelieu
et sa femme, dame d'honneur, et avec M\textsuperscript{me} d'Heudicourt
et M\textsuperscript{me} de Montchevreuil. Il avait une infinité d'amis
considérables dans tous les états, ne se contraignait pour pas un, pas
même pour M\textsuperscript{me} de Maintenon\,; ne l'avait pas qui
voulait. C'est un des premiers hommes qui aient fait connaître ce qu'on
appelle des vapeurs\,; il en était désolé, avec un tic qui à tous les
moments lui démontait tout lé visage. Il primait partout, on en riait,
mais on le laissait faire. Il était très bon ami et serviable, il a fait
sous la cheminée beaucoup de grands plaisirs, et avancé et fait même des
fortunes\,; avec cela simple, sans ambition, sans intérêt, bon homme et
honnête homme, mais fort vif, fort dangereux, et fort difficile à
pardonner, et même à ne pas poursuivre quiconque l'avait heurté. Il
était grand, maigre et blond, et à quatre-vingts ans, il se faisait
verser peu à peu une aiguière d'eau à la glace sur sa tête pelée, sans
qu'il en tombât goutte à terre, et cela lui arrivait souvent depuis
beaucoup, d'années\,; il a fort servi l'archevêque d'Arles, depuis
cardinal de Mailly, et grand nombre d'autres, rompu le cou aussi à
quelques-uns. Ce fut une perte pour ses amis, et une encore pour la
société. C'était en tout un homme fort considéré et recherché jusqu'au
bout.

M. de Rhodes, le dernier de ce nom de Pot si ancien, si distingué, et
qui eut un collier de là Toison d'or en la première promotion que
Philippe le Bon fit à l'institution de cet ordre\,; il avait été grand
maître des cérémonies comme ses pères pour qui Henri III fit cette
charge. Fort de la cour et du grand monde, extrêmement galant, et avec
grand bruit, qui fit chasser M\textsuperscript{lle} de Tonnerre de la
chambre des filles de M\textsuperscript{me} la Dauphine. Il avait bien
servi et eut toujours beaucoup d'amis\,; c'était un grand homme fort
bien fait, avec beaucoup d'esprit et fort orné, mais un esprit trop
libre qui n'était pas fait pour la cour de Louis XIV. Aussi s'en
dégoûta-t-il et se retira-t-il à Paris, en espèce de philosophe, où il
épousa une Simiane, veuve d'un autre Simiane, dont il ne laissa qu'une
fille qui n'eut point d'enfants du prince d'Isenghien, de laquelle on a
vu la mort, il n'y a pas longtemps. Rhodes mourut avant la vieillesse,
mais rongé de la goutte depuis fort longtemps. C'est de lui et des
Gesvres qu'on a dit que l'ouvrage valait mieux que l'ouvrier.

Le maréchal de Villars perdit en ce même temps sa mère, tante paternelle
du feu maréchal de Bellefonds. C'était une petite vieille ratatinée,
tout esprit et sans corps, qui avait passé sa vie dans la meilleure
compagnie, et qui y vécut avec toute sa tête et sa santé jusqu'à sa mort
à quatre-vingt-cinq ou six ans. Elle était salée, plaisante, méchante\,;
elle s'émerveillait plus que personne de l'énorme fortune de son fils\,;
elle le connaissait, et lui recommandait toujours de beaucoup parler de
lui au roi, et jamais à personne\,; elle avait beau se contraindre, le
peu de cas qu'elle faisait de lui perçait\,; elle avait des apophtegmes
incomparables, et ne semblait pas y toucher.

Gacé, depuis le maréchal de Matignon, perdit sa femme qui passait sa vie
fort renfermée chez elle\,; elle était fort vertueuse, horriblement
laide, riche, et Bertelot, sœur de Plénœuf, de qui j'aurai lieu de
parler. Qui aurait cru qu'un nom si vil eût fait dans la suite la
fortune des deux fils qu'elle laissa\,?

La vieille Tingry les suivit de près à Versailles, où elle ne sortait
presque plus de sa chambre. J'ai expliqué qui elle était et sa
singulière histoire à propos du procès de M. de Luxembourg. Elle vécut
longtemps fort délaissée, et dans de grands scrupules sur ses vœux, et
d'avoir changé son voile contre un tabouret.

La veuve sans enfants du duc Max. de Bavière, sœur de M. de Bouillon, ne
survécut presque pas son mari, de la mort duquel j'ai parlé, il n'y a
pas longtemps, et sans enfants, comme je l'ai dit.

Congis, ancien capitaine aux gardes, espèce d'officier général hébété,
et en qui il n'y avait jamais eu grand'chose, mourut employé à la
Rochelle sous le maréchal de Chamilly. Il avait le gouvernement et
capitainerie des Tuileries et son fils la survivance. Il valait encore
moins que son père. Le roi voulut qu'il en accommodât Catelan pour peu
de chose, qu'il voulut dédommager de la Muette et du bois de Boulogne,
donnés à Armenonville, et à son fils, comme je l'ai dit lorsque le comte
de Toulouse acheta Rambouillet.

Laubanie ne jouit pas longtemps de la gloire d'avoir si bien défendu
Landau et de la récompense qu'il en avait eue. Sa grand'croix de
Saint-Louis fut donnée à Maupertuis, lieutenant général et capitaine des
mousquetaires gris. Comme il n'était pas commandeur, cette grâce passa
pour une distinction très particulière. Les capitaines de mousquetaires
étaient bien éloignés alors de penser à être chevaliers de l'ordre.

La duchesse de Montbazon, mère du prince de Guéméné, femme du duc de
Montbazon, mort fou, enfermé à Liège, belle-sœur du chevalier de Rohan,
qui eut la tête coupée devant la Bastille à la fin de 1674, belle-fille
de la belle et célèbre Montbazon qu'on a vue avoir commencé par son
obscur tabouret d'abord la princerie des Rohan et du frère de la fameuse
duchesse de Chevreuse, de la seconde duchesse de Luynes, et de M. de
Soubise. La duchesse de Montbazon était fille posthume, unique du second
mariage du premier maréchal de Schomberg, et de la seconde fille de M.
de La Guiche, grand maître de l'artillerie, ainsi nièce de la duchesse
d'Angoulême\,; elle était sœur de père du second maréchal de Schomberg
qui fut duc et pair d'Halluyn, par son mariage, et de cette sainte et
illustre duchesse de Liancourt, à laquelle elle ressembla si peu. La vie
de cette duchesse de Montbazon fut obscure, et ses mœurs et sa tête fort
mal timbrée avaient beaucoup fait parler d'elle. Elle avait
soixante-seize ans\,; elle s'avisa de faire exécuteur de son testament
le duc de La Rochefoucauld, avec qui elle n'avait jamais eu grand
commerce, et qui se mêlait fort à peine de ses propres affaires. Il
avait épousé la petite-fille, héritière de la duchesse de Liancourt, sa
sœur.

M\textsuperscript{me} de Polignac, seul reste de la maison de Rambures
avec M\textsuperscript{me} de Caderousse sa sœur. Elle avait été fille
d'honneur de M\textsuperscript{me} la Dauphine, et depuis son mariage,
chassée de la cour pour avoir été trop bien avec Monseigneur\,; et M. de
Créqui hors du royaume pour avoir été trop bien avec elle dans le temps
qu'il était leur confident. Elle s'en consola à Paris où, avec un mari
qui eut toujours pour elle des égards jusqu'au ridicule, et pour qui
elle n'en eut jamais le plus léger, elle mena une vie fort libre, et
joua tant qu'elle put le plus gros jeu du monde. Elle eut à la fin
permission de se montrer à la cour, où elle ne parut que très rarement
et des instants. Le Bordage, à qui la paresse et la passion du jeu
avaient fait quitter promptement le service, était de toutes les parties
chez elle, et partout où elle allait. Il en devint passionné, quoique
fort accusé de n'avoir pas de quoi l'être. C'était une créature d'esprit
et de boutades, qui ne se mettait en peine de rien que de se divertir,
de ne se contraindre sur quoi que ce fût, et de suivre toutes ses
fantaisies. Elle joua tant et si bien, qu'elle se ruina sans ressource,
et que, ne pouvant plus vivre ni peut-être se montrer à Paris, elle s'en
alla au Puy dans les terres de son mari. La tristesse et l'ennui
(quelques-uns l'ont accusée d'un peu d'aide) l'y firent tomber bientôt
fort malade. Dès que le Bordage l'apprit, il y courut, et presque
aussitôt après son arrivée il fut témoin de sa triste mort. Il en fut si
outré de douleur, qu'il avala tout ce qu'il fallut d'opium pour le tuer,
se jeta dans sa voiture, et ordonna qu'on le menât droit chez lui en
Bretagne. Il n'eut pas fait grand chemin, que l'opium opéra. Ses valets,
sur le soir, s'en aperçurent qu'il était comme mort et tout près de
passer. Leur surprise et quelque manège qu'ils avaient vu leur fit
deviner ce que ce pouvait être. Dans l'incertitude, ils le secouèrent et
lui firent avaler du vinaigre tant qu'ils purent, puis tout ce qu'ils
purent trouver de spiritueux, et avec beaucoup de peine et de temps le
réchappèrent. Il le trouva si mauvais dès qu'il put être revenu à soi,
qu'ils le veillèrent de bien près de peur de récidive, et, malgré lui,
le ramenèrent à Paris où ils avertirent ses amis et des médecins. Cette
aventure fit grand bruit, et plut extrêmement aux dames. Il fut
longtemps sans se pouvoir consoler, et les médecins sans le pouvoir
guérir. Il languit ainsi plus d'une année, et reprit après son jeu et sa
vie accoutumée. Le singulier est qu'à plus de soixante-dix ans il la
mène encore sans avoir été un moment incommodé depuis.

\hypertarget{chapitre-xii.}{%
\chapter{CHAPITRE XII.}\label{chapitre-xii.}}

1706

~

{\textsc{Baguettes du parlement baissées à Dijon chez M. le Prince.}}
{\textsc{- Baronnies de Languedoc réelles, non personnelles.}}
{\textsc{- Deux cent mille livres de brevet de retenue à Bullion.}}
{\textsc{- Cardinal de Janson arrivé de Rome.}} {\textsc{- Mariage de
des Forts avec la fille de Bâville.}} {\textsc{- Foucault cède à son
fils l'intendance de Caen.}} {\textsc{- Fortune de l'abbé de La Bourlie
en Angleterre.}} {\textsc{- Galanterie du roi à Marlborough.}}
{\textsc{- Verbaum arrêté allant aux ennemis.}} {\textsc{-
Faux-sauniers.}} {\textsc{- Orry à Paris\,; ne retourne plus en
Espagne\,; frise la corde de près\,; puis président à mortier au
parlement de Metz.}} {\textsc{- La reine douairière d'Espagne conduite
de Tolède à Bayonne.}} {\textsc{- Mort de Fontaine-Martel et sa
dépouille.}} {\textsc{- Caractère, conduite, extraction et dégoût de
Saint-Pierre.}} {\textsc{- Ma façon d'être avec M. le duc d'Orléans.}}
{\textsc{- M\textsuperscript{lle} de Sery fait légitimer le fils qu'elle
avait de M. le duc d'Orléans, et se fait appeler M\textsuperscript{me}
le comtesse d'Argenton par lettres patentes.}} {\textsc{- Curiosités sur
l'avenir très singulières.}}

~

Le roi jugea au conseil de dépêches deux affaires assez singulières\,;
la première qui tenait fort au cœur à M. le Prince entre lui et le
parlement de Dijon, qui venant le saluer à son arrivée, pour tenir les
états de Bourgogne, faisait marcher ses huissiers avec leurs baguettes
hautes dans le logis de M. le Prince, qui, de son côté, prétendait que,
représentant le roi dans la province dont il était gouverneur, les
baguettes des huissiers du parlement ne pouvaient entrer chez lui que
baissées. Cela fut ordonné ainsi, dont ce parlement fut fort mortifié.

L'autre paraissait tout à fait sans fondement. Mérinville, dont le père
était le seul lieutenant général de Provence, et qui fut chevalier de
l'ordre en 1661, avait été forcé par la ruine de ses affaires de vendre
à Samuel Bernard, le plus fameux et le plus riche banquier de l'Europe,
sa terre de Rieux qui est une des baronnies des états de Languedoc. Ces
états ne voulurent pas souffrir que Bernard prît aucune séance dans leur
assemblée, comme n'étant pas noble par lui-même, et incapable, par
conséquent, de jouir du droit de la terre qu'il avait acquise. Sur cela,
Mérinville prétendit demeurer baron des états de Languedoc sans terre,
comme étant une dignité personnelle. Il fut jugé qu'elle était réelle,
attachée à la terre, et Mérinville évincé avec elle de la qualité de
baron, et de tout droit de séance, et d'en exercer aucune fonction, sans
que pour cela l'incapacité personnelle de l'acquéreur fût relevée. Son
fils vient enfin de la racheter, malgré les enfants de Bernard, qui ont
été condamnés par arrêt de la lui rendre pour le prix consigné.

Bullion eut en même temps deux cent mille livres sur son gouvernement du
Maine et du Perche. Il était déjà assez étrange que son frère eût eu
l'agrément de l'acheter, et que celui-ci l'eût eu après sa mort, sans
donner à un homme si riche un brevet de retenue qui assurait presque ce
gouvernement à sa famille après lui.

Le cardinal de Janson arriva de Rome. Le roi lui fit mille amitiés qu'il
méritait bien, et lui fit prêter, le lendemain 14 juillet, le serment de
grand aumônier de France.

Des Forts, que nous verrons plus d'une fois figurer en premier en
finance, fils unique de Pelletier qui avait les fortifications, et qui
lui avait donné sa place d'intendant des finances, épousa à Montpellier
la fille de Bâville. Les Lamoignon crurent faire un grand honneur à la
fortune des Pelletier par cette alliance, qui parurent les croire sur
leur parole. On a vu, il n'y a pas longtemps, sur le premier président
Lamoignon, père de Bâville et du président à mortier, combien il y avait
peu qu'ils avaient quitté la plaidoirie et le barreau, où ils n'étaient
pas même anciens, pour entrer dans la magistrature.

Foucault, conseiller d'État, obtint la rare permission du roi de quitter
à son fils l'intendance de Caen, auquel on verra faire en son temps des
personnages dangereux et extravagants en France et en Espagne. Sans une
raison de cette nature, je ne m'amuserais pas à gâter mon papier de ces
bagatelles. Foucault, grand médailliste, était fort protégé du P. de La
Chaise, qui l'était aussi.

On sut que les Anglais avaient fait l'abbé de La Bourlie lieutenant
général dans leurs troupes, avec six mille livres de pension, et
vingt-quatre mille livres pour son équipage, et qu'ils l'avaient sur
leur flotte avec Cavalier, qui, à la fin, après avoir rôdé en France
depuis sa soumission et son accommodement, s'était donné à eux. J'ai
avancé, quoique de fort peu, quelques-unes de ces petites choses pour ne
les pas oublier et pour n'en pas interrompre de plus intéressantes,
qu'il faut maintenant raconter après avoir achevé encore quelques
bagatelles.

Le roi fut, si content du procédé du duc de Marlborough, à l'égard de
tous nos prisonniers, qu'il permit à sa prière que Vanbauze, qui avait
Reims pour prison, allât pour trois mois chez lui à Orange. On a vu en
son lieu que ce lieutenant général, et grand et bon partisan, avait été
pris en Italie. On était fort mécontent de sa conduite et de ses
discours, et le roi, qui eut peine à consentir à ce congé, le fit valoir
à Marlborough. En même temps Verbaum, premier ingénieur du roi
d'Espagne, fut mis dans la citadelle de Valenciennes, comme il allait se
rendre au camp des ennemis. On prit aussi quantité de faux sauniers en
divers endroits du royaume, qui marchaient armés par troupes, et
trouvaient partout protection pour cette contrebande. On en envoya
quantité aux îles d'Amérique.

Orry était arrivé à Versailles et y avait suivi Vaset et les pierreries
d'Espagne de fort près. C'était pour solliciter des secours d'argent
dans cette extrémité des affaires. Il vit longtemps le roi dans son
cabinet le 15 juillet. Mais dans les six semaines qu'il demeura ici sur
le pied de retourner en Espagne, Amelot et le duc de Berwick mandèrent
que la commotion y était si générale et si grande contre lui, qu'il
serait fort nuisible de l'y renvoyer. En effet ses hauteurs, sa dureté,
sa brutalité, sa grossièreté, le mensonge continuel dont, en toutes
sortes d'affaires, il faisait une profession ouverte, l'avaient, rendu
si odieux que personne ne voulait plus traiter avec lui. Il en avait usé
avec Amelot comme il avait fait avec Puységur, et son effronterie avait
si peu de bornes que le duc de Berwick m'a conté que ce qu'il lui
promettait pour le lendemain, et quelquefois pour deux heures après, ne
s'exécutait point, et qu'il niait de l'avoir promis, tellement que
Berwick, qui ne le voyait jamais que pour affaires indispensables, prit
enfin le parti de lui porter chaque demande sur du papier et de lui
faire écrire et signer au bas sa réponse. Avec cela encore il manquait
de parole. On lui rapportait le papier, il ne pouvait plus nier, mais
faisait la gambade et répondait qu'il n'avait pu résister au maréchal,
sachant bien qu'il ne, pourrait exécuter ce qu'il promettait. Avec cette
conduite, tout périssait, excepté sa bourse.

Quand il fut résolu qu'il ne retournerait point, il fut question de lui
faire rendre compte de deux millions comptants qu'il avait touchés ici
dans ces six semaines pour le payement des troupes en Espagne. Ce compte
fut tel que le roi le voulut faire pendre. Il en fut à deux doigts.
M\textsuperscript{me} de Maintenon, qui sentit combien cette catastrophe
porterait sur la protection que M\textsuperscript{me} des Ursins ne
cessait de lui donner, et sur l'intime liaison toujours subsistante
entre eux, détourna le coup par Chamillart, et fit si bien dans la
suite, toujours pour couvrir et soutenir M\textsuperscript{me} des
Ursins, qu'on lui donna pour le décrasser et le réhabiliter une charge
de président à mortier au parlement de Metz, qu'il garda pour ces mêmes
raisons, mais qu'il n'exerça point, parce qu'il ne savait mot de lois ni
de jurisprudence. Il a laissé deux fils qui sont sa vive image. Qui
croirait qu'en titre et en effet on les ait rendus les arbitres et les
maîtres des finances du roi et de la fortune de tous ses sujets\,?

Ce fut un coup hardi à Amelot, avec qui Orry était fort brouillé,
d'avoir empêché son retour, Mais la conduite, la capacité et la
réputation de ces deux hommes étaient si diamétralement opposées, l'un
en vénération et en amour à toute l'Espagne et aux troupes, l'autre en
dernière horreur, que M\textsuperscript{me} des Ursins n'osa se fâcher
pour cette fois, n'en vécut pas moins bien avec Amelot et avec Berwick,
alors tous deux si nécessaires, ne put pas même leur en savoir un trop
mauvais gré, et se rabattit à sauver son ami de la corde, pour sauver sa
propre réputation à elle-même.

Avant de rentrer à Madrid, et dès que le roi d'Espagne s'en revit le
maître, il jugea à propos de se délivrer de la reine douairière
d'Espagne, dont la conduite avait été plus que suspecte dans tous les
temps. Le roi, par la considération de la mémoire de Charles II qui
l'avait appelé à sa couronne par son testament, et duquel elle était
veuve, n'avait pas voulu lui faire éprouver les rigueurs de la retraite
dans un monastère sans y voir personne et sans en sortir, qui est la
destinée que l'usage d'Espagne impose aux reines veuves, lorsqu'un fils
sur le trône ne les en dispense pas par son autorité. Celle-ci n'avait
point d'enfants. Elle était sœur de l'impératrice veuve de l'empereur
Léopold, et mère de l'empereur Joseph et de l'archiduc. On a vu combien,
du vivant et dans les fins de Charles II, cette princesse était active
pour les intérêts de l'empereur, et intimement unie avec tous les
seigneurs espagnols attachés particulièrement à la maison d'Autriche.
Philippe V, qui avait raison de ne la pas laisser à Madrid, lui donna le
choix d'une autre demeure. Elle désira d'aller à Tolède dans le beau
palais que Charles-Quint y avait rétabli, et dont les superbes restes
font déplorer l'incendie qui le détruisit à la retraite des troupes de
l'archiduc de cette ville, un peu après ce temps-ci. La conduite de la
reine douairière n'avait pas démenti son inclination pendant cette
dernière prospérité de l'archiduc son neveu, tellement qu'une des
premières choses que le roi d'Espagne jugea à propos de faire aussitôt
son espèce de rétablissement fut de l'éloigner tout à fait. Il chargea
donc le duc d'Ossone, l'un de ses capitaines des gardes qui l'avait
toujours suivi, de prendre cinq cents chevaux, d'aller à Tolède, de voir
en arrivant la reine douairière, de lui dire que le roi d'Espagne la
trouvait là trop proche des armées pour y demeurer tranquillement, et
qu'il souhaitait que, sans aucun délai, elle allât trouver la reine à
Burgos. La reine douairière parut fort affligée et fort interdite de ce
compliment, chercha des excuses et des délais, mais le duc d'Ossone mêla
si bien la fermeté avec le respect qu'il ne lui donna que vingt-quatre
heures, au bout desquelles il la fit partir avec tout ce qu'elle avait
là autour d'elle, et au lieu de Burgos, la fit conduire à Vittoria.
Pendant ce voyage, on avait dépêché au roi pour avoir ses ordres sur le
lieu de la frontière et de France où on la mènerait. Pau fut choisi pour
la commodité et l'agrément du château et des jardins\,; mais la reine
douairière, informée enfin du lieu où elle allait, demanda Bayonne par
préférence et l'obtint. Le duc de Grammont qui y était lui céda sa
maison et la reçut avec toutes sortes d'honneurs. Elle y a passé plus de
trente ans. J'aurai occasion de parler d'elle dans la suite.

Fontaine-Martel était mort, mangé de goutte, ne laissant qu'une fille
encore enfant. Il était frère d'Arcy, dont j'ai parlé, qui avait été
gouverneur de M. le duc d'Orléans, et qui avait valu à Fontaine-Martel
la place de premier écuyer de M\textsuperscript{me} la duchesse
d'Orléans. Elle était obsédée des Saint-Pierre, et par eux toujours
aigrie sur celle des Suisses qu'avait eue Nancré. Ils firent tant auprès
d'elle qu'elle se fit une véritable affaire d'obtenir cette place de son
premier écuyer pour Saint-Pierre, et M. le duc d'Orléans la lui donna
pour avoir repos, à condition que Saint-Pierre ne se présenterait pas
devant lui. Quelque déshonorante que fût cette condition, Saint-Pierre
et sa femme n'étaient pas gens à lâcher prise. La place était utile et
pleine de commodités, elle honorait fort Saint-Pierre, elle lui donnait
un état de consistance qu'il n'avait pas\,; il la reçut donc avec
avidité et tint des propos et une conduite à l'égard de M. le duc
d'Orléans plus qu'indécents.

C'était un petit noble tout au plus, de basse Normandie, qui ne s'était
jamais assis devant la vieille duchesse de Ventadour, mère de la
maréchale de Duras, quand il allait lui faire sa cour à Sainte-Marie
dont il était voisin. Pour achever, il n'y eut manèges qu'il ne fît, et
chose qu'il ne mît en œuvre pour faire aller sa femme à Marly, et par
conséquent pour la faire manger, et entrer dans les carrosses.
M\textsuperscript{me} la duchesse d'Orléans le voulut prendre au point
d'honneur, à cause de la charge. On allégua l'exemple de
M\textsuperscript{me} de Fontaine-Martel qui y avait été admise sans
difficulté. Le roi tint bon toute sa vie, car ils ne se lassèrent point
d'y prétendre. Il répondit que, quand le premier écuyer de
M\textsuperscript{me} la duchesse d'Orléans serait un homme de qualité
comme l'était Fontaine-Martel, il savait la différence des domestiques
des petits-fils de France d'avec ceux des princes du sang\,; mais que,
pour un premier écuyer tel que Saint-Pierre, il était étonné que cela se
pût imaginer, moins encore proposer. Il n'y eut peut-être que les deux
dernières années de la vie du roi tout au plus que, rebutés cent et cent
fois, ils se le tinrent pour dit.

La Saint-Pierre se fourrait partout, divertissait le monde et soi-même
tant qu'elle pouvait, avec un air étourdi, mais point du tout méchante
ni glorieuse. Le mari était un faux Caton, bien glorieux, bien
présomptueux, bien insolent, jusqu'à ne prendre pas la peine de voir le
roi, de dépit de Marly, quoique ne bougeant de Versailles, méchant et
dangereux avec force souterrains, et un froid silencieux et indifférent
copié sur d'O, mais avec beaucoup d'esprit. Son nom était Castel. Les
trois tantes paternelles du maréchal de Bellefonds avaient épousé en
1642 {[}la première{]} un Castel\,; la seconde un Cadot, qui sont les
Sebeville\,; la troisième fut mère du maréchal de Villars. Voilà une
parenté médiocre. On sait en Normandie quels sont les Gigault\,; mais le
surprenant est que la mère de ces trois femmes était Aux Épaules, bonne
et ancienne maison éteinte, dont était aussi la mère de la duchesse de
Ventadour, mère de la maréchale de Duras, qui n'en rabattait rien pour
cela avec les Saint-Pierre.

S'il n'est pas temps encore de parler du personnel de M. le duc
d'Orléans, je ne puis différer de dire de quelle façon j'étais avec lui
depuis que j'étais entré dans son commerce, de la façon dont je l'ai
raconté en son lieu. L'amitié et la confiance pour moi était entière,
j'y répondis toujours avec le plus sincère attachement. Je le voyais
presque toutes les après-dînées à Versailles, seul dans son entresol. Il
me faisait des reproches quand le hasard rendait mes visites plus rares,
et il me permettait de lui en parler en toute liberté. Aucun chapitre ne
nous échappait, il se répandait sur tous avec moi, et il trouvait bon
que je ne lui cachasse rien sur lui-même. Je ne le voyais qu'à
Versailles et à Marly, c'est-à-dire à la cour, et jamais à Paris. Outre
que je n'y étais presque point, et que, quand j'y allais pour y coucher
une nuit, et rarement deux, c'était pour des devoirs ou des affaires\,;
ses compagnies, ses parties, la vie qu'il menait à Paris ne me convenait
point. Je m'étais mis tout d'abord sur le pied de n'avoir aucun commerce
avec personne du Palais-Royal, ni de ses compagnies de plaisir, ni avec
ses maîtresses. Je n'en voulus pas avoir davantage avec
M\textsuperscript{me} la duchesse d'Orléans que je ne voyais jamais
qu'aux occasions de cérémonie et de devoirs indispensables, fort rares,
et une minute, et je ne me mêlai jamais de quoi que ce fût de leurs
maisons. Je crus toujours qu'une autre conduite là-dessus me serait fort
importune, et ne me mènerait qu'à des tracasseries, de sorte que je n'en
voulus jamais entendre parler.

Le soir même qu'il fut déclaré général pour l'Italie, je le suivis du
salon chez lui, où nous causâmes longtemps tous deux. Il m'apprit qu'on
avait dépêché à Marsin, en Flandre, où il était encore avec ce qu'il
avait amené au maréchal de Villeroy, qui ne l'avait pas attendu pour sa
bataille, ordre de se porter sur-le-champ de sa personne sur le Rhin y
prendre le commandement de l'armée, et en même temps à Villars d'en
partir, et de sa personne aller par la Suisse à l'armée d'Italie, qu'il
commanderait sous lui, d'où M. de Vendôme ne devait point partir qu'ils
ne fussent arrivés l'un et l'autre, et n'eussent conféré avec lui, et
qu'il n'était général qu'à condition, pour ce commandement, de ne rien
faire que de l'avis du maréchal, et quoi que ce soit au contraire, dont
le roi en le nommant venait d'exiger sa parole. Il en sentit moins le
poids que la joie de se voir arrivé à ce qu'il avait tant désiré toute
sa vie, et sans l'avoir demandé, et lorsque depuis si longtemps il ne
l'espérait plus et n'y songeait plus. M. le prince de Conti se
contraignit, et fit fort bien le soir dans le salon.
M\textsuperscript{me} la Duchesse, qui y jouait, ne prit pas la peine de
quitter ni d'aller à M. le duc d'Orléans\,: elle lui cria, comme il
passait à portée, qu'elle lui faisait son compliment, d'un air piqué. Il
passa sans répondre. M. le Duc n'était pas encore de retour des états de
Bourgogne. Les jours suivants, M. le duc d'Orléans voulut que j'entrasse
avec lui en beaucoup de choses. Je crus ne pourvoir lui rendre un
meilleur service, à Chamillart et aux affaires, que de lui bien et
nettement dire l'obligation qu'il avait à Chamillart de le faire
servir\,; de lui bien faire entendre que, quelle que fût sa
disproportion d'avec lui, un ministre demeurait toujours le maître, et
faisait enrager les plus grands princes quand il voulait\,; que
l'honneur, la reconnaissance, l'intérêt de sa gloire et de ce qu'il
allait manier, exigeait entre eux un concert, une union, une franchise
entière sur tout, une exclusion de tout genre de fripons, qui, pour
pécher en eau trouble et pour leurs intérêts particuliers, voudraient
semer de la défiance et les éloigner l'un de l'autre. Je lui représentai
qu'il ne pouvait douter de Chamillart, du caractère droit et vrai dont
il était, qui l'ayant mis à la tête d'une puissante armée, ne tenant
qu'à lui de le laisser oisif comme il était, n'oublierait rien pour se
maintenir dans la bienveillance qu'il devait se promettre de ce
service\,; qu'une réflexion si naturelle le devait continuellement tenir
en garde contre ceux qui, sûrement ou jaloux ou ennemis de l'un et de
l'autre, voudraient lui grossir les soupçons, les mécontentements, le
chagrin, qui pouvaient naître avec le temps par le manquement
involontaire de beaucoup de choses, qui ne se faisait que trop sentir en
beaucoup d'occasions partout. Il reçut avec amitié et avec plaisir ces
considérations, m'expliqua fort au long ses instructions et ses ordres,
et m'ordonna de lui écrire souvent et librement sur lui-même.

Il était depuis longtemps amoureux de M\textsuperscript{lle} de Sery.
C'était une jeune fille de condition, sans aucun bien, jolie, piquante,
d'un air vif, mutin, capricieux et plaisant. Cet air ne tenait que trop
ce qu'il promettait. M\textsuperscript{me} de Ventadour, dont elle était
parente, l'avait mise fille d'honneur auprès de Madame\,; là elle devint
grosse, et eut un fils de M. d'Orléans. Cet éclat, la fit sortir de chez
Madame. M. le duc d'Orléans s'attacha à elle de plus en plus. Elle était
impérieuse et le lui fit sentir\,; il n'en était que plus amoureux et
plus soumis. Elle disposait de beaucoup de choses au Palais-Royal, cela
lui fit une petite cour et des amis\,; et M\textsuperscript{me} de
Ventadour, avec toute sa dévotion de repentie et ses vues, ne cessa
point d'être en commerce étroit avec elle, et ne s'en cachait pas. Elle
fut bien conseillée. Elle saisit ce moment brillant de M. le duc
d'Orléans pour faire reconnaître et légitimer le fils qu'elle en avait,
aujourd'hui par la régence de son père devenu grand prieur de France,
général des galères, et grand d'Espagne, avec des abbayes. Mais
M\textsuperscript{lle} de Sery ne se contenta pas de cette légitimation.
Elle trouva indécent d'être publiquement mère et de s'appeler
mademoiselle. Nul exemple pour lui donner le nom de Madame\,; c'était un
honneur réservé aux filles de France, aux filles duchesses femelles, et
depuis l'invention de Louis XIII que j'ai rapportée en son lieu, pour
M\textsuperscript{lle} d'Hautefort, aux filles dames d'atours. Ces
obstacles n'arrêtèrent ni la maîtresse ni son amant. Il lui fit don de
la terre d'Argenton, et força la complaisance du roi, quoique avec
beaucoup de peine d'accorder des lettres patentes portant permission à
M\textsuperscript{lle} de Sery de prendre le nom de madame et de
comtesse d'Argenton. Cela était inouï. On craignit les difficultés de
l'enregistrement. M. le duc d'Orléans, prêt à partir et accablé
d'affaires, alla lui-même chez le premier président et chez le procureur
général, et l'enregistrement fut fait. Son choix pour l'Italie avait été
reçu avec le plus grand applaudissement de la ville et de la cour. Cette
nouveauté ralentit cette joie et fit fort crier\,; mais un homme bien
amoureux ne pense qu'à satisfaire sa maîtresse et à lui tout sacrifier.

Tout se conçut, se fit et se consomma à cet égard sans que lui et moi
nous nous en dissions un seul mot. Je fus fâché de la chose, et qu'il
eût terni un départ si brillant par une singularité si bruyante et si
déplacée. Mais ce fut tout, et je me fus fidèle à ce que je m'étais
proposé, dès le moment que je rentrai en commerce avec lui, de ne lui
parler jamais de sa maison, de son domestique ni de ses maîtresses. Il
se doutait bien que je n'approuverais pas ce qu'il faisait pour
celle-là\,; il se garda bien de m'en ouvrir la bouche en aucun temps.

Mais voici une chose qu'il me raconta dans le salon de Marly, dans un
coin où nous causions tête à tête, un jour que, sur le point de son
départ pour l'Italie, il arrivait de Paris, dont la singularité vérifiée
par des événements qui ne se pouvaient prévoir alors m'engage à ne la
pas omettre. Il était curieux de toutes sortes d'arts et de sciences,
et, avec infiniment d'esprit, avait eu toute sa vie la faiblesse si
commune à la cour des enfants d'Henri II, que Catherine de Médicis avait
entre autres maux apportée d'Italie. Il avait tant qu'il avait pu
cherché à voir le diable, sans y avoir pu parvenir, à ce qu'il m'a
souvent dit, et à voir des choses extraordinaires, et savoir l'avenir.
La Sery avait une petite fille chez elle de huit ou neuf ans, qui y
était née et n'en était jamais sortie, et qui avait l'ignorance et la
simplicité de cet âge et de cette éducation. Entre autres fripons de
curiosités cachées, dont M. le duc d'Orléans avait beaucoup vu en sa
vie, on lui en produisit un, chez sa maîtresse, qui prétendit faire voir
dans un verre rempli d'eau tout ce qu'on voudrait savoir. Il demanda
quelqu'un de jeune et d'innocent pour y regarder, et cette petite fille
s'y trouva propre. Ils s'amusèrent donc à vouloir savoir ce qui se
passait alors même dans des lieux éloignés, et la petite fille voyait,
et rendait ce qu'elle voyait à mesure. Cet homme prononçait tout bas
quelque chose sur ce verre rempli d'eau, et aussitôt on y regardait avec
succès.

Les duperies que M. le duc d'Orléans avait souvent essuyées rengagèrent
à une épreuve qui pût le rassurer. Il ordonna tout bas à un de ses gens,
à l'oreille, d'aller sur-le-champ à quatre pas de là, chez
M\textsuperscript{me} de Nancré, de bien examiner qui y était, ce qui
s'y faisait, la position et l'ameublement de la chambre, et la situation
de tout ce qui s'y passait, et, sans perdre un moment ni parler à
personne, de le lui venir dire à l'oreille. En un tournemain la
commission fut exécutée, sans que personne s'aperçût de ce que c'était,
et la petite fille toujours dans la chambre. Dès que M. le duc d'Orléans
fut instruit, il dit à la petite fille de regarder dans le verre qui
était chez M\textsuperscript{me} de Nancré et ce qu'il s'y passait.
Aussitôt elle leur raconta mot pour mot tout ce qu'y avait vu celui que
M. le duc d'Orléans y avait envoyé. La description des visages, des
figures, des vêtements, des gens qui y étaient, leur situation dans la
chambre, les gens qui jouaient à deux tables différentes, ceux qui
regardaient ou qui causaient assis ou debout, la disposition des
meubles, en un mot tout. Dans l'instant M. le duc d'Orléans y envoya
Nancré, qui rapporta avoir tout trouvé comme la petite fille l'avait
dit, et comme le valet qui y avait été d'abord l'avait rapporté à
l'oreille de M. le duc d'Orléans.

Il ne me parlait guère de ces choses-là, parce que je prenais la liberté
de lui en faire honte. Je pris celle de le pouiller à ce récit et de lui
dire ce que je crus le pouvoir détourner d'ajouter foi et de s'amuser à
ces prestiges, dans un temps surtout où il devait avoir l'esprit occupé
de tant de grandes choses. «\,Ce n'est pas tout, me dit-il\,; et je ne
vous ai conté cela que pour venir au reste\,;» et tout de suite me conta
que, encouragé par l'exactitude de ce que la petite fille avait vu de la
chambre de M\textsuperscript{me} de Nancré, il avait voulu voir quelque
chose de plus important, et ce qui se passerait à la mort du roi, mais
sans en rechercher le temps qui ne se pouvait voir dans ce verre. Il le
demanda donc tout de suite à la petite fille, qui n'avait jamais ouï
parler de Versailles, ni vu personne que lui de la cour. Elle regarda et
leur expliqua longuement tout ce qu'elle voyait. Elle fit avec justesse
la description de la chambre du roi à Versailles, et de l'ameublement
qui s'y trouva en effet à sa mort. Elle le dépeignit parfaitement dans
son lit, et ce qui était debout auprès du lit ou dans la chambre, un
petit enfant avec l'ordre tenu par M\textsuperscript{me} de Ventadour,
sur laquelle elle s'écria parce qu'elle l'avait vue chez
M\textsuperscript{lle} de Sery. Elle leur fit connaître.
M\textsuperscript{me} de Maintenon, la figure singulière de Fagon,
Madame, M\textsuperscript{me} la duchesse d'Orléans,
M\textsuperscript{me} la Duchesse, M\textsuperscript{me} la princesse de
Conti\,; elle s'écria sur M. le duc d'Orléans\,: en un mot, elle leur
fit connaître ce qu'elle voyait là de princes et de domestiques,
seigneurs ou valets. Quand elle eut tout dit, M. le duc d'Orléans,
surpris qu'elle ne leur eût point fait connaître Monseigneur, Mgr le duc
de Bourgogne, M\textsuperscript{me} la duchesse de Bourgogne, ni M. le
duc de Berry, lui demanda si elle ne voyait point des figures de telle
et telle façon. Elle répondit constamment que non, et répéta celles
qu'elle voyait. C'est ce que M. le duc d'Orléans ne pouvait comprendre
et dont il s'étonna fort avec moi, et en rechercha vainement la raison.
L'événement l'expliqua. On était lors en 1706. Tous quatre étaient alors
pleins de vie et de santé, et tous quatre étaient morts avant le roi. Ce
fut la même chose de M. le Prince, de M. le Duc et de M. le prince de
Conti qu'elle ne vit point, et vit les enfants des deux derniers, M. du
Maine, les siens, et M. le comte de Toulouse. Mais jusqu'à l'événement
cela demeura dans l'obscurité.

Cette curiosité achevée, M. le duc d'Orléans voulut savoir ce qu'il
deviendrait. Alors ce ne fut plus dans le verre. L'homme qui était là
lui offrit de le lui montrer comme peint sur la muraille de la chambre,
pourvu qu'il n'eût point de peur de s'y voir\,; et au bout d'un quart
d'heure de quelques simagrées devant eux tous, la figure de M. le duc
d'Orléans, vêtu comme il l'était alors et dans sa grandeur naturelle,
parut tout à coup sur la muraille comme en peinture, avec une couronne
fermée sur la tête. Elle n'était ni de France, ni d'Espagne, ni
d'Angleterre, ni impériale. M. le duc d'Orléans, qui la considéra de
tous ses yeux, ne put jamais la deviner\,; il n'en avait jamais vu de
semblable. Elle n'avait que quatre cercles, et rien au sommet. Cette
couronne lui couvrait la tête.

De l'obscurité précédente et de celle-ci, je pris occasion de lui
remontrer la vanité de ces sortes de curiosités, les justes tromperies
du diable que Dieu permet pour punir des curiosités qu'il défend, le
néant et les ténèbres qui en résultent au lieu de la lumière et de la
satisfaction qu'on y recherche. Il était assurément alors bien éloigné
d'être régent du royaume et de l'imaginer. C'était peut-être ce que
cette couronne singulière lui annonçait. Tout cela s'était passé à Paris
chez sa maîtresse, en présence de leur plus étroit intrinsèque, la
veille du jour qu'il me le raconta, et je l'ai trouvé si extraordinaire
que je lui ai donné place ici, non pour l'approuver, mais pour le
rendre.

\hypertarget{chapitre-xiii.}{%
\chapter{CHAPITRE XIII.}\label{chapitre-xiii.}}

1706

~

{\textsc{Marsin, au refus de Villars, va commander l'armée d'Italie sous
M. le duc d'Orléans, qui part pour l'Italie.}} {\textsc{-
M\textsuperscript{me}s de Savoie, et incontinent après M. de Savoie,
sortis de Turin défendu par le comte de Thun.}} {\textsc{- Folles
courses de La Feuillade après le duc de Savoie.}} {\textsc{- Duc
d'Orléans passe au siège dont il est peu content.}} {\textsc{- Mauvaise
conduite de La Feuillade, fort haï.}} {\textsc{- Duc d'Orléans joint
Vendôme et n'en peut rien tirer.}} {\textsc{- Vendôme à Versailles.}}
{\textsc{- Vendôme part pour Flandre, avec une lettre du roi, pour
donner l'ordre et commander à tous les maréchaux de France.}} {\textsc{-
Villeroy à Versailles sans avoir vu Vendôme, et ne voit point
Chamillart, avec qui il se brouille, et tombe en disgrâce.}} {\textsc{-
Guiscard, sans lettre de service, retiré chez lui\,; seul sans nouvelles
lettres de service.}} {\textsc{- Puységur à Versailles et en Flandre.}}
{\textsc{- Traitement des ducs en pays étrangers.}} {\textsc{-
Usurpation de rang de l'électeur de Bavière.}} {\textsc{- Traitements
entre lui et M. de Vendôme.}} {\textsc{- Villars, quoique affaibli,
prend l'île du Marquisat, où Streff est tué.}} {\textsc{- Caraman
assiégé dans Ménin et le rend.}} {\textsc{- Jolie action du chevalier du
Rosel.}} {\textsc{- Ath pris par les ennemis.}} {\textsc{- Séparation
des armées en Flandre.}} {\textsc{- Le roi, amusé sur le voyage de
Fontainebleau, ne le fait point cette année.}} {\textsc{- Kercado,
maréchal de camp, tué.}} {\textsc{- Talon, Polastron, Rose, colonels,
morts en Italie, et le prince de Maubec, colonel de cavalerie.}}

~

On sut bientôt le changement qui regardait le commandement de l'armée
d'Italie sous M. le duc d'Orléans. Villars n'en voulut point tâter\,: il
ne s'accommoda point de prendre l'ordre de M. de Vendôme, et aussi peu
d'être sous un jeune prince. Il était parvenu aux richesses et aux plus
grands honneurs. Sans balancer, il leur remit le marché à la main, et
répondit tout net que le roi était le maître de lui ôter le commandement
de l'armée du Rhin, le maître de l'employer et de ne l'employer pas,
mais que d'aller en Italie il ne pouvait s'y résoudre, et qu'il
suppliait le roi de l'en dispenser. Un autre que l'heureux Villars eût
été perdu. De lui ou des conjonctures, tout fut trouvé bon. Le même
courrier lui fut renvoyé avec ordre de demeurer à la tête de son armée,
et un autre à Marsin portant, dès qu'il y serait arrivé (et qu'on ne
savait où prendre par les chemins), de s'en aller en Italie par la
Suisse, au lieu de Villars. Le roi exigea de M. le duc d'Orléans la même
parole à l'égard de celui-ci qu'il lui avait fait donner pour l'autre.
Il l'entretint longtemps à Marly, le mercredi matin, 30 juin. M. le duc
d'Orléans prit congé et s'en alla à Paris, d'où il partit le lendemain
avec vingt-huit chevaux et cinq chaises pour arriver en trois jours à
Lyon, et pousser de là, sans s'arrêter, en Italie.

M\textsuperscript{me}s de Savoie sortirent de bonne heure de Turin et se
retirèrent à Coni. M. de Savoie reçut assez mal les offres de sûreté
pour tous les lieux où elles voudraient aller, que La Feuillade lui
envoya faire de la part du roi. Il répondit sèchement qu'elles étaient
bien où elles étaient. Lui-même quitta Turin à la fin de juin. Il en
laissa le commandement au comte de Thun, qui ne s'en acquitta que trop
bien, et qui longtemps depuis a été gouverneur du Milanais. M. de Savoie
emmena toute sa cour, ses équipages et ses trois mille chevaux, et n'y
en laissa que cinq cents et vingt hussards. Il se mit à courir le pays
dans l'opinion que La Feuillade le suivrait et se distrairait du siège
pour tâcher de le prendre. C'est en effet ce qui arriva. Il laissa le
commandement du siège à son ami Chamarande, qui fut sa dupe toute sa
vie, et se mit aux champs. Il alla s'amuser devant Quérasque, et envoya
d'Estaing prendre Asti qui, depuis la méprise de son secrétaire, était
demeuré aux ennemis, et oh lui-même avait échoué, comme on l'a vu
ci-devant.

Avec ces détachements, il ne restait que quarante bataillons devant
Turin, qui y fatiguaient fort et y avançaient fort peu. On prit
prisonniers dans Mondovi le prince de Carignan, ce fameux muet, et toute
sa famille\,; et sur sa parole, on les conduisit à Raconis, sa maison de
plaisance, où il demanda une garde à La Feuillade. En même temps
M\textsuperscript{me}s de Savoie, qui de Coni étaient allées à Oneille,
se retirèrent à Savane. La Feuillade, lassé de perdre son temps à courre
après du vent, revint au siège et lâcha Aubeterre aux trousses de M. de
Savoie, qui, pour ralentir le siège, se montrait de loin, puis se
cachait et changeait continuellement de retraite et de route. Il pensa
pourtant plus d'une fois y être attrapé, et cependant menait une vie
errante, misérable et périlleuse. Aubeterre battit son arrière-garde et
prit un fils du comte de Soissons, un capitaine des gardes de M. de
Savoie et une vingtaine d'officiers. Là-dessus La Feuillade, follement
buté à la capture de M. de Savoie, et qui n'en voulait pas laisser
l'honneur à un autre, quitta encore le siège et se remit après\,; mais
M. de Savoie se moquait de lui. Ce prince ne laissa pas de se trouver
longtemps dans les plus fâcheuses extrémités qu'il soutint avec un grand
art et un grand courage. Cette conduite de La Feuillade harassa toute sa
cavalerie, et mit à bout son infanterie, par tous les divers
détachements qu'il en fit à droite et à gauche, et par la fatigue trop
redoublée de celle qui restait au siège. C'était une étrange folie que
voler le papillon aux dépens de l'objet si principal de prendre Turin,
et si pressé qu'une heure était précieuse dans la crainte de l'arrivée
du prince Eugène, à qui ces lenteurs donnèrent tout le temps qui lui fut
nécessaire\,; et la négligence, la paresse, l'opiniâtreté, l'incurie de
M. de Vendôme pour un pays qu'il allait quitter, toutes les facilités
dont il sut bien profiter pour passer le Pô malgré lui, et lui donner le
second tome de M. de Staremberg, et par le même chemin qu'il vint au
secours de M. de Savoie, quoique fort arriéré, et toutes les rivières
gardées, les passa et devança M. de Vendôme qui revenait de cette belle
course de Trente, et arriva à temps de sauver M. de Savoie, comme je
l'ai marqué en son temps.

On avait beau presser le siège par des courriers redoublés, le temps
perdu ne se pouvait regagner\,; et Chamillart fut obligé de mander à son
gendre le mauvais effet de ses courses par monts et par vaux après un
fantôme qui ne se montrait que pour le séduire et qui lui échappait
toujours. Personne n'osait dire un mot de ce qu'il pensait à La
Feuillade. Dreux, son beau-frère, y fut si mal reçu qu'il ne s'y commit
plus, et il s'en brouilla avec Chamarande qui, comptant sur l'âge,
l'expérience et l'ancienne amitié, s'était hasardé de lui dire tête à
tête sa pensée avec grande mesure\,; sa sagesse et sa douceur évita
l'éclat et le dehors, mais on s'aperçut bientôt du refroidissement qui
ne se raccommoda plus. Le pauvre Chamarande y perdit son fils à la tête
du régiment de la reine que lui-même avait eu avant lui.

M. le duc d'Orléans passa au siège. La Feuillade le reçut magnifiquement
et lui montra tous les travaux. Il le mena aux attaques et lui fit tout
voir. Le prince ne fut content de rien. Il trouva qu'on n'attaquait
point par où il aurait voulu, et fut en cela de même avis que Catinat
qui connaissait si bien Turin, que Vauban qui l'avait fortifié, que
Phélypeaux qui y avait demeuré des années, et tous trois sans s'être
concertés. Il ne le fut pas davantage des travaux, et il trouva le siège
fort peu avancé. Il ménagea pourtant fort La Feuillade, mais il ne crut
pas lui devoir sacrifier le succès. Il fit donc changer et ordonna le
changement de beaucoup de choses\,; mais, dès qu'il fut parti, La
Feuillade remit tout, de son autorité, en son premier état, continua de
pousser sa pointe, et toujours sans consulter qui que ce fût, depuis le
commencement jusqu'à la fin. Sa conduite impérieuse, le peu d'accès
qu'il donnait auprès de lui, sa hauteur avec les officiers, même
généraux, et ses propos durs avec l'audace d'un étourdi qui compte
éblouir par sa valeur et tout permis au gendre du tout-puissant
ministre, le firent détester de toute son armée, et mirent les officiers
généraux et particuliers en humeur et en usage de s'en tenir exactement
et avec précision à leur fait et à leur devoir, sans se soucier de la
besogne ni daigner remédier, ni rien faire, sur quoi que ce fût, à rien,
quelque nécessité qu'ils y vissent, par pique, par dégoût, et par la
crainte aussi qu'on leur demandât de quoi ils se mêlaient. Avec un tel
général, qui avait mal enfourné, qui manquait par l'impossibilité de ce
que Vauban avait cru nécessaire, et secouru de la sorte, ce n'était pas
de quoi prendre Turin. On prit de temps en temps quelques ouvrages
extérieurs, dont les nouvelles venues par des courriers étaient bien
vantées à la cour et faisaient sans cesse tout espérer. Mais nos mines
allaient si mal, que La Feuillade s'en plaignait lui-même par ses
lettres, et l'artillerie y était servie avec les mêmes défauts et par
les mêmes raisons qu'elle l'avait été à Barcelone, et que j'ai
expliquées sur ce siège-là.

M. le duc d'Orléans joignit M. de Vendôme sur le Mincio, le 17 juillet,
avec lequel il conféra tant qu'il put, non pas à beaucoup près tant
qu'il voulut, moins encore autant qu'il était nécessaire. Le prétendu
héros venait de faire des fautes irréparables. Le prince Eugène venait
de passer le Pô presque devant lui\,; on ignorait ce que seraient
devenus douze de nos bataillons postés au delà du Pô, près de l'endroit
où il avait passé\,; il avait pris tous les bateaux que nous avions sur
ce fleuve, et il fallait pourtant en faire un pont pour passer l'armée
et suivre les ennemis. Vendôme craignit donc que ses fautes ne fussent
aperçues. Il voulait que son successeur en demeurât chargé. D'autre part
il attendait Marsin. Son orgueil le retenait pour le plaisir de donner
l'ordre à un maréchal de France, et jouir du billet du roi qu'il avait
obtenu. En cette situation, impatient, fuyant les conférences, les
abrégeant quand il ne pouvait les éviter, il ne put éviter le perçant
des yeux du prince qui s'appliquait à pénétrer l'état d'une besogne qui
devenait sienne et qui désormais intéressait son honneur. Il acheva sur
les lieux de découvrir à revers tout ce qu'il avait déjà aperçu en
éloignement, et y ajouta beaucoup d'autres connaissances qu'il ne
dissimula point, quoique avec modestie, et sur lesquelles Vendôme ne put
rien alléguer de bon ni même d'apparent. Enfin Marsan arriva, et, sa
dignité flétrie, Vendôme partit sans délai.

Aussitôt après, M. d'Orléans tenta un petit combat avec Médavy par un
autre côté, qui aurait déconcerté la marche des ennemis, et qui eût
infailliblement réussi, si Goïto ne se fût pas misérablement rendu au
moment que Marsin y allait lui-même pour le dégager. L'affaire manquée,
M. d'Orléans alla en poste rejoindre M. de Vendôme, arrêté, de concert
avec lui, à Mantoue, pour y donner des ordres dont ils étaient convenus.
Cette course fut pour lui proposer de faire descendre un pont à Crémone,
qu'à son insu il avait commandé et fait rassembler. Il n'y avait que peu
de troupes ennemies qui eussent encore passé le Pô. Malgré les plus
opiniâtres assurances de Vendôme, leur armée avait rendu inutiles les
obstacles qu'il avait cru mettre à toutes les rivières. Elles les
avaient passées, et même le canal Blanc pour gagner le Piémont. En vain
M. d'Orléans voulut-il persuader cette vérité à M. de Vendôme, et qu'ils
passeraient le Pô avec la même facilité\,; Vendôme, plus ferme que
jamais, n'y voulut jamais entendre. Il savait bien que tant qu'il était
en Italie, il y était le maître, et qu'à l'ordre près qu'il recevait du
prince, celui-ci était engagé au roi de ne décider de rien.

Comme ils en étaient sur cette dispute, il leur arriva des nouvelles
d'un parti qu'ils avaient sur les ennemis. Elles portaient qu'un petit
parti ennemi avait passé le Pô. Là-dessus Vendôme s'écrie que pour cinq
ou six coquins ce n'était pas merveilles. Comme il triomphait ainsi,
autres nouvelles, coup sur coup, du même partisan, qui mandait que toute
l'armée avait passé. Vendôme, qui venait d'assurer qu'elle ne s'y
hasarderait pas, paya de son effronterie ordinaire, et avec un air
également gai et libre, et ce front qui ne rougissait de rien\,: «\,Eh
bien\,! dit-il, ils sont passés, je n'y puis que faire\,; ils ont bien
d'autres obstacles à surmonter avant de se rendre en Piémont.\,» Et tout
de suite se tournant à M. le duc d'Orléans\,: «\,Vos ordres, lui dit-il,
monsieur, car je n'ai plus que faire ici, et je pars demain matin.\,» Il
tint parole. M. d'Orléans, confus pour Vendôme, ne voulut pas ajouter
les reproches à ceux de la chose même. Il se contenta de lui dire que,
puisqu'il l'avait si opiniâtrement jeté dans cet extrême inconvénient,
en soutenant toujours ce passage impossible et le laissant ouvert, il
devait bien au moins l'aider à s'en tirer avant que s'en aller. À force
de persécution il accorda vingt-quatre heures, qui furent employées à
visiter des postes et à donner divers ordres. Les vingt-quatre heures
expirées, rien ne put retenir Vendôme. Il s'en fut au plus vite,
laissant au duc d'Orléans à soutenir tout le poids de ses lourdes
fautes. Toute l'armée en était témoin, et plusieurs officiers généraux
de ce qui se venait de passer en dernier lieu. M. d'Orléans, qui
connaissait le terrain, se garda bien de tomber sur Vendôme dans ses
dépêches, mais il ne pallia point aussi la situation critique dans
laquelle il le laissait. Il attendit à Mantoue La Feuillade pour
s'aboucher avec lui sur les partis et les mesures à prendre, et les
troupes qu'il pourrait lui envoyer de son siège.

Vendôme arriva le samedi dernier juillet à Versailles. Il salua le roi à
la descente de son carrosse. Il fut reçu en héros réparateur\,; il
suivit le roi chez M\textsuperscript{me} de Maintenon, où il demeura
longtemps avec lui et Chamillart. Il y vanta le bon état où il avait
laissé toutes choses en Italie avec une audace sans pareille, et assura
que le prince Eugène ne pourrait jamais secourir Turin. Le dimanche il
fut voir Monseigneur à Meudon, et travailla après longtemps chez
Chamillart. Le lundi 2 août, M. de Vendôme fut longtemps seul avec le
roi dans son cabinet. Il en reçut une lettre de sa main, portant ordre à
tous les maréchaux de France de prendre l'ordre de lui, et de lui obéir
partout. C'est où M. du Maine et lui en voulaient venir sans patente, et
où ils arrivèrent enfin par degrés, contre le goût et la volonté du
roi\,; et de cette sorte sans patente, M. de Vendôme, quoique sans
mention de sa naissance, fut mis en parfait niveau avec les princes du
sang. Il prit congé transporté d'aise, s'en alla coucher à Clichy, d'où
il partit le lendemain pour Valenciennes. Le maréchal de Villeroy, qui
s'était tenu fort obscurément à Saint-Amand, reçut en même temps son
congé, et partit aussitôt pour revenir. Il ne vit ni ne rencontra M. de
Vendôme.

Ce retour fut bien différent de ceux de toutes les précédentes années.
Il arriva à Versailles le vendredi 6 août, et vit le roi chez
M\textsuperscript{me} de Maintenon\,; cela fut court et sec.~Il obtint
sans peine de différer quelques jours à prendre le bâton, sur ce que son
équipage n'était pas arrivé, et qu'il avait beaucoup d'affaires. Il
était dans son quartier de capitaine des gardes. Il s'en retourna
promptement à Paris, ne vit point Chamillart, et acheva de gâter ses
affaires par se plaindre hautement de lui. Ce n'était plus le temps où
le langage, les grands airs et les secouements de perruque passaient
pour des raisons, la faveur qui soutenait ce vide était passée.
Chamillart n'était pas cause qu'il eût formellement désobéi aux ordres
réitérés de ne se commettre à rien avant la jonction de Marsin\,; ce
n'était pas lui qui lui avait fait choisir un si étrange terrain pour
combattre et si connu pour tel\,; qui lui avait fait faire une
disposition si étrange\,; qui lui avait tourné la tête ensuite, et qui
lui avait fait abandonner toute la Flandre par une terreur panique, que
rien ne put rassurer, pour quatre mille hommes perdus en tout et pour
tout à Ramillies. Ses clameurs ne furent écoutées que de quelques amis
particuliers par compassion plus que par persuasion. Personne ne se
voulut brouiller avec Chamillart pour un général en disgrâce en si
lourde faute.

Villeroy, déchu de sa faveur et du commandement des armées, perdit toute
l'écorce qui l'avait fait briller, et ne montra plus que le tuf.
L'abattement, l'embarras succéda aux grands airs et aux sons des grands
mots. Son quartier lui fut pesant à achever. Le roi ne lui parlait que
pour donner l'ordre et pour des choses de sa charge. Il pesait au roi,
il le sentait, et plus encore que chacun s'en apercevait. Il n'osait
ouvrir la bouche, il ne fournissait plus à la conversation, il ne tenait
plus le dé. Son humiliation était marquée dans toute sa contenance\,; ce
n'était plus qu'un vieux ballon vidé, dont tout l'air qui l'enflait
était sorti. Dès que son quartier fut fini, il s'en alla à Paris et à
Villeroy, et jusqu'à ce qu'il recommençât l'année suivante, on le vit
très rarement et très courtement à la cour, où le roi ne lui disait pas
un mot. M\textsuperscript{me} de Maintenon en eut pitié, mais ce fut
tout jusqu'au temps où elle crut en avoir affaire. Il la voyait pourtant
chez elle quand il venait à Versailles\,; cette petite distinction le
soutenait à ras de terre.

Il n'est pas temps de s'étendre davantage sur ce roi de théâtre. Il eut
un autre dégoût. Guiscard était son protégé\,; il était beau-frère de
Langlée, qui ne bougeait à la cour de chez M. le Grand, et chez qui le
maréchal de Villeroy et la meilleure compagnie était tous les jours à
Paris en fêtes et au plus gros jeu du monde. Par le changement de
général, il fallut à tous les officiers généraux de nouvelles lettres de
service\,; Guiscard, premier lieutenant général de l'armée de Flandre,
fut le seul qui n'en eut point. On prétendait que la tête lui avait
tourné à Ramillies et depuis, comme au maréchal. Cette disgrâce porta à
plomb sur ce dernier, qui, ne pouvant se justifier ni se soutenir
lui-même, ne put être d'aucun secours à son ami. Guiscard, se voyant
sans emploi à l'armée, prit le parti de s'en venir chez lui à Magny,
terre qu'il avait achetée en Picardie de la succession du duc de
Chaumes, qu'il avait fort ajustée, et à qui il avait fait donner le nom
de Guiscard. Il y fut plusieurs mois solitaire, et obtint enfin une
audience du roi, pour laquelle il arriva de chez lui. Elle fut courte et
sèche, et tout aussitôt il retourna d'où il était venu, où il demeura
encore fort longtemps.

Le roi avait fait revenir Puységur d'Espagne, où il s'accommodait
médiocrement du droit et du sec d'un général qu'il avait vu longtemps
lui faire presque sa cour en Flandre, tandis qu'il faisait tout dans
l'armée sous M. de Luxembourg. Le roi l'entretint longtemps et le
renvoya en Flandre.

M. de Vendôme, en partant de Paris pour Valenciennes, avait écrit à
l'électeur de Bavière qu'il attendrait là ses ordres pour l'aller
trouver où il lui manderait. Le roi était convenu avec lui de la manière
dont il vivrait avec M. de Vendôme, duquel la naissance lui était plus
chère que les rangs de son royaume.

Les généraux en chef des armées du roi, lorsqu'ils étaient maréchaux de
France et qu'ils avaient vu des électeurs ou leur avaient écrit, ne leur
avaient jamais dit ni écrit que \emph{monsieur}. Ils avaient eu la main
chez eux et un siège égal, leur avaient donné l'\emph{altesse
électorale} et reçu l'\emph{excellence}. Villars n'en sut pas tant et
vécut avec l'électeur de Bavière comme s'il n'eût pas été maréchal de
France\,: de la cour on ne songea pas à l'en avertir. Marsin, après lui,
en usa de même\,; Tallard aussi, pour le peu de temps qu'il y fut. Le
mal venait de plus loin. Boufflers en Flandre avait tout gâté le
premier\,: non seulement il était maréchal de France et général d'armée,
mais il était duc. Jamais avant lui aucun duc n'avait vécu avec les
électeurs qu'en égalité entière. La main, sièges égaux, service égal à
table, la main chez eux et partout les mêmes honneurs. Le
\emph{monseigneur} à dire et à écrire jamais imaginé, \emph{altesse
électorale} rarement, \emph{excellence} de même.

Ces faits ne sont pas douteux\,; on en voit des restes dans les
\emph{Voyages} de Montconis, qui conduisit le duc de Chevreuse, fils du
duc de Luynes en quelques-uns. Il remarque cette égalité parfaite à
Heidelberg\,; qu'à la vérité l'électeur palatin se tint au lit se
prétextant malade, apparemment pour éviter la main\,; mais il donna à
dîner dans son lit au duc de Chevreuse, traité et servi comme
l'électeur, les mêmes honneurs militaires et civils qu'à l'électeur à
son arrivée et dans tout le traitement de son séjour, et le prince
électoral lui faisant les honneurs partout à la place de son père. Ces
\emph{Voyages} où cela est bien exprimé sont entre les mains de tout le
monde. Il remarque aussi que le peu des autres électeurs dans les États
desquels ils passèrent y firent rendre au duc de Chevreuse toutes sortes
d'honneurs, mais s'absentèrent, en sorte qu'avec des prétextes et des
excuses, ils évitèrent de le voir. Il n'y avait que la main qui les
tînt, et ne faisaient point de difficulté sur le reste.

Celle de la main était nouvelle, j'en expliquerai la raison dans un
moment. Le duc de Rohan-Chabot, qui fut depuis gendre de M. de Vardes,
alla voyager fort jeune. Sur le point de partir, M. de Lyonne, ministre
et secrétaire d'État des affaires étrangères, lui envoya un compliment
d'excuse, et le prier de passer chez lui. M. de Rohan y fut. M. de
Lyonne lui dit que le roi ne le voulait pas laisser partir sans une
instruction sur sa conduite à l'égard des princes chez lesquels il
passerait, et qu'il s'étonnait que lui, ou les personnes qui le
conduisaient, n'y eussent pas songé eux-mêmes. Il l'avait faite, et la
lui remit signée de lui. Elle portait ordre du roi de ne voir aucun
électeur qu'avec la main, et l'égalité entière pour toutes sortes
d'honneurs chez eux, à plus forte raison tous les autres princes,
excepté le seul duc de Savoie, duquel il prétendrait toutes les mêmes
choses que des électeurs, excepté la main. C'était une déférence
nouvelle, que le roi voulut bien accorder aux alliances si proches, et à
la réputation de tête couronnée, dont ses ambassadeurs obtinrent une
grande partie du rang, et l'eurent enfin entier partout bien des années
avant la personne de leur maître. En effet, le duc de Rohan eut tout à
Turin sans ménagement et sans la moindre difficulté, excepté la main\,;
en tout le reste, égalité entière de siège, du traitement et du service
à table, et de tous les autres honneurs. Il commença par l'Italie. La
vérité est que les électeurs évitèrent de le voir comme ils firent pour
M. de Chevreuse. Ils étaient en prétention et en usage de précéder les
ducs de Savoie\,; ils ne voulurent pas être moins distingués que lui, et
c'est ce qui forma leur difficulté de continuer à donner la main aux
ducs. M. de Savoie, plusieurs années avant qu'être roi de Sicile, et
enfin de Sardaigne, par la paix d'Utrecht, passa un carnaval à Venise,
où se trouva aussi l'électeur de Bavière, père de celui-ci, qui le
précéda toujours. M. de Savoie en voulut faire difficulté d'abord, il en
obtint le réciproque d'\emph{altesse royale} pour l'\emph{altesse
électorale}, que l'électeur ne lui avait pas voulu accorder, et avec
cette bagatelle se trouva partout avec l'électeur, et lui céda partout.
Dès lors pourtant les ambassadeurs de Savoie avaient partout le rang
d'ambassadeurs de tête couronnée.

Pour revenir donc à ce dont ces remarques nécessaires m'ont écarté, la
légèreté française, et le peu d'état que les ministres postérieurs du
roi lui avaient appris à faire des rangs de son royaume, et l'ignorance
où les plus intéressés sont en possession de vivre là-dessus, fit que
ces maréchaux, et Boufflers même duc, laissèrent prendre à l'électeur de
Bavière tout ce qu'il voulut, et sans y songer le traitèrent de
\emph{monseigneur} comme ses sujets faisaient, et à leur exemple fort
sottement nos troupes. Le maréchal de Villeroy, aussi léger qu'eux, mais
plus instruit, n'avait pas songé à la manière dont ils vivaient avec
l'électeur\,; quand il eut à y vivre lui-même, et qu'il fut arrivé, il
se trouva étrangement scandalisé. Il dépêcha un courrier au roi, qui fit
visiter les dépêches anciennes et les registres. Il trouva que le
maréchal de Villeroy avait raison, mais en même temps, embarrassé d'un
changement si marqué après l'exemple des autres, il se persuada que le
temps où l'électeur venait de perdre ses États par sa fidélité dans son
alliance n'était pas celui de mortifier son usurpation sur son rang. Il
sacrifia celui des ducs et des généraux de ses armées, maréchaux de
France, à cette idée de générosité, et Villeroy eut ordre de ne rien
prétendre et de ne rien innover. Pour Vendôme, M. du Maine y prit
d'autant plus garde, qu'il le voulait à toutes mains distinguer de tout
ce qui n'était pas prince du sang. Le roi fit donc convenir l'électeur
que Vendôme ne lui dirait et ne lui écrirait que \emph{monsieur}, et que
partout leurs sièges seraient égaux, que Vendôme prendrait toujours
l'ordre de lui. Tout le reste fut abandonné, en sorte que Vendôme même
eut beaucoup moins que n'avaient les ducs avec les électeurs avant
l'usurpation de l'électeur de Bavière, et la sottise et l'ignorance de
ceux sur lesquels il la fit. Il ne donna point d'\emph{altesse} à
Vendôme, lequel aussi ne voulut point d'\emph{excellence}, et donna
toujours l'\emph{altesse électorale}. Nous verrons dans peu jusqu'à quel
point cet abandon du rang des ducs avec les électeurs porta sur la
dignité du roi même et de sa couronne.

On fit venir en Flandre un gros détachement de l'armée du maréchal de
Villars, qui le trouva fort mauvais, fit raser les lignes de la Lauter,
et raccommoder celles de la Mutter. Il se plaignit de la faiblesse où on
le laissait, et qu'il arrivait tous les jours de nouvelles troupes au
prince Louis de Bade. Il ne laissa pas de s'emparer de l'île dite du
Marquisat, au delà du fort Louis, et d'y établir un pont qui communique
du fort à l'île. Streff, maréchal de camp fort estimé, fit et lui
proposa ce projet. Il y fut tué sur un bateau où il voulut être, quoique
le maréchal s'y opposât, parce que cette attaque se faisait avec trop
peu de troupes pour un maréchal de camp\,; ce fut grand dommage. On y
perdit près de deux cents hommes, et les ennemis beaucoup plus.

Caraman avait été mis dans Menin pour le défendre, avec douze bataillons
de vieilles troupes, deux nouveaux, et un régiment de dragons, la
plupart à pied. Spaar, maréchal de camp, mort depuis sénateur de Suède,
et fort bon officier général, y était sous lui, et pour brigadier
Beuzeval, capitaine suisse, qui a depuis négocié avec réputation en
Pologne et dans le nord longtemps, y épousa une parente de la reine, et
est mort longtemps depuis lieutenant général et colonel du régiment des
gardes suisses, homme à deux mains, d'esprit, de manège et de tête.
Beully, qui avait été dans la gendarmerie et qui avait acheté ce
gouvernement de la famille de Pracontal, y était avec eux, et sous eux,
tout gouverneur qu'il était\,; malgré ce dégoût, il y demeura et y fit
fort bien. Ils tinrent trois semaines de tranchée ouverte, obtinrent une
très honorable capitulation, sortirent le 25 août, et furent conduits à
Douai. M. de Vendôme voulut rassembler son armée, mais il ne tarda pas à
la remettre comme avait fait le maréchal de Villeroy. Il se tint
cependant à Lille, puis à Saint-Amand sous prétexte de prendre des eaux.
Il sut que Marlborough avait projeté un grand fourrage auprès de
Tournai. Vendôme en avertit le chevalier du Rosel, qui était à Tournai.
En effet, le 16 août, huit mille hommes bordèrent un ruisseau qui tombe
dans l'Escaut, et s'appelle Chin, qu'il fit passer à douze cents
chevaux. Du Rosel sortit aussitôt avec neuf escadrons de carabiniers et
quatre-vingts dragons, passa à la tête du ruisseau hors du feu de cette
infanterie, battit les douze cents chevaux qui étaient en diverses
troupes, en tua deux cents, en prit deux cent cinquante, emmena à
Tournai quatre cents chevaux de ce fourrage, et parmi les prisonniers
Cadogan, favori de Marlborough et lors brigadier de cavalerie, qui, pour
favoriser la retraite de ce général qui se trouvait s'être trop avancé,
fit ferme tant qu'il put avec cinquante dragons à la tête d'un pont. M.
de Vendôme renvoya aussitôt Cadogan au duc de Marlborough galamment sur
sa parole. L'action de du Rosel fut vive et bien entendue, mais ce fut
aussi à quoi se bornèrent les exploits du nouveau général, qui, loin de
réparer ou de soutenir les affaires de Flandre, y vit de ses places
promener les ennemis de tous côtés, et prendre ce qui fut à leur
convenance. Ils finirent par le siège d'Ath, qu'ils prirent le 3
octobre, et les cinq bataillons qui étaient dedans prisonniers de guerre
après trois semaines de tranchée ouverte, et dix jours après, les armées
se séparèrent en Flandre, et la campagne finit.

Le roi comptait sur le voyage de Fontainebleau. M\textsuperscript{me} la
duchesse de Bourgogne était grosse et y devait aller en bateau. Ce
voyage déplaisait fort aux médecins, et bien autant à Chamillart, fort
court et fort pressé de dépenses indispensables, qui regrettait avec
raison celle de ce voyage qui était toujours grande.
M\textsuperscript{me} de Maintenon, pressée de ces deux côtés, résolut
d'amuser le roi, de retarder le voyage, enfin à l'extrémité de le
rompre. Sur les fins la plupart des gens instruits comprirent qu'il
était rompu. Le roi ne s'en doutait pas le moins du monde. Il avait été
reculé à deux reprises\,; il devait partir de Meudon\,; il alla voir de
ce lieu l'église nouvelle des Invalides qui fut fort admirée, où le
cardinal de Noailles officia devant lui, et donna ensuite à dîner à Mgr
le duc de Bourgogne, qui alla faire ses prières à Notre-Dame et à
Sainte-Geneviève, et voir ensuite la Sorbonne où il fut reçu par
l'archevêque de Reims, proviseur. Le lendemain de cette visite de
l'église des Invalides, Clément, soutenu de Fagon, déclara au roi que
M\textsuperscript{me} la duchesse de Bourgogne ne pouvait aller à
Fontainebleau sans se mettre en plus évident hasard. Cela fâcha fort le
roi, il disputa, les autres étaient bien instruits, il n'y gagna rien.
Avec dépit il décida qu'au lieu d'aller le lendemain à Fontainebleau, il
retournerait à Versailles, que Monseigneur et M\textsuperscript{me} la
princesse de Conti iraient à Fontainebleau, que lui-même y ferait un
voyage de trois semaines, et parut chagrin quelques jours. On le laissa
se repaître de ce voyage de trois semaines, on le recula, et enfin on le
rompit comme on avait fait le grand, mais sous prétexte que ce n'était
pas la peine pour si peu. Il n'y eut donc que Monseigneur qui vit
Fontainebleau cette année, et sa petite cour, où M. le duc de Berry le
fut voir et chasser. Ils n'osèrent y demeurer longtemps et s'en
revinrent auprès du roi.

Kercado, maréchal de camp, fut tué devant Turin. Polastron, fils du
lieutenant général, dont j'ai parlé de la mort naguère, et qui était
colonel de la couronne, Talon, fils et père des deux présidents à
mortier, et Rose, tous deux colonels, y moururent. Ce dernier était
petit-fils de Rose, secrétaire du cabinet, dont j'ai parlé en son lieu,
et laissa plus d'un million à sa sœur, femme de Portail, mort longtemps
depuis premier président. Pluveaux, maître de la garde-robe de M. le duc
d'Orléans, y mourut aussi de maladie peu de jours après, et quantité de
subalternes et d'anciens et bons officiers qui menaient les corps. Le
prince de Maubec, fils du prince d'Harcourt qui depuis un an avait un
régiment de cavalerie, mourut aussi à Guastalla\,; il n'était point
marié.

\hypertarget{chapitre-xiv.}{%
\chapter{CHAPITRE XIV.}\label{chapitre-xiv.}}

1706

~

{\textsc{M. le duc d'Orléans, sous la tutelle de Marsin, empêché par lui
d'arrêter le prince Eugène au Taner\,; chiffres.}} {\textsc{- Armée de
M. le duc d'Orléans à Turin.}} {\textsc{- Mauvais état du siège et des
lignes.}} {\textsc{- Conduite pernicieuse de La Feuillade.}} {\textsc{-
M. le duc d'Orléans empêché par Marsin de disputer la Doire, pois de
sortir des lignes et d'y combattre.}} {\textsc{- Conseil de guerre
déplorable.}} {\textsc{- M. le duc d'Orléans cesse de donner l'ordre et
de se mêler de rien.}} {\textsc{- Cause secrète de ces contrastes.}}
{\textsc{- Dernier refus de Marsin.}} {\textsc{- M. le duc d'Orléans, à
la prière des soldats, reprend le commandement sur le point de la
bataille.}} {\textsc{- Étrange abusement de Marsin.}} {\textsc{- Triple
désobéissance et opposition formelle de La Feuillade à M. le duc
d'Orléans.}} {\textsc{- Bataille de Turin.}} {\textsc{- Belle action de
Le Guerchois lâchement abandonné.}} {\textsc{- M. le duc d'Orléans veut
faire retirer l'armée en Italie.}} {\textsc{- Frémissement des officiers
généraux, qui, par leurs ruses, leur audace, leur désobéissance, le
forcent enfin à la retraite en France.}} {\textsc{- Motif d'une si
étrange conduite.}} {\textsc{- La nouvelle de la bataille portée au
roi.}} {\textsc{- Désordres de la retraite sans aucuns ennemis.}}
{\textsc{- Chaîne des causes de désastre devant Turin et de ses
suites.}} {\textsc{- Mort de Marsin prisonnier\,; son extraction, son
caractère.}} {\textsc{- La Feuillade, de négligence ou de dessein, prive
M. le duc d'Orléans de la communication avec l'Italie par Ivrée.}}
{\textsc{- Prises de la Feuillade avec Albergotti.}} {\textsc{-
Désespoir feint ou vrai de La Feuillade.}} {\textsc{- Origine de
l'amitié de M. le duc d'Orléans pour Besons, qui le demande.}}
{\textsc{- Besons le joint venant des côtes de Normandie.}}

~

M. le duc d'Orléans, abandonné à lui-même par M. de Vendôme, et ce qui
fut bien pis, à la tutelle du maréchal de Marsin, laissa un corps à
Médavy pour donner ordre aux convois et à toutes choses, subordonné au
prince de Vaudemont qui ne bougeait de milan, rassembla tout ce qui
était séparé de son armée, envoya demander par deux fois un corps de
cavalerie à La Feuillade, qu'il eut grand'peine à obtenir. Après avoir
observé les ennemis quelques jours, il résolut de se poster entre
Alexandrie et Valence pour leur empêcher le passage du Taner\footnote{Le
  Tanaro, affluent du Pô.}, ou les réduire à un combat. Ce passage était
le seul par lequel ils pussent pénétrer. Ne le point tenter, c'était
abandonner le secours de Turin\,; le vouloir forcer, c'était s'exposer à
un combat si désavantageux qu'il y avait une espèce d'évidence qu'ils
n'y pourraient jamais réussir.

Le prince le proposa au maréchal et ne le put persuader. D'en donner la
raison, c'est à quoi il ne faut pas prétendre, puisque Marsin n'en
allégua pas même d'apparente. Il était maîtrisé par La Feuillade qui
désirait ardemment de se voir rapproché par l'armée. Marsin ne songeait
qu'à satisfaire le gendre du tout-puissant ministre et à lui plaire.
Tous deux ne voyaient pas qu'empêcher le secours de Turin, c'était tout
faire, même pour le succès personnel de ce gendre fatal.

Tandis que le prince et le maréchal en étaient sur cette dispute, un
courrier du prince Eugène à l'empereur fut enlevé par un de nos partis,
et ses dépêches étaient en chiffres, comme on peut bien le juger. Le
prince eut beau feuilleter les siens, il n'en trouva point de
semblables. Marsin, venu de Flandre par l'Alsace et la Suisse, n'avait
garde d'en avoir. On envoya à Vaudemont qui manda n'avoir point ce
chiffre. Il fallut donc dépêcher un courrier au roi qui se trouva
l'avoir oublié au fond d'une cassette. Le courrier le rapporta, mais
quand\,? Le soir même de la bataille de Turin. Les dépêches déchiffrées
à Versailles et rapportées avec le chiffre du roi contenaient un grand
raisonnement du prince Eugène à l'empereur, précisément le même que
celui que M. le duc d'Orléans avait fait à Marsin. Il se terminait à
déclarer que si ce prince se postait où il l'avait si opiniâtrement
proposé à Marsin, il était extravagant, c'était le terme de la lettre,
de tenter ce passage, impraticable de passer le Taner ailleurs, qu'ainsi
il se trouverait réduit à se résoudre à tout sur la perte de Turin qu'il
ne pourrait empêcher après avoir fait tout le possible, et à la
supporter sans y ajouter celle de l'armée impériale, inévitable, et par
cela même inutile pour sauver Turin, en essayant follement de forcer un
passage inattaquable. Telle fut la justification ou plutôt l'éloge de M.
le duc d'Orléans par le prince Eugène à l'empereur dans une dépêche la
plus secrète, que le roi et son ministre virent de la première main,
puisque, faute de chiffre, elle leur avait été envoyée pour la
déchiffrer. Tel fut le désespoir que le roi et son ministre durent
ressentir d'avoir donné de si fatales brassières à un prince qui en
avait si peu besoin, et encore de si mauvaises.

Marsin donc n'ayant pu être persuadé, ce fut au duc d'Orléans à céder,
peu à peu à s'approcher de Turin et à joindre l'armée du siège. Il y
arriva le 28 août au soir. La Feuillade, désormais sous deux maîtres
présents, semblait devoir devenir plus docile\,; mais devenu si
rapidement général en chef, et d'une si importante armée, il ne songea
qu'à se conserver l'effective autorité. Il n'avait besoin que de Marsin,
sans lequel il n'ignorait pas que le prince ne pouvait rien. Avec
celui-ci il n'eût pas trouvé son compte. Sa fortune ne dépendait pas de
Chamillart, il n'avait d'objet que le succès d'où dépendait sa gloire,
et s'il eût été le maître, rien ne l'eût détourné de ce double objet. La
Feuillade se tourna donc uniquement à se saisir du maréchal, et il prit
sur lui un ascendant si fort qu'à l'ordre près qu'il donnait après
l'avoir reçu du prince, tout le reste demeura visiblement à La
Feuillade, au grand malheur de la France.

Le but commun était bien de prendre Turin, mais la manière d'y parvenir
et les moyens formèrent des contestations sans nombre. M. le duc
d'Orléans fut d'abord justement scandalisé que La Feuillade eût changé
tout ce qu'il avait réformé et ordonné à son passage au siège, allant
joindre M. de Vendôme. Cela lui parut si essentiel pour le succès qu'il
le fit rétablir, quoique avec douceur et modestie. En effet, avec le
chemin couvert pris, il se pouvait dire qu'il ne trouva aucun progrès au
siège. La Feuillade avait perdu des contre-gardes et d'autres ouvrages
qu'il avait pris, et qui avaient coûté plusieurs ingénieurs et beaucoup
de monde. Rien n'avançait, et de plus, on ne savait par où s'y prendre
pour avancer. La Feuillade, devenu de mauvaise humeur de son peu de
succès, s'était rendu inabordable, et s'était acquis une telle haine des
officiers généraux et particuliers, qu'ils ne se souciaient plus, pas
un, des événements. M. le duc d'Orléans reconnut les postes et les
travaux du siège\,; il visita les lignes et le terrain par où le prince
Eugène pouvait venir et tenter le secours. Il fut mal content de tout ce
qu'il remarqua au siège, il trouva les lignes mauvaises, très
imparfaites, très vastes et très mal gardées.

Il recevait cependant des avis de toutes parts que l'armée impériale
s'avançait, résolue de tenter le secours. Il voulut marcher à elle et se
saisir des passages de la Doire pour y faire à la vérité moins sûrement
et moins bien qu'à ceux du Taner, mais mieux au moins que dans des
lignes si étendues, si mal faites et si impossibles à garder partout. Il
trouva la même opposition pour la Doire qu'il avait éprouvée pour le
Taner. Marsin prétendit qu'en s'éloignant du siège, on pourrait jeter de
la poudre dans la place qui en manquait, dont on ne pouvait douter parce
qu'on avait trouvé plusieurs peaux de bouc qui en étaient pleines
nageant sur le Pô, qu'on y avait prises, et qui y avaient été jetées
dans l'espérance que le courant de l'eau les porterait aux assiégés. Le
fait était vrai, mais la réponse aisée. Ce que craignait Marsin était
incertain, et il ne l'était pas que ces poudres jetées dans la place
n'en différeraient que peu la prise et ne la pourraient empêcher si le
prince Eugène l'était de la secourir. Cette évidence de raisons fut
inutile\,; jamais Marsin ne se laissa entamer.

Les ennemis s'approchant toujours, le prince pressa le maréchal de
sortir des lignes telles que je les ai décrites, et qui ne se pouvaient
garder, de présenter la bataille au prince Eugène, avec tous les
avantages qui se trouveraient perdus dans des lignes nouvellement
tracées, point achevées, et d'une étendue qui ne se pouvait garder. Le
prince Eugène marchait depuis longtemps par des pays si ruinés, que son
armée n'en pouvait plus\,; qu'il était impossible qu'il pût subsister
vis-à-vis de la nôtre sans laisser périr la sienne de misère\,; qu'il ne
hasarderait peut-être pas de l'exposer en rase campagne à l'impétuosité
française, et en ce cas, qu'il abandonnerait le secours de Turin, qui
tomberait après nécessairement\,; que, s'il donnait la bataille, rien
n'était plus différent pour des Français que la donner aussi de leur
côté, d'attaquer et de se manier en terrain libre, ou de ne faire que se
défendre derrière de mauvaises lignes qui seraient percées de tous les
côtés\,; de plus, si les troupes, harassées du prince Eugène étaient
battues, elles se trouveraient sans retraite entre notre armée et la
Savoie, dont nous étions maîtres, ayant été obligées à faire ce grand
tour, parce que tout l'autre côté était inaccessible.

Marsin, gourmandé par La Feuillade, répondit que toutes ces raisons
étaient véritables, mais que le parti proposé par le prince ne se
pouvait prendre qu'en fortifiant l'armée des quarante-six bataillons
qu'Albergotti avait sur la hauteur des Capucins, par où la place
pourrait alors recevoir quelques secours. Cela était vrai, et vrai
encore, que rien de plus inutile qu'une armée sur cette hauteur à rien
faire qu'à la garder de petites tentatives, à quoi peu de bataillons
auraient suffi, et qui cependant avait porté un grand affaiblissement au
reste des troupes du siège, À cette raison du maréchal la réponse était
la même qu'à celle des poudres. Ce secours à jeter par la hauteur des
Capucins dégarnie était incertain, il ne pouvait être grand, il ne
pouvait être préparé ni appuyé d'aucunes troupes, et si, avec ce
secours, le prince Eugène se trouvait réduit à n'oser combattre ou être
battu, Turin était sans ressource, et avec ce peu de secours jeté par
les Capucins, était pris à l'aise quinze jours plus tôt ou plus tard.

Cette dispute s'échauffa tellement que Marsin consentit à un conseil de
guerre où tous les lieutenants généraux furent appelés. La matière y fut
débattue. Mais La Feuillade, gendre favori du ministre arbitre de la
fortune de tout homme de guerre, et Marsin, dépositaire, disait-on, du
secret, n'avaient garde de n'être pas suivis. Le seul d'Estaing parla en
homme d'un courage libre (M. le duc d'Orléans ne l'oublia jamais), et
seul aussi y acquit de l'honneur. Albergotti, Italien raffiné, prévit la
honte et l'orage, et se tint à son poste sous prétexte de l'éloignement.
Tous les autres opinèrent servilement, de sorte que ce remède rendit le
mal incurable. M. le duc d'Orléans protesta devant tous des malheurs qui
en allaient arriver, déclara que, n'étant maître de rien, il n'était pas
juste qu'il essuyât l'affront que la nation allait recevoir, et le sien
particulier encore, demanda sa chaise de poste, et à l'instant voulut
quitter l'armée. Marsin, La Feuillade et les plus distingués de ce
conseil de guerre, mirent tout en œuvre pour l'arrêter. Revenu enfin de
ce premier mouvement, content peut-être d'avoir marqué sa fermeté
jusqu'à ce point, et si fortement manifesté combien peu l'événement
imminent lui pouvait être imputé, il consentit à demeurer. Mais en même
temps il s'expliqua qu'il ne se mêlerait plus du tout du commandement de
l'armée, jusque-là même qu'il refusa de donner l'ordre et qu'il renvoya
tout à Marsin, à La Feuillade et à quiconque en voudrait prendre le
soin. Il l'exécuta de la sorte, sans pouvoir être ramené. Le fin d'une
opiniâtreté si funeste était la folle espérance, uniquement fondée sur
la grandeur du désir, que le prince Eugène n'oserait attaquer les
lignes\,; que, se retirant ainsi, Turin serait pris, non par l'armée du
duc d'Orléans, non par sa victoire, non par son fait, mais par le siège
et les lignes dont La Feuillade avait eu la direction comme général, et
par conséquent n'en partagerait la gloire avec personne. Tel est le vrai
fait, qui, soutenu de captieuses raisons, et soutenu de tout le feu
d'une bouillante et puissante jeunesse, asservit Marsin et finit par
égorger la France. Tel fut l'état des choses pendant les trois derniers
jours de ce siège désastreux. Le duc d'Orléans, dépossédé par lui-même,
souvent chez soi, quelquefois se promenant, écrivit fortement au roi
contre le maréchal, en lui rendant un compte exact de toutes choses, fit
lire sa lettre à Marsin, la lui laissa, et le chargea de l'envoyer par
le premier courrier qu'il dépêcherait, n'en voulant plus envoyer
lui-même, comme n'étant plus rien dans l'armée.

La nuit du 6 au 7, qui fut le jour de la bataille, quoiqu'il ne se mêlât
plus de quoi que ce fût, il ne laissa pas d'être réveillé par un billet
qu'on lui apporta d'un partisan qui lui mandait que le prince Eugène
attaquait le château de Pianezze pour y passer la Doire, qu'il était
assuré qu'il marcherait aussitôt après à lui pour l'attaquer. Malgré son
dépit et sa résolution, le prince se lève, s'habille à la hâte, va
lui-même chez Marsin qui dormait tranquillement dans son lit, l'éveille,
lui montre le billet qu'il venait de recevoir, lui propose de marcher
aux ennemis à l'heure même, de les attaquer, de profiter de leur
surprise et d'un ruisseau difficile qu'ils avaient à passer, s'il les
trouvait déjà maîtres du château de Pianezze et en marche pour venir sur
lui. La supputation du temps et du chemin n'était pas douteuse.
Saint-Nectaire, longtemps depuis chevalier de l'ordre, et fort entendu à
la guerre, arriva en ce moment de dehors chez Marsin. Il confirma l'avis
du partisan et appuya l'avis du prince\,; niais il était résolu dans les
décrets éternels que la France serait frappée au cœur ce jour même.

Le maréchal fut inébranlable, tout ce qui allait à sortir des lignes
était proscrit par la raison secrète que j'en ai expliquée. Il maintint
que l'avis était faux, que le prince Eugène ne pouvait arriver si
promptement sur eux, et conseilla à M. le duc d'Orléans de s'aller
reposer sans avoir jamais voulu donner aucun ordre. Le prince, plus
piqué et plus dégoûté que jamais, se retira chez lui, bien résolu de
tout abandonner aux aveugles et aux sourds qui ne voulaient rien voir ni
entendre.

Peu après qu'il fut rentré dans sa chambre, les avis vinrent de toutes
parts de l'approche du prince Eugène. Il ne s'en ébranla point.
D'Estaing et quelques autres officiers généraux qui vinrent chez lui le
forcèrent malgré lui de monter à cheval. Il s'avança négligemment au
petit pas le long de la tête du camp. Tout ce qui se passait depuis
quelques jours avait fait trop de bruit pour que toute l'armée n'en fût
pas instruite, jusqu'aux soldats. Son rang, la justesse et la fermeté de
ses avis, dont les vieux soldats ne sont pas incapables d'être
quelquefois bons juges, ce que plusieurs d'entre eux se souvenaient de
lui avoir vu faire à Leuze, à Steinkerque, à Neerwinden, les faisait
murmurer de ce qu'il ne voulait plus commander l'armée. Comme il passait
donc de la sorte à la tête des camps, un soldat de Piémont l'appela par
son nom, et lui demanda s'il leur refuserait son épée. Ce mot fit plus
que n'avaient pu les officiers généraux qui lavaient été tirer de chez
lui. Il répondit au soldat qu'il la lui demandait de trop bonne grâce
pour en être refusé, et mettant à l'instant à ses pieds tant de
mécontentements si vifs et si justes, il rie pensa plus qu'à secourir
Marsin et La Feuillade malgré eux-mêmes.

Mais il n'était plus possible de sortir des lignes, quand bien même ils
y auraient consenti. L'armée ennemie commençait à paraître, et s'avança
si diligemment, que le temps manqua pour achever les dispositions.
Marsin, plus mort que vif, voyant ses espérances trompées, abîmé dans
les réflexions qui n'étaient plus de saison, parut comme un homme
condamné, incapable de donner aucun ordre à propos. Les vides étaient
grands dans les lignes. M. le duc d'Orléans envoya chercher les
quarante-six bataillons d'Albergotti, qui, sur cette hauteur des
Capucins, demeuraient également éloignés et inutiles contre la place et
contre le prince Eugène. Mais La Feuillade, bien plus craint et obéi que
le prince, avait défendu à Albergotti de bouger, et il ne bougea malgré
les ordres réitérés de M. le duc d'Orléans. Il y renvoya encore les
chercher\,; en même temps La Feuillade leur envoya défendre de marcher,
et ils ne bougèrent encore. Cependant le duc d'Orléans, pour remplir un
peu les intervalles de la première ligne si dégarnie, y mêla des
escadrons avec les bataillons, et la fortifia en affaiblissant sa
seconde ligne, comptant toujours que les quarante-six bataillons
d'Albergotti allaient arriver. En attendant, il envoya hâter d'autres
troupes un peu éloignées de passer un petit pont et de venir à lui
garnir les lignes. Mais La Feuillade encore poussé de je ne sais quel
démon, et qui sut cet ordre, s'en alla lui-même se mettre sur ce petit
pont et les arrêter. La désobéissance fut telle que M. le duc d'Orléans,
ayant lui-même commandé à un officier qui menait un escadron du régiment
d'Anjou de le faire marcher, il le refusa, sur quoi le prince lui
balafra le visage et le fit dire au roi.

L'attaque, commencée sur les dix heures du matin, fut poussée avec une
incroyable vigueur et soutenue d'abord de même. Langallerie, qui avait
fort servi le prince Eugène dans la marche, ne lui fut pas moins utile
dans l'action. Il perça le premier par des intervalles que le petit
nombre de nos troupes laissait ouverts. Le prince Eugène y courut avec
des troupes\,; d'autres intervalles où on ne put suffire donnèrent
entrée à d'autres troupes. Marsin, vers le milieu du combat, reçut un
coup qui lui perça le bas-ventre et lui cassa les reins\,; {[}il fut{]}
pris en même temps et conduit en une cassine éloignée. La Feuillade
courait éperdu partout, s'arrachant les cheveux et incapable de donner
aucun ordre. Le duc d'Orléans les donna tous, mais fort mal obéi. Il fit
des merveilles, toujours dans le plus grand feu avec un sang-froid qui
voyait tout, qui distinguait tout, qui le conduisait partout où il avait
le plus à remédier et à soutenir par son exemple qui animait les
officiers et les soldats. Blessé d'abord assez légèrement vers la
hanche, ensuite près du poignet dangereusement et très douloureusement,
il fut inébranlable. Voyant que tout commençait à s'ébranler, il
appelait les officiers par leur nom, animait les soldats de la voix, et
mena lui-même les escadrons et les bataillons à la charge. Vaincu enfin
par la douleur, et affaibli par le sang qu'il perdait, il fut contraint
de se retirer un peu pour se faire panser. À peine en donna-t-il le
temps, et retourna où le feu était le plus vif. Mais le terrain,
l'ordre, la discipline, tout semblait de concert pour confondre les
Français.

Trois fois Le Guerchois, avec sa brigade de la vieille marine, avait
repoussé les ennemis avec beaucoup de carnage, encloué leur canon, et
trois fois réparé la bataille, lorsque, affaibli par tout ce qu'il avait
perdu d'officiers et de soldats, il manda à la brigade voisine qui le
devait soutenir de s'avancer pour faire front avec la sienne, et
l'empêcher d'être débordé par un plus grand nombre de bataillons frais
qu'il voyait venir à lui pour la quatrième fois. Cette brigade et son
brigadier, desquels il faut ensevelir la mémoire, le refusèrent tout
net.

Ce fut le dernier moment du peu d'ordre qu'il y eut en cette bataille.
Tout ce qui suivit ne fut que trouble, confusion, débandement, fuite,
déconfiture. Ce qu'il y eut de plus horrible, c'est que les officiers
généraux et de tout caractère, j'en excepte bien peu, plus en peine de
leur équipage et de la bourse qu'ils avaient faite par leur pillage,
l'augmentèrent plus qu'ils ne s'y opposèrent, et furent pis qu'inutiles.

M. le duc d'Orléans, convaincu enfin qu'il était désormais impossible de
rétablir cette malheureuse journée, se tourna à y laisser le moins qu'il
se pourrait. Il retira son artillerie légère, ses munitions, tout ce qui
était au siège et aux travaux les plus avancés, songea à tout avec une
si grande présence d'esprit que rien ne lui échappa. Enfin, ramassant
autour de lui ce qu'il put d'officiers généraux, il leur exposa
courtement, mais avec justesse, qu'il n'était plus temps que de penser à
la retraite, et à prendre le chemin d'Italie, que par ce parti ils y
demeureraient maîtres, enfermeraient l'armée victorieuse autour de
Turin, lui empêcheraient tout retour en Italie, la feraient périr dans
un pays entièrement ruiné et désolé, dans l'impossibilité d'y subsister
et d'en sortir, encore moins de s'y réparer, tandis que l'armée du roi,
lui fermant la communication de tout secours, se trouverait dans un pays
abondant où ils seraient les plus forts, à portée de tout et de tout
entreprendre avec temps et loisir.

Cette proposition effaroucha au dernier point des esprits peu rassurés,
et qui espéraient au moins ce fruit de leur désastre, qu'il leur
procurerait le retour si désiré en France, pour y porter leur argent,
dont ils s'étaient gorgés à toutes mains en Italie. La Feuillade, à qui
tant de raisons devaient fermer la bouche, se mit si bien à combattre
cet avis, que le prince, poussé à bout d'une effronterie si soutenue,
lui imposa {[}silence{]} et fit parler les autres. D'Estaing fut encore
le seul qui appuya l'avis de l'Italie. Le débat tint du désordre de la
journée, et de l'abattement où la blessure de M. le duc d'Orléans
l'avait mis. Il le finit en leur disant que le temps ni le lieu
n'étaient pas susceptibles d'une plus longue dispute\,; que las enfin
d'avoir eu tant de raison et si peu de créance, il s'en voulait faire
croire à son tour maintenant qu'il était libre, et donna l'ordre de
marcher au pont et de se retirer en Italie. Il n'en pouvait plus. Son
corps et son esprit s'épuisaient également. Après avoir marché quelque
temps, il se jeta dans sa chaise de poste. Il continua ainsi la marche,
et traversa le Pô sur le pont, entendant derrière lui des officiers
généraux qui murmuraient tout haut du parti qu'il prenait, désespérés de
se revoir en Italie, et sans communication avec la France qui leur
tenait si fort au cœur. Ce bruit alla même si loin, surtout de l'un
d'entre eux, que le duc d'Orléans, trop justement irrité, ne put
s'empêcher de passer sa tête par la portière, de lui reprocher sa
maîtresse par son nom, et de lui dire que, pour ce qu'il faisait à la
guerre, il ferait mieux de rester avec elle\,; cette sortie fit taire
chacun.

Mais il était arrêté que l'esprit d'erreur et de vertige déferait seul
notre armée et sauverait les alliés. Comme on débouchait le pont, du
côté d'Italie, d'Arennes, major général et officier général, vint à
toute bride devers la tête du corps d'Albergotti. Il présenta un
officier à M. le duc d'Orléans, lui dit que les ennemis occupaient les
passages par où il était indispensable de passer. Sur les questions du
prince, l'officier l'assura que ce poste était bien retranché, occupé
par le régiment de la Croix-Blanche, dont entre autres il avait bien
reconnu les drapeaux, et qu'il se croyait sûr aussi d'y avoir reconnu la
personne de M. le duc de Savoie. Malgré un rapport si positif, le
prince, en trop juste défiance après tout ce qu'il avait vu et entendu
sur ce parti d'Italie, voulut qu'on continuât la marche, quitte à
revenir si les passages se trouvaient occupés de manière à ne pouvoir
forcer et passer. On continua, et en attendant on envoya les
reconnaître. Les officiers généraux n'en voulurent pas être les dupes.
Le chemin vers nos Alpes était sans danger. Ils le firent prendre, et
depuis continuer, à ce qu'on avait de vivres et de munitions, tellement
qu'après une demi-journée de marche, et des rapports des passages fort
équivoques, on avertit M. le duc d'Orléans qu'il n'avait ni vivres ni
munitions, qui, ayant pris et continué la route du côté de France, lui
rendait celle d'Italie impossible, que d'ailleurs on lui maintenait
toujours fermée par les ennemis. La rage et le désespoir de tant de
criminelles désobéissances, pour ne pas dire de trahisons redoublées,
jointes à la douleur de sa blessure et à la faiblesse où il se trouvait,
le firent retomber au fond de sa chaise, et dire qu'on allât donc où on
voudrait et qu'on ne lui en parlât plus.

Telle est l'histoire de la catastrophe d'Italie. On sut depuis que tout
le rapport de cet officier, mené par d'Arennes, était entièrement
controuvé\,; qu'il n'y avait personne dans aucun passage pour disputer
celui d'Italie, pas même le moindre obstacle, et pour combler les
regrets, l'avantage que Médavy remporta deux jours après, par lequel, en
arrivant, M. le duc d'Orléans se fût trouvé maître absolu de toute la
Lombardie, et d'acculer sans ressource le prince Eugène entre lui et la
Savoie que nous tenions. C'est ce qui combla la douleur de ce prince en
arrivant à Oulx, au milieu des Alpes, où il était en sûreté entre ses
quartiers, ne pouvant passer outre par l'état de sa blessure.

Saint-Léger, un des premiers valets de chambre de M. le duc d'Orléans,
dépêché au roi avec cette cruelle nouvelle, arriva à Versailles, le
mardi 14 septembre, avant le lever du roi, et annonça Nancré avec le
détail.

L'armée, dans ce subit retour, marcha donc à colonne renversée sur
Pignerol. Ce changement de disposition lit que quantité d'équipages qui,
sans le savoir, se trouvèrent à l'arrière-garde, furent pillés ou perdus
la nuit dans la montagne. Albergotti, dont, comme on l'a vu, les troupes
n'avaient pas combattu, fut chargé de cette arrière-garde, et la fit
très bien nonobstant la nuit et la longueur de la queue, l'embarras des
défilés continuels et la confusion de la nuit. Du côté des ennemis il
n'eut pas la moindre inquiétude.

Comblés d'une joie d'autant plus grande qu'elle était moins espérée, ils
se contentèrent de leurs succès qu'ils avaient encore peine à croire.
Leur armée n'en pouvait plus. Elle n'eut donc garde de songer à troubler
la retraite. On a vu que l'artillerie, les munitions et tout ce qui
était dans les postes les plus avancés du siège avait été entièrement
retiré, sans aucun obstacle. On a su positivement depuis que le prince
Eugène avait tout à fait pris le parti de cesser l'attaque et de faire
sa retraite, si Le Guerchois eût soutenu la quatrième et dernière charge
dont j'ai parlé, à laquelle il succomba et fut pris par l'insigne
lâcheté du brigadier et de la brigade qui refusa de le secourir. On sut
encore que Turin n'avait pas pour plus de quatre jours de poudre. Enfin
rien ne manqua pour les transporter de la joie la plus complète, et nous
de la plus cuisante douleur.

Il ne fallut pas moins qu'un enchaînement de miracles pour produire un
si grand effet, dont un seul manqué, et lequel de tous que ce pût être,
emportait la ruine de l'entreprise. Vendôme, comme on l'a vu, en eut le
premier déshonneur, que Marsin consomma et que La Feuillade combla. Le
siège mal enfourné pour les attaques, languissamment poussé par les
folles courses de La Feuillade\,; les rivières et le Pô passés par la
négligence de Vendôme\,; l'obstacle du Taner, qui était invincible,
méprisé par Marsin, pour le faux intérêt de La Feuillade\,; la folie de
se mettre dans des lignes mal faites, imparfaites, la plupart à peine
tracées et d'une étendue à ne les pouvoir garder\,; l'opiniâtreté de ne
vouloir pas aller au-devant des ennemis, sur ce château de Pianezze,
harassés et qu'on y aurait surpris dans l'embarras de passer un ruisseau
difficile\,; le servile succès de ce conseil de guerre\,; l'inutilité de
quarante-six bataillons, c'est-à-dire d'une armée entière, et pour le
siège, et pour la garde des lignes, et pour le combat\,; la triple
désobéissance de La Feuillade pour arrêter ces troupes aux Capucins,
malgré deux ordres exprès de M. le duc d'Orléans, et la troisième
d'avoir arrêté d'autres troupes sur ce petit pont, que ce prince avait
envoyé chercher en diligence pour garnir ses lignes\,; l'insigne
confiance de Marsin, et son opiniâtreté jusqu'à l'instant de l'arrivée
du prince Eugène, tout cela conduit par le seul intérêt de La Feuillade
de ne partager pas sa conquête avec M. le duc d'Orléans, et la crainte
de Marsin, subjugué par le gendre, de déplaire au beau-père\,; enfin,
pour dernier coup, la lâcheté si punissable de ce refus de secours à Le
Guerchois et à sa brigade, qui fut le dernier assommoir qui détermina la
victoire d'une part, le désordre et la fuite de l'autre\,; voilà la
chaîne de tant d'incroyables miracles pour la délivrance de Turin.

Après, pour la retraite\,: la révolte, l'intérêt lâche et pécuniaire des
officiers généraux\,; la supposition de d'Arennes ou de son officier\,;
l'envoi clandestin des vivres et des munitions par les Alpes, pour
rendre toute autre retraite impossible\,; un concert continuel de
mauvaise foi, de désobéissance, pour ne pas dire de trahison\,; ce sont
d'autres miracles qui sauvèrent l'Italie, Turin dans les suites, et
l'armée victorieuse qui serait périe avec la place faute d'issue, de
vivres et de secours. À tout cela, qui peut méconnaître la main de Dieu
toute-puissante, mais qui peut douter du crime de ceux de nos François
qui en ont été les agents\,?

Marsin, gagnant cette cassine éloignée où il fut conduit, demanda une
seule fois si M. le duc d'Orléans était tué. Arrivé là avec un aide de
camp et deux ou trois domestiques, il envoya chercher un confesseur,
dicta quelque chose sur ses affaires, mit dans un paquet pour M. le duc
d'Orléans la lettre que ce prince avait écrite au roi contre lui, et
qu'il lui avait lue et confiée pour l'envoyer lui-même, ne voulut plus
ouïr parler que de Dieu, et mourut dans la nuit. On trouva parmi ses
papiers des misères innombrables, et un amas de vœux plus que
surprenants, un désordre immense dans ses affaires, et des dettes que
six fois plus de bien qu'il n'en avait n'eût jamais payées.

C'était un extrêmement petit homme, grand parleur, plus grand courtisan,
ou plutôt grand valet, tout occupé de sa fortune, sans toutefois être
malhonnête homme, dévot à la flamande, plutôt bas et complimenteur à
l'excès que poli, cultivant avec un soin qui l'absorbait tous ceux qui
pouvaient le servir ou lui nuire, esprit futile, léger, de peu de fond,
de peu de jugement, de peu de capacité, dont tout l'art et le mérite
allait à plaire. Il était moins que rien, du pays de Liège. Son père,
qui était capitaine, s'avança de bonne heure au service de France, y
épousa une Balzac, suivit le parti de M. le Prince dont il fut estimé,
changea aisément de parti selon son intérêt, se donna aux Espagnols,
courtisa si bien Charles II lorsqu'il était à Bruxelles, qu'il en eut la
Jarretière, au scandale des Anglais, et parvint à tout dans le
militaire, au service d'Espagne, dans lequel il mourut d'assez bonne
heure. Il ne laissa que ce fils que sa mère éleva en France et l'y
attacha. On a vu sa fortune et sa catastrophe. Il n'était point marié et
point vieux.

Dans une si cruelle retraite, l'armée manqua de pain, qui fut le comble
de ses malheurs. M. le duc d'Orléans, bien qu'outré de corps et
d'esprit, était le seul qui songeât à tout et qui n'était soulagé par
personne. Il s'arrêta pour attendre la queue de ses troupes et leur
fournir du pain. Dès qu'il y en eut de cuit, il en fit prendre à un gros
détachement avec lequel il ordonna à Vibraye de s'aller saisir du
château de Bar, passage unique qui conservait la communication et le
retour en Italie par Ivrée. La Feuillade, qui s'était chargé de ce
détail, voulut aller avec le détachement, le retarda à partir de deux
jours, et n'oublia qu'à lui faire prendre le pain qui lui était destiné.
Il fallut donc s'arrêter dès le second jour pour en envoyer quérir. Il
est difficile de comprendre le dépit de M. le duc d'Orléans, qui était
dans son lit et qui comptait le détachement bien loin, d'apprendre ce
retardement et cet oubli du pain qui l'arrêtait encore, et la
promptitude avec laquelle il y remédia. Le pain arrivé, le détachement
continua sa route, mais il ne marcha pas longtemps sans être averti que
les ennemis s'étaient emparés du château et du passage, de manière à
n'en pouvoir être dépostés, et qu'ils l'avaient prévenu de vingt heures,
tellement que ce fut au retardement de la Feuillade et à son incroyable
négligence sur ce pain que ce dernier malheur fut encore dû. La
Feuillade n'eut donc de parti à prendre que celui de retourner sur ses
pas.

Peu de jours avant la bataille, il avait fort maltraité Albergotti, qui
s'était licencié sur la lenteur du siège, à n'approuver pas les courses
du général après le duc de Savoie. Quelques gens se mirent entre-deux.
Dès le lendemain, l'Italien, fort en peine sur Chamillart, alla chez son
gendre le prier d'oublier ce qui s'était passé la veille.

La Feuillade, arrivant de ce beau détachement à Oulx, y trouva M. le duc
d'Orléans dans un état périlleux, qui le devint bien davantage par tous
les soins qu'il se donnait à reposer, assurer, nourrir et raccommoder
ses troupes avec des peines et des dépenses extrêmes, par le peu de
secours qu'il recevait de la cour, ne respirant que de rentrer en
Italie. La Feuillade se trouvant dans la chambre de M. le duc d'Orléans
avec Albergotti et d'autres, ce prince, de nouveau outré du succès de ce
détachement, ne put s'empêcher de leur reprocher à tous deux leur
désobéissance à demeurer sur la hauteur des Capucins. Tous deux
voulurent répondre\,; mais M. le duc d'Orléans, qui n'avait pu retenir
cette plainte, et le reproche trop véritable qu'ils étaient cause de la
perte de la bataille, et qui se sentait assez ému pour se craindre
soi-même à la réplique, les pria qu'il n'en fût pas parlé davantage.
Sassenage et le peu d'autres qui se trouvèrent à la ruelle du lit les en
écartèrent, et les poussèrent grommelant l'un contre l'autre, et dont la
voix s'élevait à mesure qu'ils s'éloignaient du lit. Ils n'étaient pas
au bout de la chambre qu'Albergotti dit assez vivement à La Feuillade
que c'était lui seul que ce reproche du prince pouvait regarder, puisque
lui n'avait fait qu'obéir à ses ordres de lui La Feuillade\,; sur quoi
celui-ci lui répondit net que cela n'était pas vrai, le poussa en même
temps et mit la main à l'épée. Albergotti, rougissant de colère,
marmotta entre ses dents et recula deux pas. Sassenage, Saint-Frémont et
quelques autres se jetèrent entre-deux, les tirèrent hors de la chambre,
et leur demandèrent s'ils savaient en quel lieu ils étaient, et si la
tête leur avait tourné. M. le duc d'Orléans, de dedans ses rideaux, ou
n'entendit pas, ou n'en fit jamais semblant. Chacun emmena son homme,
fort en peine de ce qui arriverait après, mais il ne se passa rien entre
eux en aucun temps. La valeur d'Albergotti ne fut jamais douteuse, mais
il était Italien, et La Feuillade était le gendre bien-aimé de
Chamillart, qui ne laissa pas, quoique fort brave aussi, d'être fort
aise que l'autre se montrât si bonne personne. Cette aventure ne laissa
pas de leur faire grand tort à tous deux, non sur la valeur, car leurs
preuves étaient faites et complètes, mais sur l'honneur\,: à l'un
d'avoir osé démentir une vérité trop connue à toute l'armée, et qui en
avait été la perte dans le temps de la bataille\,; à l'autre de l'avoir
avalé et digéré si doux.

Cependant La Feuillade, hors de soi de tant d'affreuses sottises
entassées, dépêche un courrier à Chamillart, lui envoie la démission de
son gouvernement de Dauphiné, et lui mande qu'il est indigne de son
estime, des grâces du roi et de voir le jour\,; le lendemain, obtient
permission de M. le duc d'Orléans de s'en aller à Antibes profiter de
l'occasion de quelques bâtiments qui passaient à Gènes, pour se rendre
de là auprès de Médavy, et là, servant sous ses ordres et se mettant à
tout, se rendre digne qu'on oubliât ses fautes. Chamillart, toujours
également affolé de son gendre, lui renvoya son courrier et sa démission
qu'il s'était bien gardé de montrer, le caressa par sa réponse,
l'encouragea et lui remit la cervelle. Ceux qui surent cette
désespérade, ne doutèrent pas qu'elle ne fût un jeu pour faire pitié à
son beau-père et au roi même, qu'il comptait bien qu'il ne saurait rien
de sa démission, au moins qu'à coup sûr pour lui. En même temps, M. le
duc d'Orléans reçut des réponses et des ordres favorables à son désir de
repasser en Italie. Il était tenu à Chamillart, était content d'avoir
humilié La Feuillade, à la vérité content à bon marché. Il lui envoya un
courrier pour lui apprendre les ordres qu'il venait de recevoir,
l'empêcher de s'embarquer et le faire revenir à Briançon, où il allait
dès qu'il pourrait être transporté, et repasser avec l'armée, plutôt que
s'en aller seul et devant par Gênes. La Feuillade, ravi de se voir moins
mal avec ce prince qu'il n'avait lieu de le croire, ne se le fit pas
dire deux fois et s'en alla à Briançon.

Ce fut où Besons joignit M. le duc d'Orléans. Il avait commandé sous lui
la réserve, puis avait été mis par le roi auprès de lui lorsqu'il avait
commandé la cavalerie. M. le duc d'Orléans avait pris de l'estime et de
l'amitié pour lui. Il servait cette année sur les côtes de Normandie,
parce que sa santé ne lui avait pas permis mieux. M. le duc d'Orléans le
demanda au roi qui le lui accorda, et Besons en meilleure santé et
flatté de ce souvenir, l'alla trouver le plus tôt qu'il lui fut
possible.

\hypertarget{chapitre-xv.}{%
\chapter{CHAPITRE XV.}\label{chapitre-xv.}}

1706

~

{\textsc{Promptitude incroyable avec laquelle j'apprends les malheurs
devant Turin.}} {\textsc{- Nancré apporte le détail pour la bataille de
Turin.}} {\textsc{- Mort de Murcé de ses blessures\,; fadaises sur lui
par rapport à M\textsuperscript{me} de Maintenon.}} {\textsc{- Victoire
de Médavy en Italie sur le prince de Hesse, depuis roi de Suède.}}
{\textsc{- Médavy chevalier de l'ordre\,; autres récompenses.}}
{\textsc{- M\textsuperscript{me} de Nancré et d'Argenton à Grenoble.}}
{\textsc{- On ne pense plus à repasser en Italie, qui se perd.}}
{\textsc{- M. le duc d'Orléans à Versailles.}} {\textsc{- Ce qu'il pense
de La Feuillade et ses officiers généraux.}} {\textsc{- La Feuillade
perdu et rappelé.}} {\textsc{- La Feuillade et le cardinal Le Camus.}}
{\textsc{- La Feuillade salue le roi\,; très mal reçu.}} {\textsc{-
Électeur de Cologne incognito à Paris et à Versailles.}} {\textsc{- Mort
de Saint-Pouange.}} {\textsc{- Chamillart grand trésorier de l'ordre.}}
{\textsc{- Mort de M\textsuperscript{me} de Barbezieux.}} {\textsc{-
Mort de Boisfranc.}} {\textsc{- Survivance de Maréchal à son fils\,;
alarme des survivanciers.}} {\textsc{- M\textsuperscript{me} de La
Chaise à Marly en absence de M\textsuperscript{me} la duchesse de
Bourgogne et de Madame.}} {\textsc{- Dispute entre le duc de Tresmes et
M. de La Rochefoucauld pour le chapeau du roi.}} {\textsc{- Piété de Mgr
le duc de Bourgogne.}} {\textsc{- Le roi de Suède, victorieux en Saxe, y
dicte la paix au roi Auguste.}} {\textsc{- Sa glorieuse situation et sa
lourde faute.}} {\textsc{- Patkul et sa catastrophe.}} {\textsc{-
Stanislas reconnu roi par la France\,; mécontents et leurs progrès.}}
{\textsc{- Mariage arrêté de l'archiduc avec une princesse de
Wolfenbuttel.}} {\textsc{- Facilité des princes protestants à se faire
catholiques pour des avantages, et sa véritable cause.}} {\textsc{-
Succès et séparation des armées en Espagne.}} {\textsc{- Secours
d'argent à l'archiduc.}} {\textsc{- Conférences refusées par les alliés
sur la paix.}} {\textsc{- Villars et le duc de Noailles de retour.}}
{\textsc{- Le roi entretient le prince de Rohan sur la bataille de
Ramillies.}} {\textsc{- Surville et La Barre accommodés, le premier
demeurant perdu.}} {\textsc{- M\textsuperscript{me} le Châtillon\,; sa
famille, son caractère, sa conduite\,; quitte Madame et y demeure.}}
{\textsc{- Mariage du fils de Livry avec une fille du feu prince
Robert\,; grâces du roi à cette occasion.}} {\textsc{- M. de
Beauvilliers cède son duché, etc., à son frère, et le marie à la fille
unique de feu Besmaux.}} {\textsc{- Conduite admirable de la duchesse de
Beauvilliers.}} {\textsc{- Bergheyck à Versailles\,; son caractère et sa
fortune.}} {\textsc{- Vendôme de retour.}} {\textsc{- Grand prieur à
Gênes.}} {\textsc{- Ridicule de M\textsuperscript{me} de Maintenon sur
Courcillon.}}

~

J'étais allé passer un mois à la Ferté, j'y recevais les nouvelles
d'Italie que M. le duc d'Orléans me faisait envoyer avec soin, et des
lettres de sa main quand il ne voulait pas que ce qu'il me mandait
passât par d'autres. J'étais donc pleinement instruit des malheurs qui
s'y préparaient, et fort inquiet, lorsqu'un gentilhomme arrivant de
Rouen chez son frère, tout auprès de chez moi, y vint comme nous nous
promenions M\textsuperscript{me} de Saint-Simon et moi dans le parc avec
du monde, et nous raconta le désastre de Turin avec les circonstances
exactes sur M. le duc d'Orléans, sur le maréchal de Marsin, et sur tout
le reste, telles que le roi les apprit trois jours après seulement, par
le courrier qui en porta la nouvelle (et moi, quatre jours, par mes
lettres de la cour et de Paris), sans que nous ayons jamais pu
comprendre comment il était possible que cette triste nouvelle eût été
portée avec une si extrême diligence, pour ne pas dire incroyable, sans
que ce gentilhomme nous le voulût dire, sinon d'en fortement appuyer la
certitude, et sans que nous l'ayons jamais revu depuis, car il mourut
fort tôt après. Je fus vivement touché de ce malheur arrivé entre les
mains de M. le duc d'Orléans, quoiqu'elles en fussent parfaitement
innocentes. La fièvre me prit, je m'en allai à Paris, sans m'arrêter à
Versailles pour éviter l'empire de sa faculté.

Nancré, dépêché avec le détail, y arriva presque en même temps. Quoique
je ne le connusse point du tout, je lui envoyai dire que j'étais hors
d'état de l'aller trouver et que je le priais de venir chez moi. Il y
vint aussitôt. Il avait ordre de nie voir\,; nous fûmes deux bonnes
heures tête à tête. Il m'apprit que le roi rendait une pleine justice à
son neveu, et me pressa de lui écrire sans nul ménagement, je n'en eus
pas besoin. Le public équitable, la cour même, malgré ses jalousies,
décernèrent des lauriers à sa défaite, et l'élevèrent d'autant plus que
la fortune l'avait voulu abaisser. Ce fait est aussi mémorable que
singulier, et je ne crois pas qu'il y ait d'exemple de tant et de si
unanimes louanges dans un malheur aussi complet. Tout le cri tomba sur
Marsin, et nonobstant Chamillart, sur La Feuillade.

Quoique les ennemis, contents de leurs succès, ne se fussent opposés à
rien de la retraite, il est pourtant vrai que le gros canon de batterie
ne put être emmené. L'abbé de Grancey, premier aumônier de M. le duc
d'Orléans, médiocre pauvre, mais fort brave et fort bon homme, fut tué à
deux pas derrière lui, sur quoi le comte de Roucy disait que ce prêtre
abbé mourrait de joie s'il pouvait savoir qu'il a été tué. Villiers et
La Bretonnière, maréchaux de camp, Bonelles, fils de Bullion, colonel
d'infanterie, Kercado, mestre de camp du Dauphin-étranger\footnote{Le
  régiment \emph{Dauphin-étranger} était composé d'étrangers, comme le
  Royal-Allemand, le Royal-Pologne, etc.}, très bon sujet, et à qui
j'avais vendu ma compagnie, lui jeune cornette dans le même régiment, et
assez d'officiers y furent tués\,; et Murcé, lieutenant général, mourut
de ses blessures, prisonnier à Turin. On n'y perdit pas plus de quinze
cents hommes, mais beaucoup de blessés et de prisonniers.

Murcé était frère de M\textsuperscript{me} de Caylus, aussi disgracié de
corps et d'esprit que sa sœur avait l'un et l'autre charmants. Il était
donc fils de Villette, lieutenant général de mer, cousin germain de
M\textsuperscript{me} de Maintenon, et tous sous sa protection la plus
particulière. Celui-ci était brave, et point mauvais officier, mais
gauche, bête, inepte au dernier point. Il avait avec nous, en Allemagne,
un jeûne valet qui le suivait toujours, qu'il appelait Marcassin, et qui
se moquait de lui à cœur de journée. C'était l'année que
M\textsuperscript{me} la duchesse de Bourgogne vint en France. Il arriva
à Murcé trois grands malheurs dont il se plaignit amèrement à toute
l'armée son cheval isabelle était mort, Marcassin l'avait quitté, et sa
femme n'était point femme d'honneur, il voulait dire dame du palais.
Marivault et Montgon le faisaient valoir\,; c'était une farce
continuelle de le voir avec eux, leurs questions, leur moqueuse
admiration, leurs panneaux et ses sottises. Il avait épousé la fille du
lieutenant général de Chaumont en Bassigny\,; il l'avait menée à
Strasbourg, où il avait été employé l'hiver {[}comme{]} brigadier\,;
elle était laide, sotte et dévote à merveilles\,; il n'y avait qu'un
ménage de gâté. Elle faisait ses dévotions fort souvent, et la veille
voulait coucher seule. Murcé s'en plaignait et rendait compte à tout le
monde du calendrier de sa femme. Il priait à manger chez lui par
grades\,; et un homme de grade différent des conviés qui s'y présentait
quelquefois pour s'en divertir était sûrement éconduit, et Murcé lui en
disait la raison. Tant de fadaises, et d'un Murcé, pourront surprendre
ici, mais voici pourquoi je les ai mises. Murcé était une espèce de La
Feuillade de M\textsuperscript{me} de Maintenon. Elle le croyait un
homme merveilleux\,; il lui rendait compte des choses et des personnes
de l'armée, elle le consultait sur ce qu'il pensait qu'on devait
exécuter. Il montrait souvent de ses lettres qui marquaient en effet une
confiance qui faisait pitié. Il était craint et ménagé, et il a souvent
servi et nui à bien des gens\,; de là on peut juger à qui on avait
affaire, et en grande partie de ce qu'était M\textsuperscript{me} de
Maintenon.

Le 9 septembre, c'est-à-dire le surlendemain de la bataille de Turin,
Médavy marcha avec neuf mille hommes au secours de Castiglione delle
Stivere, que le prince héréditaire de Hesse-Cassel assiégeait avec douze
mille hommes, lequel a depuis été roi de Suède. Il laissa huit cents
hommes dans la ville qu'il avait prise, leva ses quartiers de devant le
château, et vint au-devant de Médavy dans une belle plaine, qui de son
côté marcha aussi à lui. Notre cavalerie, débordée par celle des
ennemis, fut d'abord un peu en désordre\,; il fut augmenté par la fuite
que prirent quatre régiments d'infanterie de Milanais et de
Napolitains\,; Sebret, qui commandait une brigade en seconde ligne, alla
les remplacer sans attendre d'ordre. Médavy fit mettre l'épée à la main
à toute son infanterie\,; elle essuya toute la décharge de l'infanterie
ennemie, la chargea ensuite et la défit entièrement. La cavalerie
ennemie, voyant l'infanterie défaite, s'enfuit. On leur tua deux mille
hommes, on leur en prit quinze cents, tout leur canon et beaucoup
d'étendards et de drapeaux. Médavy y perdit aussi du monde, le chevalier
de Verac, Grammont de Franche-Comté, Renepont, du Cheilar, tous quatre
mestres de camp, et d'Hérouville, colonel d'infanterie, blessé à mort.
Outre ces prisonniers, on eut les huit cents hommes laissés dans la
ville. Médavy fit passer le Mincio au prince de liesse, et le,
poursuivit jusqu'à l'Agide\,; il lui tua encore du monde, prit des
traîneurs dans cette poursuite, et reprit Goïto. Ce fut un étrange
contraste avec Turin, et un grand renouvellement de douleur sur la
retraite en France au lieu de l'avoir faite en Italie. Médavy en fut
fait sur-le-champ chevalier de l'ordre\,; Saint-Pater et Dilon, ses deux
maréchaux de camp, lieutenants généraux\,; Grancey, son frère, qui avait
apporté la nouvelle, maréchal de camp\,; et Sebret, qui apporta le
détail, brigadier.

Sur ce succès, Vaudemont rassembla ce qu'il avait de troupes, manda à
Médavy de le venir joindre avec les siennes, fit mine de vouloir
défendre le Tésin, s'en fit fête par un courrier, et manda que c'était
pour conserver la ville de Milan, qui prétend avoir droit de se rendre
sans blâme à quiconque a passé cette rivière. Vaudemont ajoutait qu'il
avait voulu envoyer Colmenero rendre compte de toutes choses, mais qui
s'était trouvé mal sur le point de partir. Colmenero n'avait garde de
venir. Il avait été gouverneur du château de Milan, l'était d'Alexandrie
alors, et ami intime de Vaudemont. Vendôme l'avait fort vanté au roi\,;
c'était un bon officier, mais dont l'âme était de la trempe de celle de
Vaudemont, et qui le montra bien dans la suite. Toutes ces fanfaronnades
de Vaudemont ne servirent qu'à amuser le roi, qui ne se lassa jamais
d'en être la dupe.

Le prince Eugène, entré dans Turin, et M. de Savoie au comble de sa joie
la plus inespérée de se revoir dans Turin, ne s'amusèrent point aux
réjouissances. Ils ne pensèrent qu'à profiter d'un succès inouï\,; ils
reprirent rapidement toutes les places du Piémont et toutes celles de
Lombardie que nous occupions. Le château de Casal fut leur dernière
conquête. Vaudemont et Médavy, retirés dans Mantoue, ne purent empêcher
ces fruits de la bataille de Turin, et de la retraite de l'armée en
France. Elle était pourtant encore de quatre-vingt-quinze bataillons, en
bon état ceux qui venaient de Lombardie, mais ceux du siège fort
délabrés\,; six régiments de dragons, mais à pied\,; et à l'égard de la
cavalerie, quatre à cinq mille chevaux.

Jamais bataille ne coûta moins de soldats que celle de Turin, jamais de
retraite plus tranquille de la part des ennemis ni laissée plus à choix,
jamais suites plus affreuses ni plus rapides. Ramillies, avec une perte
légère, coûta les Pays-Bas espagnols et partie de ceux du roi, par la
terreur et le tournoiement de tête du seul maréchal de Villeroy, et
celle de Turin coûta toute l'Italie par l'ambition de La Feuillade, la
servitude de Marsin, l'avarice, des ruses, les désobéissances des
officiers généraux contre M. le duc d'Orléans, qui seul voulut et
s'opiniâtra à trois reprises à se retirer en Italie, ce qui était libre,
aisé et d'une suite victorieuse à réparer, plus que le malheur qui
venait d'arriver, vaincu par l'artifice et le concert de La Feuillade et
des officiers généraux, pour n'en rien dire de plus, dont l'audace et
les moyens furent aidés par l'épuisement et les souffrances de la
blessure de M. le duc d'Orléans. On assembla fort diligemment mille
mulets en Provence et en Languedoc pour M. le duc d'Orléans\,; on lui
envoya de l'argent, des chevaux, des armes, huit mille tentes.

Nancré retourné vers M. le duc d'Orléans, qui avait été extrêmement mal
de sa blessure, la nouvelle M\textsuperscript{me} d'Argenton et
M\textsuperscript{me} de Nancré, veuve sans enfants du père de celui
dont je viens de parler, et dans l'intimité la plus étroite avec lui,
s'en allèrent ensemble chacune dans une chaise de poste le plus
secrètement qu'elles purent à Lyon, et de là se cacher dans une
hôtellerie à Grenoble. M. le duc d'Orléans n'y était pas encore arrivé.
Il sut en chemin cette équipée, il en fut très fâché, et leur manda
qu'il ne les verrait point, et de s'en retourner. Être arrivées de Paris
à Grenoble et s'en retourner bredouille était chose fort éloignée de
leur résolution, elles l'attendirent. Savoir sa maîtresse si près de soi
et lui tenir rigueur, l'amour ne le put jamais permettre. Sur les sept
ou huit heures du soir, les affaires du jour vidées et la représentation
finie, il ferma ses portes, s'enfonça dans son appartement, et par les
derrières d'un escalier dérobé arrivèrent les femelles, et soupèrent
avec lui et deux ou trois de leurs plus familiers. Cela dura ainsi cinq
ou six jours, au bout desquels il les renvoya, et repartirent. Ce voyage
ridicule fit grand bruit. Le public en murmura, fâché véritablement de
cette tache sur sa gloire personnelle\,; les envieux, ravis de pouvoir
rompre le silence qu'ils avaient été forcés de garder, parmi lesquels M.
le Duc et M\textsuperscript{me} la Duchesse se signalèrent. Quelque
résolution que j'eusse prise de ne lui parler jamais de ses maîtresses,
il m'avait écrit avec trop d'ouverture, dès que sa blessure le lui avait
permis, pour qu'il me le fût de demeurer dans le silence quand tout
criait si haut. Il reçut ma lettre en même temps qu'une autre que
Chamillart lui écrivit de la part du roi, qui par ménagement n'avait pas
voulu le faire lui-même, pour lui conseiller de renvoyer ces femmes et
l'avertir du mauvais effet de leur voyage. Toutes deux ne furent reçues
qu'après leur départ, lequel en fut toute la réponse.

M. le duc d'Orléans visita ses troupes le plus qu'il put dans leurs
quartiers, quoique mal rétabli encore, et y répandit avec choix beaucoup
d'argent. Il travailla fort à examiner ce qui était possible pour
rentrer en Italie, et envoya Besons bien instruit des moyens et des
difficultés pour en rendre compte au roi, et recevoir ses ordres. Le
fruit de ce voyage fut de ne plus songer à faire repasser l'armée de M.
le duc d'Orléans en Italie, au moins jusqu'au printemps. Besons demeura,
et un simple courrier porta cette résolution finale à M. le duc
d'Orléans, qui, malgré toutes les difficultés qu'il y voyait lui-même,
ne laissa pas d'en être fort touché. Pendant ce temps-là l'Italie s'en
allait par pièces. Chivas, la ville de Casal, Pavie, Pizzighettone,
Alexandrie, etc., s'étaient rendues au duc de Savoie ou au prince
Eugène, qui était dans Milan déclaré gouverneur général du Milanais, et
qui bientôt après fut maître des châteaux de Milan, de Casal et de
Tortone.

On envoya les quartiers d'hiver pour l'armée de M. le duc d'Orléans, et
ce prince arriva à Versailles le lundi 8 novembre, sur la fin du dîner
du roi, qui avait pris médecine, et dînait dans son lit à deux heures et
demie, comme il faisait toujours les jours qu'il la prenait. On ne peut
être mieux reçu du roi qu'il le fut, et de tout le monde. Il fut voir
monseigneur aussitôt après à Meudon, et soupa avec le roi à l'ordinaire.

Dès qu'il fut ce jour-là même débarrassé du plus gros, j'allai chez lui.
Nancré me saisit en y entrant, et, sans me donner un instant, se mit à
se disculper d'avoir conseillé et machiné ce misérable voyage de ces
deux femmes. Il suivit M. le duc d'Orléans, qui me menait dans son
entresol, et voulut encore s'en laver devant moi en sa présence. Je le
croyais trop sensé pour l'avoir fait, mais le monde n'en avait pas jugé
de même. Ce fut alors que M. le duc d'Orléans me remercia avec effusion
de cœur de la franchise avec laquelle je lui avais écrit sur ce voyage.
Il m'avoua que fâché d'abord, puis tenté les sachant en même lieu que
lui, il avait succombé avec les précautions que j'ai rapportées. «\,Et
voilà, monsieur, lui répondis-je, la sottise, en l'interrompant. --- Il
est vrai, me répliqua-t-il, mais qui est-ce qui n'en fait jamais\,?»

Nancré sortit, et, la porte fermée, nous entrâmes bien avant en matière.
Je le mis au fait des choses de la cour qui le regardaient, et de l'état
présent du reste que les lettres, bien que chiffrées, n'avaient pu
comporter. Lui ensuite me parla en gros des choses principales d'Italie,
parce que, réciproquement affamés, nous ne pouvions encore tomber aux
détails que nous discutâmes depuis. Il me fit une étrange peinture des
officiers généraux de son armée, telle en tous points que j'ai tâché de
la rendre, mais plus affreuse encore, et des malheurs, pour en parler
sobrement, qui, entassés les uns sur les autres, avaient causé tous ceux
de Turin. Il me représenta La Feuillade comme un jeune homme impérieux,
enivré de présomption et d'ambition sans mesure, détesté des officiers
généraux et particuliers, des troupes et du pays\,; plein d'esprit, de
valeur, de fantaisies et de vues, qui voyant beaucoup d'abord était
incapable aussi de rien voir au delà de ce premier coup d'œil, de
souffrir aucun avis de personne bien loin de se rendre jamais sur rien,
par conséquent incapable d'apprendre jamais d'autrui, et fort peu de
soi-même, parce que l'action chez lui précédait toujours la réflexion\,;
brillant sans nulle solidité, dangereux à l'excès à la tête de quelque
chose, se piquant surtout de savoir mieux toutes choses que les gens du
métier. Ce prince ajouta qu'il le croyait perdu, de la manière dont le
roi lui en avait parlé, et dont il lui paraissait qu'il le connaissait.
Il me dit qu'il avait fait son possible pour pallier ses fautes, encore
qu'elles fussent énormes, et telles que je les ai expliquées, et qu'il
ne se fût pas mis en état de le mériter, mais qu'il avait cru devoir
rendre ce change à son beau-père\,; que le roi l'avait même grondé de
l'avoir trop excusé, et que cet article était le seul sur lequel il lui
eût parlé d'un air aigre et sévère. Il ajouta qu'il avait laissé La
Feuillade en Dauphiné, dans l'espérance que ses lettres, soutenues de
ses bons offices à son arrivée, lui en conserveraient le commandement\,;
que Chamillart, qui n'osait trop en parler au roi, l'avait prié d'y
insister, mais qu'il n'avait osé aller trop avant là-dessus, après ce
que le roi lui avait dit, de manière qu'il était persuadé que La
Feuillade allait être rappelé. Diverses autres conversations semblables
m'instruisirent à fond, et je ne laissai pas de l'être aussi par
quelques-uns des officiers généraux et particuliers, à leur arrivée de
cette armée.

Il faut achever tout de suite ce qui la regarde. On ne fut pas longtemps
à quitter toute pensée de retour en Italie. On ne songea plus qu'à une
défensive nécessaire vers les Alpes, et à grossir l'armée d'Espagne de
ce qui se tirerait de celle-ci pour essayer d'y recouvrer quelque
supériorité. Peu de jours après ce retour, La Feuillade reçut ordre de
revenir, et Giraudan, lieutenant général, de commander en sa place en
Savoie et en Dauphiné, avec deux maréchaux de camp sous lui, Vallière à
Chambéry, et Muret à Fenestrelle. Quelque peu d'apparence qu'il y eût à
le laisser à Grenoble, cet ordre lui fut si amer, que pour n'omettre
aucune sorte de sottise, de folie et d'audace, il se mit dans la tête de
le faire révoquer, dépêcha courriers sur courriers à son beau-père, et
s'y cramponna quinze jours durant, jusque-là que le roi {[}fut{]} outré
de cette lenteur à lui obéir\,; et Chamillart, dans le dernier embarras,
ne savait plus que devenir. Enfin un dernier courrier qu'il lui dépêcha
le fit partir, au grand contentement de la ville et de la province, dont
il n'avait pas acquis les cœurs. Dès en y arrivant la première fois, il
s'était brouillé avec le cardinal Le Camus, qui, sur une mascarade assez
étrange qu'il donna, fut sur le point de l'excommunier dans toutes les
formes solennelles. Il fallut des ordres réitérés du roi pour l'en
empêcher, et à La Feuillade de se conduire d'une autre sorte.

Il fut plusieurs jours à Paris sans oser venir à Versailles. Chamillart
obtint enfin du roi la permission pour lui de le saluer, et même chez
M\textsuperscript{me} de Maintenon, pour éviter la réception publique,
et par un reste de traitement de général d'armée, desquels il arriva le
dernier, le lundi 13 de décembre. Chamillart, allant travailler avec le
roi chez M\textsuperscript{me} de Maintenon, l'y mena. Sitôt que le roi
le vit entrer avec son gendre en laisse, il se leva, alla à la porte,
et, sans leur donner le temps de prononcer un mot, dit à La Feuillade
d'un air plus que sérieux\,: «\,Monsieur, nous sommes bien malheureux
tous deux\,;» et dans l'instant tourna le dos. La Feuillade, de dedans
la porte qu'il n'avait pas eu loisir de dépasser, ressortit
sur-le-champ, sans avoir osé dire un seul mot. Jamais depuis le roi ne
lui parla\,; il fut longtemps même à permettre à Monseigneur de le mener
à Meudon, et à souffrir qu'il allât à Marly à cause de sa femme. On
remarquait qu'il détourna toujours les yeux de dessus lui. Telle fut la
chute de ce Phaéthon. Il vit bien qu'il n'avait plus d'espérance\,; il
vendit ses équipages, et dit assez publiquement, oubliant apparemment
qu'il avait voulu aller sous Médavy, et ce qu'il avait dit et écrit
là-dessus, qu'après avoir commandé les armées, il ne pouvait plus servir
en ligne de lieutenant général\,; et toutefois dans cet état de
disgrâce, il n'y eut sorte de moyens qu'il ne tentât, de bassesses qu'il
ne fît pour se raccrocher. Il eut celle de se plaindre de son sort et de
faire son apologie à chacun qui ne s'en souciait guère, et après s'être
fait envier et craindre, il se fit mépriser sans faire pitié. Je ne
crois pas qu'il y ait eu de plus folle tête, ni de plus radicalement
malhonnête homme jusque dans les moelles des os. Retournons maintenant à
ce qui est demeuré en arrière pour ne pas interrompre le récit de toute
cette catastrophe d'Italie, qui suivit de bien près celle de Barcelone
et de Flandre.

La fantaisie avait pris à l'électeur de Cologne d'aller voyager à Rome.
Il n'avait plus d'États à lui où se tenir\,; il aimait mieux se promener
que le séjour de nos villes de Flandre. Il arriva donc à Paris, au
milieu de septembre, tout à fait incognito, et logea chez son envoyé.
Dix ou douze jours après, il alla dîner chez Torcy, à Versailles, puis
attendre l'heure de son audience dans l'appartement de M. le comte de
Toulouse. Il ne voulut point être accompagné de l'introducteur des
ambassadeurs. Torcy le mena dans le cabinet du roi par les derrières,
suivi des trois ou quatre de sa suite les plus principaux. Les
courtisans ayant les entrées, qui voulurent, étaient dans le cabinet
avec Monseigneur et Mgrs ses fils. Le roi, toujours debout et découvert,
le reçut avec toutes les grâces imaginables, et en lui nommant ces trois
princes, ajouta\,: «\,Voilà votre beau-frère, vos neveux et moi, qui
suis votre proche parent\,; vous êtes ici dans votre famille.\,» Après
un peu de conversation, il le mena par la galerie chez
M\textsuperscript{me} la duchesse de Bourgogne, qui le reçut debout, et
qu'il ne salua point, à cause de la présence du roi devant qui elle ne
baise personne. Il fut ensuite chez Madame, qui s'avança au-devant de
lui dans sa chambre. Elle le baisa et causa fort longtemps avec lui en
allemand. Il vit après M\textsuperscript{me} la duchesse d'Orléans dans
son lit, qui le baisa. La visite fut courte. Il ne s'assit nulle part.
De là il alla faire un tour dans les jardins, et partit de chez Torcy
pour s'en retourner à Paris. Huit jours après, il vint de Paris entendre
la messe du roi dans une autre travée de la tribune, et le vit après
seul dans son cabinet, avant le conseil. Il se promena dans les jardins
jusqu'au dîner chez Torcy. Il vit ensuite M\textsuperscript{me} la
duchesse de Bourgogne, qui était au lit. Mgr le duc de Bourgogne s'y
trouva, et, contre l'ordinaire de ces sortes de visites, la conversation
fut vive et soutenue, toujours debout l'un et l'autre. Peu de jours
après, il vit encore le roi dans son cabinet, se promena dans les
jardins, s'amusa dans le cabinet des médailles, dîna chez M. de
Beauvilliers, et s'en retourna à Paris. La semaine suivante, il revint
voir le roi dans son cabinet avant le conseil. Le maréchal de Boufflers
lui donna à dîner, d'où il alla chez M\textsuperscript{me} la duchesse
de Bourgogne, et y eut une longue conférence avec Mgr le duc de
Bourgogne, debout, en un coin de la chambre. Avant de retourner à Paris,
il fut voir M. le duc de Berry.

De ce voyage, il changea son dessein d'aller à Rome, où, pour son rang
avec les cardinaux et sa personne, dans la situation où il était avec
l'empereur, et nos troupes hors d'Italie, au corps de Médavy près, il
n'aurait pu être que fort indécemment. Le roi lui prêta pour une nuit
l'appartement du duc de Grammont, qui était à Bayonne. Torcy, chez qui
il avait dîné à Paris, le mena voir Trianon et lui donna à souper à
Versailles, puis le mena par le petit degré droit dans le cabinet du
roi, où il le trouva sortant de table avec ce qui de sa famille y était
à ces heures-là, privance qui n'avait jamais encore été accordée à
personne, et dont il fut fort touché. Le roi lui dit qu'il voulait qu'il
le vît au milieu de sa famille, où il n'était point étranger, et dans
son particulier. Il avait à son cou une croix de diamants très belle
pendue à un ruban couleur de feu, qu'avant souper Torcy lui avait
présentée de la part du roi. Il prétendait pouvoir porter l'habit des
cardinaux comme archichancelier de l'empire pour l'Allemagne. Il était
vêtu de court, en noir, souvent avec une calotte rouge, quelquefois
noire. Les bas variaient de même. Il était blond, avec une fort grosse
perruque et assez longue, cruellement laid, fort bossu par derrière, un
peu par devant, mais point du tout embarrassé de sa personne ni de son
discours. Il prit tout à fait bien avec le roi, qui, le lendemain, le
vit en particulier après la messe. Après, il suivit le roi à la chasse.
L'électeur y était dans une calèche avec un de sa suite, le premier
écuyer et Torcy. Il retomba après à Marly, où il prit congé du roi pour
retourner en Flandre. Il alla voir l'électeur de Bavière à Mons, et
revint s'établir à Lille. Il avait, quelques jours auparavant, dîné à
Meudon avec Monseigneur, qui seul eut un fauteuil, et l'électeur
vis-à-vis de lui avec M. le prince de Conti au milieu des dames.

La mort de Saint-Pouange arriva tout à propos pour donner le plaisir au
roi de marquer que la disgrâce du gendre n'influait point sur le
beau-père. J'ai assez parlé ailleurs de Saint-Pouange pour n'avoir rien
à y ajouter. Il était grand trésorier de l'ordre\,; le roi décora
Chamillart de cette charge.

M\textsuperscript{me} de Barbezieux mourut à Paris après une longue
infirmité et fort jeune. Ses malheurs n'avaient point cessé depuis son
éclat avec son mari, dont la mort ne put là remettre dans le monde. Elle
ne laissa que deux filles, toutes deux mortes fort jeunes\,: l'une
duchesse d'Harcourt qui a laissé des enfants\,; l'autre, troisième femme
de M. de Bouillon, père de celui d'aujourd'hui. Elle laissa un fils
unique, mort bientôt après, de sorte que la duchesse d'Harcourt hérita
presque de tout, et leur grand-père d'Alègre de fort peu de chose.

Le vieux Boisfranc mourut aussi à quatre-vingt-sept ou quatre-vingt-huit
ans. Il était beau-père du duc de Tresmes, avec qui il demeurait. J'ai
dit ailleurs ce que c'était que ce riche financier.

Le roi donna à Maréchal la survivance de sa charge de premier chirurgien
pour son fils qui travaillait dans les hôpitaux de l'armée de Flandre.
C'était un paresseux qui ne promettait pas d'approcher de son père. Le
roi qui le sentait ne put s'empêcher de dire à ses valets que si le fils
ne se rendait pas bien capable, cela ne l'empêcherait pas de prendre un
autre chirurgien s'il perdait le père. Cette parole qui fut bientôt sue
fit grand'peur à tous les survivanciers, à pas un desquels il n'est
pourtant arrivé malheur, excepté à quelques secrétaires d'État, et comme
je l'ai dit, au fils de Congis pour les Tuileries.

Il eut une complaisance pour le P. de La Chaise tout à fait marquée. Ce
père, qui était gentilhomme, voulait être homme de qualité. Son frère,
d'écuyer de l'archevêque de Lyon, puis de commandant son équipage de
chasse, était devenu capitaine des gardes de la porte du roi par le
confesseur, et son fils avait eu sa charge après lui. Il avait épousé
une du Gué-Bagnols, riche, d'une famille de robe de Paris. Le P. de La
Chaise se mourait de douleur de ne pouvoir obtenir qu'elle allât à
Marly, et le roi, malgré son faible pour lui, ne se pouvait résoudre à
faire manger sa nièce avec M\textsuperscript{me} la duchesse de
Bourgogne, et à la faire entrer dans ses carrosses. Il arriva cette
année que le roi voulant aller faire la Saint-Hubert à Marly, la
grossesse de M\textsuperscript{me} la duchesse de Bourgogne l'empêcha de
pouvoir être du voyage, qui, à cause de cela, ne fut que du mercredi au
samedi, et qu'en même temps Madame se trouva si enrhumée qu'elle n'y put
aller. Le roi trouva que c'était là son vrai ballot, qu'il ne trouverait
de longtemps, et le saisit. Il nomma donc M\textsuperscript{me} de La
Chaise pour Marly, à qui, par conséquent, cela n'acquit aucun droit pour
manger ni pour les carrosses, et qui aussi, n'y fut jamais admise. Mais
cette délicatesse n'était pas aperçue de tous, au lieu qu'aller à Marly
se sut partout. Le P. de La Chaise fut ravi. Cette adresse fut un
nouveau crève-cœur pour Saint-Pierre, dont la femme ne put même en cette
sorte parvenir à aller à Marly, et un peu de dépit à
M\textsuperscript{me} la duchesse d'Orléans de pouvoir moins pour la
femme de son premier écuyer si hautement portée par elle que le P. de La
Chaise pour sa nièce.

Ce Marly produisit une querelle assez ridicule. Il faisait une pluie qui
n'empêcha pas le roi de voir planter dans ses jardins. Son chapeau en
fut percé, il en fallut un autre. Le duc d'Aumont était en année, le duc
de Tresmes servait pour lui. Le portemanteau\footnote{Dans l'ancienne
  monarchie, il y avait douze officiers \emph{portemanteaux} attachés à
  la maison du roi. Leurs fonctions consistaient à garder le chapeau,
  les gants, la canne et l'épée du roi et à les lui présenter lorsqu'il
  les demandait. Un de ces officiers suivait toujours le roi à la chasse
  avec un portemanteau garni de linge, tel que chemises, mouchoirs, etc.}
du roi lui donna le chapeau, il le présenta au roi. M. de La
Rochefoucauld était présent. Cela se fit en un clin d'œil. Le voilà aux
champs, quoique ami du duc de Tresmes. Il avait empiété sur sa charge,
il y allait de son honneur. Tout était perdu. On eut grand'peine à les
raccommoder. Leurs rangs, ils laissent tout usurper à chacun, personne
n'ose dire mot\,; et pour un chapeau présenté, tout est en furie et en
vacarme. On n'oserait dire que voilà des valets.

Pendant ce même Marly, Mgr le duc de Bourgogne cessa d'aller à la
musique, quoiqu'il l'aimât fort, et vendit les pierreries qu'il avait
eues de feu M\textsuperscript{me} la Dauphine (et il en avait beaucoup)
dont il fit donner tout l'argent aux pauvres. Il n'allait plus à la
comédie depuis quelque temps.

Le roi de Suède triomphant en Pologne, où il avait fait un roi à son
gré, écarté les Moscovites et réduit l'électeur de Saxe à une abdication
dans toutes les formes, mena son armée en Saxe, dont outre la
subsistance il tira des trésors. Dresde, Leipzig, toute la Saxe subit le
joug\,; la souveraine se retira à Bayreuth chez son père. La paix signée
en secret, le roi Auguste, forcé par le reste de son parti en Pologne à
qui il n'avait osé l'avouer, attaqua un corps de Suédois commandé par le
général Mardefeld, fort inférieur, qu'il défit. Mardefeld y perdit trois
mille hommes, et se retira en Silésie, dont l'empereur n'osa se fâcher.
Là-dessus le roi de Suède éclata comme contre un manque de foi insigne.
C'est ce qui lui fit imposer au roi Auguste les conditions les plus
humiliantes, et achever de ruiner ses pays par tout ce qu'il en exigea.
Il dicta la paix par laquelle, outre beaucoup d'autres détails, il le
fit consentir à abandonner tout ce qu'il lui restait de partis, et la
Pologne avec la Lituanie à Stanislas, à en quitter le titre et ne porter
plus que celui de roi-électeur, de souffrir toute l'armée suédoise en
Saxe, aux dépens du pays jusqu'au mois de mai, c'est-à-dire six grands
mois encore, de livrer ce qu'il avait en Saxe de troupes moscovites et
de renoncer à toute alliance avec le czar, de remettre en liberté les
deux Sobieski, fils du feu roi de Pologne, enfin de lui envoyer pieds et
poings liés le général Patkul, auquel incontinent après il fit couper
publiquement la tête.

Ce Patkul était passé en Pologne sur ce que, étant député à Stockholm de
la noblesse de Livonie poussée à bout par la chambre des révisions qui
ruina la Suède sous le précédent règne et en anéantit l'ancienne
noblesse, et dont les exactions, et ceux qui les exerçaient étaient
encore plus insupportables, il avait parlé avec tant de liberté qu'il
avait été obligé de s'enfuir. C'était un homme de tête, de ressource et
de grand courage, qui était fort suivi et fort accrédité dans son pays,
lequel était outré contre la Suède, et plus encore contre ses ministres.
Patkul, n'espérant plus de sûreté sous cette domination, ne songea qu'à
se venger de la Suède. Il persuada, au roi Auguste d'entrer en Livonie
et d'y appeler les Moscovites. Le succès répondit à ce qu'il s'en était
proposé. Aucun général ennemi ne nuisit plus que lui aux Suédois. Il en
encourut une haine si personnelle que le roi de Suède ne voulut point de
paix qu'avec une condition expresse qu'il lui serait livré. Il le fut,
il lui en coûta la vie sur un échafaud, et au roi de Suède un
obscurcissement à sa gloire. Elle lui avait dressé un tribunal en Saxe
qui imposa des lois à tout le Nord, à une partie très vaste de
l'Allemagne, à l'empereur même, qui n'osa lui rien refuser et à qui il
demanda des restitutions et d'autres choses fort dures. Il était en
posture d'être le dictateur de l'Europe et de faire faire la paix à son
gré sur la succession d'Espagne\,; toutes les puissances en guerre
avaient recours à lui. Il était mieux avec la France et plus enclin à
elle qu'à pas une des autres, qui toutes, malgré leurs succès contre la
France, le craignirent ainsi placé en Allemagne, au point d'en passer
par tout ce qu'il eût voulu plutôt que de risquer de l'y voir avancer
avec son armée et se déclarer contre elles. Les plus grands rois sont
malheureux. Piper était son unique ministre qui l'avait toujours
suivi\,; il avait toute sa confiance. Tout occupé de troupes, de
subsistances, de guerre, il ne donnait aux affaires d'État qu'une
attention superficielle, emporté par cette passion de héros et par
l'amour de la vengeance. L'empereur et l'Angleterre gagnèrent Piper à
force d'argent et d'autres promesses. Piper vendu de la sorte, se servit
de ces deux passions de son maître pour le tirer de Saxe et le faire
courir après le czar pour le détrôner comme il avait fait le roi
Auguste. Rien ne le put détourner d'une si hasardeuse folie. L'objet et
le péril qui y était attaché furent pour lui un double attrait. Piper
l'y nourrit et l'y précipita. Le traître y périt dans les cachots des
Moscovites\,; et son maître, qui ne s'en sauva que par des miracles, et
qui en fit depuis du plus grand courage de cœur et d'esprit, ne fit que
palpiter depuis, et ne figura plus en Europe, comme on le verra en son
temps.

Bonac, qui était à Dantzig chargé des affaires du roi en Pologne, eut
ordre d'aller reconnaître et complimenter de sa part le nouveau roi
Stanislas, qui fut reconnu de l'empereur et de presque toute l'Europe.
Cromstrom, envoyé de Suède, avait donné part au roi, de la part du roi
de Suède et de celle du roi Stanislas dont il avait reçu une lettre de
créance, de son avènement à la couronne de Pologne et de l'abdication du
roi Auguste, électeur de Saxe.

Les mécontents inquiétaient toujours extrêmement l'empereur qu'ils
pensèrent prendre à la chasse, à deux lieues de Vienne, où ils brûlèrent
des villages. Ils avaient pris Gratz, qui fut repris sur eux, sans
qu'ils fissent pour cela une diversion moins embarrassante. Ils finirent
l'année par battre le général Heusler et lui tuer quatre mille hommes.

Le mariage de l'archiduc fut arrêté à la fin d'octobre avec la princesse
de Wolfenbüttel, de même maison que l'impératrice régnante lors, et que
le duc d'Hanovre, depuis roi d'Angleterre. Elle était luthérienne, et on
l'instruisit pour embrasser la religion catholique. Les protestants
croient que les catholiques se sauvent dans leur religion\,; ils l'ont
avoué longtemps, et ne l'ont nié depuis que pour se dérober à la force
de l'argument qui s'en tire contre eux. Quand je dis protestants,
j'entends luthériens et calvinistes. C'est cette persuasion qu'ils
conservent qui les rend faciles à embrasser et à faire embrasser la
religion catholique à leurs enfants, quand ils y trouvent des avantages,
principalement pour des mariages qui ne se pourraient pas faire
autrement\,; et la raison contraire fait qu'il n'y a point d'exemple
d'aucun prince catholique qui se soit fait protestant, ni qui l'ait
souffert à ses enfants, pour quelque mariage ou quelque autre avantage
que ç'ait pu être.

La campagne finit en Espagne, après beaucoup de petites places rendues
ou emportées, par la prise de Carthagène. La garnison, qui n'était que
d'un régiment de cavalerie et un d'infanterie, avec trois mille paysans
armés, sous un maréchal de camp espagnol, se rendit au duc de Berwick
prisonnière de guerre, et la vie sauve seulement aux bourgeois. Il s'y
trouva soixante-quinze pièces de canon, dont trente de fonte et trois
mortiers. Bey prit quelques jours après Alcantara par escalade, sur une
garnison aussi nombreuse que ses troupes, dont il ne perdit que trois ou
quatre soldats. Il trouva tout le canon qu'on y avait perdu Après ces
exploits, les armées se séparèrent et entrèrent en quartier d'hiver.
Presque toutes ces conquêtes furent rançonnées, et valurent beaucoup
d'argent comptant au roi d'Espagne. Peterborough, qui voltigeait souvent
d'Angleterre en Espagne, en Italie, en Portugal et par toute l'Europe,
porta en ce même temps un secours de cent cinquante mille pistoles à
l'archiduc dans le royaume de Valence, des contributions que le prince
Eugène venait de tirer du Milanais et des pays voisins. Le roi, en ce
même temps, fit entrer le duc d'Albe dans son cabinet après sa messe
avant le conseil. Il lui dit qu'il avait cru devoir faire proposer des
conférences aux ennemis pour établir une bonne paix\,; qu'ils les
avaient refusées\,; qu'ainsi il ne fallait plus songer qu'à la guerre,
et l'espérer plus heureuse la campagne prochaine qu'elle ne l'avait été
celle-ci. Le duc d'Albe, qui, dans la situation d'alors, craignait fort
ces conférences, sortit du cabinet du roi extrêmement soulagé.

Ce qu'il y avait d'Impériaux à Hagenbach sous Thungen ayant repassé le
Rhin à la mi-novembre, Villars sépara son armée pour entrer en quartiers
d'hiver. Il fit un tour sur la Sarre pour en visiter les places, et
arriva à la cour les premiers jours de décembre. Le duc de Noailles
revint en même temps de Roussillon.

Le prince de Rohan étant arrivé des premiers de Flandre, le roi
l'entretint longtemps dans son cabinet sur la bataille de Ramillies et
ses suites. On ne put attribuer cette confiance qu'à sa qualité de fils
de M\textsuperscript{me} de Soubise. Il s'y était comporté avec
valeur\,; mais c'était un homme à qui il n'en fallait pas demander
davantage. Il savait moins de guerre que de cour, où avec un esprit fort
médiocre il avait merveilleusement profité des leçons de son habile
mère.

Surville était sorti de la Bastille à la fin du temps que les maréchaux
de France avaient ordonné, et le roi avait mandé au duc de Guiche de
ramener La Barre de l'armée avec lui. Il le lui présenta en arrivant, et
tout de suite le roi le fit entrer dans son cabinet. Là, il lui dit
qu'il avait eu un démêlé avec Surville, où il n'avait aucun tort\,; que
Surville avait été puni\,; que lui était un vieil officier dont la
réputation était établie depuis fort longtemps\,; qu'ainsi il lui
demandait, comme à son ami, qu'il lui sacrifiât son ressentiment, et si
cela ne suffisait pas, comme roi et comme son maître\,; mais qu'il
croyait qu'il aimerait mieux s'en tenir à la première partie, et qu'il
désirait qu'il le fît de bonne grâce, lorsqu'ils seraient accommodés par
les maréchaux de France. On peut juger quelle fut la réponse et la
conduite de La Barre à un discours aussi rare -dans la bouche d'un grand
roi, et à un petit particulier de sa sorte. Les maréchaux de France les
accommodèrent huit jours après, mais Surville demeura perdu.

M\textsuperscript{me} de Châtillon, dame d'atours de Madame, demanda à
se retirer. Elle conserva mille écus de deux mille qu'elle avait, ses
logements du Palais-Royal et de Versailles, et une place de dame de
Madame, comme la maréchale de Clérembault et la comtesse de Beuvron en a
voient eu depuis la mort de Monsieur. Elle était sœur cadette de la
duchesse d'Aumont, et se piquèrent toute leur vie d'une union intime\,:
toutes deux du nom de Brouilly, filles du marquis de Piennes, chevalier
de l'ordre en 1661, mort gouverneur de Pignerol en 1676, n'ayant laissé
que ces deux filles d'une Godet des Marais, ce qui, dans la faveur de M.
de Chartres, Godet des Marais aussi et leur oncle, leur servit fort
auprès de M\textsuperscript{me} de Maintenon. C'étaient deux fort
grandes personnes, les mieux faites de la cour\,; M\textsuperscript{me}
d'Aumont plus belle, M\textsuperscript{me} de Châtillon, sans beauté,
bien plus aimable\,; toutes deux mariées par amour. M. de Châtillon, qui
était l'homme de France le mieux fait, et dont la figure fit sa fortune
chez Monsieur, en obtint, malgré Madame, cette place de dame d'atours
quand M\textsuperscript{me} de Durasfort mourut, qui l'avait été lorsque
M\textsuperscript{me} de Gordon la quitta, qui l'avait été auparavant de
feu Madame\,; et pour tout accommoder, le roi permit que Madame eût une
seconde dame d'atours, laquelle voulait opiniâtrement
M\textsuperscript{me} de Châteauthiers, une de ses filles d'honneur, que
cette place fit appeler madame. L'amour ne dura que peu d'années entre
M. et M\textsuperscript{me} de Châtillon. Ils se brouillèrent et se
séparèrent avec éclat, et quoique dans la nécessité de passer leur vie
dans les mêmes lieux par leurs charges, et de se rencontrer tous les
jours, ils ne se raccommodèrent jamais. M\textsuperscript{me} de
Châtillon n'avait jamais été trop bien avec Madame. Elle était
extrêmement du grand monde et importunée de l'assiduité. Avec un esprit
médiocre, elle prétendait en avoir beaucoup, et devenait ridicule en
étalant du bien-dire et de l'écorce de science tant qu'elle pouvait\,;
flatteuse, moqueuse et méchante. Elle et sa sœur étaient bien avec
Monseigneur et fort des amies de M\textsuperscript{me} la princesse de
Conti de tout temps. Jamais on ne vit un plus beau couple ni de si grand
air que M. et M\textsuperscript{me} de Châtillon.

Livry, qui avait quatre cent mille livres de brevet de retenue sur sa
charge de premier maître d'hôtel du roi, en eut soixante mille livres
d'augmentation et la survivance de capitainerie de Livry pour son fils
en le mariant à la fille du feu président Robert. Desmarets, grand
fauconnier, avait épousé l'autre. Ce président Robert, qui l'était de la
chambre des comptes, était fort proche parent de M. de Louvois,
longtemps intendant d'armée, homme d'esprit, capable et d'honneur, mais
qui aimait tant son plaisir que M. de Louvois n'en put rien faire.
C'était le plus gros et le plus noble joueur du monde, et l'homme de sa
sorte le plus mêlé avec la meilleure compagnie. Il était mort il y avait
longtemps.

M. de Beauvilliers avait deux frères du second mariage de son père,
qu'il avait élevés avec ses enfants, et qui étaient tous quatre à peu
près de même âge. L'aîné voulut être d'Église, et y voulut persévérer
lorsque les deux fils de M. de Beauvilliers moururent. Le cadet était à
Malte pour faire ses caravanes\,; M. de Beauvilliers, qui n'avait plus
que lui, l'en fit revenir pour en faire désormais son fils unique. Il
arriva\,; M. et M\textsuperscript{me} de Beauvilliers conjointement lui
firent de grandes donations, et M. de Beauvilliers lui céda son duché,
lui fit prendre le nom de duc de Saint-Aignan et le maria à la fille
unique de Besmaux, extrêmement riche. Sa mère était fille de Villacerf,
son père était mort jeune. Besmaux, père de celui-là, était un
gentilhomme gascon qui avait été capitaine des gardes du cardinal
Mazarin, et depuis très longtemps gouverneur de la Bastille, où il
s'était extrêmement enrichi. Il avait toujours conservé de la
considération du roi et de la confiance personnelle. Avant qu'être
riche, il avait marié sa fille à Saumery, sous-gouverneur des princes,
parla protection et le choix de M. de Beauvilliers. C'est celle dont
j'ai parlé à l'occasion de M. de Duras. Sa nièce, héritière sans père ni
mère et le vieux Besmaux mort il y avait longtemps, dépendait de sa
tante paternelle et de Villacerf, premier maître d'hôtel de
M\textsuperscript{me} la duchesse de Bourgogne, son oncle maternel.

Le mariage fut donc bientôt fait. M. et M\textsuperscript{me} de
Beauvilliers les prirent chez eux à Versailles comme leurs enfants\,;
M\textsuperscript{me} de Beauvilliers les traita de même. La conduite
toujours suivie qu'elle eut avec eux fut le chef-d'œuvre de l'amitié
conjugale. Elle se livra à cette éducation avec un courage héroïque. Je
l'ai vue bien des fois, étant seul avec elle les soirs, les envoyer
chercher sur le point que le plus court et le plus intime particulier
allait arriver pour souper, que les grosses larmes lui tombaient des
yeux, m'avouer ce que lui coûtait le souvenir de la mort de ses enfants,
renouvelé à tous moments par le fils et la belle-fille postiches\,; puis
recogner ses larmes, pour qu'on ne s'en aperçut point, eux surtout, me
les louer, dire que ce n'était pas leur faute si elle avait perdu ses
enfants\,; que, si ce n'était pas une ressource pour elle, c'en était
toujours une pour M. de Beauvilliers, ce qui était tout pour elle\,; et
dès qu'ils arrivaient, leur faire cent caresses et toutes les amitiés
possibles. Elle les traita toute sa vie comme ses véritables enfants et
les mieux aimés, avec un intérêt en eux et des soins qui ne se peuvent
exprimer\,; M. de Beauvilliers de même. Toutes ces dispositions se
firent de concert avec M. de Mortemart et M\textsuperscript{me} sa mère,
pour ne préjudicier point aux droits de sa femme, fille de M. et de
M\textsuperscript{me} de Beauvilliers, qu'ils ne conservèrent que trop
scrupuleusement.

Bergheyck arriva de Flandre sur la fin de novembre. Chamillart le logea,
le défraya et le présenta le soir au roi, chez M\textsuperscript{me} de
Maintenon. D'abord baron, puis comte (à dire vrai, ni l'un ni l'autre
qu'à la mode de nos ministres), c'était un homme de Flandre et de
meilleure famille qu'ils ne sont d'ordinaire, qui avait travaillé dans
les finances des Pays-Bas sur la fin de Charles II, que l'électeur de
Bavière y trouva fort employé, et qu'il y continua à la mort du roi
d'Espagne. Sa capacité et sa droiture donna confiance en lui\,; sa
fidélité et son zèle y répondirent, avec beaucoup d'esprit, de sens, de
lumière, de justesse, une grande facilité de travail et d'abord,
beaucoup de douceur avec tout le monde et dans la manière de gouverner,
une grande modestie, un entier désintéressement et beaucoup de vues. Il
se pouvait dire un homme très rare, et qui avait une connaissance
parfaite non seulement des finances, mais de toutes les affaires des
Pays-Bas, et de tout ce qui y était et pouvait y être employé\,; avec
tous ces talents grand travailleur et fort appliqué, et qui avait une
exactitude et une simplicité en tout singulière. Il fut bientôt mis au
timon des affaires de ces pays-là pour l'Espagne.

C'était un homme qui ne s'avançait jamais, qui ne parlait jamais aussi
contre sa pensée, mais ferme dans ses avis, et qui les mettait en tout
leur jour, obéissant après qu'il avait dit toutes ses raisons, tout
comme s'il les eût suivies et non pas des ordres contraires ou
différents de ce qu'il avait cru et exposé comme meilleur. Il fut
longtemps en première place. Il vécut plusieurs années content et retiré
depuis l'avoir quittée, et ne se mêlant plus de rien\,; fort homme de
bien, point du tout riche, et n'ayant jamais rien fait pour sa famille.
On aurait tiré de lui de grands et d'utiles services si on l'avait
toujours cru, surtout sur les fins, et qu'on s'en fût servi jusqu'au
bout de sa longue et intègre vie. Il fut peu à Versailles et point à
Paris, travailla fort avec Chamillart, et vit le roi en particulier avec
lui et tête à tête. Chamillart l'aimait fort et tous nos ministres et
nos généraux, et le roi le traitait avec amitié et distinction. Il ne
paraissait point en public dans les divers voyages qu'il fit à la cour.
Même dans sa retraite il conserva beaucoup de considération en Flandre,
où il fut universellement aimé, estimé, honoré et regretté. Ce sont de
ces trésors que les rois savent rarement connaître, et dont il est plus
rare encore qu'ils ne se dégoûtent pas. Ses voyages ici étaient rares et
toujours fort courts.

M. de Vendôme, après avoir visité les places maritimes de Flandre et
tout ce voisinage de la mer, arriva à Versailles les premiers jours de
décembre, et entretint le roi longtemps. Il fut bien reçu parce qu'il
était M. de Vendôme, mais la différence fut entière d'avec ses deux
derniers retours. Ce restaurateur n'avait rien redressé en Flandre, il y
avait laissé faire aux ennemis tout ce qu'ils avaient voulu. On ne
revenait point d'Italie et on revenait de Flandre. Ceux qui en
arrivaient n'avaient point reconnu le héros auquel ils s'étaient
attendus\,: ils n'y avaient trouvé que hauteur démesurée, propos en tout
genre qui l'étaient encore plus, mais qui ne tenaient rien, une paresse
qui allait jusqu'à l'incurie, une débauche qui étonnait les moins
retenus. Réunis avec ceux qui revenaient d'Italie, ils ne se trouvèrent
pas de différents avis. Le masque tomba\,; mais comme le roi, toujours
prévenu et voulant encore plus l'être donnait le ton à tous, que les
appuis de Vendôme étaient connus et craints, et que le nombre des sots
et des gens bas est toujours le plus grand, Vendôme, déchu de tout en
effet, demeura toujours héros en titre. Son frère ne fut pas longtemps à
Rome sans s'y ennuyer. Il n'y trouva ni complaisance ni considération\,;
ses prétentions de rang l'écartèrent et le séparèrent\,; sa réputation,
secondée de la vie qu'il y mena et dont il ne pouvait et n'eût même
daigné se défaire, le fit mépriser. Il s'en alla à Gènes où il espéra
être mieux reçu et vivre plus à son aise.

Je me garderais bien de barbouiller ce papier de l'opération de la
fistule que maréchal fit à Courcillon, fils unique de Dangeau, en sa
maison de la ville à Versailles, sans l'extrême ridicule dont elle fut
accompagnée. Courcillon était un jeune homme fort brave, qui avait un
des régiments du feu cardinal de Fürstemberg qui valait fort gros. Il
avait beaucoup d'esprit et même orné, mais tout tourné à la
plaisanterie, à bons mots, à méchanceté, à impiété, à la plus sale
débauche, dont cette opération passa publiquement pour être le fruit.

Sa mère dont j'ai parlé à l'occasion de son mariage, était dans la
privance de M\textsuperscript{me} de Maintenon la plus étroite\,; toutes
deux seules de la cour et de Paris ignoraient la vie de Courcillon.
M\textsuperscript{me} de Dangeau, qui l'aimait passionnément, était fort
affligée et avait peine à le quitter des moments. M\textsuperscript{me}
de Maintenon entra dans sa peine, et se mit à aller tous les jours lui
tenir compagnie au chevet du lit de Courcillon, jusqu'à l'heure que le
roi allait chez elle, et très souvent dès le matin y dîner.
M\textsuperscript{me} d'Heudicourt, autre intime de
M\textsuperscript{me} de Maintenon et dont j'ai parlé aussi, y fut
admise pour les amuser, et presque point d'autres. Courcillon les
écoutait, leur parlait dévotion et des réflexions que son état lui
faisait faire\,; elles de l'admirer et de publier que c'était un saint.
La d'Heudicourt et le peu d'autres qui écoutaient tous ces propos, et
qui connaissaient le pèlerin qui quelquefois leur tirait un bout de
langue à la dérobée, ne savaient que devenir pour s'empêcher de rire, et
au partir de là ne pouvaient se tenir d'en faire le conte tout bas à
leurs amis. Courcillon, qui trouvait que c'était bien de l'honneur
d'avoir M\textsuperscript{me} de Maintenon tous les jours pour
garde-malade, et qui en crevait d'ennui, voyait ses amis quand elle et
sa mère étaient parties les soirs, leur en faisait ses complaintes le
plus follement et le plus burlesquement du monde, et leur rendait en
ridicule ses propos dévots et leur crédulité, tellement que, tant que
cette maladie dura, ce fut un spectacle qui divertit toute la cour, et
une duperie de M\textsuperscript{me} de Maintenon dont personne n'osa
l'avertir, et qui lui donna pour toujours une amitié et une estime
respectueuse pour la vertu de Courcillon qu'elle citait toujours en
exemple, et dont le roi prit aussi l'impression, sans que Courcillon se
souciât de cultiver de si précieuses bonnes grâces après sa guérison,
sans qu'il en rabattît quoi que ce fût de sa conduite accoutumée, sans
que M\textsuperscript{me} de Maintenon s'aperçût jamais de rien, sans
que pour ses négligences même à son égard elle se refroidît des
sentiments qu'elle avait pris pour lui. Il faut le dire, excepté le
manège sublime de son gouvernement et avec le roi, c'était d'ailleurs la
reine des dupes.

\hypertarget{chapitre-xvi.}{%
\chapter{CHAPITRE XVI.}\label{chapitre-xvi.}}

1706

~

{\textsc{Oublis.}} {\textsc{- Procès intentés par le prince de Guéméné
au duc de Rohan sur le nom et armes de Rohan.}} {\textsc{- Matière de ce
procès.}} {\textsc{- Cause ridicule de ce procès.}} {\textsc{- Parti que
le duc Rohan devait prendre.}} {\textsc{- Excuse du roi, en plein
chapitre, des trois seuls ducs ayant l'âge, non compris dans la
promotion de 1688.}} {\textsc{- Raisons de l'aversion du roi pour le duc
de Rohan.}} {\textsc{- Raison secrète qui fait raidir le duc de Rohan à
soutenir ce procès.}} {\textsc{- Éclat du procès.}} {\textsc{- Conduite
de M\textsuperscript{me} de Soubise.}} {\textsc{- qui le fait évoquer
devant le roi.}} {\textsc{- Conseil curieux où le procès se juge.}}
{\textsc{- Le duc de Rohan gagne entièrement son procès avec une
acclamation publique.}} {\textsc{- Licence des plaintes des Rohan, qui
les réduisent aux désaveux et aux excuses de Mgr le duc de Bourgogne et
au duc de Beauvilliers.}} {\textsc{- Le roi sauve le prince de Guéméné
d'un hommage en personne au duc de Rohan\,; qui l'accorde au roi par
procureur pour cette fois.}} {\textsc{- Branche de Gué de L'Isle, ou du
Poulduc, de la maison de Rohan, attaquée par M\textsuperscript{me}
Soubise, maintenue par arrêt contradictoire du parlement de Bretagne.}}
{\textsc{- Persécution au P. Lobineau, bénédictin, et mutilation de son
\emph{Histoire de Bretagne}.}}

~

Quelque soin que j'aie pris jusqu'à cet endroit, non seulement de ne
dire que la plus exacte vérité, mais de la ranger encore dans l'ordre
précis des temps où sont arrivées les choses que j'estime mériter d'être
écrites, il faut avouer qu'il m'en est échappé deux\,: l'une sur la
maison de Rohan, l'autre sur la maison de Bouillon, la première de 1703,
l'autre aussi de la même année. Il faut donc avant d'aller plus loin
réparer cette faute dès que je m'en aperçois.

On se souviendra de ce quia été expliqué (t. II, p.~137) sur la maison
de Rohan, et les divers degrés d'art et de fortune qui l'ont portée au
rang dont elle jouit maintenant. Il faut parler de la première érection
du vicomté de Rohan en duché-pairie en faveur du célèbre duc de Rohan,
gendre de l'illustre premier duc de Sully, du mariage de sa fille unique
avec Henri Chabot, et de la seconde érection de Rohan en faveur de cet
Henri Chabot, enfin du procès intenté par la maison de Rohan au duc de
Rohan, fils unique de ce mariage, pour faire quitter à ses puînés le nom
et les armes de Rohan, qui est l'oubli qu'il s'agit de réparer.

Le premier et célèbre duc de Rohan était mort en 1636. Sa veuve le
survécut jusqu'en 1660, parfaitement huguenots l'un et l'autre jusqu'à
leur mort. Henri IV érigea le vicomté de Rohan en duché-pairie en faveur
de cet Henri de Rohan en 1603, enregistré la même année aux parlements
de Paris et de Bretagne\,: L'érection porta cette clause\,: \emph{que la
ligne masculine venant à manquer, la qualité de duc et pair demeurerait
éteinte}. Elle eut son effet par la mort de ce même duc de Rohan qui ne
laissa qu'une fille unique née en 1617, qui était peut-être alors la
plus grande héritière qui fût dans le royaume. Cette raison et celle de
la religion dont elle était fit toute la difficulté de son mariage du
vivant de son père, et fort longtemps depuis. Le duc de Rohan, et depuis
lui la duchesse sa veuve, ne la voulaient donner qu'à un huguenot comme
eux. Tantôt il ne se trouvait point de parti sortable pour elle dans
cette religion, tantôt ceux qui auraient été écoutés avaient l'exclusion
du roi, ensuite de la reine régente, qui voulaient ôter ces grands
établissements de terres en Bretagne à la religion prétendue réformée,
dans une province si voisine de l'Angleterre, environnée de la mer de
trois côtés, et à qui les temps permettaient encore d'être jalouse de
ses privilèges. À ces difficultés il s'en était joint une autre qui
arrêta des prétendants. Ce fut le procès de ce Tancrède\footnote{Le
  procès de Tancrède contre M\textsuperscript{lle} de Rohan avait été
  jugé par le parlement le 26 février 1646. L'arrêt lui défendit de
  prendre le nom de fils du feu duc de Rohan.} qui se prétendait son
frère légitime de père et de mère, dont le procès a été trop célèbre et
trop connu pour s'arrêter ici à l'expliquer, et qui ne se termina que
par sa mort, arrivée, sans avoir été marié, au combat du faubourg
Saint-Antoine, en 1649.

M\textsuperscript{lle} de Rohan s'ennuyait cependant d'un célibat auquel
elle ne voyait point de fin, sous l'aile d'une mère jalouse et sévère.
On était en 1646 au milieu des troubles de la régence\,; elle avait
vingt-huit ans. Elle trouva Henri Chabot, seigneur de Saint-Aulaye, fort
à son gré, qui était un des hommes de France le mieux fait et le plus
agréable et qui n'avait qu'un an plus qu'elle, arrière-petit-fils de Guy
Chabot, seigneur de Jarnac, si connu par ce fameux duel auquel il tua
François de Vivonne, seigneur de la Châteigneraie, en champ clos, 10
juillet 1547, en présence du roi Henri II et de toute sa cour.
Saint-Aulaye était dans l'intime confiance de Gaston et de M. le Prince,
qui le servirent si bien dans un temps où ils pouvaient presque tout,
qu'ils firent ce grand mariage malgré la duchesse de Rohan, qui n'avait
rien à dire sur l'alliance, mais qui se récriait sur les biens et sur
les établissements, dont en effet Saint-Aulaye n'avait aucun, et qui
était encore plus outrée de voir sa fille, qu'elle avait si longtemps
réservée à quelque grand parti de sa religion, épouser, avec tant de
grands biens, un catholique dénué de tous ceux de la fortune. Elle eut
beau crier et s'opposer, sa fille avait vingt-huit ans\,: appuyée de
Monsieur, de M. le Prince, et de l'autorité de la reine régente, elle
fit à sa mère des sommations respectueuses et se maria.

Les puissants protecteurs de cet heureux époux firent valoir ces fureurs
de la mère et de plusieurs de ses proches, trop bien fondées sur la
nudité de l'époux. Par là ils lui procurèrent des lettres, en décembre
1648, d'érection nouvelle du duché-pairie de Rohan, pour lui et pour les
enfants mâles qui naîtraient de ce mariage. Ils lui avaient aussi fait
donner promesse du premier gouvernement de province qui viendrait à
vaquer\,; il eut celui d'Anjou en 1647. Cette érection ne put être sitôt
enregistrée à cause des troubles de la cour et de l'État. Dans
l'intervalle, la reine et le cardinal Mazarin, mécontents de Gaston et
de M. le Prince, s'en prenaient entre autres au nouveau duc de Rohan et
empêchaient l'enregistrement. On sait de quelle façon cette affaire fut
à la fin consommée malgré la cour, absente de Paris au fort des
troubles. Un lundi 15 juillet 1652, Monsieur et M. le Prince menèrent le
duc de Rohan à la grand'chambre, où ils avaient déjà fait deux fois la
même tentative, mais à cette troisième ils vinrent à bout avec autorité
de faire enregistrer l'érection et de faire prêter le serment, et
prendre place à M. de Rohan tout de suite en qualité de duc et pair de
Rohan.

Il n'en jouit, pas longtemps et mourut trois ans après, à trente-neuf
ans, 27 février 1655, après avoir beaucoup figuré dans tous les troubles
et les intrigues de son temps. Il laissa un fils unique, qui est le duc
de Rohan dont il s'agit ici, la belle et florissante
M\textsuperscript{me} de Soubise, M\textsuperscript{me} de Coetquen et
la seconde femme du prince d'Espinoy, grand'mère du duc de Melun, en qui
cette branche s'est éteinte, et bientôt après cette grande et illustre
maison de Melun.

Il fallait expliquer tout cela avant que venir au fait, et il est encore
nécessaire de dire qu'outre que le duc de Rohan n'était pas d'humeur
accorte et facile, comme on l'a vu à l'occasion de notre procès de M. de
Luxembourg, il avait un ancien levain contre M\textsuperscript{me} de
Soubise, qui les a tenus mal ensemble toute leur vie, même dans les
intervalles de leurs raccommodements. Leur mère, qui était Rohan, avait
toujours marqué une prédilection fort grande pour M\textsuperscript{me}
de Soubise, sa fille aînée, et par amitié pour elle, et peut-être encore
plus pour l'avoir mariée à M. de Soubise, Rohan comme elle. Outre la
jalousie et les aigreurs que cette prédilection avait fait naître, le
duc de Rohan était persuadé que sa mère avait fait à M. et à
M\textsuperscript{me} de Soubise tous les avantages directs et indirects
qu'elle avait pu à ses dépens. M. de Soubise dans ces temps-là était
fort pauvre, M. de Rohan devait être extrêmement riche, et cela des
biens de la maison de Rohan\,; sa mère en représentait l'aîné bien
qu'elle ne la fût pas. Jean II, pénultième vicomte de Rohan, d'aîné en
aîné, direct de la maison de Rohan, laissa deux fils et deux filles\,:
l'aîné, vicomte de Rohan après son père, mourut sans enfants de
Françoise de Daillon du Lude\,; le second, déjà sacré évêque de
Cornouailles, succéda au vicomte de Rohan et à tous les biens. Les deux
filles épousèrent deux Rohan\,: l'aînée le second fils du fameux
maréchal de Gié, la cadette le seigneur de Guéméné, dont la branche
était aînée de celle de Gié, mais qui en biens n'en fut que la cadette,
parce que la belle-fille du maréchal de Gié, comme l'aînée de
M\textsuperscript{me} de Guéméné, emporta la vicomté de Rohan et tous
les biens de la maison. Or, l'arrière-petit-fils de ce mariage de
l'héritière de la branche aînée de Rohan avec le second fils du maréchal
de Gié fut le duc de Rohan, père de l'héritière qui épousa le Chabot,
seigneur de Saint-Aulaye, père du duc de Rohan dont il s'agit, et qui,
comme on l'a dit, n'avait rien ou presque rien vaillant. Cette grande
inégalité de biens, avec cette grande héritière qu'il épousait, lui fit
imposer la loi par son contrat de mariage, \emph{que les enfants qui en
naîtraient porteraient à toujours, et à leur postérité, le nom et les
armes de Rohan}, ce qui fut exécuté sans difficulté aucune, jusqu'au
temps dont je vais parler.

Immédiatement avant la rupture de l'Angleterre, après l'avènement de
Philippe V à la couronne d'Espagne, le duc de Rohan envoya ses deux
aînés se promener en Angleterre. L'aîné portait le nom de prince de
Léon, l'autre celui de chevalier de Rohan. Ils firent à Londres une
dépense convenable à leur qualité\,; ils furent fort accueillis en cette
cour, et y virent familièrement tout ce qui y était le plus distingué.
En même temps, le prince de Guéméné se trouva aussi à Londres, celui
même dont j'ai fait mention à propos de notre procès contre M. de
Luxembourg, ce qui me dispensera de le dépeindre ici de nouveau.
L'oisiveté, l'ennui lui avaient fait passer la mer pour acheter des
chevaux. Il vivait à Londres comme à Paris, dans l'avarice et
l'obscurité, sans y voir qui que ce fût qui eût ni nom, ni emploi, ni
figure. Le contraste du brillant du prince de Léon et du chevalier de
Rohan le piqua à travers sa stupidité, sans toutefois vouloir rien faire
de tout ce qui le pouvait mettre dans une meilleure compagnie et le
faire considérer. Il était l'aîné de la maison de Rohan\,; l'extrême
bêtise n'empêche pas l'orgueil\,; il s'imagina que son nom de Guéméné le
faisait ignorer, tandis que celui de Rohan procurait au chevalier de
Rohan et à son frère toutes les prévenances dont il n'avait éprouvé
aucune, dans le souvenir qu'il supposa que les Anglais avaient du
célèbre duc de Rohan, et de la figure qu'il avait faite dans les guerres
de la religion, et Soubise, son frère, mort chez eux. Plein de ce dépit,
il repassa la mer, et conçut le dessein de faire quitter le nom et les
armes de Rohan aux enfants du duc de Rohan.

Il lui fallut du temps pour consulter ce projet et pour le mettre en
exécution. Il n'y a si mauvaise affaire qui ne trouve des avocats avides
de gagner, et qui se soucient peu des suites. Il ne manqua pas de
ceux-là\,; et, quand il crut pouvoir commencer ce procès, il éclata en
mauvaise humeur sur son voyage, et envoya un exploit au duc de Rohan,
sans aucune civilité préalable. Cet exploit concluait à ce que ses
enfants et leur postérité eussent à quitter le nom et les armes de
Rohan, lui seul pouvant porter l'un et l'autre à cause de son titre de
duc de Rohan, et après lui son fils aîné seulement, et ainsi
successivement. M. de Rohan ne s'attendait à rien moins, et avec la loi
du contrat de mariage de son père, exécutée plus de soixante ans durant
sans difficulté ni contradiction de personne, il avait raison de se
croire hors d'atteinte et de tout trouble à cet égard.

Un homme plus raisonnable que lui, et qui eût senti moins gauchement sa
grandeur originelle, aurait eu beau jeu en cette occasion. Les Chabot
sont connus dès avant 1050 avec des fiefs et dans les fonctions des
grands seigneurs d'alors. Leurs grandes terres, leurs grandes alliances
actives et passives, leurs grands emplois jusqu'aux officiers de la
couronne inclusivement, se sont longuement soutenus dans les diverses
branches de cette maison\,; et quelque illustre que soit celle de Rohan,
il n'y avait que des biens immenses pour un cadet Chabot, qui n'en avait
point, qui pût le soumettre à quitter son nom pour aucun autre, car pour
les armes, ils ont toujours conservé au moins leurs chabots\footnote{Les
  chabots sont de petits poissons qui ont la tête grande, large et
  plate, et dont le corps va toujours se rétrécissant de la tête à la
  queue. La maison de Rohan-Chabot les plaça dans un quartier de ses
  armes, ou, comme on dit en style de blason, en écartela ses armes.} en
écartelure. M. de Rohan avait donc un bon personnage à faire, beau et
honnête à tout événement\,: c'était d'aller avec sa plus proche famille,
et quelques amis pour témoins dignes de foi, chez M. de Guéméné, lui
témoigner sa reconnaissance du joug de son nom dont il voulait bien le
délivrer, lui porter le contrat de mariage de son père, et lui dire que
ces contrats étant les lois fondamentales des familles, et celui-là le
plus spécialement honoré de l'autorité du roi, ils n'étaient ni l'un ni
l'autre parties capables d'y donner atteinte, mais qu'il était prêt de
l'accompagner pour demander au roi conjointement qu'il lui plût ratifier
leur commun désir par un acte de sa puissance, et prêt encore de
présenter à même fin avec lui soit au roi, soit au parlement, toutes
requêtes pour y parvenir\,; le presser ensuite d'en venir à l'effet, se
presser soi-même d'en obtenir le succès et de se montrer en effet ravi
d'espérer de pouvoir reprendre son nom et ses armes, pousser même la
chose jusqu'à faire biffer par autorité juridique le nom de Rohan de son
contrat de mariage, et de celui de ses trois sœurs, et de tous les
principaux actes de lui et d'elles.

Par cette conduite, point d'aigreur, point de procédés, une hauteur
accablante par son seul poids, et de laquelle pourtant M. de Guéméné
agresseur, ni les siens, ne se pouvaient plaindre. Si la chose
réussissait, joug ôté à M. de Rohan rendu à son nom et à ses armes assez
anciennes et illustres pour en être jaloux, et assez connues pour
telles, pour qu'au lieu de blâme, le monde lui en eût su gré, avec un
rejaillissement désagréable pour le nom et les armes qu'il se prêtait si
volontiers à secouer. Si, au contraire, les liens de la loi du contrat
de mariage étaient trouvés inextricables par le roi et par les
tribunaux, la honte de l'entreprise serait retombée, sur le seul M. de
Guéméné doublement, et pour l'avoir hasardée contre toute raison et
possibilité, et pour avoir donné lieu à M. de Rohan de témoigner sans
injure le peu de compte qu'il faisait du nom et des armes de Rohan, en
comparaison d'être restitué au sien.

Mais une hauteur tranquille, simple, sortie de la nature des choses,
sans mélange d'honneur et de vanité mal placée, n'était pas pour naître
de M. de Rohan. Il aima mieux s'abaisser et s'avilir même en croyant
faussement se relever, et s'exposer à un affront véritable pour la
fantaisie de crier faussement à l'affront.

Une autre considération devait encore venir à l'appui d'un parti si
noble et si raisonnable. On a vu (t. II, p.~156) et en d'autres endroits
de ces Mémoires quel était le crédit de M\textsuperscript{me} de
Soubise. Elle et son frère se haïssaient parfaitement, et il ne pouvait
ignorer que le roi ne l'aimait pas mieux. Outre le courant de la vie où
il avait toujours essuyé des dégoûts, il ne pouvait pas oublier
l'étrange déclaration du roi au chapitre de l'ordre de 1688, où les
chevaliers de cette grande promotion furent nommés. Le roi, peiné de
l'injustice qu'il faisait aux ducs, en faveur de la maison de Lorraine,
mais dont l'engagement était pris de longue main, et pour parvenir à ce
qu'il souhaitait le plus, comme on l'a vu (t. Ier, p.~9), voulut bien ne
pas dédaigner de faire aux ducs une excuse publique des trois seuls
d'entre eux ayant l'âge qu'il n'avait pas compris dans la promotion, et
d'en dire les raisons. C'était MM. de Ventadour, de Brissac, mon
beau-frère et frère de la maréchale de Villeroy, et M. de Rohan. Du
premier, le roi dit qu'il n'avait pas voulu exposer son ordre dans les
cabarets et les mauvais lieux de Paris\,; du second, qu'il n'avait pu se
résoudre à le prostituer en des lieux encore plus infâmes, et cela en
plein chapitre de l'ordre\,; de M. de Rohan enfin, que pour celui-là il
n'y avait rien à dire, sinon qu'il ne l'avait jamais aimé, et qu'il
fallait au moins lui en passer un. Cela fut net. Outre que le duc de
Rohan était un homme d'esprit et d'une humeur fort désagréable, le roi
qui voulait qu'on regardât les charges, surtout celles qui
l'approchaient de plus près, comme le souverain bonheur, ne lui avait
jamais pardonné d'avoir rompu son mariage avec la fille unique du duc de
Créqui pour faire celui de la fille unique de Fardes. Le roi aimait fort
le duc de Créqui, et lui avait accordé la survivance de sa charge de
premier gentilhomme de sa chambre, pour son gendre, et Tardes était
exilé en Languedoc depuis longtemps, pour avoir manqué personnellement
au roi en chose essentielle, qui ne le lui pardonna jamais.
M\textsuperscript{me} de Soubise, de plus, n'avait pas aidé à faire
revenir le roi pour son frère. Elle était toute Rohan, et enivrée du
rang qu'elle avait procuré à son mari et à ses enfants. Par toutes ces
raisons, il n'était pas douteux qu'elle ne fût en cette occasion pour M.
de Guéméné contre son frère, et que ce crédit de plus sur le roi aussi
mal disposé qu'il était, et sur les ministres, qui tous la craignaient
et la ménageaient infiniment, ne devînt fort dangereux à la cause du duc
de Rohan.

Mais le temps des chimères était arrivé\,; il en était monté une dans la
tête du duc de Rohan qui ne se découvrit que quelque temps après, comme
il sera remarqué en son lieu, qui, toute folle qu'elle put être,
l'entraîna dans le soutien du nom et des armes de Rohan, pour ses
enfants et leur postérité. Piqué de n'avoir point été chevalier de
l'ordre, il aurait voulu faire croire la fausseté de ce que
M\textsuperscript{me} de Soubise avait fait écrire sur les registres de
l'ordre, au lieu de ce que le roi avait commandé qui y fût mis, et que
j'ai remarqué (t. II, p.~159), et persuader qu'il avait suivi le sort
des Rohan. De là avec les années, il se mit peu à peu dans la tête de
prétendre le même rang, dont ils jouissent, parce que sa mère lui en
avait apporté tous les biens. Sa mère, étant fille, n'avait jamais été
assise\,; sa mère n'était l'aînée de la maison de Rohan que par les
biens\,; avant la comédie de \emph{Georges Dandin}, où M. et
M\textsuperscript{me} de Sotenville prétendirent que le ventre
anoblissait, on n'en avait jamais vu former de prétention. Mais comme
l'expérience en plusieurs montre qu'en vieillissant les prétentions et
les chimères avaient de nos jours fait fortune, M. de Rohan espéra le
même succès de la sienne et ses enfants, comme nous le verrons après
lui. Jusqu'à présent elle n'a pas encore réussi.

Quoi qu'il en soit de ce qui conduisit le duc de Rohan, il se mit aux
hauts cris de l'injure qui lui était faite, et ne pensa qu'à la
repousser, et à se maintenir dans le droit acquis par le contrat de
mariage de son père. L'instance se lia avec le plus grand éclat et
l'aigreur la moins ménagée. Au commencement de la rupture,
M\textsuperscript{me} de Soubise conserva une sorte de pudeur. Le nom
qu'elle avait pris dans son contrat de mariage et dans tous les actes où
elle avait parlé depuis jusqu'alors la fit nager un temps entre deux
eaux. Son frère ne se contentait point de cette espèce de neutralité,
qui, pour dire le vrai, n'en avait que l'apparence. Il se fâcha, les
étoupes entre eux n'étaient pas difficiles à rallumer.
M\textsuperscript{me} de Soubise fit semblant d'être entraînée par
l'autorité de son mari et par l'intérêt de ses enfants. Elle leva le
masque, se mit à la tête du conseil de M. de Guéméné, et fit avec lui
cause commune à découvert. Son crédit engagea le roi à évoquer l'affaire
à sa propre personne, qui déclara en même temps qu'il joindrait le
conseil des finances à celui des dépêches pour la juger en sa
présence\,; et commit le bureau du conseil des parties\footnote{Voy.,
  sur le conseil des parties, t. Ier, notes, p.~445. Le \emph{bureau} de
  ce conseil désigne ici les membres chargés d'instruire le procès et
  d'en faire le rapport.} de M. d'Aguesseau pour l'instruire, et être
ensuite des juges dans son cabinet avec les deux conseils. Tout cela ne
multipliait guère les juges que de ce bureau\,; encore d'Aguesseau
était-il du conseil des finances. Par là M\textsuperscript{me} de
Soubise n'avait affaire qu'aux quatre secrétaires d'État pour le conseil
des dépêches, au chancelier et au duc de Beauvilliers qui étaient de
tous, à Pelletier de Sousy et à d'Aguesseau pour le conseil des finances
dont ils étaient conseillers, à Desmarets et à Armenonville, qui y
entraient comme directeurs des finances, aux trois conseillers d'État du
bureau de M. d'Aguesseau, et au maître des requêtes rapporteur. Tout
était donc la cour, son pays et son règne, hors les trois derniers,
desquels encore elle espérait bien qu'aucun ne voudrait déplaire au roi,
dont l'inclination était assez publique, surtout le rapporteur, qui,
comme tous les maîtres des requêtes, avait une fortune à faire, à
obtenir une intendance, et par ce chemin à parvenir à une place de
conseiller d'État, qui est le bâton de maréchal de France du métier.
Monseigneur et Mgr le duc de Bourgogne, qui entrait dans tous les
conseils, devaient aussi être juges.

Les écrits volèrent donc de part et d'autre. Le public en fut avide,
même les pays étrangers. La maison de Rohan y perdit. Sans oser attaquer
la maison de Chabot, elle voulut s'élever au-dessus de toute noblesse,
en princes qui étaient d'une classe hors du niveau. Cette hauteur,
destituée de toutes preuves, irrita et les véritables princes et ceux
qui ne l'étaient pas, et donna un grand cours et une grande faveur aux
mémoires du duc de Rohan, qui, sans attaquer aussi la maison de Rohan,
mit sa chimère en pièces, et sans aucune réponse qui eut la moindre
apparence ni le plus léger soutien. Il fallut avoir recours à des
mensonges, à des contradictions qui étaient incontinent et cruellement
relevés, et qui augmentèrent la partialité et l'indignation publique.
Beaucoup de gens, paresseux jusqu'alors d'approfondir, et faciles à
croire sur parole, virent clair sur cette princerie. Le plus fâcheux fut
que Mgr le duc de Bourgogne, qui lisait tout de part et d'autre, avec
l'application d'un homme qui veut s'instruire pour faire justice, fut
mis au fait de ce qu'il importait tant à l'état où les Rohan s'étaient
élevés de laisser ignorer à un prince qui devait régner, et qui aimait
l'ordre et la vérité, et que le roi même ne laissa pas, dans le cours de
l'affaire, d'être détrompé de bien des choses essentielles que
M\textsuperscript{me} de Soubise lui avait de longue main peu à peu
inculquées.

Cependant toute la faveur pendant l'instruction fut pour
M\textsuperscript{me} de Soubise. Il ne s'y fit pas un seul pas sans
prendre l'ordre du roi, qui pressa ou qui retarda l'affaire à son gré.
Enfin, tout étant prêt, le roi donna une après-dînée entière au jugement
de cette cause, où Monseigneur ne voulut pas se donner la peine de se
trouver. Le coadjuteur de Strasbourg, depuis cardinal de Rohan, touché
de la faiblesse de leurs écrits, en donna, sur la fin, un de sa façon
dont il espéra des merveilles. Il ne s'y trouva que du fiel peu mesuré,
peu séant et sans aucun nouvel appui, qui acheva de révolter le monde de
tous états qui ne cachait plus sa partialité pour le duc de Rohan.

La veille du jugement, la maréchale de La Mothe, grand'mère de la
princesse de Rohan, à la tête de toute cette famille, se trouva à la
porte du cabinet du roi, au retour de sa messe, pour lui présenter un
nouveau mémoire. Le coadjuteur se promenait, en attendant, par la
galerie avec un grand air de confiance et de supériorité, en fils de la
fortune et de l'amour, dans la maison maternelle. Il y débitait entre
autres choses qu'on ne devait pas être surpris, si ceux de sa maison, si
fort relevés par leur naissance au-dessus de la noblesse du royaume,
étaient jaloux de leur nom, et le souffraient impatiemment à d'autres.
La cour était fort grosse. Le marquis d'Ambres, qui l'écoutait avec son
silence ordinaire, n'y put enfin résister, et de son ton de fausset et
son air audacieux\,: «\,Cela s'appelle, lui dit-il, soutenir une odieuse
cause par des propos encore plus odieux\,;» et lui tourna le dos. Cette
sortie publique et si peu ménagée, que la contenance et l'air des
nombreux assistants applaudirent, déconcerta tellement le jeune et beau
prélat, qu'il ne répliqua pas une seule parole, et qu'il n'osa plus
haranguer.

Le lendemain le même cortège se présenta à l'entrée des juges à la porte
du cabinet du roi, et vis-à-vis le duc de Rohan, uniquement accompagné
de la duchesse sa femme et de leur fils aîné. Le duc de Rohan avait
supplié le roi que l'affaire au moins fût jugée sans milieu et sans
retour, et avait eu pour réponse sèche qu'on lui ferait justice. À la
connaissance qu'on avait de tous les, personnages qui devaient être
juges, leurs opinions étaient déjà conjecturées, on ne s'y trompa que de
ce qu'il fallut précisément pour former l'arrêt. On voyait encore que
celles qui seraient pour le duc de Rohan ne seraient que faiblement
énoncées par des gens conduits par leur conscience, mais accoutumés à se
tenir dans le terme étroit du devoir, sans s'affectionner jamais, et
moins encore vouloir prévaloir. Les juges entrés, le roi alla à
Chamillart, avec qui il avait le plus de familiarité, et lui demanda
tout bas pour qui il serait. Chamillart lui répondit à l'oreille pour
M\textsuperscript{me} de Soubise\,; car, depuis quelque temps M. de
Guéméné était effacé, et cette affaire ne s'appelait plus que celle du
duc de Rohan et de M\textsuperscript{me} de Soubise.

Dès que tous furent en place, avant que le rapporteur eût ouvert la
boucha\,: «\, Messieurs, dit le roi, je dois justice à tout le monde, je
veux la rendre exactement dans l'affaire que je vais juger\,: je serais
bien fâché d'y commettre aucune injustice\,; mais pour de grâce, je n'en
dois à personne, et je vous avertis que je n'en veux faire aucune au duc
de Rohan.\,» Et tout de suite, passant les yeux sur toute la séance, il
commanda au rapporteur de commencer. On peut juger de l'impression de ce
préambule si peu usité, et quel aussi en put être le dessein. L'affaire
dura six heures de suite. Le roi avait dîné exprès de fort bonne heure,
pour donner tout le temps, et n'avoir pas à y revenir. Le rapporteur
parla deux heures avec une netteté et une précision dont ils furent tous
charmés. Il n'omit rien de part et d'autre\,; tout fut mis également
dans le plus grand jour, et pesé de même. La conclusion surprit fort la
compagnie, elle fut entièrement en faveur du duc de Rohan. Les quatre
conseillers d'État du bureau parlèrent ensuite avec éloquence et
véhémence. Il y en eut d'accusés de cacher avec art ce qu'il y avait de
faible dans leur raisonnement, qui ne laissa pas de balancer fort celui
du rapporteur, et qui pensa entraîner tous les autres.

D'Aguesseau doux, faible, non de capacité ni d'expression, mais
d'habitude, et naturellement fort timide et fort défiant de soi-même,
avait une conscience tendre, épineuse, qui émoussait son savoir, et
arrêtait la force de son raisonnement. Son opinion était donc toujours
comme mourante sur ses lèvres, et peu capable d'en entraîner d'autres,
quoique toujours parfaitement approfondie et judicieuse. On ne doutait
donc pas qu'en cette occasion il ne se montrât plus timide encore qu'à
l'ordinaire. La surprise fut grande de voir cet homme si modeste,
souvent jusqu'à l'embarras, pressé sans doute par sa conscience et par
la considération du danger du lieu pour ce qu'il croyait juste,
s'énoncer avec un poids nouveau, et saisir une autorité inconnue, avec
laquelle il soutint, cinq quarts d'heure durant, le droit du duc de
Rohan, même avec des raisons qui avaient échappé au rapporteur. Il
conclut par une péroraison qu'il adressa au roi, sur ce que cette cause
était la sienne, celle de la mémoire de la reine sa mère, celle de la
religion\,; sur la part que le roi et la reine mère avaient eue au choix
de M. de Saint-Aulaye par bille de Rohan, et à leur contrat de mariage,
auquel, par cette raison, leur signature ne pouvait être considérée
comme un simple honneur, ainsi qu'aux autres contrats de mariage, mais
comme une autorisation formelle de toutes les clauses contenues en
celui-ci, dont on ne pouvait attaquer aucune sans contester la validité
de l'autorité royale. Il fit souvenir le roi des raisons d'État et de
religion qui lui avaient fait prendre tant de part en ce mariage, et il
finit en interpellant le roi des vérités qu'il avançait.

Le roi convint à l'heure même de tout ce qu'il venait de dire sur ce
mariage, et loua succinctement le beau discours de d'Aguesseau. Les
autres juges opinèrent ensuite, entre autres Chamillart qui, à la grande
surprise du roi, après ce qu'il lui avait dit entrant au conseil, fut
pour le duc de Rohan, entraîné comme il l'avoua au roi, au sortir de la
séance, par la force et le torrent de d'Aguesseau. Le duc de
Beauvilliers opina succinctement pour le duc de Rohan, mais très
fortement contre sa coutume. Jusque-là tout se trouva tellement balancé,
que le duc de Rohan ne l'emportait que de deux voix. Restaient à parler
M. le chancelier et Mgr le duc de Bourgogne, et le roi après à
prononcer.

La vérité me force à en dire une que je vaudrais taire, dont le fond put
n'être pas mauvais par l'intime persuasion, mais dont l'écorce au moins,
et la façon de soutenir ce qu'on pense être juste, parut passer le but.
Le chancelier était ami intime de M\textsuperscript{me} de Soubise. Il
considéra qu'opinant pour M. de Guéméné, Mgr le duc de Bourgogne ferait
l'arrêt\,; il résolut de l'emporter de vive force\,; au lieu d'opiner en
peu de mots sur une affaire si longuement débattue, et si fort disputée
et éclaircie, il fit un long discours avec tout l'esprit, la force, la
subtilité possible, qui parut moins d'un chancelier que d'un avocat de
réplique. Puis, se rabattant peu à peu sur son dessein, il s'adressa par
diverses questions au jeune prince, lui répétant souvent avec art\,: que
peut-on objecter à ceci\,? que peut-on répondre à cela\,? quelle sortie
de cet autre\,? pour étourdir sa conscience délicate, en essayant
d'étouffer ses lumières, au cas qu'il ne fût pas de son avis, et
peut-être encore en le provoquant ainsi, l'accabler de l'embarras de lui
répondre, et le réduire par l'insuffisance d'entrer en lice contre
lui\,: il s'y trompa.

Mgr le duc de Bourgogne avait étudié à fond les mémoires de part et
d'autre, écouté attentivement le rapporteur, d'Aguesseau, et toutes les
opinions. Il s'était surtout appliqué à celle du chancelier, qui dura
une grosse heure. Quand il eut fini, le prince prit la parole, d'abord
avec sa retenue ordinaire, mais incontinent après avec une décision
précise qui sentait l'indignation, et qui semblait avoir pénétré la
poitrine du chancelier. Il suivit la route qu'il lui avait tracée en
s'adressant à lui. «\,Ce que je vous répondrai, monsieur, lui dit-il
tout à coup, à ce que vous venez de dire, c'est que je ne trouve pas de
question en ce procès, et que je suis surpris de la hardiesse de la
maison de Rohan à l'entreprendre.\,» Passant ensuite un regard sur toute
la compagnie, il reprit toute l'affaire avec exactitude, justesse et
précision, et appuya sur les principaux points et les raisons
principales de d'Aguesseau, du rapporteur et des autres en les citant,
qui avaient opiné pour le duc de Rohan. Fixant ensuite un regard perçant
sur le chancelier, il discuta les raisons fondamentales de son avis,
dont il mit en évidence le captieux et les sophismes. Retombant après
sur les nouvelles raisons que d'Aguesseau avait apportées, et sur
l'autorisation du contrat de mariage, par la signature du roi, il
soutint les premières, mais il combattit cette dernière, et déclara
qu'il ne croyait point que l'autorité des rois pût s'étendre jusque sur
les lois des familles, qu'il ne tenait pour inviolables que lorsque d'un
consentement mutuel elles avaient été faites par elles-mêmes, comme il
était arrivé en celles dont il s'agissait, et de plus confirmées par une
exécution aussi paisible et aussi longue. Il parla une heure et demie,
et se fit admirer par la force et la sagesse de son discours, et par la
profonde instruction qu'il y montra. Il le termina par les mêmes paroles
qui l'avaient commencé, par quelques-unes sur la naissance illustre et
ancienne des Chabot, et par quelque chose de plus animé contre les
Rohan, qu'il ne s'était permis dans toute son opinion. De cette manière
il fit l'arrêt.

Restait le roi à prononcer, qui, depuis ce peu de mots à d'Aguesseau sur
son opinion, avait gardé un profond mais très attentif silence\,;
personne n'avait que voix consultative en sa présence. Il avait donc le
choix de deux partis\,: l'un de se rendre à la pluralité en deux mots,
comme il avait coutume de faire, laquelle n'était que de deux voix\,;
l'autre parti, qu'il n'a pris que trois ou quatre fois au plus en sa
vie, était d'user de sa pleine puissance, et de prononcer en faveur du
prince de Guéméné.

Il ne fit ni l'un ni l'autre, et en prit un troisième pour la première
fois. Au lieu de se tourner vers le chancelier, pour lui déclarer sa
volonté, il regarda un moment en silence toute la compagnie, et fit un
discours d'un quart d'heure, plein de dignité et de justesse. Il honora
de son souvenir et de ses louanges le précis de l'avis des deux
différentes opinions de ceux qu'il trouvait avoir le mieux parlé,
surtout du rapporteur et de d'Aguesseau, et marqua de la complaisance
pour le discours de son petit-fils. Opinant ensuite en juge ordinaire,
il exposa sommairement les raisons qui l'avaient le plus touché, blâma,
mais avec une modération qui se sentait de son penchant, l'entreprise de
MM. de Rohan, insista sur la justice de la cause du duc de Rohan, et fit
sentir que, lorsqu'il était question de justice, il était bien aise de
la rendre. Enfin, se tournant au chancelier, il lui commanda de dresser
l'arrêt avec le duc de Rohan, de ne lui refuser rien de ce qui pouvait
le rendre plus net, plus décisif, le plus hors d'atteinte d'aucun
retour, en quelque sorte que ce pût être, et qu'à l'avenir, il ne pût
jamais se trouver ni lieu ni prétexte de ne plus ouïr parler de la
question.

Cette action du roi surprit infiniment. On crut que, voyant en effet la
justice et la cause y tourner, instruit qu'il se disait tout haut que
M\textsuperscript{me} de Soubise, l'ayant pour jugé, il n'était pas
possible qu'elle perdît, et ayant promis implicitement le matin même au
duc de Rohan que l'affaire serait jugée sans milieu et sans retour, il
avait été bien aise de montrer qu'il ne faisait acception de personne en
justice, que lui-même la croyait du côté du duc de Rohan, qu'il lui
avait voulu tenir une parole si fraîchement donnée, épargner au
rapporteur, qui naturellement devait dresser l'arrêt, tout ce qu'il
aurait à y essuyer de points et de virgules, et de pis encore de la part
des Rohan\,; son parti pris, tenir le chancelier de court, après ce
qu'il en avait entendu en opinant, et se délivrer lui-même des demandes
et de l'importunité de M\textsuperscript{me} de Soubise, sur un arrêt où
il ne voulait plus toucher.

Pendant ce long conseil, les Rohan séparément répandus faisaient des
visites dans. Versailles, tenaient les plaids chez la maréchale de La
Mothe, et le jeune coadjuteur, pour marquer une pleine confiance, jouait
tranquillement à l'hombre chez la chancelière. Le duc de Rohan s'était
retiré chez lui à la ville, sa femme dans un cabinet de
M\textsuperscript{me} d'O au château\,; leur fils aîné allait et venait.
Il était près de huit heures du soir quand le conseil leva. Le duc de
Rohan était revenu chez le roi, résolu d'essuyer l'événement\,; aucun
des Rohan n'y parut. Ils sentaient l'extrême révolte du public contre
eux sur cette affaire, ils le craignirent. En effet tout l'appartement
du roi n'était qu'une foule que la curiosité intéressée y avait
assemblée. Jusqu'à la cour de Marbre en était remplie pour savoir
l'événement, par les fenêtres qui étaient ouvertes, de ceux qui étaient
dans les appartements. Mgr le duc de Bourgogne sortit le premier. M. de
Rohan qui était à la porte lui demanda son sort. Comme il ne répondit
rien, le duc lui demanda au moins s'il était jugé. «\,Oh\,! pour cela
oui, répondit le prince, et jugé sans milieu ni retour.\,» Et tout
aussitôt se tournant au chancelier qui le suivait, lui demanda si on ne
pouvait pas dire le jugement. Le chancelier ayant répondu qu'il n'y
avait nulle difficulté à le dire, le prince se retourna au duc de
Rohan\,: «\,Puisque cela est lui dit-il, monsieur, vous avez gagné
entièrement, et je suis ravi de vous l'apprendre.\,» Le duc s'inclina
fort, par respect, et en même temps Mgr le duc de Bourgogne l'embrassa,
et ajouta qu'il en était aussi aise que lui-même, et qu'il n'avait
jamais vu un si méchant procès.

Au premier mot de jugement rendu, l'antichambre, et tout aussitôt le
reste de l'appartement, retentit des cris de joie et de battements de
mains, auxquels la cour de Marbre répondit jusqu'à l'indécence, vu le
respect des lieux. On criait tout haut\,: «\,Nous avons gagné, ils ont
perdu\,!» et cela se répéta sans nombre. Le roi devait aller se promener
à pied dans ses jardins, et descendre par son petit degré dans la cour
de Marbre pour y aller. À grand'peine le duc de Rohan, quoique
généralement peu aimé et considéré, put-il gagner ce petit degré à
travers les embrassades, les félicitations et les redoublements des cris
de joie, à mesure qu'il était aperçu.

Le roi reçut ses remerciements avec tout l'accueil et les grâces qu'il
s'était bien proposés, en opinant contre sa coutume, comme il avait
fait. Le soir, M. de Rohan étant chez Mgr le duc de Bourgogne, où il y
avait grand monde, ce prince lui parla encore de son affaire. Il ne
feignit point de lui dire qu'il avait été pour lui de tout son cœur, et,
baissant un peu la voix, que c'était une chose indigne et odieuse.

Le lendemain au soir, M\textsuperscript{me} de Soubise, supérieure aux
événements et au cri public, vint attendre le roi peu accompagnée, comme
il allait passer chez M\textsuperscript{me} de Maintenon. Elle lui
demanda que l'arrêt fût communiqué à M. de Guéméné avant d'être signé,
et l'obtint sur-le-champ, nonobstant les ordres qu'on vient de voir que
le roi, en décidant, avait donnés au chancelier. Il en résulta des
discussions, où à la fin le duc de Rohan ne perdit rien.

Rien n'égala l'amertume des Rohan. Ils ne la purent si bien contenir
qu'il ne leur échappât des plaintes aigres contre le duc de
Beauvilliers, qui s'était, disaient-ils, rendu maître des voix de tous
ses amis au conseil, et qui avait instruit Mgr le duc de Bourgogne à y
faire un plaidoyer contre eux. La chose était bien éloignée de
l'austérité des mœurs de M. de Beauvilliers, mais la vérité était que
ses amis, excepté Desmarets, avaient, par un hasard qui n'avait de
source qu'en leurs seules lumières, tous été pour le duc de Rohan. Dette
licence, qui fut relevée, mit M. et M\textsuperscript{me} de Soubise et
leurs enfants dans une grande peine. Il fallut s'excuser, se dédire, en
venir aux justifications, aux déguisements, aux pardons avec le prince
et le gouverneur. Le soulèvement général les toucha profondément,
surtout l'abandon des Bouillon leurs semblables, qui ne voulurent point
participer avec eux au déchaînement public, et les propos des Lorrains,
qui, parents des Chabot et toujours en dépit de similitude avec des
seigneurs qui ne sont pas comme eux de maison souveraine, ne les
épargnèrent pas en cette occasion.

Il s'en présenta bientôt une autre, qui les jeta dans un cruel embarras.
Guéméné relevait en juveigneur du duc de Rohan, qui, pour les biens,
représentait l'aîné de la maison. Le prince de Guéméné n'en avait point
rendu de foi et hommage, et jusqu'alors M. de Rohan l'avait souffert. À
cet éclat il saisit féodalement cette terre, qui est de quinze mille
livres de rente. Nul moyen de s'y opposer ni d'en empêcher l'effet, qui
est la perte entière des fruits, c'est-à-dire de la totalité du revenu,
que par rendre la foi et hommage. Pour la rendre, il fallait que le
prince de Guéméné allât en personne en Bretagne se mettre à genoux, sans
épée ni chapeau, devant le duc de Rohan, lui prêter foi et hommage en
cet état, et pour cette fois n'en pas avoir la main chez lui. C'est à
quoi le duc de Rohan le voulut réduire, et y tint ferme, quoi qu'on pût
employer auprès de lui.

Dans cette presse, le roi fut longtemps sollicité de les tirer de ce
mauvais pas, et le roi longtemps à s'en défendre, sur ce qu'il ne se
mêlait point d'affaires particulières. M\textsuperscript{me} de Soubise
obtint pourtant que le roi demandât quelques délais. Mais c'était
toujours à recommencer, c'était traîner le lien, il fallait une
délivrance. À la fin, M\textsuperscript{me} de Soubise fit tant
d'efforts, que le roi fit pour elle ce qu'il n'avait jamais fait\,: il
s'abaissa à demander grâce au duc de Rohan pour le prince de Guéméné,
lui expliquant qu'il ne lui commandait rien, qu'il n'exigeait même rien,
mais qu'il la lui demandait comme ferait un particulier, et avec toutes
sortes d'honnêtetés, comme un plaisir qui lui serait sensible. Le duc de
Rohan, après avoir bien expliqué au roi ce dont il s'agissait, et voyant
qu'il insistait toujours, accorda enfin que l'hommage se rendrait pour
cette fois par procureur au sien, et répéta bien au roi, et après tout
le monde, que c'était au roi, non au prince de Guéméné, qu'il
l'accordait.

M\textsuperscript{me} de Soubise, si heureuse et si accréditée en tout,
ne l'était pas sur le nom de Rohan. Elle aurait pu se souvenir de la
leçon qu'elle avait reçue là-dessus en Bretagne pour s'épargner celle
qui lui fut donnée à Versailles. Il y avait en Bretagne une branche de
la maison de Rohan sortie d'Éon, cinquième fils d'Alain VI, vicomte de
Rohan et de Thomasse de La Roche-Bernard sa femme, connue sous le nom de
Gué de L'Isle, dont Éon de Rohan avait épousé l'héritière, puis du
Poulduc, depuis que Jean de Rohan, cinquième génération d'Éon, eut
dissipé tous ses biens, dont les générations qui suivirent ne purent se
relever. M\textsuperscript{me} de Soubise, mariée en 1663, ne tarda pas
à plaire, et, comme on l'a vu (t. II, p.~155 et suiv.), à faire par sa
beauté son mari prince, dont la première femme n'avait jamais été assise
ni prétendu l'être. En faveur et en puissance de plus en plus, cette
branche de Poulduc lui déplut fort. Sa chute de biens et le médiocre
état où elle se trouvait réduite en Bretagne par des alliances
proportionnées à sa décadence, ne permettaient pas à la nouvelle
princesse de songer à la poulier\footnote{Vieux mot qu'emploie plusieurs
  fois Saint-Simon dans le sens de hisser avec une poulie. Les
  précédents éditeurs ont cru devoir le remplacer par le verbe
  \emph{pousser}.}, au rang que ses beaux yeux avaient conquis. D'un
autre côté, il était bien fâcheux pour des princes de si nouvelle
impression de voir traîner en Bretagne leur nom et leurs armes à des
gens qui n'avaient aucune distinction, et qui demeuraient un monument
vivant de leur commune origine rien moins que souveraine, ni que
supérieure aux premières maisons de leur pays, quelque ancienne et
illustre qu'elle fût.

Isaac de Rohan, seigneur du Poulduc, dans la paroisse de Saint-Jean de
Beverlay, diocèse de Vannes, quatrième descendant de celui qui s'était
ruiné, et neuvième descendant d'Éon, puîné d'Alain VI, vicomte de Rohan,
était, depuis ce père commun de toute la maison de Rohan, c'est-à-dire
depuis plus de trois cent cinquante ans, en possession paisible du nom
et des armes de Rohan, reconnu jusqu'alors par tous ceux de cette maison
pour en être, ainsi qu'eux-mêmes, sans nulle difficulté en aucun temps,
avec toute la Bretagne pour témoin de leur naissance. Cela était
extrêmement incommode.

Isaac de Rohan, seigneur du Poulduc, fils d'une Kerbalot, mari d'une
Kerpoësson, se trouvait sans appui comme sans biens et sans alliances.
On crut, avec de l'argent et du crédit, pouvoir lui enlever son état et
le faire passer pour un bâtard ou pour un usurpateur. Dans cette
confiance, il fut attaqué sur son nom et ses armes. On espéra qu'il
n'oserait se défendre, ou qu'avec des moyens on l'introduirait à céder.
On se trompa sur ces deux points, et on ne s'abusa pas moins sur un
troisième, qui fut de s'être flatté de n'avoir affaire qu'à un homme
sans secours. Le nom et le crédit de M. et de M\textsuperscript{me} de
Soubise eurent beau paraître à découvert, ce fut un soulèvement général
dans toute la Bretagne. La vérité y excita tout le monde, l'oppression
attira l'indignation, tous les alliés de cette branche se démenèrent et
attirèrent à eux tout le reste de la noblesse. Du Poulduc produisit ses
titres devant le parlement de Bretagne, et y obtint, le 21 janvier 1669,
un arrêt contradictoire qui le maintint dans la possession de son état
du nom, maison et armes de Rohan, depuis lequel cette branche n'y a plus
été troublée, et y subsiste encore jouissant et usant de cette
possession.

Ces aventures ne découragèrent point des gens qui, non contents du rang
qu'ils avaient obtenu, voulaient absolument être princes. Ils avaient
tenté une descendance chimérique d'un Conan Mériadec qui n'exista
jamais, prétendu roi de Bretagne dans les temps fabuleux. Le nom et les
macles\footnote{Les macles sont, en style de blason, des espèces de
  losanges percées à jour. La maison de Rohan porte neuf macles d'or sur
  champ de gueules (rouge), avec la devise\,: \emph{sine macula} (sans
  tache).} de Rohan ne ressemblaient en rien au nom ni aux armes de
Bretagne\,; aucun titre qui les en pût approcher\,; nul moyen de sortir
de la dernière race des ducs, issus par mâles de la branche de Dreux de
la maison de France. Celle de Rohan, si connue, si ancienne, si illustre
en Bretagne, n'en était jamais sortie avant Louis XI, et on a vu dans ce
que j'en ai rapporté qu'elle n'y a jamais eu de distinction ni
d'avantages sur les autres grandes maisons du pays, ni par leurs aînés,
ni par leurs cadets, que ceux du rang de la vicomté de Rohan aux états,
plus que balancé par celui de Laval, ou plutôt de Vitré, c'est-à-dire
rang de terre, non de naissance, quoique gendres et beaux-frères des
ducs de Bretagne, et grandement établis en grands biens, en premiers
emplois et en hautes alliances.

Un bénédictin, nommé Lobineau, fit en ces temps-ci une \emph{Histoire de
Bretagne}. M. de Strasbourg y voulut faire insérer ce qui lui convenait.
Le moine résista et souffrit une persécution violente et même publique,
sans qu'il fût possible de le vaincre\,; mais enfin, las de tourments et
menacé de pis encore, il vint à capitulation. Ce fut de retrancher tout
ce qui pouvait déplaire et nuire aux prétentions. Ces retranchements
furent infinis\,; il les disputa pourtant pied à pied avec courage\,;
mais à la fin, il fallut céder et insérer faussement du Mériadec, malgré
tout ce qu'il put dire et faire pour s'en défendre. Il s'en plaignit à
qui le voulut entendre\,; il fut bien aise, pour sa réputation, que la
violence ouverte de ces mutilations et de ces faussetés ajustées par
force ne fût pas ignorée. Il en encourut pour toujours la disgrâce des
Rohan, qui surent lui en faire sentir la pesanteur jusque dans le fond
de son cloître, et qui ne s'en sont jamais lassés.

L'abbé de Caumartin, mort évêque de Blois, à qui le moine disait tout,
me l'a conté dans le temps, outre que la chose devint publique. Avec ces
mutilations, l'ouvrage parut fort défiguré, sans quoi il n'eût jamais vu
le jour. Ceux qui s'y connaissent trouvèrent que c'était un grand
dommage, parce qu'ils l'estimèrent excellent et fort exact d'ailleurs.
Venons maintenant à l'autre oubli qui regarde MM. de Bouillon.

\hypertarget{chapitre-xvii.}{%
\chapter{CHAPITRE XVII.}\label{chapitre-xvii.}}

1706

~

{\textsc{Chambre de l'Arsenal contre les faussaires.}} {\textsc{- Maison
de La Tour.}} {\textsc{- M\textsuperscript{lle} de Limeuil.}} {\textsc{-
Vicomte de Turenne La Tour, dit le maréchal de Bouillon.}} {\textsc{-
Sedan\,; son état\,; ses seigneurs.}} {\textsc{- Sedan acheté par
Éverard III de La Mark.}} {\textsc{- Bouillon acquis par MM. de La
Mark.}} {\textsc{- Folle déclaration de guerre du seigneur de Sedan, La
Marck, à Charles-Quint.}} {\textsc{- Sedan mouvant de Mouzon.}}
{\textsc{- Rang personnel de duc obtenu par le maréchal de Fleuranges La
Marck, seigneur de Sedan et Bouillon.}} {\textsc{- Son fils se donne le
premier le titre de prince de Sedan.}} {\textsc{- Bouillon\,; son
état\,; point duché\,; mouvant de Liège, auparavant de Reims.}}
{\textsc{- M. de Bouillon, seigneur de Bouillon plus que très
précaire.}} {\textsc{- Comte de Maulevrier, oncle paternel de
l'héritière, précède, sa vie durant, le maréchal de Bouillon partout.}}
{\textsc{- Comte de Braine.}} {\textsc{- Marquis de Mauny.}} {\textsc{-
Seigneurs de Lumain.}} {\textsc{- Comte de La Marck.}} {\textsc{-
Sommaire jusqu'à MM. de La Tour.}} {\textsc{- Maréchal de Bouillon La
Tour\,; titres qu'il prend, et ses deux infructueuses prétentions.}}
{\textsc{- Duc de Bouillon et son échange.}} {\textsc{- M. de Turenne.}}
{\textsc{- Change adroitement donné sur le titre de maréchal ou de
vicomte de Turenne.}} {\textsc{- Vicomté de Turenne.}} {\textsc{- Époque
du changement de style des secrétaires d'État et avec les secrétaires
d'État.}} {\textsc{- Qualité de prince absolument refusée à MM. de
Bouillon, au contrat de mariage de M. d'Elbœuf avec
M\textsuperscript{lle} de Bouillon.}} {\textsc{- Qualité de prince au
tombeau de M. de Turenne défendue par le roi\,; pourquoi point
d'épitaphe ni de nom.}} {\textsc{- Époque et raison du mort
\emph{Auvergne} ajouté au nom de La Tour.}} {\textsc{- Cartulaire de
Brioude.}} {\textsc{- \emph{Histoire de la maison d'Auvergne}, par
Baluze.}} {\textsc{- Le cardinal de Bouillon fait faire le cartulaire et
cette Histoire.}} {\textsc{- De Bar arrêté pour faussetés.}} {\textsc{-
Bouillon sollicitent pour de Bar.}} {\textsc{- Aveu du duc de Bouillon
au roi pour arrêter l'affaire, et de l'abbé d'Auvergne aux juges.}}
{\textsc{- De Bar, convaincu, s'avoue en plein tribunal fabricateur du
cartulaire, qui est déclaré faux, et lui faussaire.}} {\textsc{- Cause
et singularité de la peine infligée à de Bar.}} {\textsc{-
\emph{Histoire de la maison d'Auvergne}, par Baluze, publiée aussitôt
après.}}

~

On a vu (t. III, p.~210, 211) qu'en 1702, Matignon avait gagné un
terrible procès au parlement de Rouen contre un va-nu-pieds qui en fut
pendu, après lui avoir donné des années des plus cuisantes peines, qui
se prétendait son aîné, lui demandait tout son bien sur des titres de
tous les âges, qui avaient paru incontestables, et dont à la fin la
fausseté fut reconnue, et par lui-même avouée à la potence. Il semble
qu'il y ait dans de certains temps des modes de crimes comme d'habit. Du
temps de la Voysin et de la Brinvilliers, ce n'étaient qu'empoisonneurs,
contre lesquels on fit une chambre expresse qu'on appela \emph{ardente}
parce qu'elle les condamnait au feu. En celui dont je parle, ce fut une
veine de faussaires, qui devinrent si communs qu'il fut établi une
chambre composée de conseillers d'État, de maîtres des requêtes et de
conseillers au parlement, qui tint ses séances à l'Arsenal, uniquement
pour juger ces sortes d'accusations et de procès. Cela suffira
maintenant jusqu'à ce que j'aie expliqué ce qui en arriva à la maison de
Bouillon, mais qu'il faut traiter de plus haut et l'expliquer avec
l'étendue uniquement nécessaire pour l'entendre.

La maison de la Tour, originaire de la province d'Auvergne, bonne,
ancienne, bien alliée, heureuse en grandes successions de traverses, et
en quelques mariages dont l'événement lui a donné un éclat de hasard,
n'avait jamais eu ni prétendu aucune distinction particulière, et avait
toujours roulé d'égale avec les Montboissiers, les Montmorin, les
Saillant, et les premières maisons de leur commune province. On a vu (t.
Ier, p, 218), à propos du dauphiné d'Auvergne que le roi empêcha
Monsieur de vendre au cardinal de Bouillon, ce que c'est que cette
terre, et ce que c'est aussi que le comté d'Auvergne qui a été plus
d'une fois dans la maison de La Tour, et y est encore\,: toutes deux
terres toutes ordinaires et très distinctes de la province d'Auvergne.

François III de La Tour, vicomte de Turenne, mort en 1557, ne prétendit
pas plus que ses pères, quoique gendre du connétable Anne de
Montmorency. Lui et M\textsuperscript{lle} de Limeuil étaient enfants
des deux frères. Elle était fille d'honneur de la reine Catherine de
Médicis, trop connue par le malheur qui lui arriva. Je la cite ici pour
montrer par son emploi combien il était alors peu question chez MM. de
La Tour des prétentions que les troubles de l'État, où ils ont toujours
figuré contre les trois rois de la branche de Bourbon, leur ont fait
prospérer, après avoir pris naissance dans la faveur et la protection
d'Henri IV.

Henri de La Tour, vicomte de Turenne, fils de François III, et de la
fille du connétable pince de Montmorency, si connue sous le nom de
maréchal de Bouillon, est le premier qui ait eu des chimères. Henri IV,
qu'il avait bien servi, le lit premier gentilhomme de sa chambre, charge
dont il fit depuis sa cour à Marie de Médicis dans sa régence, en la
vendant au maréchal d'Ancre et en en tirant des avantages. Henri IV,
content de ses services de plus en plus, voulut faire sa fortune, et
s'assurer en même temps d'une frontière jalouse en la mettant entre les
mains d'un de ses plus affidés serviteurs. Il ne réussit que trop pour
ses intérêts à l'une, et fut cruellement trompé sur la suite qu'il en
attendait. Il fit le vicomte de Turenne maréchal de France, pour épouser
l'héritière de Sedan, Bouillon, Raucourt et Jametz. Le mariage se fit en
octobre 1591. Elle mourut à Sedan, 15 mai 1594, en couches d'un fils
mort en naissant, et ne laissa aucun enfant. Le maréchal de Bouillon
prétendit garder tout ce que possédait sa femme, en vertu d'un testament
fait par elle en sa faveur, pièce qu'il ne montra jamais parce qu'elle
n'exista jamais. Henri IV, par les mêmes raisons qui lui avaient fait
faire ce mariage, soutint l'usurpation, contre l'oncle paternel, de
l'héritage, qui n'en put avoir justice. On voit dans tous les Mémoires
et les histoires de ces temps combien Henri IV lui-même eut à s'en
repentir, et sa postérité après lui, et que l'époque de la souveraineté
du maréchal de Bouillon fut celle de son ingratitude et de ses
perfidies, desquelles ses enfants héritèrent avec ces mêmes biens.

Il s'était fait huguenot de bonne heure. Il se remaria en 1595 à une
fille du fameux Guillaume, prince d'Orange, qui, fondateur de la
république des Provinces-Unies, fut touché d'avoir un gendre puissant
dans les Ardennes et dans le parti huguenot en France. Dans cette
posture, il se trouvait beau-frère de Frédéric IV, électeur palatin, qui
avait épousé une autre fille du même prince d'Orange en 1593, dont il
eut le malheureux roi de Bohême, l'électrice de Brandebourg, et nombre
d'autres enfants. Tant de moyens et d'élévation étrangère, joints à tout
l'esprit, la capacité, le courage et l'ambition nécessaires à les faire
valoir, lui firent trouver trop étroites les bornes de sujet et de
particulier, et le jetèrent dans tous les complots dont les histoires
sont pleines. En même temps l'état de seigneur français, quant au rang,
ne lui déplut pas moins, et il forma là-dessus des prétentions qui ne
lui furent pas heureuses. Elles ne pouvaient porter sur sa naissance,
qui n'avait jamais eu, ni rang, ni distinction, ni préférence au-dessus
des autres seigneurs sans dignités, ni imaginé d'en prétendre, non pas
lui-même avant qu'il fût parvenu à cette fortune. Il ne les pouvait
tirer de la maison de La Marck dont il n'était pas, et dont l'héritière
ne lui avait point laissé d'enfants. Il essaya donc de les établir sur
sa qualité de prince souverain de Sedan. Avant de voir combien peu elles
lui réussirent, il est bon de voir quel fut l'état de ses prédécesseurs
à Sedan.

Adolphe, comte de La Marck, épousa en 1332 Marguerite de Clèves, et
devint par elle comte de Clèves. Il fit la branche aînée qui se divisa
en deux\,: les aînés furent ducs de Clèves et de Juliers, etc.\,; les
cadets s'établirent en France, y furent ducs de Nevers et comtes d'Eu,
et fondirent par deux sœurs héritières dans Gonzague, qui furent ducs de
Nevers, et par la suite durent l'héritage de Mantoue à la fermeté et à
la valeur personnelle de la protection de Louis XIII, et dans Guise qui
eurent Eu.

Le frère cadet de cet Adolphe fut Éverard III de La Marck, qui épousa en
1410 Marie, fille de Guillaume de Braquemont, seigneur de Sedan et de
Florenville, et de Marie de Campremy. M\textsuperscript{me} de
Braquemont était veuve en premières noces de Louis d'Argies, seigneur de
Béthencourt. Elle avait un frère duquel Éverard III de La Marck, son
mari, acheta en 1424 les seigneuries de Sedan et de Florenville, et fit
commencer la forteresse de Sedan en 1446. Jean, son fils, fit achever la
forteresse de Sedan dont il avait la seigneurie avec plusieurs autres,
et fut un des chambellans de Charles VII. Son frère, Louis de La Marck,
seigneur de Florenville, fut conseiller de René d'Anjou, roi de Sicile.
Jusqu'ici nul vestige de principauté ni de souveraineté dans la
seigneurie de Sedan ni de Florenville, qualifiées simplement de
seigneuries, ni dans les seigneurs de Braquemont, ni dans ceux de La
Marck qui l'achetèrent. On n'a jamais vu vendre ni acheter une
souveraineté entre des particuliers. Sedan relevait constamment de
Mouzon\,; sa situation dans les Ardennes et sur un bord jaloux de
frontière, avec la forteresse qui y fut bâtie, mirent ses seigneurs en
état de nager entre la France et la maison d'Autriche par le fait et la
commodité du lieu, non par aucun droit d'indépendance. Un souverain
n'eût pas été un des chambellans de Charles VII, ni son frère un des
conseillers d'un roi en peinture tel que fut le bon roi René, duc
d'Anjou, un moment de Lorraine, et comte de Provence.

Ce Jean de La Marck eut trois fils qui eurent postérité Robert Ier,
seigneur de Sedan, Fleuranges et Jametz\,; Éverard qui fit la branche
d'Aremberg, éteinte en son petit-fils, fondue dans la maison de Ligne\,;
et le fameux Guillaume, dit \emph{le Sanglier d'Ardenne}, un des
chambellans de Louis XI, qui fit soulever les Liégeois contre Charles,
dernier duc de Bourgogne et contre Louis de Bourbon, évêque de Liège,
qu'il tua en 1482. Toutes ces guerres, où il s'était rendu redoutable,
finirent l'année suivante, 1483, par le traité de Tongres, fait avec
Jean de Horn, évêque de Liège, et les états du pays, qui, pour les
dépenses qu'il avait faites à leur défense, lui donnèrent en payement le
duché de Bouillon, fief mouvant de Liège. Guillaume s'en accommoda avec
son frère aîné, Robert Ier de La Marck, seigneur de Sedan. Il tomba peu
après entre les mains de Maximilien d'Autriche, depuis empereur et
grand-père de Charles-Quint. Maximilien lui fit faire son procès à
Maestricht, où il eut la tête coupée en juin 1485. Ce Sanglier d'Ardenne
portait le nom de seigneur de Lumain, qu'il laissa à sa branche. C'est
l'unique qui subsiste aujourd'hui de toute cette grande, ancienne et
illustre maison de La Marck. Le comte de La Marck d'aujourd'hui, connu
par ses ambassades et chevalier de l'ordre, est son sixième descendant
en droite ligne.

Après avoir vu l'acquisition de Sedan, le marché et la donation de
Bouillon, revenons à Jean Ier de La Marck, seigneur de Sedan, qui eut le
duché de Bouillon de Guillaume son frère. Charles VIII le prit sous sa
protection, lui, son fils aîné et ses terres, contre Maximilien Ier,
archiduc d'Autriche, etc., par des lettres de 1486, qui, tout honorables
qu'elles lui sont, n'ont pas le moindre trait à souveraineté ni
principauté. Robert II, son fils, duc de Bouillon, seigneur de Sedan,
Fleuranges et Jametz, fut chevalier de Saint-Michel et compris dans les
traités de paix entre Charles VIII et Maximilien Ier, roi des Romains,
fait à Senlis en 1493, et de Cambrai en 1508, mais comme un seigneur de
frontière, sans rien qui sente la souveraineté. Depuis, ce Robert, après
avoir bien servi en France, se tourna pour la maison d'Autriche. Il en
fut plus mal content qu'il n'avait été de la France. Il s'y raccommoda,
puis s'outrecuida jusqu'à dénoncer la guerre à l'empereur par un héraut,
en pleine diète à Worms. Charles-Quint en rit, prit toutes ses places,
le ruina, et Sedan ne fut sauvé que par la guerre qui s'alluma entre la
France et l'empereur. Une pareille déclaration de guerre ne se prendra
jamais pour un titre de souveraineté, quand il est seul, le premier et
fondé sur aucun autre titre. Son fils et son petit-fils, tous deux du
nom de Robert, tous deux ducs de Bouillon, seigneurs de Sedan, etc.,
furent tous deux maréchaux de France. Le dernier des deux acheta
Raucourt, en 1549, de Charles de Luxembourg, vicomte de Martigues, et,
l'année suivante, il alla ambassadeur de France à Rome, auprès de Jules
II\footnote{Il y a dans le manuscrit Jules II\,; mais c'est évidemment
  une erreur pour Jules III, qui fut pape de 1550 à 1555.}. Ce n'était
pas l'emploi d'un souverain\,; aussi Bouillon était-il très constamment
mouvant de Liège, et Sedan de Mouzon, comme on le voit encore par les
lettres patentes de Charles VII en 1454, comme souverain de Mouzon, d'où
Sedan relevait, et par le jugement des jugeurs de Mouzon, rendu en 1455,
en conformité de ces lettres.

Ce dernier maréchal était connu sous le nom de maréchal de Fleuranges
plus que sous celui de maréchal de Bouillon. Il avait épousé la fille
aînée de la fameuse Diane de Poitiers et de son défunt mari Louis de
Brézé, comte de Maulevrier, grand sénéchal de Normandie. Il fut marié
quatorze ans sans avoir aucun rang en France, non plus que ses pères.
Henri II, dans le fort de ses amours et du crédit de Diane de Poitiers,
la fit duchesse de Valentinois, en 1548\,; et ce même crédit obtint
quatre ans après le rang de duc, en France, au maréchal son gendre, duc
de Bouillon, personnellement pour lui et pour sa femme par conséquent.
Il mourut en 1556, Henri II en 1559 et la maréchale de Fleuranges, qui
depuis ce rang ne s'appelait plus que la duchesse de Bouillon, en 1574.
Deux fils naquirent de ce mariage et plusieurs filles, dont l'aînée fut
la première femme du dernier connétable de Montmorency, et mère des
duchesses de Ventadour et d'Angoulême\,; les deux fils furent le duc de
Bouillon et le comte de Maulevrier, tous deux sans aucun rang ni
prétention.

Ce duc de Bouillon est le premier des seigneurs de Sedan qui en ait
changé le titre en celui de prince de son autorité particulière. Il fut
capitaine des Cent-Suisses de la garde du roi, céda, avec protestation
et promesse du roi de récompense, le château de Bouillon à l'évêque de
Liège avec quelques dépendances, conformément au traité du
Cateau-Cambrésis, 1559. Il épousa, en 1558, la fille aînée du premier
duc de Montpensier, sœur de cette abbesse de Jouars, défroquée et
huguenote, en 1572, qui épousa, en 1574, le fameux prince d'Orange
Guillaume, tué à Delft, 1584, dont elle eut la seconde femme du maréchal
de Bouillon La Tour, veuve de l'héritier de Sedan. Le duc de Bouillon
mourut en 1574. La princesse de Bourbon-Montpensier, sa femme, en 1587,
dont il laissa deux fils et une fille. Le cadet mourut sans alliance, en
1587, portant le nom de comte de La Marck. L'aîné, duc de Bouillon et
prince de Sedan, etc., mort à Genève sans alliance, le 1er janvier 1588,
à vingt-six ans, ayant par son testament institué sa sœur unique son
héritière universelle, à laquelle il substitua le duc de Montpensier,
frère de leur mère, et à celui-ci le prince de Dombes, son fils, leur
cousin germain\,; ainsi Charlotte de La Marck, eut Bouillon, Sedan, etc.
C'est elle à qui on fit épouser Henri de La Tour, vicomte de Turenne et
maréchal de France, si connu sous le nom de maréchal de Bouillon. Elle
était née à Sedan, à la fin de 1574, mariée à la fin de 1591, et mourut
en 1594, sans enfants, comme il a été dit, à Sedan, dont elle n'était
jamais sortie.

De cette courte analyse il résulte, que des huit générations de La Marck
qui ont possédé Sedan, dont les six dernières ont du Bouillon aussi,
aucune n'a eu ni prétendu aucun rang ni distinction à ces titres, ni à
ceux de leur naissance\,; que le seul dernier maréchal, grand-père de
l'héritière, a eu le rang personnel de duc par le crédit de sa
belle-mère, et qu'ils ont eu des charges et des emplois, que des princes
ou gens qui voudraient l'être n'auraient pas acceptés\,; que Sedan est
un fief mouvant du domaine de Mouzon, que c'est le père de l'héritière
qui le premier a changé, sans titre aucun et de son autorité privée, le
titre de seigneur de Sedan, que ses prédécesseurs avaient toujours pris,
en celui de prince de Sedan, et que la folie qu'eut le père du premier
maréchal de La Marck de déclarer la guerre à Charles-Quint ne leur donne
aucun droit de souveraineté, non plus que la protection accordée par
lettres de nos rois, ni la mention faite d'eux dans les traités de paix,
comme de tous autres seigneurs particuliers des frontières qui touchent
les dominations différentes\,; que Sedan relevait des archevêques de
Reims comme seigneurs de Mouzon, sans aucune difficulté, avant que le
roi se fût accommodé de ce domaine\,; enfin que Sedan, possédé par la
maison de Jausse en Brabant, ensuite par celle de Barbançon, seigneurs
de Bossu, après par celle de Braquemont, fut enfin vendu à celle de La
Mark, comme on a vu plus haut. Voilà pour Sedan. Raucourt, Jametz, etc.,
n'eurent jamais rien de particulier. Ce n'est pas la peine de s'y
arrêter.

Bouillon est une ancienne seigneurie démembrée du comté d'Ardenne, que
le célèbre Godefroy de Bouillon eut de sa mère Ide. Il était fils
d'Eustache, comte de Boulogne, et fut investi du duché de la basse
Lorraine. Comme il était duc, on l'appela le duc Godefroy de Bouillon,
parce qu'on était accoutumé auparavant à le nommer Godefroy de Bouillon,
selon la mode du temps pour les cadets de leur partage, et cette terre
n'a pas eu d'autre titre de passer et d'être dite le duché de Bouillon.
Godefroy, allant à la Terre sainte, où il devint si célèbre, vendit
Bouillon à Albert, évêque de Liège\,; et Alberon, depuis son successeur,
acquit, en 1127, de Renaud, archevêque de Reims, tout le fief que
l'église de Reims avait à Bouillon. C'était apparemment la mouvance. Au
moins ne prétendra-t-on pas qu'une terre sans titre et démembrée du
comté d'Ardenne fût une souveraineté. On a vu ci-devant comment elle a
passé des évêques de Liège de la maison de La Marck. Mais cette église
ni les états de Liège n'ont jamais cédé, non seulement la mouvance, mais
la propriété\,; et à travers les guerres et les traités jusqu'à celui de
Ryswick exclusivement, ils l'ont toujours revendiquée.

M. de Bouillon, fils du maréchal et frère aîné de M. de Turenne, et
petit-fils maternel du grand Guillaume, prince d'Orange, se trouvant
gouverneur de Maestricht pour les Hollandais, se fit craindre des
Liégeois, avec qui il traita, en 1641, sans prendre la qualité de duc de
Bouillon dans l'acte qu'il passa avec eux, et renonça à toutes
prétentions sur Bouillon et ses dépendances pour cent cinquante mille
florins, qu'il acheva de toucher, en 1658, sans avoir pourtant cessé de
porter le même nom\,; et au traité des Pyrénées, il ne se parla plus de
Bouillon, possédé par les Liégeois. Ils prirent parti pour l'empereur,
en 1676, contre le roi. Les François prirent Bouillon, que le roi donna,
en 1678, au duc de Bouillon, fils de celui dont on vient de parler, qui,
sans aucun titre de souveraineté possible, comme on vient de le voir, y
établit une cour souveraine. Cette entreprise fit une grande difficulté
à la paix de Nimègue, mais à la fin les Liégeois cédèrent et
protestèrent\,; et il fut dit que la possession demeurerait à M. de
Bouillon, et que la question de la propriété serait décidée par des
arbitres. Oncques depuis il n'en a été parlé.

On voit donc combien Bouillon est éloigné de pouvoir être une
souveraineté, et à quel étrange titre M. de Bouillon en jouit. Il n'est
pas nécessaire de s'y étendre davantage. En aucun temps depuis, les
évêques, le chapitre et les états de Liège auraient été mal reçus à
disputer Bouillon, quoique payé tant de fois, et de plus de leur ancien,
domaine, au fils de celui à qui ils l'avaient si bien payé la dernière,
à qui Louis XIV l'avait donné après l'avoir pris sur eux, et qui lui a
toujours accordé sa protection pour le garder. La suite de ce qu'est
devenu Bouillon, pour n'être pas interrompue, nous a conduits jusqu'à
Louis XIV et à son grand chambellan. Avant de parler de la maison de
celui-ci, il faut achever ce qui regarde celle de La Marck.

On a vu ci-devant que l'héritière de Sedan et Bouillon avait un oncle
unique, frère cadet de son père. Il portait le nom de comte de
Maulevrier, et prit le nom de duc de Bouillon après la mort de sa nièce,
en 1594. Il n'eut jamais ni ne prétendit aucun rang, servit Charles IX
et Henri III en leurs guerres, fut capitaine des Cent-Suisses de la
garde, et chevalier de l'ordre, le dernier décembre 1578, qui est la
première promotion qui ait été faite.

Les ducs de, Nevers-Gonzague, Mercœur, frère de la reine, femme d'Henri
III, Uzès-Crussol, et Aumale-Lorraine étaient en ce rang de leurs duchés
à la tête de la promotion. Le comte de Maulevrier y eut le
vingt-quatrième rang, c'est-à-dire le vingtième parmi les gentilshommes,
et n'en eut que trois après lui. Il marcha entre M. d'Estrées, père du
premier maréchal et de la belle Gabrielle, et M. d'Entragues, père de la
marquise de Verneuil, c'est-à-dire entre les deux pères des deux trop
fameuses maîtresses d'Henri IV. Il lutta longtemps contre le maréchal de
Bouillon pour l'héritage de sa nièce. On a encore les factums et les
écrits qu'il publia sur l'usurpation qui lui était faite et sur les
incroyables dénis de justice et les violences qu'il essuyait par
l'autorité d'Henri IV et les artifices du maréchal. De guerre lasse et
désespérant de pouvoir obtenir de jugement en aucun tribunal, qui tous
se trouvaient fermés pour lui par une suite continuelle de violences, il
transigea avec le maréchal de Bouillon, 25 août 1601\,; et l'une des
conditions de la transaction confirmée par le roi fut qu'il précéderait
en tous lieux le maréchal de Bouillon pendant sa vie, ce qui lui fut
exactement tenu, et mieux que les articles pécuniaires avec lesquels il
courut longtemps sans succès. Avec cette préséance sur le maréchal de
Bouillon, et le nom de duc de Bouillon qu'il prit à la mort de sa nièce,
il ne prétendit jamais aucun rang, comme on l'a dit, il demeura parmi
les gentilshommes dans les cérémonies de l'ordre, comme il y avait été
reçu, et il mourut en septembre 1622, à quatre-vingt-quatre ans, ayant
été ainsi quarante-quatre ans chevalier de l'ordre.

D'une Averton, sa première femme, il n'eut qu'une fille, mariée à
Comblisy, fils du secrétaire d'État Pinart. Sa seconde femme {[}était{]}
fille de Gilles de La Tour, seigneur de Limeuil, et de Marguerite de La
Cropte, et sœur de M\textsuperscript{lle} de Limeuil, fille d'honneur de
Catherine de Médicis, qui la chassa pour être accouchée du fait du
prince de Condé dans la garde-robe de cette reine à Lyon, et de laquelle
j'ai dit un mot plus haut. Le comte de Maulevrier eut Henri-Robert de La
Marck, comte de Braine\,; Louis de La Marck, marquis de Mauny\,;
Alexandre de La Marck, abbé de Braine et d'Igny, qui ne figura point,
non plus qu'un quatrième, mort sans enfants d'une Rennequin.

Le comte de Braine prit, à la mort de son père, le nom de duc de
Bouillon, et poursuivit ses droits sur la succession de sa cousine aussi
peu heureusement que son père. Il fut aussi capitaine des Cent-Suisses
de la garde. Il trouva dans les deux puissants et célèbres fils du
maréchal de Bouillon, mort un an après son père, de quoi être tenu dans
l'obscurité. Il mourut, depuis longtemps retiré en sa maison de Braine,
quelques mois après l'autre duc de Bouillon La Tour, la même année 1652,
à soixante-dix-sept ans. De Marguerite d'Autun, sa première femme, il ne
laissa que des filles qui finirent cette branche. L'une épousa M. de
Choisy-L'Hôpital, l'autre M. de La Boulaye-Eschallart, dont les enfants
héritèrent des biens de cette branche éteinte, en prirent le nom et les
armes, et ont fini en la duchesse de Duras, mère de la princesse de
Lambesc et de la comtesse d'Egmont. Je ne parle point de la troisième
femme du comte de Maulevrier, ni des deux dernières de ce comte de
Braine, qui n'ont point eu d'enfants.

Le marquis de Mauny, frère puîné du comte de Braine, fut chevalier de
l'ordre en 1619, le cinquante et unième de la promotion, c'est-à-dire le
trente-neuvième parmi les gentilshommes. Huit autres le suivirent, dont
le quatrième fut le marquis de Marigny, depuis comte de Rochefort,
Alexandre de Rohan, frère cadet du duc de Montbazon, oncle paternel de
la connétable de Luynes, depuis la célèbre duchesse de Chevreuse. Le
marquis de Mauny fut premier écuyer de la reine Anne d'Autriche, et
capitaine des gardes du corps de la dernière compagnie en 1621, après M.
de La Force, jusqu'en 1627, que M. de Brézé-Maillé lui succéda, qui
était beau-frère du cardinal de Richelieu et fut maréchal de France, à
qui M. d'Aumont, aussi maréchal de France depuis, succéda en 1632. Le
marquis de Mauny mourut capitaine des gardes, sans enfants d'Isabelle
Jouvenel, fille du baron de Traynel, chevalier de l'ordre.

Toute cette branche éteinte, il ne resta plus de toute la maison de La
Mark, que celle de Lumain plus haut expliquée, sortie du Sanglier
d'Ardenne\,; elle demeura aux Pays-Bas de Liège et de Westphalie, et
s'allia dans ces provinces, excepté Guillaume de La Marck, second fils
de ce fameux Sanglier, qui fut un des chambellans de Louis XII, et
capitaine des Cent-Suisses de sa garde. Lui, son fils unique et ses deux
filles se marièrent en France\,; et son fils, qui n'eut point d'enfants,
finit cette courte branche.

Ernest, cinquième descendant direct du Sanglier, fut premier comte de
Lumain. Il eut un fils d'une Hohenzollern, mort longtemps après lui sans
postérité, mais Ernest épousa en secondes noces Catherine-Richard
d'Esche\,; je ne sais même si ce put être de la main gauche\footnote{On
  appelle en Allemagne \emph{mariage de la main gauche} ou \emph{mariage
  morganatique} l'union légitime d'une personne de haute qualité avec
  une personne de condition inférieure.}, comme ils parlent en
Allemagne, tant la naissance était disproportionnée. Il en laissa deux
fils et deux filles, l'une religieuse à Liège, l'autre mariée en fille
de mère de fort peu. Le cadet des deux fils mourut obscur sans
alliance\,; l'aîné redevint baron de Lumain par le triste mariage dont
il était sorti. Mais l'empereur le réhabilita et le fit même comte de
l'empire. Il mourut en 1680 et laissa trois fils de Catherine-Charlotte,
fille du comte de Wallenrode, qui se remaria au comte de Fürstemberg,
neveu du cardinal de Fürstemberg. C'est cette comtesse de Fürstemberg
qui gouverna et pilla le cardinal de Fürstemberg tant qu'il vécut, qui
en fit après sa mort une longue et sérieuse pénitence, et de laquelle
j'ai parlé sur la coadjutorerie de Strasbourg. Elle n'eut point
d'enfants de son second mari. Venue et fixée en France avec le cardinal
de Fürstemberg qu'elle ne quitta jamais, elle amena deux de ses fils et
laissa le dernier en Allemagne, où il est devenu lieutenant
feld-maréchal des armées impériales. L'aîné mourut de bonne heure à
Paris sans alliance, ayant un régiment qui fut donné au second, beau et
bien fait, et qui ressemblait au cardinal de Fürstemberg comme deux
gouttes d'eau. C'est le comte de La Marck qui a épousé une fille du duc
de Rohan, de la mort de laquelle j'ai parlé, qui était debout à la cour
sans nulle prétention, et qui a laissé un fils. Le comte de La Marck,
fort employé aux négociations, était ambassadeur de France auprès du
fameux roi de Suède\,; et dans son camp lorsqu'il fut tué. Il est devenu
lieutenant général et fut fait chevalier de l'ordre en 1624, le
quarante-deuxième de la promotion, c'est-à-dire le vingt-quatrième parmi
les gentilshommes, dont il eut huit autres après lui. Il alla longtemps
depuis ambassadeur en Espagne, d'où il est revenu grand d'Espagne et
chevalier de la Toison d'or, à l'occasion du mariage de Madame, fille
aînée du roi, avec l'infant don Philippe, troisième fils du roi
d'Espagne.

En voilà assez, ce semble, pour demeurer persuadé que Sedan ni Bouillon
ne furent jamais principautés, duchés, encore moins souverainetés\,; que
l'un et l'autre sont demeurés à MM. de Bouillon La Tour, très
précairement, pour ne pas dire fort étrangement\,; qu'aucun seigneur de
ces deux terres n'a été ni prétendu être souverain, jusqu'au père de
l'héritière\,; et que pas un d'eux, ni avant ni depuis, n'a eu de rang
en France, ni pas un de leur maison, ni n'en ont prétendu, si on excepte
le seul maréchal de Fleuranges qui, par le crédit de la duchesse de
Valentinois, maîtresse d'Henri II, sa belle-mère, eut personnellement
rang de duc. Tel a été l'état des choses à cet égard jusqu'au vicomte de
Turenne, Henri de La Tour, devenu maréchal de Bouillon. Aux pays
étrangers il n'en a pas été différent, en aucun desquels Sedan ni
Bouillon n'ont jamais passé pour ni souveraineté ni pour principautés\,;
aucun de leurs seigneurs n'a été reconnu en aucune cour de l'Europe pour
souverain ni même pour prince, et n'a prétendu aucun rang ni aucune
distinction comme tels en pas une. Voyons maintenant ce qu'en a su faire
le maréchal de Bouillon La Tour et sa postérité.

Les étranges moyens par lesquels ils sont parvenus au rang et aux biens
dont ils jouissent, et aux grands établissements de toutes les sortes
qu'ils ont su se procurer, remplissent nombre de volumes qui sont entre
les mains de tout le monde. Je me renferme ici à ce qui est de mon
sujet, faits qu'ils ont pris et prendront grand soin d'étouffer autant
qu'il leur sera possible. Il n'y en a que deux du maréchal de Bouillon
en France. Gendre du fondateur des Provinces-Unies, comme à la tête du
parti huguenot en France, beau-frère de l'électeur palatin, oncle de ses
enfants, par conséquent de l'infortuné roi de Bohême et de l'électrice
de Brandebourg, tranchant par la voie de fait de souverain de Sedan et
de Bouillon, par l'argent, la faveur et toute la protection d'Henri IV,
bientôt après par ceux de ses ennemis contre ce monarque et contre son
fils, parmi des entreprises et des abolitions continuelles, il voulut
essayer de se procurer un rang qui répondît à tant de grandes choses. Il
n'en eut jamais aucun en France. Il n'y eut que les distinctions
communes à tous les maréchaux de France. Il se trouva à l'assemblée des
notables à Rouen, où Henri IV était présent et en fit l'ouverture. Le
maréchal de Bouillon s'avisa de s'aller mettre dans le banc des ducs,
qui l'en firent sortir\,; sa ressource fut de s'aller placer à la tête
de celui des maréchaux de France, dont il se trouva l'ancien, mais il
sentit toute la mortification d'une tentative si peu heureuse.

L'autre fait arriva au baptême de Louis XIII, que Henri IV fit faire
très solennellement. Il nomma le maréchal de Bouillon, quoique huguenot,
pour porter un des honneurs\footnote{Dans certaines cérémonies, comme le
  sacre, le baptême des princes, leurs funérailles, etc., on appelle
  \emph{honneurs} les principales pièces qui servent à la cérémonie,
  comme la couronne, le sceptre, l'épée, etc., pour le sacre\,; le
  cierge, le chrémeau, l'aiguière, etc., pour les baptêmes.}, car il n'y
a point de difficulté avec les huguenots pour le baptême, lorsqu'il ne
s'agit pas d'être parrain. Le maréchal qui se vit au rang de maréchal de
France pour l'honneur qui lui était destiné à porter, se rabattit à
supplier Henri IV de lui permettre de n'en porter aucun, ce qu'il obtint
fort aisément. Il se contenta de ces deux tentatives, et n'osa pas se
commettre à en entreprendre davantage, dans les intervalles qu'il passa
à la cour. Il prit toujours dans ses titres la qualité de prince
souverain de Sedan, de duc souverain de Bouillon, et ne signa jamais ni
actes ni lettres que simplement Henri de La Tour. Pour sa femme, elle
passa toute sa vie à Sedan, où il mourut en mars 1623, et elle en
septembre 1643, aussi ambitieuse et guère moins habile que son mari.

Leurs enfants furent les deux célèbres frères, le duc de Bouillon et le
vicomte de Turenne, la duchesse de la Trémoille, la comtesse de Roucy La
Rochefoucauld, mère du comte de Roye, mort retiré en Angleterre, la
marquise de Duras, mère des maréchaux de Duras et de Lorges, et du comte
de Feversham, M\textsuperscript{me} de La Moussaye-Goyon, comme les
Matignon, dont la branche s'est éteinte, et dont les filles furent
M\textsuperscript{me}s de Montgomery et du Bordage, et
M\textsuperscript{lle} de Bouillon, morte en 1662 sans alliance.

Les deux fils ne furent ni moins ambitieux, ni moins habiles, ni moins
remuants que leur père. Leurs vies, dont les histoires de leur temps
sont remplies, ne furent de même qu'un cercle d'entreprises et
d'abolitions, et leur union, leur concert, leur mutuel appui,
incomparables. Ce qui devait coûter la tête à M. de Bouillon lui procura
ce qu'il n'eût pas eu en récompense s'il eût sauvé l'État. Le cardinal
Mazarin voulut s'attacher deux frères de ce mérite\,; il eut peur de
celui du cadet qu'il ne tenait pas, et de ses alliances étrangères s'il
livrait l'aîné au supplice. Il le changea aux plus grands honneurs et
aux plus solides biens, et se les acquit par de si prodigieux bienfaits
qu'il sacrifia à l'appui qu'il en espérait contre les puissances
ennemies qui, sous l'aveu de Gaston et de M. le Prince, le voulaient
chasser pour toujours du royaume. Il fit donc faire un échange de Sedan
et de Bouillon, dont M. de Bouillon se réserva l'utile, et ne céda que
la souveraineté, qui n'exista jamais que de fait, et depuis si peu, et
qu'il n'était plus en situation de soutenir, au lieu de laquelle il eut
le comté d'Évreux avec les bois et les dépendances, qui valaient plus de
trois cent mille livres de rentes, et les duchés d'Albret et de
Château-Thierry, avec la dignité de duc et pair et le rang nouveau des
princes étrangers en France. Il eut ainsi les apanages de deux fils de
France, et celui qu'avait Henri IV avant d'être roi de France. Quelque
ordinaire que fût la terre qui porte le nom de comté d'Auvergne, et
quelque distincte, et totalement, qu'elle fût de la province d'Auvergne
dans laquelle elle est située, M. de Bouillon la voulut avoir, et le
cardinal Mazarin eut la complaisance de la retirer des mains où elle
était pour la comprendre dans l'échange.

Il fut fait en mars 1651, lors des plus grands troubles, et M. de
Bouillon mourut à Pontoise à la suite de la cour, où il pouvait tout sur
la reine et sur le cardinal Mazarin, 9 août 1652, étant dans le conseil
le plus intime, et sur le point d'être déclaré surintendant des
finances. Il n'avait pas encore cinquante ans\,; son père en avait vécu
soixante-huit. Sa femme belle, vertueuse, courageuse, ambitieuse et fort
habile, fille du comte de Berghes gouverneur de Frise, ne le survécut
que de cinq ans. C'est ce duc de Bouillon qui a commencé à être prince
en Italie avant que l'être devenu en France par son échange. Il y
commanda les troupes du pape, dont il obtint à Rome le traitement de
souverain, et eut un tabouret devant lui. Il sut bien faire valoir
depuis cette grande distinction ailleurs où elle lui aplanit beaucoup de
choses\,; mais toutefois le parlement de Paris, épouvanté de l'immensité
de l'échange, et qui d'ailleurs ne connaît de princes que ceux du sang,
ne put se résoudre d'en faire l'enregistrement, qui n'est pas encore
consommé aujourd'hui\,; mais en attendant, MM. de Bouillon ont toujours
joui depuis des biens et des honneurs.

M. de Turenne dont les actions, la réputation et les menées avaient tant
contribué à porter sa maison jusqu'où elle était à la mort de son frère
aîné, singulièrement modeste sur ses grandes qualités jusqu'à
l'affectation, suprêmement glorieux, délicat et attentif sur sa
prétendue qualité de prince, et la cachant toutefois sous une simplicité
d'habits, de meubles et d'équipages, dont l'ombre faisait sortir
davantage le tableau, n'oublia rien dans la suite de sa vie pour
confirmer de plus en plus cette nouvelle principauté, et augmenter les
établissements de sa famille. Son frère avait laissé cinq fils et quatre
filles, c'était bien des princes et des princesses pour l'être si
nouvellement. M. de Turenne, dont les services et la capacité militaire
et politique avaient porté la considération et le crédit au comble, les
sut bien pourvoir pour la plupart. Il acheva le mariage projeté dès le
vivant du cardinal Mazarin d'une des Mancini ses nièces avec le duc de
Bouillon son neveu, qu'il appuya ainsi du duc de Vendôme, de la comtesse
de Soissons, de chez qui le roi ne bougeait lors et qui était le centre
de la cour, de l'alliance si proche du prince de Conti, et aux pays
étrangers du duc de Modène et du connétable Colonne, avec de grands
biens.

Le duc de Joyeuse, père du dernier duc de Guise, qui eut l'honneur
d'épouser M\textsuperscript{lle} d'Alençon, était mort en 1654, ne
laissant que ce fils âgé de quatre ans, et les charges de grand
chambellan et de colonel général de la cavalerie vacantes. C'était alors
le fort de l'autorité de M. de Turenne à la cour. Il la venait de sauver
à Bléneau des mains de M. le Prince, accouru secrètement de Guyenne, et
qui enlevait subitement le roi, la reine et le cardinal Mazarin, sans la
diligence et la profonde science militaire de M. de Turenne. Il chassa
d'autour de Paris enfin, et de Paris même, M. le Prince par le combat du
faubourg Saint-Antoine, qui fut réduit à se retirer en Flandre, et dont
le parti tomba tout à fait dans le royaume. La gloire de M. de Turenne
s'accrut de nouveau en 1653 par la prise de Rethel et de Mouzon. Enfin
en 1654, il força les lignes d'Arras, où M. le Prince était en personne,
qui eut grand'peine à se retirer, et qui laissa toute l'artillerie, les
munitions et les bagages qu'il avait menés à ce siège. En ce point de
gloire, et de nécessité qu'on se crut avoir de lui, il voulut la
dépouille du duc de Joyeuse, et le cardinal Mazarin la lui donna. Il
prit pour soi la charge de colonel général de la cavalerie, et pour le
duc de Bouillon celle de grand chambellan, qui n'avait alors que treize
ans.

On peut juger si M. de Turenne sut faire en entier sa charge dans la
cavalerie et s'y rendre le maître. Pour son neveu, outre la grandeur de
l'appui de l'office de la couronne qu'il lui procura, qui, par la place
qu'elle donne partout jusque dans les lits de justice auprès du roi, le
tirait d'embarras partout avec son idée de prince souverain, dont il
prenait toujours la qualité. Quoique cédée au roi par l'échange, une
charge si intime et qui approche le roi de si près en tous lieux, et à
toutes les heures les plus particulières, était d'un grand usage à un
homme de l'âge de M. de Bouillon, et qui n'avait que trois ans moins que
le roi, et nous verrons bientôt qu'elle a sauvé MM. de Bouillon du
naufrage.

M. de Turenne, si magnifiquement récompensé, continua ses exploits. Il
prit le Quesnoy, Landrecies, Condé, Saint-Guillain en 1655\,; l'année
1656 parut encore plus savante, quoique avec moins de brillant. En 1657
le roi assiégeant Dunkerque, et M. le Prince, et don Juan d'Autriche
ayant amené toutes leurs forces pour délivrer cette importante place, M.
de Turenne les délit à la bataille des Dunes, dont la prise de
Dunkerque, et d'autres suites encore, furent le prix. Il fallut une
nouvelle récompense à de nouveaux services, et si importants. L'épée de
connétable était bien le but du modeste héros, mais la timidité du
cardinal Mazarin ne put se résoudre à la mettre entre des mains si
puissantes et si habiles. Le souvenir de ce qu'avaient pu les derniers
connétables de Montmorency et leurs prédécesseurs, le souvenir même de
M. de Lesdiguières faisaient encore peur à la cour. Elle en sortit par
renouveler en faveur de M. de Turenne la charge de maréchal général des
camps et armées de France, imaginée et créée pour M. de Lesdiguières,
lorsque le duc de Luynes, abusant de la jeunesse de Louis XIII qui
n'avait lors que dix-sept ans et n'avait encore pu voir le jour par
l'éducation qu'on lui avait donnée que parle trou d'une bouteille, se
fit connétable. Ce fut à Montpellier, le 7 avril 1660, que M. de Turenne
reçut cette charge de la main du roi qui y était avec la reine sa mère,
le cardinal et toute sa cour, allant à Bordeaux pour son mariage.

Alors M. de Turenne supérieur aux maréchaux de France qu'il commandait
tous, cessant de l'être lui-même, mais n'étant pas connétable, et ne
pouvant en porter les marques, ne voulut plus de celles de maréchal de
France, dont il quitta les bâtons à ses armes, et le titre de maréchal,
qu'il avait toujours porté depuis plus de dix-sept ans qu'il l'était,
pour reprendre celui de vicomte de Turenne qu'il avait porté avant
d'être maréchal de France. Il signa tout court Turenne ou Henri de La
Tour, dans tous les temps de sa vie\,; ainsi il n'y changea rien. Dans
les suites on prit le change, et MM. de Bouillon y ont donné cours tant
qu'ils ont pu. On se persuada qu'il avait toujours méprisé l'office de
maréchal de France, qu'il n'en avait point pris ni le nom ni les marques
à ses armes, comme étant au-dessous du rang et de la qualité de prince.
Il n'y avait pourtant qu'à se souvenir du maréchal de Bouillon son père,
souverain d'effet et de fait, sinon de droit, et les deux maréchaux de
La Mark et de Fleuranges, père et fils, tous deux seigneurs de Sedan et
de Bouillon. Mais le gros du monde ne va pas si loin, et pour peu qu'on
ait lu quelques pages, on est étonné des idées qu'on voit prendre pied.

M. de Turenne obtint pour la vicomté de Turenne, qui avait déjà de
grands droits, de nouveaux privilèges qu'il fit augmenter par degrés.
Sous prétexte de l'inimitié ouverte qui était entre lui et M. de Louvois
déjà fort puissant par lui-même, outre l'appui du chancelier son père,
il délivra cette vicomté de tout logement et de tout passage de gens de
guerre, et par la connivence de M. Colbert, son ami, de tout le pouvoir
des maltôtiers, même des intendants. En un mot, ces droits devinrent des
droits régaliens\footnote{Les \emph{droits régaliens}, ou droits qui
  étaient semblables à ceux des rois, étaient à l'époque féodale, le
  droit de faire la guerre, de rendre la justice, de battre monnaie, et
  de percevoir les impôts. La plupart des seigneurs jouissaient des
  droits régaliens.} que sa mémoire a toujours maintenus, mais si à
charge au dedans du royaume, et si voisins de la souveraineté, que le
conseil de Louis XV, profitant du désordre des affaires de M. de
Bouillon et de son mécontentement des principaux de sa vicomté, l'a
achetée quatre millions de lui, et a cru avec raison qu'il faisait une
mauvaise affaire et le roi une fort bonne.

Parlant de M. de Louvois, voici une anecdote dont M. de Turenne sut
profiter. Les secrétaires d'État avaient toujours écrit aux ducs
\emph{monseigneur}, et c'est aux soins et à l'autorité de ceux de cette
époque qu'est due l'adresse de l'avoir fait réformer dans les lettres
imprimées. Le pur hasard a laissé en existence trois lettres des 2
novembre 1663, 13 septembre 1665, 5 février 1666, de M. Colbert, alors
ministre et contrôleur général des finances, qui avait le même
cérémonial que les secrétaires d'État, et qui le fut en 1669, à mon père
à Blaye, qui lui écrit \emph{monseigneur} dessus, dedans et au bas, en
marquant son nom. M. de Louvois, monté au comble de crédit et d'orgueil,
fit entendre au roi que ce style ne pouvait convenir à ceux qui par
leurs charges donnaient ses ordres et écrivaient en son nom. Il le
changea donc, mais il n'osa toucher à la maison de Lorraine, toute
brillante du grand mariage de M. de Guise, de la mémoire toute récente
du comte d'Harcourt, de la faveur de M. le Grand son fils, ni s'exposer
aux cris de M\textsuperscript{lle} de Guise si haute et si considérée,
moins encore à ceux de Monsieur possédé par le chevalier de Lorraine. Ce
fut un des fruits des quatorze érections de duchés-pairies de 1663, et
des quatre autres de 1665, et du peu de concert et de force des ducs
anciens et nouveaux.

M. de Turenne, averti à temps de cette entreprise, fut trouver le roi et
cria si haut et avec tant d'autorité contre un complot fait par son
ennemi pour l'humilier, et de l'exception de la maison de Lorraine à
l'égalité du rang et des honneurs de laquelle il avait été élevé, qu'il
obtint que sa maison conserverait le \emph{monseigneur} des secrétaires
d'État, ce que celle de Rohan n'eut pas, quoiqu'en pareil rang que MM.
de Bouillon\,; et quelque crédit qu'ait eu M\textsuperscript{me} de
Soubise, jamais dans la suite elle ne l'a pu emporter.

Pour achever l'anecdote des secrétaires d'État, M. de Louvois n'en
demeura pas en si beau chemin. Le même prétexte de flatterie, quelque
grossière qu'elle fût, lui fit obtenir du roi que tout ce qui ne serait
ni duc, ni prince, ni officier de la couronne, lui écrirait
\emph{monseigneur}, ce qui de lui passa aux autres secrétaires d'État,
et le rare fut qu'il ne le prétendit que des gens de qualité, et point
du clergé ni de la robe. Beaucoup de gens distingués le refusèrent et
furent perdus. M. de Louvois les poursuivit partout, et le roi y ajouta
toutes les marques de disgrâce\,: ces exemples, qui n'en manquèrent
aucun, soumirent enfin tout le monde, et il n'y eut plus personne qui ne
portât ce joug, auxquels les secrétaires d'État ajoutèrent encore
l'inégalité des souscriptions pour tout ce qui n'était pas titré. Cela a
duré jusqu'à l'éclipse des secrétaires d'État à la mort de Louis XIV.

M. de Turenne maria le comte d'Auvergne, son neveu, à la fille unique et
seule héritière du prince de Hohenzollern, marquis de Berg-op-Zoom par
sa femme. Cette grande terre en Hollande avec beaucoup d'autres biens,
avec une alliance étrangère, entée sur celle de la mère et la
grand'mère, parut au vicomte un établissement pour son neveu cadet, qui
pouvait en son temps avoir de grands avantages. Il ne tarda pas à lui
faire accorder ses survivances de la charge de colonel général de la
cavalerie et de son gouvernement de Limousin. On a vu (t. II, p.~164)
avec quelle adresse lui et son troisième neveu mirent le roi en
situation de leur offrir pour lui sa nomination au cardinalat, et de
s'en croire quitte à bon marché en la lui donnant, et la charge de grand
aumônier deux ans après. C'est-à-dire qu'il fut cardinal à vingt-cinq
ans, et grand aumônier à vingt-sept. Tels furent les établissements que
M. de Turenne procura à sa maison, à ses trois neveux et à soi-même.
Mais parmi tant de splendeur, il reçut quelques déplaisirs. Ses deux
derniers neveux, enflés d'une situation si brillante, furent tous deux
tués en duel\,; et il eut la douleur que, mariant leur sœur à M.
d'Elbœuf, jamais M. de Lorraine ne voulurent passer à la future ni aux
siens les qualités de prince et de princesse. Le mariage en fut rompu,
puis renoué, mais avec la même opiniâtreté de la part des Lorrains. À la
fin, M. de Turenne céda, et conclut le mariage avec la douleur du bruit
que cela fit dans le monde. Il trouva depuis le moyen de marier son
autre nièce, sœur de celle-ci, à un frère de l'électeur de Bavière, l'un
et l'autre morts sans enfants. Je ne sais si la maison de Bavière eut la
même délicatesse que la maison de Lorraine, ni si celle-ci l'a soutenue
au contrat de mariage de M. de Bouillon, père de celui-ci, avec sa
troisième femme, fille du comte d'Harcourt, dit depuis le comte de
Guise.

M. de Turenne acheva sa vie avec la même gloire et la même autorité
auprès du roi, et la termina comme chacun sait. La majesté de ses
obsèques et de sa sépulture n'eut aucun rapport à sa naissance ni à tout
ce qu'il avait acquis d'extérieur. Ce fut la récompense de ses vertus
militaires et de la mort qui les couronna par un coup de canon à la tête
de l'armée. Le roi défendit même très expressément que la qualité de
prince fût employée nulle part à Saint-Denis\,; et c'est ce qui a fait
que ses neveux, qui lui ont fait faire dans cette église un superbe
mausolée dans une chapelle magnifique, n'y ont fait mettre aucune
épitaphe, en sorte qu'à voir ce tombeau, on ne peut conjecturer que
c'est celui de M. de Turenne que par sa figure qui ressemble à tous ses
portraits, et par ses armes qui n'ont d'autre ornement que la couronne
de duc et des trophées. Il n'y a même aucun vers, aucune louange, parce
qu'on n'a osé mettre cette précieuse qualité de prince, et qu'on n'a pas
voulu montrer qu'on l'évitait.

C'est du temps de ces deux fameux frères, que le nom d'Auvergne a peu à
peu été joint à celui de La Tour. Il y a en Limousin, en Dauphiné et en
d'autres provinces des maisons de La Tour, qui ne sont point de
celle-ci, et qui toutes ont des armes différentes les unes des autres,
et n'ont aucune parenté entre elles. Ce mot d'Auvergne s'ajouta d'abord,
comme pour distinction et pour montrer de laquelle on parlait\,; après,
cela devint équivoque, l'attachement à ce mot pour s'en faire un nom
découvrit le projet. Le cardinal de Bouillon se prétendit sorti par mâle
des anciens comtes de la province d'Auvergne, cadets des ducs de
Guyenne, et n'omit rien pour trouver à Cluni, qui est de la fondation de
ces princes, de quoi appuyer cette chimère. Elle lui venait sans doute
de plus loin. On a vu l'affectation avec laquelle ils voulurent avoir
par l'échange cette terre particulière, qui a été ailleurs plus d'une
fois expliquée, et qu'on appelle le comté d'Auvergne. Le second fils du
duc de Bouillon, qui fit l'échange, en porta le nom. Ils espérèrent la
confusion dans l'esprit du gros du monde du titre d'une terre médiocre,
ordinaire, et tout à fait sans distinction, et particulière, avec celui
du titre de la province même, et persuader ainsi leur origine des
anciens comtes de la province d'Auvergne, puisqu'ils en portaient le nom
et le titre, comme la plupart des gens sont infatués que les Montmorency
sont les premiers barons du royaume, parce qu'ils prennent le titre de
premiers barons de France, c'est-à-dire de la France proprement dite
comme province, qui est grande comme la main, autour de Montmorency et
de l'abbaye de Saint-Denis, dont Montmorency relevait, et que de sa
situation on appelle Saint-Denis en France.

C'était donc non plus simplement déplaire, mais offenser le cardinal de
Bouillon et les siens, que de parler de leur maison sous le seul nom de
La Tour, comme leurs pères l'avaient toujours pris et signé uniquement
partout\,; il fallut dire La Tour d'Auvergne, jouant sur le mot, et se
garder surtout de l'expression trop claire de La Tour en Auvergne, qui
ne se pardonnait point. Ils avaient enfin compris le peu de sûreté d'un
rang qui se peut ôter comme il a pu être donné, avec la différence que
le dernier est justice et raison\,; d'un rang sans prétexte de
naissance, puisque leurs pères n'y avaient jamais prétendu, et n'avaient
jamais été distingués de tous les autres seigneurs qui n'avaient ni
dignité ni office de la couronne\,: ils ne pouvaient se dissimuler à
eux-mêmes que la possession, même légitime, de Sedan ni de Bouillon
n'avait jamais donné ni fait prétendre aucun rang ni distinction en
France, et nulle part en Europe\,; qu'ils ne sortaient pas même des
possesseurs légitimes\,; enfin de quelle façon leurs père et grand-père
les avaient eues. Le grand parti de rang qu'ils en avaient su tirer leur
paraissait donc mal assuré dans un temps ou dans un autre\,; et quoique
ce rang, même pour les maisons vraiment souveraines, fût inconnu en
France jusqu'aux Guise, à qui il fallut tant d'adresse, de puissance, et
de degrés pour l'établir, par conséquent très susceptible d'y tomber,
c'en était tout un autre danger pour des seigneurs particuliers
distingués depuis si peu, et à si peu de titre, ou plutôt de prétexte,
et qui bien loin de voir encore aujourd'hui l'aîné de leur maison un
véritable souverain depuis tant de siècles comme est le duc de Lorraine,
n'en pouvaient montrer la moindre apparence chez eux en aucun temps.

Dans cette angoisse une fortune inespérable les vint trouver. Un vieux
cartulaire de l'église de Brioude, enterré dans l'obscurité de plusieurs
siècles, fut présenté au cardinal de Bouillon. Ce titre avait les plus
grandes marques de vétusté, et contenait une preuve triomphante de la
descendance masculine de la maison de La Tour des anciens comtes
d'Auvergne, cadets des ducs de Guyenne. Le cardinal de Bouillon fut
moins surpris que ravi d'aise d'avoir entre ses mains une pièce de si
bonne mine, car c'était là le point, plus que ce qu'elle témoignait. De
longue main, pour sa réputation d'abord, après pour sa chimère, il
s'était attiré tout ce qu'il avait pu de savants en antiquités. De tous
temps les jésuites lui étaient dévoués, comme lui à eux sans mesure, et
parmi tous les démêlés que son abbaye de Cluni lui avait causés avec ses
religieux, il avait eu grand soin de ménager les savants des trois
congrégations françaises de l'ordre de Saint-Benoît.

Baluze qui avait formé la belle et immense bibliothèque de M. Colbert,
qui protégea toujours les lettres et les sciences, s'était fait un grand
nom en ce genre et beaucoup d'amis, pour avoir été souvent
l'introducteur des savants auprès de ce ministre, et le canal des
grâces. Il avait soutenu sa réputation depuis la mort de son maître par
plusieurs ouvrages qu'il avait donnés au public. Le cardinal de Bouillon
se l'était attaché par des pensions et par des bénéfices. Son fort était
de démêler l'antiquité historique et généalogique, et ses découvertes et
sa critique étaient estimées. Ce n'était pas qu'on le crût à toute
épreuve\,; sa complaisance pour cet autre maître le déshonora. Il fit
une généalogie de la maison d'Auvergne, c'est-à-dire de La Tour, dont le
nom peu à peu se supprimait pour faire place au postiche, et il la fit
descendre de mâle en mâle des anciens comtes d'Auvergne, cadets des ducs
de Guyenne.

La fausseté veut être bien concertée, mais il est dangereux qu'elle la
soit trop. Il faut attraper un milieu avec adresse pour tromper avec un
dehors de simplicité qui surprenne et qui impose. Ce fut l'écueil contre
lequel toute cette belle invention se brisa. Rien de plus semblable au
cartulaire que cette nouvelle généalogie par ses découvertes, ignorées
jusqu'alors, et quoique cette pièce la dût être entièrement pendant la
composition de l'ouvrage, puisqu'elle ne devait pas encore être trouvée,
l'un et l'autre se montra prêt en même temps. Néanmoins, il fut jugé
plus expédient de produire le cartulaire le premier, et d'en attendre le
succès avant de publier \emph{l'Histoire de la maison d'Auvergne}.

Pour le mieux assurer, le cardinal de Bouillon joua le modeste, et fit
difficulté d'ajouter foi à une pièce si décisive. Il en parla en
confiance à ce qu'il put de savants avec doute, en les priant de bien
examiner, et de ne le laisser pas prendre pour dupe, et toutefois
ajoutait avec un air de désir et de complaisance, que cette descendance
était de tout temps l'opinion et la tradition de sa maison, quoique (et
voilà une belle contradiction) jusqu'au maréchal de Bouillon, elle ne
fût pas tombée dans la pensée d'aucun d'eux, et que, si elle était née
pour la première fois dans celle de son père et de son oncle, comme il y
a lieu de le soupçonner par leur affectation d'avoir cette terre appelée
le comté d'Auvergne, et la jonction du mot d'Auvergne au nom de La Tour,
au moins n'avaient-ils osé s'en laisser entendre avec toute la
splendeur, la gloire, le crédit, l'autorité dont ils avaient joui.
D'autres sortes de savants subalternes et mercenaires, aussi consultés
pour avoir lieu de les faire admettre à l'examen de la pièce par les
premiers et avec eux, furent bien endoctrinés par Baluze à dire ce qu'il
fallait à propos, et lui-même à découvert paya du poids de sa réputation
et de toute l'adresse de son esprit dès longtemps préparée sur une
matière si importante et si jalouse.

Soit que les véritables examinateurs y fussent trompés, soit qu'ils se
fussent laissé séduire, soit, comme il y a plus d'apparence, qu'ils
vissent bien ce qui en était, mais qu'ils ne voulussent pas se faire un
cruel ennemi du cardinal et de toute sa maison pour chose qui, au sens
de ces gens obscurs qui ne connaissent que leurs livres, ne blessait
personne et n'importait à personne, ils prononcèrent en faveur du
cartulaire, et le P. Mabillon, ce bénédictin si connu dans toute
l'Europe par sa science et par sa candeur, laissa entraîner son opinion
par les autres.

Avec de tels suffrages que ce dernier couronnait, le cardinal de
Bouillon ne feignit plus de parler à l'oreille de ses amis de sa
précieuse découverte, et surtout de bien étaler tout ce qu'il avait fait
et toutes les précautions qu'il avait prises pour n'y être pas trompé.
Par ce récit, il comptait d'en constater entièrement la vérité, et de
ses amis la nouvelle en gagna d'autres, et bientôt la ville et la cour,
comme il se l'était bien proposé. Chacun lui fit des compliments d'une
si heureuse découverte, la plupart pour se divertir de la mine qu'il
leur ferait. Ce fut un chaos plutôt qu'un mélange de la vanité la plus
outrée et de la modestie la plus affectée, et d'une joie immodérée qui
éclatait malgré lui. Il fallait, pour la vraisemblance, garder quelque
interstice entre la publication de cette découverte et celle de
l'\emph{Histoire d'Auvergne}, pour en rompre la cadence autant qu'il se
pourrait aux yeux du public.

Le malheur voulut que de Bar, ce va-nu-pieds qui avait, disait-on,
déterré ce cartulaire, et qui l'avait présenté au cardinal de Bouillon,
fut arrêté dans cet intervalle, et mis en prison pour faussetés, par
ordre de la chambre de l'Arsenal. Cet événement fit quelque bruit qui
intrigua les Bouillon, mais qui rendit leur cartulaire fort suspect et
fit mettre force lunettes pour l'examiner. Des savants sans liaison avec
les Bouillon le contestèrent, et tant fut procédé que de Bar, arrêté
pour d'autres faussetés, fut poussé sur celle-ci. La Reynie, si
redoutable aux vrais criminels par ses lumières et sa capacité, et par
l'expérience des prisonniers de la Bastille et de Vincennes dans sa
charge de lieutenant de police, si longtemps mais si intègrement
exercée, et en magistrat des anciens temps, présidait en chef à la
chambre de l'Arsenal, et fit subir à de Bar divers interrogatoires sur
le cartulaire de Brioude. Il se défendit le mieux qu'il put, mais il
laissa échapper des choses délicates qui le firent resserrer et presser
de nouveau.

Alors l'alarme se mit dans la maison de Bouillon, près de voir éclater
la fourberie. Il n'est rien qu'ils ne fissent pour en parer le coup,
d'abord sourdement par la honte de paraître, mais voyant que le tribunal
ne relâchait rien de la rigueur de l'examen, la douleur et le bruit des
savants qu'ils avaient trompés, et le cri public, ils se mirent à
solliciter ouvertement pour de Bar, et à y employer tout leur crédit. À
la fin, l'inflexibilité de La Reynie et l'indignation qui échappait aux
autres magistrats de la chambre de l'Arsenal, les réduisit à un parti
extrême. M. de Bouillon, que le roi aimait, lui avoua qu'il ne voudrait
pas répondre que son frère, le cardinal, n'eût été capable, à leur insu
à tous, d'essayer à constater des faits incertains\,; et, prenant le roi
par ce qui le touchait le plus, qui était la confiance, il ajouta que,
se mettant ainsi entre ses mains sur une chose si délicate, il le
suppliait d'arrêter cette affaire par bonté pour ceux qui n'y avaient
point trempé, qui n'étaient coupables que d'une crédulité trop confiante
pour un frère, et de leur faire au moins la grâce de les sauver de la
flétrissure d'y être nommés en rien. Le roi avec plus d'amitié pour M.
de Bouillon que de réflexion à ce qu'il devait de réparation à l'injure
publique, voulut bien prendre ce parti.

Cependant l'abbé d'Auvergne, longtemps depuis cardinal au scandale
public le plus éclatant et le plus éclaté, sollicitant de toutes ses
forces, n'eut pas honte de dire aux juges, pour les toucher, à peu près
ce que M. de Bouillon dit au roi.

De Bar enfin, atteint et convaincu d'avoir fabriqué ce cartulaire de
l'église de Brioude, ne fut point poussé par delà l'aveu qu'il en fit en
plein tribunal, pour éviter, par ordre du roi à La Reynie, qu'il ne
parlât du cardinal, et peut-être de quelques autres Bouillon. Le
cartulaire fut déclaré faux et fabriqué par ce faussaire, et par la
raison susdite, de Bar, par le même arrêt, ne fut point condamné à mort,
mais à une prison perpétuelle, parce que les autres faussetés sur
lesquelles il fut d'abord arrêté n'étaient rien en comparaison de
celle-ci. On peut comprendre que cette aventure fit un grand éclat\,;
mais ce qui ne se comprend pas si aisément, c'est que MM. de Bouillon,
qui en devaient être si embarrassés, osèrent, quinze mois après,
demander à M. le chancelier l'impression de l'\emph{Histoire de la
maison d'Auvergne}, et que M. le chancelier l'accorda. Les réflexions
seraient trop fortes et m'écarteraient de mon sujet. Il en est seulement
de dire que le monde en fut étrangement scandalisé, et qu'un aussi gros
ouvrage et si recherché, dont le fondement unique était ce cartulaire,
qui parut aussi promptement après l'éclat, ne sembla à personne avoir
été fait et achevé qu'avec le cartulaire même, et par conséquent aussi
faux que lui. C'est le jugement qui en fut universellement porté, qui
déshonora Baluze jusqu'à faire rompre avec lui beaucoup de savants et
plusieurs de ses amis, et qui mit le comble à la confusion de cette
affaire. On verra en son temps ce que ce beau livre devint.

Après avoir réparé ces deux oublis, l'un sur la maison de Rohan, l'autre
sur celle de Bouillon, revenons d'où nous sommes partis.

\hypertarget{chapitre-xviii.}{%
\chapter{CHAPITRE XVIII.}\label{chapitre-xviii.}}

1707

~

{\textsc{Année 1707.}} {\textsc{- Retranchement d'étrennes et de partie
de la pension de M\textsuperscript{me} de Montespan.}} {\textsc{- Mort
de Cauvisson\,; sa dépouille.}} {\textsc{- Survivance de secrétaire
d'État au fils de Chamillart.}} {\textsc{- Visites inusitées chez
Chamillart.}} {\textsc{- Bassesse de du Bourg.}} {\textsc{- Mort du roi
de Portugal.}} {\textsc{- Mort et famille du prince Louis de Bade.}}
{\textsc{- Grandeurs de Marlborough.}} {\textsc{- Entrevues étranges.}}
{\textsc{- Électeur de Cologne sacré, etc.}} {\textsc{- Naissance du
second duc de Bretagne.}} {\textsc{- Mort de Saint-Hermine.}} {\textsc{-
Mort de M\textsuperscript{me} de Montgon.}} {\textsc{-
M\textsuperscript{me} de La Vallière dame du palais.}} {\textsc{-
Mariage de Gondrin avec une fille du maréchal de Noailles.}} {\textsc{-
Mort du comte de Grammont\,; son caractère.}} {\textsc{- Mort de La
Barre.}} {\textsc{- Mort de M\textsuperscript{me} de Frontenac\,; sa
famille, etc.}} {\textsc{- Mort de M\textsuperscript{lle} de Goello\,;
sa famille.}} {\textsc{- Mort du chevalier de Gacé.}} {\textsc{- Mines
inutilement cherchées aux Pyrénées.}} {\textsc{- Retour et personnage de
M\textsuperscript{me} de Caylus à la cour.}} {\textsc{- Union de
l'Écosse avec l'Angleterre.}} {\textsc{- Marquis de Brancas et de Bay.}}
{\textsc{- Port-Mahon repris pour Philippe V.}} {\textsc{- Envoi
d'argent de Mexique par le duc d'Albuquerque.}} {\textsc{- Prise
considérable en mer sur les Anglais.}}

~

La situation pressée des affaires qui avait fort augmenté les dépenses
de la guerre par tout ce qu'on avait perdu de troupes et de terrain,
avait obligé le roi, depuis deux ou trois ans, à diminuer, puis, à
retrancher les étrennes qu'il donnait aux fils et aux filles de France,
qui se montaient fort haut. Le trésor royal lui apportait tous les
premiers jours de l'an pour les siennes trente-cinq mille louis d'or de
quelque valeur qu'ils fussent. Cette année, 1707, il s'en retrancha dix
mille. La cascade en tomba sur M\textsuperscript{me} de Montespan.
Depuis qu'elle eut quitté la cour pour toujours, le roi lui donnait
douze mille louis d'or tous les ans, sur quelque pied qu'ils fussent\,;
d'O était chargé de lui en porter trois mille tous les trois mois. Cette
année, le roi lui manda par le même qu'il ne pouvait plus lui en donner
que huit mille. M\textsuperscript{me} de Montespan n'en témoigna pas la
moindre peine\,; elle répondit qu'elle n'en était fâchée que pour les
pauvres, à qui, en effet, elle donnait avec profusion.

D'Alègre en eut de meilleures\,; ce fut une des trois lieutenances
générales de Languedoc, vacante par la mort subite de Cauvisson\,; sans
enfants, sortant de dîner chez M. le Grand à Versailles. J'ai parlé de
lui lorsque M. du Maine lui fit donner cette charge.

Chamillart en eut encore de plus considérables. Ce fut la survivance de
sa charge de secrétaire d'État pour son fils unique de dix-huit ans. Le
prétexte fut d'épargner au père trois ou quatre heures de signatures par
jour, mais dans le fait, le roi était aussi libéral des survivances de
ces importantes charges qu'avare de toutes les autres. Il ne voulait
être servi par de fort jeunes gens que dans ses principales affaires, et
croyait montrer qu'il n'avait besoin que de soi-même pour les gouverner.
Cette même raison lui fit faire d'étranges choix en ce genre,
indépendamment des survivances dont les suites ont été cruelles pour
l'État et pour lui. Cette grâce fut un surcroît de disgrâces pour le
maréchal de Villeroy, qui, non seulement n'avait pas voulu voir
Chamillart à son retour, et avait rompu hautement avec lui, mais avait
défendu au duc de Villeroy de le voir, dont Chamillart avait été peiné,
et le roi l'avait trouvé très mauvais. Dans l'esprit de lui plaire,
Monseigneur et M. le duc de Berry allèrent l'après-dînée voir
M\textsuperscript{me} Chamillart et faire compliment à toute la
famille\,; et M\textsuperscript{me} la duchesse d'Orléans qui, fort mal
à propos, comme je l'ai remarqué ailleurs, ne faisait plus de visite,
quitta cette morgue pour cette fois, et alla aussi voir
M\textsuperscript{me} Chamillart.

Bientôt après, le fils de Chamillart alla visiter les places frontières
de Flandre et d'Allemagne. Le comte du Bourg, longtemps depuis maréchal
de France, n'eut pas honte de s'offrir et fut accepté pour lui servir de
mentor en ce voyage. On ne lui en pouvait choisir un meilleur\,; la
merveille fut que tous les honneurs pareils, ou plus grands que ceux
qu'aurait reçus un prince du sang, ne tournèrent point cette jeune
cervelle, qui conserva toute sa raison\,; et cet écolier, pour le bien
dire, revint doux, modeste, officieux et respectueux comme s'il n'eût
pas été fils du ministre favori et secrétaire d'État lui-même. Il se fit
aimer partout.

La mort du roi de Portugal fit un deuil de six semaines. Il n'avait eu
qu'une fille morte sans alliance devant lui, de sa première femme.
L'histoire de leur mariage et de la catastrophe du roi son frère est si
connue, que je n'en dirai rien ici. Il laissa plusieurs enfants de sa
seconde femme, sœur de l'impératrice, et fille et sœur de l'électeur
palatin, duc de Neubourg\footnote{La phrase de Saint-Simon peut paraître
  étrange dans sa forme elliptique, et c'est probablement ce qui a
  engagé les précédents éditeurs à la modifier. Cette phrase s'explique,
  cependant, facilement par la généalogie de la reine de Portugal\,:
  Marie-Sophie-Élisabeth, seconde femme de Pierre II, roi de Portugal,
  était fille de Philippe-Guillaume, électeur palatin, et sœur de Jean
  Guillaume qui, en 1690, succéda à son père dans la dignité d'électeur
  palatin.}.

Un moindre prince, mais de plus grande réputation, mourut en même
temps\,; le prince Louis de Bade, à cinquante-deux ans. Il était fils de
Ferdinand-Maximilien, marquis de Bade, qui ne fit jamais parler de lui
et de la fille de la princesse de Carignan, dernière princesse du sang
de la branche de Bourbon-Soissons. Maximilien-Ferdinand l'avait épousée
à Paris en 1653, et en eut deux ans après le prince Louis de Bade dont
le roi fut le parrain. La princesse de Bade fut dame du palais de la
reine plusieurs années, sans prétention ni distinction d'avec les
duchesses et les princesses établies en France, et n'en eut jamais,
faisant sa semaine et son service auprès de la reine comme les autres
dames du palais titrées, et roulant avec elles. Elle fut à la fin
chassée, avec la princesse de Carignan, sa mère, pour des intrigues trop
anciennes pour avoir place ici. Le prince de Bade, médiocrement content
de sa femme, se retira dans ses États an 1658, y emmena son fils, et y
mourut l'année suivante d'un coup de fusil qui lui cassa le bras comme
il s'appuyait dessus. Le prince Herman de Bade, son frère cadet, s'était
attaché à l'empereur. Il devint premier commissaire impérial à la diète
de Ratisbonne, gouverneur de Javarin, maréchal de camp général,
président du conseil de guerre, la meilleure tête et le plus autorisé du
conseil intime de l'empereur. Ce fut l'émule du fameux duc de Lorraine,
qu'il barra, abaissa et tint éloigné en Tyrol tant qu'il put. Il ne fut
point marié et mourut en 1691. Ce fut lui qui, prit soin de son neveu et
qui l'attacha à l'empereur. Il devint maréchal de camp général comme son
oncle, et gagna sur les Turcs, en Hongrie, les importantes batailles de
Jagodina, de Nissa, de Vidin et de Salankmen, où le grand vizir
Cuprogli\footnote{Il s'agit ici de Mustapha Cuprogli. Nous avons suivi,
  pour le nom de cette famille célèbre, l'orthographe de Saint-Simon. On
  écrit quelquefois Koprogli, Kiuperli et Kioprili.} et plus de vingt
mille Turcs demeurèrent sur la place. Il commanda presque toujours
depuis les armées impériales du Rhin, et passa justement pour un des
plus grands capitaines de son siècle.

Il avait épousé en 1690, une des deux filles du dernier des ducs de
Saxe-Lauenbourg, sœur de la veuve du dernier des grands-ducs de
Toscane-Médicis, qui, pour le dire en passant, était la première et la
plus ancienne maison d'Allemagne. Il en laissa deux fils et une fille.
L'aîné, accordé à notre reine, et le mariage près d'être célébré, la
princesse de Bade apprit la mort du fils unique du prince de
Schwartzenberg, qui, par un cas fort rare en Allemagne, laissait sa sœur
unique héritière de fort grands biens. Notre reine fut congédiée après
avoir demeuré quelque temps auprès de la princesse de Bade pour la
former à son gré comme sa future belle-fille\,; son mariage rompu et
celui de la fille de Schwartzenberg fait. Quelque temps après sa
célébration, la princesse de Bade, qui était dévote, alla voir le prince
de Schwartzenberg, et fit si bien auprès de lui, qu'elle lui fit
reprendre sa femme avec qui il était fort mal depuis longtemps, et qui
vivait hors de chez lui. De ce raccommodement vint un fils qui réduisit
la jeune princesse de Bade à l'état ordinaire, pour les biens, de toutes
les filles des bonnes maisons d'Allemagne, dont sa belle-mère eut grand
mal au cœur. Le cadet du jeune prince de Bade fut destiné à l'Église, et
leur sœur épousa M. le duc d'Orléans, et est morte en couches de M. le
duc de Chartres. Elle s'était extrêmement fait aimer, et fut fort
regrettée. Sa vie en ce pays-ci, malgré sa douceur, son esprit et sa
vertu, n'avait pas été heureuse.

En ce même commencement d'année le duc de Marlborough, à qui l'empereur
avait donné une belle et riche terre en Allemagne, et qu'il avait fait
prince de l'empire, fut déclaré vicaire général de l'archiduc aux
Pays-Bas. Cela surprit fort à cause de la différence de sa religion, et
de la part de la maison d'Autriche, qui se pique si fort d'être
catholique zélée, et qui couvre tant de desseins et d'exécutions de ce
manteau. Mais Marlborough refusa et ne voulut pas donner cette prise sur
lui en Angleterre pour un emploi si passager.

On eut lieu de l'être bien davantage de l'entrevue qu'eurent ensemble,
près de Leipzig, les rois de Suède et de Pologne, que le premier venait
de forcer à abdiquer, et de reconnaître le roi Stanislas Lesczinski à sa
place, et qui vivait en souverain à ses yeux en Saxe dont il tirait des
trésors. Ce ne fut pas tout\,; pour combler l'étonnement, il y eut
incontinent après une autre entrevue entre ces deux rois de Pologne.

L'électeur de Cologne qui n'avait aucuns ordres voulut enfin les
recevoir. L'archevêque de Cambrai le vint trouver à Lille, et en cinq
jours de suite lui donna les quatre moindres, le sous-diaconat, le
diaconat, le fit prêtre et le sacra évêque. Il se plut fort après aux
fonctions ecclésiastiques, surtout à dire la messe et à officier
pontificalement.

M\textsuperscript{me} la duchesse de Bourgogne accoucha d'un duc de
Bretagne fort heureusement et fort promptement le samedi, 8 janvier, un
peu avant huit heures du matin. La joie fut grande, mais le roi, qui en
avait déjà perdu un, défendit toutes les dépenses qui avaient été faites
à sa naissance, et qui avaient infiniment coûté. Il écrivit au duc de
Savoie pour lui donner part de cet événement, malgré la guerre et
l'excès des mécontentements, et il en reçut une réponse de conjouissance
et de remerciement.

Saint-Hermine, frère de la comtesse de Mailly, dame d'atours de
M\textsuperscript{me} la duchesse de Bourgogne, mourut à Versailles et
fut regretté. Il était bon officier, maréchal de camp et inspecteur.
Cela donna lieu à séparer la cavalerie des dragons pour les inspections,
comme le maréchal de Tessé et le duc de Guiche l'avaient toujours
souhaité, tandis qu'ils étaient colonels généraux des dragons. Coigny,
en cela, fut plus heureux qu'eux.

M\textsuperscript{me} de Montgon, dame du palais de
M\textsuperscript{me} la duchesse de Bourgogne, mourut en Auvergne, où
elle était allée faire un tour dans la famille et les biens de son mari.
Elle était fille de M\textsuperscript{me} d'Heudicourt, desquelles j'ai
assez parlé, lorsqu'on fit la maison de M\textsuperscript{me} la
duchesse de Bourgogne, pour n'avoir rien à y ajouter, sinon qu'elle
était flatteuse, insinuante, amusante, méchante et moqueuse, et qu'elle
divertissait fort le roi, M\textsuperscript{me} de Maintenon et
M\textsuperscript{me} la duchesse de Bourgogne, qui en furent fâchées.
Elle ne laissait pas d'avoir des amis qui la regrettèrent. Sa place fut
désirée de tout ce qui s'en crut à portée. Les Noailles enfin
l'emportèrent pour leur fille, M\textsuperscript{me} de La Vallière, qui
avait seule plus d'esprit, de tête et d'intrigue que tous les Noailles
ensemble\,; aimable quand elle voulait, mais pleine d'humeur, et
naturellement brutale beaucoup plus que son père, qui ne l'était pas
peu.

Ils firent, en ce même mois de janvier, un sixième mariage qui eut de
grandes suites pour les deux familles, de leur sixième fille avec
Gondrin, fils aîné de d'Antin, qui lui donna Bellegarde pour dix mille
livres de rente, et M\textsuperscript{me} de Montespan cent mille francs
en pierreries. Les Noailles donnèrent cent mille écus en diverses choses
et dix ans de nourriture. La conduite de la duchesse de Noailles les
embarrassait fort. Ils la tenaient extrêmement recluse. Sa tête tenait
fort de celle de son père\,: sa place était une occasion continuelle de
chagrins entre la laisser aller quelquefois et l'en empêcher beaucoup
plus souvent. M\textsuperscript{me} de Maintenon en était importunée.
Ils l'obligèrent donc de la céder à sa belle-sœur. Qui eût dit au roi
que cette nouvelle dame épouserait un jour M. le comte de Toulouse, et
qu'elle ferait, sous son successeur, le personnage que nous voyons\,?

Le comte de Grammont mourut à Paris, où il n'était presque jamais, à la
fin de ce mois de janvier, à plus de quatre-vingt-six ans, ayant
toujours eu, jusqu'à quatre-vingt-cinq uns, une santé parfaite et la
tête entière, et encore depuis. Il était frère du père du maréchal de
Grammont, duquel la mère était fille du maréchal de Roquelaure, et celle
du comte de Grammont était sœur de Bouteville, décapité à Paris pour
duels, père du maréchal-duc de Luxembourg. Il s'était attaché à M. le
Prince qu'il suivit en Flandre, s'alla promener après en Angleterre et y
épousa M\textsuperscript{lle} Hamilton dont il était amoureux avec
quelque éclat, et que ses frères, qui en furent scandalisés, forcèrent
d'en faire sa femme, malgré qu'il en eût. C'était un homme de beaucoup
d'esprit, mais de ces esprits de plaisanterie, de reparties, de finesse
et de justesse à trouver le mauvais, le ridicule, le faible de chacun,
de le peindre en deux coups de langue irréparables et ineffaçables,
d'une hardiesse à le faire en public, en présence et plutôt devant le
roi qu'ailleurs, sans que mérite, grandeur, faveur et places en pussent
garantir hommes ni femmes quelconques. À ce métier il amusait et il
instruisait le roi de mille choses cruelles, avec lequel il s'était
acquis la liberté de tout dire jusque de ses ministres. C'était un chien
enragé à qui rien n'échappait. Sa poltronnerie connue le mettait
au-dessous de toutes suites de ses morsures\,; avec cela escroc avec
impudence, et fripon au jeu à visage découvert, et joua gros toute sa
vie. D'ailleurs, prenant à toutes mains et toujours gueux, sans que les
bienfaits du roi, dont il tira toujours beaucoup d'argent, aient pu le
mettre tant soit peu à son aise. Il en avait eu pour rien le
gouvernement de la Rochelle et pays d'Aunis à la mort de M. de
Navailles, et l'avait vendu depuis fort cher à Gacé, depuis maréchal de
Matignon. Il avait les premières entrées et ne bougeait de la cour.
Nulle bassesse ne lui coûtait auprès des gens qu'il avait le plus
déchirés lorsqu'il avait besoin d'eux, prêt à recommencer dès qu'il en
aurait eu ce qu'il en voulait. Ni parole, ni honneur, en quoi que ce
fût, jusque-là qu'il faisait mille contes plaisants de lui-même et qu'il
tirait gloire de sa turpitude, si bien qu'il l'a laissée à la postérité
par des Mémoires de sa vie, qui sont entre les mains de tout le monde,
et que ses plus grands ennemis n'auraient osé publier. Tout enfin lui
était permis et il se permettait tout. Il a vieilli sur ce pied-là.

J'ai parlé quelquefois de lui, et encore plus de sa femme, et j'ai
raconté le compliment cruel dont il accabla le duc de Saint-Aignan,
lorsque le duc de Beauvilliers, son fils, fut chef du conseil royal des
finances. Il ne dit pas un mot moins assommant à l'archevêque de Reims
qu'il rencontra sortant du cabinet du roi, la tête fort basse, de son
audience sur l'affaire du moine d'Auvillé que j'ai expliquée (t. IV,
p.~127). «\,Monsieur l'archevêque, lui dit-il tout haut avec un air
d'insulte, \emph{verba volant}, mais \emph{scripta manant}. Je suis
votre serviteur.\,» L'archevêque brossa et ne répondit pas un mot.

Une autre fois, le roi parlant d'un employé du nord qui était venu faire
un compliment et quelque autre chose encore, dont il s'était fort mal
acquitté, et qui venait de s'en retourner, ajouta qu'il ne comprenait
pas comment on envoyait des gens comme était celui-là. «\,Vous verrez,
sire, dit le comte de Grammont, que c'est quelque parent de ministre.\,»
Il n'y avait guère de jour qu'il ne bombardât ainsi quelqu'un.

Étant fort mal à quatre-vingt-cinq ans, un an devant sa mort, sa femme
lui parlait de Dieu. L'oubli entier dans lequel il en avait été toute sa
vie le jeta dans une étrange surprise des mystères. À la fin, se
tournant vers elle\,: «\,Mais, comtesse, me dis-tu là bien vrai\,?»
Puis, lui entendant réciter le \emph{Pater\,:} «\,Comtesse, lui dit-il,
cette prière est belle, qui est-ce qui a fait cela\,?» Il n'avait pas la
moindre teinture d'aucune religion. De ses dits et de ses faits on en
ferait des volumes, mais qui seraient déplorables si on en retranchait
l'effronterie, les saillies et souvent la noirceur. Avec tous ces vices
sans mélange d'aucun vestige de vertu, il avait débellé la cour et la
tenait en respect et en crainte. Aussi se sentit-elle délivrée d'un
fléau que le roi favorisa et distingua toute sa vie. Il était chevalier
de l'ordre, de la promotion de 1688.

La Barre mourut en ce même temps, celui dont il a été tant parlé à
propos de l'affaire qu'il eut avec Surville et qui perdit ce dernier.

Mourut aussi M\textsuperscript{me} de Frontenac, dans un bel appartement
que le feu duc du Lude, qui était fort galant, lui avait donné à
l'Arsenal, étant grand maître de l'artillerie. Elle avait été belle et
ne l'avait pas ignoré. Elle et M\textsuperscript{lle} d'Outrelaise
qu'elle logeait avec elle, donnaient le ton à la meilleure compagnie de
la ville et de la cour, sans y aller jamais. On les appelait les
Divines. En effet, elles exigeaient l'encens comme déesses, et ce fut
toute leur vie à qui leur en prodiguerait. M\textsuperscript{lle}
d'Outrelaise était morte il y avait longtemps. C'était une demoiselle de
Poitou, de parents pauvres et peu connus, qui avait été assez aimable,
et qui perça par son esprit beaucoup plus doux que celui de son amie,
qui était impérieux. Celle-ci était fille d'un maître des comptes qui
s'appelait Lagrange-Trianon. Son mari, qui, comme elle, avait peu de
bien, et, comme elle, aussi beaucoup d'esprit et de bonne compagnie,
portait avec peine le poids de son autorité. Pour l'en dépêtrer et lui
donner de quoi vivre, ils lui procurèrent, en 1672, le gouvernement du
Canada, où il fit si bien longues années, qu'il y fut renvoyé en 1689\,;
et y mourut à Québec à la fin de 1698. Son grand-père était premier
maître d'hôtel et gouverneur de Saint-Germain. Il fut chevalier de
l'ordre en 1619. Il avait marié son fils à une fille de Raymond
Phélypeaux, secrétaire d'État après son père et son frère, ayant été
auparavant trésorier de l'épargne. Cela fit Frontenac père du gouverneur
de Canada, beau-frère de MM. d'Humières et d'Huxelles. Il fallait
pourtant que ce ne fût pas grand'chose, car on trouve avec les mêmes nom
et armes un Roger de Buode, huissier de l'ordre en 1641, seigneur de
Cussy, après Paul Aubin. Ce Roger, seigneur de Cussy, mourut en 1655, et
Jean Aubin, fils de son prédécesseur, rentra dans la charge.
M\textsuperscript{me} de Frontenac était extrêmement vieille, et voyait
encore chez elle force bonne compagnie. Elle n'avait point d'enfants et
peu de bien que, par amitié, elle laissa à Beringhen, premier écuyer.

M\textsuperscript{lle} de Goello mourut peu de jours après, à plus de
quatre-vingts ans, à l'hôtel de Soubise, où elle avait logé toute sa
vie. Elle était sœur de la mère de M. de Soubise, qui avait une grande
confiance en elle, et qui en eut trois cent mille livres. C'était une
créature de tête et d'esprit. Elle était des bâtards de Bretagne, sœur
du père du comte de Vertus d'aujourd'hui, derniers de ces bâtards. Sa
sœur aînée, mère de M. de Soubise, était cette belle duchesse de
Montbazon, qui figura tant dans les troubles de la minorité de Louis
XIV, belle-mère de la fameuse duchesse de Chevreuse et du mari de cette
belle et habile princesse de Guéméné, qui, à leur aide, accrocha le
tabouret, comme je l'ai raconté (t. II, p.~153, 154)\,; et toutes trois
commencèrent le rang dont jouit la maison de Rohan, que la beauté de
M\textsuperscript{me} de Soubise a si bien su achever.

La mère de M. de Soubise et M\textsuperscript{lle} de Goello, et
plusieurs autres frères et sœurs eurent pour mère la fille du fameux La
Varenne, marmiton, puis cuisinier, après portemanteau, ensuite le
Mercure d'Henri IV, enfin employé par ce prince en affaires secrètes en
Espagne et ailleurs, et parvenu à parier\footnote{Ce mot signifie ici
  \emph{aller de pair}.} avec ses ministres, à se faire compter par les
plus grands seigneurs, et à faire rappeler les jésuites et partager la
Flèche avec eux. Sa fille fut donc grand'mère de M. de Soubise, et c'est
ce quartier qui eût empêché son fils d'être admis dans le chapitre de
Strasbourg, conséquemment d'en devenir évêque, sans le change, qui fut
donné dans les preuves que j'ai expliquées (t. II, p.~66), de supprimer
le nom de Fouquet, qui était celui de cet heureux aventurier, pour ne
produire que celui de La Varenne qu'il portait, et de ce dernier nom en
donner le change avec une ancienne maison de Poitou de ce nom de La
Varenne, avec qui MM. de Rohan n'ont jamais eu d'alliance, et dès lors
éteinte depuis fort longtemps.

Gacé, depuis maréchal de Matignon, avait un second fils, qui fut tué à
Lille vers ce temps-ci, chez une femme où il allait souvent, dont le
mari s'enfuit aussitôt après. Le père obtint le régiment de cavalerie
qu'avait ce cadet pour son troisième fils, qui était dans la marine.
C'est aujourd'hui le marquis de Matignon, chevalier de l'ordre comme son
frère, de la façon de M. le Duc, dont la femme a été dame du palais\,;
et la fille, à qui elle a donné sa place, a épousé le duc de Fitz-James.
Cette fortune, qui n'a pas été loin d'être poussée plus haut, ne s'est
pas faite sans beaucoup de manèges et, d'intrigues dans sa propre
famille et dans le monde\,; mais ces temps dépassent ceux que je me suis
proposés.

La nécessité, qui fait chercher des ressources aux rois comme aux
particuliers, avait mis en besogne un chercheur de mines, nommé Rodes,
qui crut ou qui fit accroire avoir trouvé beaucoup de veines d'or dans
les Pyrénées. Il manda en ce temps-ci à Chamillart qu'elles étaient
tellement abondantes que, moyennant dix-huit cents travailleurs qu'il
lui demandait, il fournirait un million par semaine. Cinquante-deux
millions par an était une belle augmentation de revenu. La flatterie des
gens du pays confirma une si folle avance. On y prêta ses espérances,
qui ne durèrent pas longtemps. On en fut pour de la dépense\,; on s'y
opiniâtra. Elle demeura enfin en pure perte, et on n'en parla plus.

J'ai parlé plus haut de l'exil à Paris de M\textsuperscript{me} de
Caylus, et de la pension qu'elle eut pour quitter la direction du P. de
La Tour. Tant qu'elle dura, ce fut un ange qui ne se lassait point de
prières, d'austérités, de toutes sortes de bonnes œuvres, d'une solitude
qui lui faisait pleurer amèrement le temps qu'elle croyait perdu en des
délassements avec des personnes de la plus grande piété, qui aurait pu
passer pour un temps bien employé, et auquel elle se laissait aller si
rarement. Lorsqu'elle fut en d'autres mains, l'ennui succéda au goût de
la prière, de la solitude et des bonnes œuvres. Elle se laissa aller à
des rendez-vous en borine fortune avec M\textsuperscript{me} de
Maintenon à Versailles ou à Saint-Cyr, mais sans découcher de Paris,
qu'elle avait jusqu'alors constamment refusés, puis à aller passer
quelque temps à Saint-Germain avec le duc et la duchesse de Noailles. À
la fin, M\textsuperscript{me} de Maintenon, contente de son obéissance,
la fit revenir. Elle l'avait toujours aimée\,; elle fut ravie d'avoir
lieu de finir son éloignement.

Elle eut un logement\,; mais elle demeura enfermée chez
M\textsuperscript{me} de Maintenon ou chez M\textsuperscript{me}
d'Heudicourt. Peu à peu elle s'élargit chez les Noailles à des heures
solitaires, puis de même chez M. d'Harcourt, dont la femme et feu Caylus
étaient enfants des deux sœurs. Sa beauté, ses agréments, son enjouement
revinrent. Harcourt, trouvant en elle un instrument très propre à
l'aider auprès de M\textsuperscript{me} de Maintenon, la servit auprès
d'elle pour la faire nager en plus grande eau. Elle fut des Marlys et
des particuliers du roi. Ce fut une grande complaisance de la part du
roi pour M\textsuperscript{me} de Maintenon. Jamais il n'avait aimé
M\textsuperscript{me} de Caylus\,: il avait cru s'apercevoir qu'elle
s'était moquée de lui. Quelque divertissante qu'elle fût, il n'était
point à son aise avec elle\,; et elle, qui avait senti cet éloignement,
était aussi en brassière en sa présence. Néanmoins elle fut admise à
tout. La conduite de la duchesse de Noailles lui fut confiée, la
compassion de sa captivité la lui fit adoucir, et peu à peu la remettre
sur le pied des autres femmes de la cour. Bientôt la chambre de
M\textsuperscript{me} de Caylus devint un rendez-vous important. Les
gens considérables frappaient à cette porte et se trouvèrent heureux d'y
entrer quelquefois. La dévotion enfin écoulée devint la matière des
plaisanteries de M\textsuperscript{me} de Caylus. Elle revit
M\textsuperscript{me} la Duchesse et ses anciennes connaissances avec
qui elle déplora la tristesse avec laquelle sa jeunesse s'était passée,
dont elle faisait mille contes sur elle-même, en se moquant de toutes
ses pratiques de dévotion.

Toujours attachée au duc de Villeroy et lui à elle, ils se voyaient sans
que M\textsuperscript{me} de Maintenon le trouvât mauvais, tant elle
l'avait subjuguée, et à la fin elle se fit une cour les matins de
généraux, de ministres, et de la plupart des importants de la cour, par
ricochet vers M\textsuperscript{me} de Maintenon. Au fond, elle se
moquait d'eux tous, ne pouvait rien, et si elle pouvait quelquefois
insinuer à sa tante certaines choses, elle se réservait toute pour M.
d'Harcourt et pour tous ses desseins, auxquels elle demeura livrée sans
réserve, privativement à tout le reste, parce qu'après ce qui lui était
arrivé, elle n'osa rien hasarder en faveur des Villeroy que plusieurs
années après ce retour.

Ce fut en ce temps-ci que les Anglais parvinrent à consommer la grande
affaire qu'ils se proposaient depuis tant d'années, à laquelle le prince
d'Orange avait échoué. Ce fut ce qu'ils appelèrent l'\emph{union de
l'Écosse}, et ce que plus exactement les Écossais appelèrent
\emph{réduire l'Écosse en province}. Son indépendance de l'Angleterre
dura tant que durèrent ses parlements. À force de menées, d'argent et de
persévérance, le parlement d'Écosse consentit en ce commencement d'année
à être abrogé et à ne faire plus qu'un seul parlement pour les deux
royaumes avec celui d'Angleterre, moyennant certains privilèges
particuliers maintenus, et que l'Écosse serait représentée aux
parlements d'Angleterre par douze pairs d'Écosse, élus par les pairs de
ce royaume, qui s'assembleraient pour cette élection seulement, à
Édimbourg, sous la présidence d'un pair écossais nommé par le roi, alors
par la reine Anne. Ce nombre, si inférieur à celui des pairs anglais et
dans Londres, n'était pas en état de rien balancer de ce qui se
proposerait dans les parlements. On les leurra de l'influence qu'ils
auraient, comme les pairs anglais, sur ce qui regarderait l'Angleterre
même\,; et à la fin cela passa sous la condition que le parlement
désormais ne s'appellerait plus que le parlement de la Grande-Bretagne.
Ainsi plus d'embarras du côté de l'Écosse pour le commerce ni pour
aucune partie du gouvernement, dont les Anglais devinrent entièrement
les maîtres, sans qu'on puisse comprendre comment une nation si fière,
si ennemie de l'anglaise, si instruite par ce qu'elle en avait éprouvé
dans tous les temps, si jalouse de sa liberté et de son indépendance,
put baisser la tête sous ce joug.

Le marquis de Brancas, qui servait en Espagne, vint rendre compte au roi
de l'état des troupes et des affaires militaires de ce pays-là, et
recevoir ses ordres sur la campagne prochaine. Il était destiné à servir
en Castille, dans le corps séparé que le marquis de Bay y devait
commander, lequel M. de Bay, pour le dire en passant, était un
Franc-Comtois, fils d'un cabaretier\,: c'était un homme d'esprit et de
valeur, qui avait su profiter de la rareté des sujets militaires en
Espagne, pour s'y pousser promptement par son application et par de
petits succès, et il parvint jusqu'au grade de capitaine général, qui
est le plus élevé de tous en Espagne dans les armées, et, ce qui est
énorme, à l'ordre de la Toison d'or. D'ailleurs il devint capable, bon
général, et servit fort utilement.

Tout à la fin de janvier, le frère du maréchal de Villars entra au port
Mahon avec trois vaisseaux de guerre et neuf cents soldats, mit pied à
terre sous un gros feu de canon qu'il essuya, prit cinq cents hommes qui
étaient dans la place, et avec ces quatorze cents hommes en alla
attaquer cinq mille, presque toutes milices du pays, força plusieurs
retranchements qu'ils avaient devant eux, et leur tua cinq cents hommes.
Le reste s'enfuit dans leurs villages, d'où presque tous envoyèrent
leurs armes. Il y avait plusieurs moines parmi eux qui se distinguèrent
par leur opiniâtreté. Ceux qu'on prit, on les fit tous passer par les
armes, personne n'ayant voulu servir de bourreau pour les pendre. Ainsi
toute l'île de Minorque rentra sous la domination du roi d'Espagne. Cent
cinquante Castillans de la place firent merveilles contre les rebelles.
Trois mois après, on y découvrit une conspiration du major de la place
qui la voulait livrer aux partisans de l'archiduc. Le gouverneur
espagnol, qui s'y conduisit fort bien, aidé de deux bataillons français
qui étaient dans l'île, marcha aux rebelles, les dissipa, fit pendre le
major et plusieurs de ses complices, et prit plusieurs moines qui
étaient du complot, dont il fit passer quelques-uns en France.

Peu de jours après la réduction de l'île de Minorque, il arriva à Brest
un vaisseau du Mexique dépêché par le duc d'Albuquerque, vice-roi de ce
pays, chargé de beaucoup d'argent pour le roi d'Espagne et pour les
Espagnols. Il fit partir ce secours ayant appris la nouvelle que le roi
d'Espagne était errant hors de Madrid. Pontchartrain, qui en eut l'avis,
dit un million d'écus pour le roi d'Espagne et trois millions d'écus
pour les particuliers. En même temps, le comte de Toulouse eut avis de
deux vaisseaux espagnols, au lieu d'un, chargé de trente et un millions
en argent, dont un peu plus de trois pour le roi d'Espagne, et quelque
argent et force marchandises précieuses sur deux petits vaisseaux
français qui les convoyaient. On ne démêla point entre ces deux avis
lequel était le vrai\,; j'avoue aussi que je ne suivis pas fort
curieusement cette nouvelle. Six semaines après, Duquesne-Mosnier, sorti
de Brest avec son escadre, rencontra quinze bâtiments anglais escortés
de deux vaisseaux de guerre qui s'enfuirent dès qu'ils l'aperçurent.
Duquesne coula un de ces bâtiments bas, et envoya les quatorze autres à
Brest. Ils étaient chargés de poudre, de fusils, de selles, de brides,
en un mot, de tous les besoins des troupes anglaises qui étaient en
Espagne, qui manquaient de tout et ne pouvaient rien tirer de ces choses
du Portugal, ni des pays qu'ils avaient conquis ou qui s'étaient donnés
à l'archiduc en Espagne.

\hypertarget{chapitre-xix.}{%
\chapter{CHAPITRE XIX.}\label{chapitre-xix.}}

1707

~

{\textsc{Duc de Noailles capitaine des gardes, sur la démission de son
père.}} {\textsc{- Puysieux conseiller d'État d'épée.}} {\textsc{-
Curiosités sur Poissy et ses deux dernières abbesses.}} {\textsc{- Mort
de Roquette, évêque d'Autun\,; son caractère.}} {\textsc{- Bals à la
cour\,; comédies à Sceaux et à Clagny.}} {\textsc{- Généraux d'armée\,:
Tessé en Italie battu par le parlement de Grenoble\,; Villars sur le
Rhin\,; Vendôme en Flandre\,; Berwick resté en Espagne sous M. le duc
d'Orléans\,; duc de Noailles en Roussillon.}} {\textsc{- Mot étrangement
plaisant du roi sur Fontpertuis.}} {\textsc{- Exclusion du duc de
Villeroy de servir\,; curieuse anecdote.}} {\textsc{- Rage du maréchal
de Villeroy\,; ses artifices.}} {\textsc{- Mon éloignement pour le
maréchal de Villeroy.}} {\textsc{- Faiblesse du roi pour le maréchal de
Villeroy et pour ses ministres.}} {\textsc{- Cause intime de l'extrême
haine du maréchal de Villeroy pour Chamillart.}} {\textsc{- Peu de sens
du maréchal de Villeroy.}}

~

Le maréchal de Noailles était malade dès le commencement de février\,;
son énorme grosseur et les accidents de sa maladie firent peur à sa
famille. Le roi était inexorable sur les survivances, excepté pour les
secrétaires d'État. Toute la faveur des Noailles, celle même de
M\textsuperscript{me} de Maintenon, n'avaient osé rien tenter là-dessus
en faveur du duc de Noailles. La charge de capitaine des gardes du corps
avait à cet égard l'inconvénient de plus que le roi n'y voulait que des
maréchaux de France. La compagnie de Noailles était l'écossaise, la
première, la distinguée, et le duc de Noailles n'avait que vingt-sept
ans. Ils se mirent donc tous après le maréchal de Noailles, pour
l'engager à donner sa démission et tâcher, en levant l'obstacle de la
survivance, de faire passer la charge à son fils. Ce ne fut pas chose
facile à persuader\,; mais à force d'y travailler, ils arrachèrent sa
démission et une lettre au roi en conséquence plutôt qu'ils ne
l'obtinrent. Tout était de concert avec M\textsuperscript{me} de
Maintenon. Le roi reçut l'une et l'autre le 17 février, revenant de se
promener à Marly, et passa à son ordinaire chez M\textsuperscript{me} de
Maintenon. Un peu après qu'il y fut entré, il envoya quérir le duc de
Noailles, et lui dit d'aller apprendre à son père que, suivant son
désir, il lui donnait sa charge. Dès le lendemain matin, il prêta son
serment, prit le bâton et acheva le quartier qui était le sien. Ce même
jour, qui était un vendredi (et ces jours-là point de conseil),
Puysieux, revenu de Suisse faire un tour, eut une audience du roi, à la
fin de laquelle il lui demanda une place de conseiller d'État d'épée qui
n'était pas remplie depuis fort longtemps. Le roi la lui donna
sur-le-champ et lui dit qu'il la lui destinait depuis deux ans. On a vu
plus haut (t. IV, p.~236, 375-377) quel était Puysieux et comment il
s'était mis sur le pied de ces retours de Suisse et de ces audiences,
que nul autre ambassadeur n'obtenait, et combien il en sut profiter.

M\textsuperscript{me} de Mailly, sœur de l'archevêque d'Arles, depuis
cardinal de Mailly, eut en ce même temps le beau et riche prieuré ou
abbaye de Poissy, au bout de la forêt de Saint-Germain, dont elle était
professe. Cette nomination avait été longtemps contestée\,; les
religieuses se prétendaient avoir droit d'élection, et pour en dire le
vrai, elles en avaient conservé la possession depuis le concordat. Le
voisinage de la cour qui demeurait à Saint-Germain la tenta de disposer
d'une si belle place.

En dernier lieu, le roi y avait nommé une sœur du duc de Chaulnes
l'ambassadeur. Le pape ne s'y était pas opposé, mais les religieuses
fermèrent les portes à la reine qui l'y avait conduite elle-même,
tellement que les gardes les enfoncèrent. Ce fut un vacarme horrible que
cette installation des cris, des protestations, des insultes à
l'abbesse, beaucoup de grands manques de respect à la reine, force
religieuses chassées et mises en d'autres couvents. Malgré tout cela,
M\textsuperscript{me} de Chaulnes fut bien des années sans être
paisible. C'était aussi une grosse créature qui faisait peur, et qui
ressemblait de taille et de visage à son frère comme deux gouttes d'eau,
plus abbesse, plus glorieuse, plus impertinente que toutes les abbesses
ensemble, et qui, à force d'avoir été tourmentée en arrivant, s'était
mise à faire enrager ses religieuses. Pour s'en faire plus respecter,
elle s'était avisée de se faire annoncer par quelque tourière
affectionnée tantôt M. Colbert, tant M. de Louvois ou M. Le Tellier dans
un temps où elle était avec toute la communauté où la portière la venait
avertir. Elle faisait la surprise, après l'importunée, car les visites
étaient fréquentes\,; elle allait s'enfermer dans son parloir d'où pas
une religieuse n'osait approcher pendant ces importants entretiens qui
duraient le temps qu'elle jugeait à propos, puis, toute fatiguée de
consultations et d'affaires de la cour et du monde qu'elle n'avait pas
quitté, disait-elle, pour y perdre son temps dans l'état qu'elle avait
embrassé, elle revenait se reposer avec ses religieuses de tant de soins
dont elle aurait voulu n'ouïr jamais parler, et n'être point distraite
des devoirs d'abbesse. À la fin, ces ministres revenaient si souvent et
occupaient si longtemps M\textsuperscript{me} l'abbesse que quelque
religieuse, plus avisée que les autres, commença à se douter du jeu. À
la première visite de ces messieurs, trois ou quatre montèrent en lieu
de voir dans les cours et les dehors où elles n'aperçurent point de
carrosse. Après cette épreuve le doute se fortifia, et se communiqua de
plus en plus par le redoublement de la même épreuve, et il demeura
constant parmi toutes que jamais aucun de ces ministres n'avait mis le
pied à Poissy. À la fin, l'abbesse qui se vit découverte, également
honteuse et furieuse, n'osa plus continuer la tromperie\,; mais elle en
fit payer chèrement la découverte. Son règne fut également dur et long.
Sur la fin, elle prit en aversion, et bientôt en persécution celles
qu'elle crut lui pouvoir succéder, M\textsuperscript{me} de Mailly, sur
toutes, qui par son mérite et sa parenté semblait y avoir plus de part,
et la réduisit à chercher ailleurs un repos qu'elle ne pouvait plus
goûter à Poissy. Elle se retira à Longchamp, et elle y était lorsqu'elle
fut nommée.

Pour y parvenir après M\textsuperscript{me} de Chaulnes sans rumeur et
sans dispute, le roi profita d'un accident qui était arrivé à ce beau
monastère quelque temps avant la mort de M\textsuperscript{me} de
Chaulnes. Le tonnerre avait enfoncé la voûte du chœur et mis le feu à
l'église. La fonte du plomb qui la couvrait empêcha tout secours, en
sorte que ce dommage fut extrêmement grand, et à l'église qui est
magnifique et aux lieux du monastère qui en étaient voisins. Dans
l'impossibilité où la maison se trouva de le réparer même en partie, le
roi s'en chargea à condition qu'elle lui céderait pour toujours ses
prétentions d'élire, que le pape en ferait une abbaye, et qu'il en
donnerait la collation au roi. Cela fut fait ainsi au grand regret des
religieuses, qui n'osèrent pas résister, et le pape accorda tout.
Cependant on ne se pressait pas de la part du roi de réparer les
désordres du feu. On ne s'y mit que lorsque la santé de
M\textsuperscript{me} de Chaulnes fit craindre des difficultés sur cette
non-exécution\,; alors on l'entreprit, et elle a coûté près d'un
million. Néanmoins M\textsuperscript{me} de Mailly trouva beaucoup
d'opposition. Toutes l'aimaient et l'estimaient, protestaient qu'elles
l'auraient préférée dans l'élection, mais ne pouvaient souffrir la
nomination. La vertu, la patience, la douceur, l'esprit, l'art du
gouvernement, parurent avec éclat et succès dans la nouvelle abbesse.
Elle laissa sortir les plus opiniâtres, et gagna les autres par ses
talents, son grand exemple et sa bonté\,; mais pour n'y pas revenir, dès
que le roi fut mort, les protestations, jusque-là cachées, parurent, et
il se forma un véritable procès entre M\textsuperscript{me} de Mailly et
les prétendantes au droit d'élire, opprimées, disaient-elles, par
l'autorité du feu roi. La plupart de celles qui étaient à Poissy, et qui
avaient le plus goûté le gouvernement de leur abbesse, s'y joignirent.
Elle demeura la même à leur égard. Nous jugeâmes ce procès au conseil de
régence\,; M\textsuperscript{me} de Mailly le gagna. Il n'était pas
possible qu'elle le pût perdre avec toutes les précautions qui avaient
été prises ici et à Rome pour assurer cette nomination pour toujours. À
la fin, les religieuses, vaincues par la douceur, le mérite et la
conduite de M\textsuperscript{me} de Mailly envers toutes, l'ont aimée
comme la meilleure mère, et vivent là plus heureuses, à ce qu'il en
revient même de toutes parts par elles-mêmes, qu'aucune religieuses du
royaume.

Il mourut alors un vieux évêque, qui toute sa vie n'avait rien oublié
pour faire fortune, et être un personnage. C'était Roquette, homme de
fort peu, qui avait attrapé l'évêché d'Autun, et qui à la fin, ne
pouvant mieux, gouvernait les états de Bourgogne à force de souplesses
et de manège autour de M. le Prince. Il avait été de toutes les couleurs
à M\textsuperscript{me} de Longueville, à M. le prince de Conti son
frère, au cardinal Mazarin, surtout abandonné aux jésuites. Tout sucre
et tout miel, lié aux femmes importantes de ces temps-là, et entrant
dans toutes les intrigues, toutefois grand béat. C'est sur lui que
Molière prit son Tartufe, et personne ne s'y méprit. L'archevêque de
Reims, passant à Autun avec la cour, et admirant son magnifique
buffet\,: «\,Vous voyez là, lui dit l'évêque, le bien des pauvres. ---
Il me semble, lui répondit brutalement l'archevêque, que vous auriez pu
leur en épargner la façon.\,» Il remboursait accortement ces sortes de
bourrades\,; il n'en sourcillait pas, il n'en était que plus obséquieux
envers ceux qui les lui avaient données, mais allait toujours à ses fins
sans se détourner d'un pas. Malgré tout ce qu'il put faire, il demeura à
Autun, et ne put faire une plus grande fortune. Sur la fin, il se mit à
courtiser le roi et la reine d'Angleterre. Tout lui était bon à espérer,
à se fourrer, à se tortiller. M. de Bayeux, Nesmond, les courtisait
d'une autre façon. Il ne les voyait guère, leur donnait dix mille écus
tous les ans, et fit si bien, qu'on ne l'a jamais su qu'après sa mort.

M. d'Autun, pour achever par ce dernier trait, avait une fistule
lacrymale. Peu après la mort du roi d'Angleterre, il s'en prétendit
miraculeusement guéri par son intercession. Il l'alla dire à la reine
d'Angleterre, à M\textsuperscript{me} de Maintenon, au roi. En effet,
son œil paraissait différent\,; mais peu de jours après il reprit sa
forme ordinaire, la fistule ne se put cacher. Il en fut si honteux qu'il
s'enfuit dans son diocèse, et qu'il n'a presque point paru depuis. Les
restes de son crédit et de ses manèges trompèrent vilainement l'abbé
Roquette, son neveu, qui s'était fourré dans le grand monde, qui
prêchait et qui avait passé sa vie avec lui. Il obtint sa coadjutorerie
pour un autre neveu, et l'abbé Roquette, avec ses sermons, ses
intrigues, ses cheveux blancs et tant d'espérances, n'a pu parvenir à
l'épiscopat. Il a fini chez M\textsuperscript{me} la princesse de Conti,
fille de M. le Prince, dont il se fit aumônier, et son frère son écuyer.

Il y eut tout l'hiver force bals à Marly\,; le roi n'en donna point à
Versailles, mais M\textsuperscript{me} la duchesse de Bourgogne alla à
plusieurs chez M\textsuperscript{me} la Duchesse, chez la maréchale de
Noailles et chez d'autres personnes, la plupart en masques. Elle y fut
aussi chez M\textsuperscript{me} du Maine, qui se mit de plus en plus à
jouer des comédies avec ses domestiques et quelques anciens comédiens.
Toute la cour y allait\,; on ne comprenait pas la folie de la fatigue de
s'habiller en comédienne, d'apprendre et de déclamer les plus grands
rôles, et de se donner en spectacle public sur un théâtre. M. du Maine,
qui n'osait la contredire de peur que la tête ne lui tournât tout à
fait, comme il s'en expliqua une fois nettement à M\textsuperscript{me}
la Princesse en présence de M\textsuperscript{me} de Saint-Simon, était
au coin d'une porte, qui en faisait les honneurs. Outre le ridicule, ces
plaisirs n'étaient pas à bon marché.

Cependant le roi régla les généraux et les officiers généraux de ses
armées. Le maréchal de Tessé fut déclaré dès le commencement de février
pour le commandement de l'armée destinée à repasser en Italie. Il partit
bientôt après pour le Dauphiné avec une patente de commandant en chef
dans cette province. Il y prétendit du parlement les mêmes honneurs dont
y jouit le gouverneur de la province, qui sont entre autres d'être
visité par une nombreuse députation du parlement, traité de
\emph{monseigneur} dans le compliment, et de seoir au-dessus du premier
président dans le coin du roi. Cela lui fut disputé\,; le parlement de
Grenoble députa à la cour, où ses raisons furent si, bien expliquées,
qu'il gagna l'un et l'autre point et d'autres moindres, dont le maréchal
de Tessé eut le dégoût entier. Le maréchal de Villars fut destiné pour
l'armée du Rhin et M. de Vendôme à celle de. Flandre sous l'électeur de
Bavière. Le maréchal de Berwick était demeuré en Espagne\,; M. le duc
d'Orléans, qui ne voulait pas demeurer sur sa mauvaise bouche d'Italie,
et qui voyait peu d'apparence d'y faire rentrer une armée, désira
d'aller en Espagne. Il n'aurait pu obéir à l'électeur de Bavière qu'on
ne voulait pas mécontenter en lui proposant ce supérieur. Villars avait,
comme on l'a vu, fait ses preuves de ne pas vouloir servir sous ce
prince\,; il était trop bien soutenu pour lui être sacrifié. Il ne resta
donc que l'Espagne aux dépens du duc de Berwick, sur lequel l'expérience
funeste de ce qui était arrivé avec le maréchal de Marsin fit donner au
prince l'autorité absolue. Ce fut une grande joie pour lui que de
continuer à commander une armée, et de la commander, non plus en figure,
mais en effet. Il fit donc ses préparatifs. Le roi lui demanda qui il
menait en Espagne. M. le duc d'Orléans lui nomma parmi eux Fontpertuis.
«\, Comment, mon neveu reprit le roi avec émotion, le fils de cette
folle qui a couru M. Arnauld partout, un janséniste\,! je ne veux point
de cela avec vous. --- Ma foi, sire, lui répondit M. d'Orléans, je ne
sais point ce qu'a fait la mère\,; mais pour le fils être jansénistes il
ne croit pas en Dieu. --- Est-il possible, reprit le roi, et m'en
assurez-vous\,? Si cela est, il n'y a point de mal\,; vous pouvez le
mener.\,» L'après-dînée même, M. le duc d'Orléans me le conta en pâmant
de rire\,; et voilà jusqu'où le roi avait été conduit de ne trouver
point de comparaison entre n'avoir point de religion et le préférer à
être janséniste ou ce qu'on lui donnait pour tel.

M. le duc d'Orléans le trouva si plaisant qu'il ne s'en put taire, on en
rit fort à la cour et à la ville, et les plus libertins admirèrent
jusqu'à quel aveuglement les jésuites et Saint-Sulpice pouvaient
pousser. Leur art fut que le roi n'en sut nul mauvais gré à M. le duc
d'Orléans\,; qu'il ne lui en a jamais ni parlé, ni rien témoigné, et que
Fontpertuis le suivit en toutes ses deux campagnes en Espagne. Il était
débauché et grand joueur de paume, avec de l'esprit, fort ami de Nocé,
de M. de Vergagne et d'autres gens avec qui M. le duc d'Orléans vivait
quand il était à Paris. Tout cela l'avait fait goûter à ce prince. Le
duc de Noailles {[}commandait{]} en chef en Roussillon avec trois
maréchaux de camp sous lui.

Parmi les officiers généraux nommés pour les armées, le duc de Villeroy
fut oublié, qui fut un rude coup de poignard pour lui et pour son père.
C'est un fait qui mérite d'être un peu expliqué pour réparer ce que j'ai
trop croqué en parlant du retour et de la disgrâce du père\,; et j'ai
estropié la curiosité en faveur de la brièveté. Il faut donc retourner
un moment sur mes pas.

Le maréchal de Villeroy, qui toujours frivole voulait faire le jeune et
le galant, avait, à Paris, une petite maison écartée, mode assez
nouvelle des jeunes gens. Ce fut là qu'il arriva tout droit de Flandre,
avec défenses expresses à la maréchale de Villeroy de l'y venir voir et
à tous ses amis de l'y venir chercher, et par ce bizarre procédé fit
craindre quelque dessein plus bizarre à sa famille. Harlay, premier
président, dont je n'ai eu que trop occasion de parler, était son parent
et s'en honorait fort avec tout son orgueil, et de tout temps son ami
intime. Il hasarda de forcer la barricade, il perça, après quoi il n'y
eut pas moyen de refuser la maréchale de Villeroy. Il leur avoua qu'il
avait dans sa poche les démissions de sa charge et de son gouvernement,
toutes signées, prêt à les envoyer au roi dans la résolution de ne le
voir jamais. Ce sont de ces extrémités où le dépit emporte et contre
lesquelles la volonté réclame intérieurement. Sans cette pause ridicule
dans un lieu de Paris écarté qui n'était bon qu'à s'y faire chercher, il
était tout court d'envoyer ses démissions, tout droit de sa dernière
couchée, de traverser Paris sans s'y arrêter, et d'aller à Villeroy.
C'était là être chez soi à la campagne, à portée d'y recevoir qui il eût
voulu, et point d'autres, éloigné de dix lieues de Paris et de quatorze
de la cour, dans la bienséance d'un homme outré qui s'éloigne, et dans
la décence de ne se tenir pas tout auprès des lieux d'où il attendrait
des nouvelles dans l'espérance que ses démissions lui seraient
renvoyées.

Mais c'était un homme à éclats, et à rien de sage, de suivi, ni de
solide. Il se fit donc beaucoup tirailler, puis jeta ses démissions au
feu, et s'en alla à Versailles, où il fut reçu comme je l'ai raconté.

Sa conduite sur Chamillart, que j'ai aussi rapportée, aigrit le roi de
plus en plus. Le maréchal, de plus en plus enragé de voir sa disgrâce
s'approfondir, se mit à montrer au plus de gens qu'il put des morceaux
de lettres du roi et de Chamillart, pour appuyer ce qu'il avait déjà
répandu, savoir qu'il n'avait rien fait que sur des ordres exprès, et
qu'il était cruellement dur de porter l'infortune d'une bataille à
laquelle il avait été excité, même d'une façon piquante, et qu'on lui
eût encore moins pardonné de n'avoir pas donnée. Ces propos spécieux,
soutenus de ces fragments de lettres qu'il ne montrait qu'avec un
apparent mystère pour leur donner plus de poids, commencèrent enfin à
persuader que Chamillart, abattu des mauvais succès, s'en prenait à qui
n'en pouvait répondre, et qu'embarrassé d'avoir conseillé la bataille,
il écrasait celui qui l'avait perdue, sous prétexte de l'avoir hasardée
de son chef, et abusait ainsi de sa toute puissance de ministre favori,
pour perdre un général qui avait en main de quoi le confondre pour peu
qu'il pût être écouté.

Quelque ami que je fusse de la maréchale de Villeroy, jamais je n'avais
pu m'accommoder des airs audacieux de son mari, dont jusqu'aux caresses
étaient insultantes. Il m'était quelquefois arrivé les matins, au sortir
de la galerie, de dire que j'allais chercher de l'air pour respirer,
parce que le maréchal, qui y faisait la roue, en avait fait aussi une
machine pneumatique. J'étais d'ailleurs ami intime de Chamillart, et je
devais l'être pour les services qu'il m'avait rendus, et la confiance
avec laquelle il vivait avec moi. Alarmé donc du progrès des discours du
maréchal de Villeroy, j'en parlai à l'Étang à Chamillart, qui ému contre
son ordinaire me dit qu'il était bien étrange que le maréchal, non
content d'avoir tant démérité de l'État, du roi et de soi-même,
puisqu'il s'était perdu sans raison, voulût encore entreprendre des
justifications qu'il ne pouvait douter qui ne lui tournassent à crime,
pour peu qu'elles fussent approfondies et qu'il osât le pousser assez
pour l'obliger d'en demander justice au roi, qui savait tout\,: qu'il
voulait cependant être plus sage que le maréchal, mais qu'il me voulait
faire voir, à moi, les pièces justificatives des faits dont il me
demandait le secret, et me les montrerait dès que nous serions à
Versailles. En effet, à peine y fûmes-nous de retour, que j'allai chez
lui un soir qu'il soupait seul dans sa chambre, avec du monde familier
autour de lui, comme il avait accoutumé. Dès qu'il me vit, il me pria de
m'approcher de lui, et me dit qu'il allait me tenir parole. Là-dessus il
me donna la clef de son bureau, me dit où je trouverais les dépêches
dont il m'avait parlé, et me pria de passer dans son cabinet et de les
lire avec attention.

J'en trouvai trois. Deux minutes du roi au maréchal, et une du maréchal
au roi\,; celle-là en original et signée de lui. La première du roi
portait\,: «\,Que la prudence et la circonspection trop grandes, dont
les généraux de ses armées avaient usé depuis quelque temps en Flandre,
avaient enflé le courage à ses ennemis, et leur avaient laissé croire
qu'on craignait de se commettre avec eux\,: qu'il était temps de les
faire apercevoir du contraire et de leur montrer de la vigueur et de la
résolution. Que, pour cela, il avait mandé au maréchal de Marsin de se
mettre en marche de l'Alsace avec le détachement de l'armée du maréchal
de Villars (qui était là détaillé) et de le joindre. Qu'il lui ordonnait
de l'attendre, et, après leur jonction, d'aller ensemble faire le siège
de Lewe, de telle sorte qu'il fût formé des troupes de Marsin, et, si
elles ne suffisaient pas, d'un détachement des siennes, le tout commandé
par le maréchal de Marsin, tandis qu'avec les siennes il (le maréchal de
Villeroy) observerait les ennemis\,; que, pour peu qu'ils fissent mine
de s'approcher trop du siège, il ne les marchandât pas, et que, s'il ne
se trouvait pas assez fort pour les combattre, il ne laissât au siège
que le nécessaire, et qu'avec le reste il donnât bataille.\,» Voilà
exactement le contenu de cette première lettre, que le maréchal montrait
par morceaux, s'avantageant du commencement qu'il ajustait à sa mode sur
ce qu'il s'y prétendait piqué d'honneur, incité vivement aux partis
vigoureux, mais il se gardait bien d'en montrer le reste qui faisait
voir si clairement que cette vigueur ne lui était ni prescrite ni
conseillée qu'au cas que les ennemis entreprissent de troubler le siège
de Lewe, bien moins de leur prêter le collet sans cette raison, et
encore sans avoir reçu le renfort du maréchal de Marsin.

La seconde lettre du roi ne consistait qu'en raisonnements de troupes,
revenant en deux mots au projet susdit qu'elle confirmait tel qu'il
vient d'être exposé.

La lettre du maréchal de Villeroy était datée de la veille de la
bataille. Elle contenait le détail de sa marche et de celle des ennemis,
ne parlait d'aucun dessein de les combattre, et finissait en marquant
seulement que, \emph{s'ils s'approchaient si fort de lui, il aurait
peine à se contenir}. Ce mot ne manquait rien moins qu'un dessein formé
de combattre\,; il montrait seulement une excuse prématurée de ce qui
pouvait arriver, bien éloigné de l'exécution d'un ordre qu'il prétendait
l'avoir dû piquer d'honneur. Ainsi, bien loin d'avoir reçu celui de
donner bataille dans le temps et dans la circonstance qu'il livra celle
de Ramillies, quelque victoire qu'il y eût remportée ne l'eût pas dû
garantir du blâme d'avoir hasardé le projet du siège par un événement
douteux, et de n'avoir attendu ni l'occasion seule où la bataille lui
était prescrite ni le renfort qui le devait joindre, sans lequel il ne
lui était pas permis de rien entreprendre. Il le sentit si bien
lui-même, que, dans le dessein qu'il avait conçu de combattre, sans
l'occasion du siège qui lui était ordonné, surtout sans le renfort que
lui amenait Marsin pour vaincre par ses seules forces, même à l'insu de
l'électeur de Bavière, auquel il était subordonné en toute manière,
comme au gouverneur général des Pays-Bas, au milieu desquels il était,
et comme généralissime et en faisant effectivement la fonction, il
faisait d'avance des excuses obscures, obscures, dis-je, pour ne pas
découvrir son dessein arrêté, excuses pour qu'elles se trouvassent
faites avant l'événement, mais desquelles il n'aurait pas eu besoin, si,
comme il voulut le prétendre depuis, il eût agi conformément aux ordres
qu'il avait reçus. Avec un peu de sens, il devait se contenter d'une
désobéissance aussi formelle, et devenue aussi funeste que ses fautes,
et lors de la bataille, et dans toutes ses suites, la rendirent, et se
contenir dans le silence, puisqu'il ne pouvait douter de ce qu'il avait
à perdre par le plus facile éclaircissement.

Je fus surpris jusqu'à l'indignation d'un procédé si peu droit\,; je
rapportai les clefs à Chamillart et lui dis à l'oreille ce qu'il m'en
sembla. Je lui en reparlai une autre fois plus à mon aise, parce que ce
fut tout haut, tête à tête, et alors je connus que le roi, tout piqué
qu'il était contre le maréchal, ou par son ancien goût d'habitude, ou
par la constante protection de M\textsuperscript{me} de Maintenon, ne
voulait pas l'exposer à ce que méritait une si étrange conduite\,; que,
par cette raison, il la voulait ignorer, et que Chamillart en était
lui-même si persuadé, que, quelque désir qu'il eût de pousser le
maréchal à bout là-dessus, il n'osa l'entreprendre, quoique l'ayant si
belle, ou que, s'il le hasarda, ce fut sans succès, et qu'il cacha l'un
ou l'autre sous l'apparence du mépris, que je sentis bien n'être qu'un
voile à l'impuissance.

Dans cette situation, plus je les vis tous deux irréconciliables, plus
je me mis en soin du duc de Villeroy, devenu de mes amis par sa femme,
dont je l'étais depuis longtemps. Je sondai Chamillart, je leur parlai
ensuite, et ce fut alors que je sus d'eux que le père avait défendu au
fils de voir le ministre. Un homme de guerre, quel qu'il fût, n'en pas
voir le ministre, se rompit le cou sans ressource auprès du roi,
quelques talents et quelques services qu'il eût, et ne pouvait espérer
de continuer à servir, encore moins les récompenses ni le chemin
militaire. Ils me prièrent d'en parler à Chamillart, et de tâcher de lui
faire passer cela le plus doucement qu'il me serait possible. Je le fis
deux jours après, et j'y mis tout ce qu'il me fut possible. Je trouvai
un homme doux, poli, sensible aux avances, mais, sur la visite,
ministre, et qui me dit nettement que si le duc de Villeroy n'en
franchissait le pas, il ne servirait point. J'eus beau représenter à
Chamillart la situation du fils avec le père, la déraison et l'autorité
de ce père, la délicatesse du fils qui n'en avait éprouvé que des
duretés dans sa splendeur, à ne le pas choquer dans sa disgrâce\,; rien
ne put vaincre Chamillart. Il me chargea pour le duc de Villeroy de tous
les compliments du monde, de toutes les offres, de services possibles,
hors sur la guerre, et il n'y avait que sur la guerre où il pût lui en
rendre. Faute de mieux, il me fallut contenter d'avoir rapproché les
choses, dans l'espérance qu'elles se pourraient raccommoder tout à fait.

J'allai souper en tiers avec le duc et la duchesse de Villeroy, qui
s'affligea amèrement d'une réponse si dure parmi tant de compliments.
Son mari la sentit vivement. Je lui représentai son âge, ses services,
son grade de lieutenant général, et ce à quoi l'un et l'autre le
devaient fout naturellement conduire. Je lui parlai du bâton et du
commandement des armées\,; je lui représentai qu'il rendrait douteux
l'espèce de droit qu'il pouvait prétendre de succéder à la charge de
capitaine des gardes de son père, à laquelle l'exemple du duc de
Noailles lui frayait un chemin assuré\,; que l'éclat qui avait fait
chasser M\textsuperscript{me} de Caylus avait fait une impression qui
n'était effacée que pour elle, et qui subsistait contre lui et
M\textsuperscript{me} de Maintenon, comme il n'en pouvait douter, malgré
son amitié pour le maréchal de Villeroy\,; enfin, que son père avait
travaillé trop peu solidement pour lui, et lui avait toute sa vie trop
durement appesanti le joug pour que sciemment et volontairement il se
perdît sur une chose inutile, vaine, de purs travers et de pure
fantaisie, que son père même ne devait jamais exiger de lui. En un mot,
je n'oubliai rien, ni sa femme non plus\,; mais tout fut inutile.

Le duc de Villeroy avait promis à son père, qui avait exigé sa parole.
Accoutumé à trembler devant lui comme un enfant, il n'osa la refuser\,;
il ne put se résoudre à en manquer, même en ne voyant Chamillart qu'en
secret, ce que je me faisais fort de faire passer au ministre. Il fallut
donc se réduire à essayer qu'il se contentât d'un compliment du duc de
Villeroy, chez le roi, sur ce qu'il ne le voyait point chez lui. J'en
parlai à Chamillart de toute mon affection\,; mais il me répondit que ce
qui eût été bon d'abord venait trop tard, après deux mois de retour.
J'eus recours à la maréchale de Villeroy, de laquelle j'avais reçu cent
fois de vives plaintes sur toute cette affaire\,; je la reconnus si
éloignée de s'adoucir, que je n'osai pousser mon projet. Toutefois la
solide piété qui était en elle lui fit faire quelques réflexions.
D'elle-même elle permit à son fils de tâcher à fléchir son père. Le fils
n'y gagna rien. Il trouva son père plus entêté et plus furieux que
jamais.

Le vrai motif de cette rage fut l'énoncé de la patente de M. de Vendôme
pour aller commander l'armée en Flandre en sa place. Véritablement il
appesantissait la honte du maréchal et sans nécessité, et la rendait
immortelle. Ses amis en furent avertis à temps de l'arrêter, ce qui en
augmenta le bruit, et M. le Grand, ami de Chamillart, obtint de lui que
cet endroit de la patente serait réformé et changé. Elle était déjà
scellée lorsque Chamillart l'envoya retirer du chancelier sous prétexte
que son commis l'avait mal dressée. Le chancelier, ami du maréchal, et
scandalisé pour lui, ne fit pas difficulté de la rendre, ni le commis de
lui avouer que cet énoncé injurieux était l'ouvrage de son maître,
auquel un subalterne comme lui n'eût pas osé attenter. De cette sorte
fut expédiée une autre patente, sans que l'injure de la précédente pût
s'effacer du cœur du maréchal, qui ne manqua, pas de prétextes
différents et moins humiliants pour colorer sa haine.

S'il eût su céder au temps et embrasser de bonne grâce le sauve
l'honneur que nous avons vu le roi lui présenter avec tant de bonté et
d'affection, après toutes ses fautes, il fût revenu à la cour plus
puissant et plus en faveur que jamais. On a vu (t IV, p.~59 et suiv.)
qu'au retour de sa prison de Gratz il ne tint qu'à lui d'entrer au
conseil en quittant la guerre, et le salutaire conseil que lui en donna
son ami le chevalier de Lorraine, et avec quel travers insensé il le
refusa. La maréchale de Villeroy me l'a avoué depuis avec une douleur
amère. Le bon est qu'il est certain que sans qu'il ait été depuis nulle
mention de lui communiquer aucune affaire étrangère, il voulut quitter
la guerre l'hiver qui précéda la bataille de Ramillies, et c'était alors
la quitter pour rien\,; qu'il fit tout ce qu'il put pour engager le roi
à disposer du commandement de l'armée de Flandre, et lui permettre de
demeurer auprès de lui, et qu'il ne put jamais l'obtenir. C'est ainsi
que la plus haute faveur montre ce que vaut celui qui la possède, et se
trouve toujours inférieure à quelque peu de sens que ce soit. La fin de
tout ceci fut que le duc de Villeroy ne servit plus, et que Chamillart
se rabattit sur le fils, n'ayant pu pousser à bout le père. Il en coûta
dans la suite au duc de Villeroy le bâton de maréchal de France qu'il
vit donner à de ses camarades qui ne l'avaient pas mieux mérité que lui,
et qui n'en étaient pas plus capables, mais qui avaient toujours
continué à servir.

\hypertarget{chapitre-xx.}{%
\chapter{CHAPITRE XX.}\label{chapitre-xx.}}

1707

~

{\textsc{Accablement, vapeurs, instances de Chamillart pour être
soulagé.}} {\textsc{- Sa manière d'écrire au roi, et du roi à lui.}}
{\textsc{- Réponse étonnante.}} {\textsc{- Personnes assises et debout
aux conseils.}} {\textsc{- Impôts sur les baptêmes et mariages\,;
abandonnés par les désordres qu'ils causent.}} {\textsc{- Mort de du
Chesne, premier médecin des enfants de France.}} {\textsc{- Mariage de
Mezières avec M\textsuperscript{lle} Oglthorp\,; leur famille, leur
fortune, leur caractère.}} {\textsc{- Livre du maréchal de Vauban sur le
dîme royale\,; livres de Boisguilbert sur la même matière.}} {\textsc{-
Mort du premier et exil du second.}} {\textsc{- Origine de l'impôt du
dixième.}} {\textsc{- Mort du marquis de Lusignan\,; sa maison, sa
famille, sa fortune, son caractère.}} {\textsc{- Mort de Pointis.}}
{\textsc{- Mort du chevalier d'Aubeterre.}} {\textsc{- Comte
d'Aubeterre, son neveu\,; sa fortune, son caractère, leur extraction.}}

~

Chamillart, accablé du double travail de la guerre et des finances,
n'avait le temps de manger ni de dormir. Des armées détruites presque
toutes les campagnes par des batailles perdues, des frontières
immensément rapprochées tout à coup par le tournement de têtes des
généraux malheureux épuisaient toutes les ressources d'hommes et
d'argent. Le ministre à bout de temps à en chercher et à vaquer
cependant au courant, avait plus d'une fois représenté son impuissance à
suffire à deux emplois, qui dans des temps heureux auraient même fort
occupé deux hommes tout entiers. Le roi, qui l'avait chargé de l'un et
de l'autre pour se mettre à l'abri des démêlés entre la finance et la
guerre qui l'avaient si longtemps fatigué, du temps de MM. Colbert et de
Louvois, ne put se résoudre à décharger Chamillart des finances. Il fit
donc de nécessité vertu, mais à la fin, la machine succomba. Il lui prit
des vapeurs, des éblouissements, des tournements de tête. Tout s'y
portait, il ne digérait plus. Il maigrit à vue d'œil. Toutefois il
fallait que la roue marchât sans interruption, et dans ces emplois il
n'y avait que lui qui pût la faire tourner.

Il écrivit au roi une lettre pathétique pour être déchargé. Il ne lui
dissimula rien de la triste situation de ses affaires et de
l'impossibilité où leur difficulté le mettait d'y remédier, faute de
temps et de santé. Il le faisait souvenir de plusieurs temps et de
plusieurs occasions où il les lui avait exposées au vrai par des états
abrégés\,: il le pressait par les cas urgents et multipliés qui se
précipitaient les uns sur les autres, et qui chacun demandaient un
travail long, approfondi, continu, assidu, auquel, quand sa santé le lui
permettrait, la multitude de ses occupations, toutes indispensables, ne
lui laissait pas une heure à s'y appliquer. Il finissait que ce serait
bien mal répondre à ses bontés et à sa confiance, s'il ne lui disait
franchement que tout allait périr, s'il n'y apportait ce remède.

Il écrivait toujours au roi à mi-marge, et le roi apostillait à côté, de
sa main, et lui renvoyait ainsi ses lettres, Chamillart me montra
celle-là, après qu'elle lui fut revenue. J'y vis avec grande surprise
cette fin de la courte apostille de la main du roi\,: \emph{Eh bien\,!
nous périrons ensemble}.

Chamillart en fut également comblé et désolé\,; mais cela ne lui rendit
pas les forces. Il manqua des conseils, et surtout il se dispensa de
ceux des dépêches lorsqu'il pouvait éviter d'y rapporter\,; ou s'il y
avait des affaires, le roi lui donnait d'abord la parole, qui d'ailleurs
va par ancienneté entre les secrétaires d'État, et dès qu'il avait fait
il s'en allait. La raison était qu'il ne pouvait demeurer debout, et
qu'au conseil des dépêches, tous les secrétaires d'État, même ministres,
demeurent toujours debout, tant qu'il dure. Il n'y a que les princes qui
en sont, c'est-à-dire, Monseigneur, Mgr le duc de Bourgogne, Monsieur,
lorsqu'il vivait, le chancelier\,; et s'il y a des ducs, comme M. de
Beauvilliers, qui en était, assis. Aux autres conseils, tous ceux qui en
sont s'assoient, excepté s'il y entre, comme il arrive quelquefois, des
maîtres des requêtes qui viennent rapporter quelque procès au conseil de
finances, où ils ne s'assoient jamais, et y entrent en ces occasions
avec les conseillers d'État du bureau où le même maître des requêtes
avait auparavant rapporté la même affaire. Alors, les conseillers d'État
de ce bureau opinent immédiatement après lui, assis, et coupent par
ancienneté de conseillers d'État les ministres, les secrétaires d'État
et le contrôleur général, et les uns et les autres y cèdent en tout aux
ducs et aux officiers de la couronne, lorsqu'il s'en trouve au conseil,
comme M. de Beauvilliers, qui était de tous, et les deux maréchaux de
Villeroy avant et après lui.

La nécessité des affaires avait fait embrasser toutes sortes de moyens
pour avoir de l'argent. Les traitants en profitèrent pour attenter à
tout, et les parlements n'étaient plus en état, depuis longtemps, d'oser
même faire des remontrances. On établit donc un impôt sur les baptêmes
et sur les mariages sans aucun respect pour la religion et pour les
sacrements, et sans aucune considération pour ce qui est le plus
indispensable et le plus fréquent dans la société civile. Cet édit fut
extrêmement onéreux et odieux. Les suites, et promptes, produisirent une
étrange confusion. Les pauvres et beaucoup d'autres petites gens
baptisaient eux-mêmes leurs enfants sans les porter à l'église, et se
marièrent sous la cheminée par le consentement réciproque devant
témoins, lorsqu'ils ne trouvaient point de prêtre qui voulût les marier
chez eux et sans formalité. Par là plus d'extraits baptistaires, plus de
certitude des baptêmes, par conséquent des naissances, plus d'état pour
les enfants de ces sortes de mariages qui pût être assuré. On redoubla
donc de rigueurs et de recherches contre des abus si préjudiciables,
c'est-à-dire qu'on redoubla de soins, d'inquisition et de dureté pour
faire payer l'impôt.

Du cri public et des murmures on passa à la sédition en quelques lieux.
Elle alla si loin à Cahors qu'à peine deux bataillons qui y étaient
purent empêcher les paysans armés de s'emparer de la ville, et qu'il y
fallut envoyer des troupes destinées pour l'Espagne, et retarder leur
départ et celui de M. le duc d'Orléans. Mais le temps pressait, et il en
fallut venir à mander à Le Gendre, intendant de la province, de
suspendre l'effet\,; on eut grand'peine à dissiper le mouvement du
Quercy et, les paysans armés et attroupés, à les faire retirer dans
leurs villages. En Périgord, ils se soulevèrent tous, pillèrent les
bureaux, se rendirent maîtres d'une petite ville et de quelques
châteaux, et forcèrent quelques gentilshommes de se mettre à leur tête.
Ils n'étaient point mêlés de nouveaux convertis. Ils déclaraient tout
haut qu'ils payeraient la taille et la capitation, la dîme à leurs
curés, les redevances à leur seigneur, mais qu'ils n'en pouvaient payer
davantage, ni plus ouïr parler des autres impôts et vexations. À la fin,
il fallut laisser tomber cet édit d'impôt sur les baptêmes et les
mariages, au grand regret des traitants qui, par la multitude et bien
autant par les vexations, les recherches inutiles et les friponneries,
s'y enrichissaient cruellement.

Du Chesne, fort bon médecin, charitable et homme de bien et d'honneur,
qui avait succédé auprès des fils de France à Fagon, lorsque celui-ci
devint premier médecin du roi, mourut à Versailles à quatre-vingt-onze
ans, sans avoir été marié ni avoir amassé grand bien. J'en fais la
remarque, parce qu'il conserva jusqu'au bout une santé parfaite et sa
tête entière, soupant tous les soirs avec une salade et ne buvant que du
vin de Champagne. Il conseillait ce régime. Il n'était ni gourmand ni
ivrogne, mais aussi il n'avait pas la forfanterie de la plupart des
médecins.

Mezières, capitaine de gendarmerie, estimé pour son courage et pour son
application à la guerre, épousa une Anglaise, dont il était amoureux,
qui était catholique. Elle s'appelait M\textsuperscript{lle} Oglthorp.
Elle était bien demoiselle, mais sa mère avait été blanchisseuse de la
reine, femme du roi Jacques II, et M. de Lauzun m'a dit souvent l'avoir
vue et connue dans cette fonction à Londres. Elle avait beaucoup de
frères et de sœurs dans la dernière pauvreté. Elle avait beaucoup
d'esprit insinuant, et se faisant tout à tous, méchante au dernier point
et intrigante également, infatigable et dangereuse. Elle a eu des filles
de ce mariage qui ne lui ont cédé sur aucun de ces chapitres\,; dont
elles et leur mère ont rendu et rendent encore des preuves continuelles
avec une audace, une hardiesse, une effronterie qui se prend à tout et
n'épargne rien, et qui a mené loin leur fortune.

Mezières était un homme de fort peu, du nom de Béthisy, dont on voit
l'anoblissement assez récent. Il y a eu une maison de Béthisy, avec qui
il ne le faut pas confondre, qui peut-être n'est pas encore éteinte.
Avec cette naissance, la figure en était effroyable\,; bossu devant et
derrière à l'excès, la tête dans la poitrine au-dessous de ses épaules,
faisant peine à voir respirer, avec cela squelette et un visage jaune
qui ressemblait à une grenouille comme deux gouttes d'eau. Il avait de
l'esprit, encore plus de manège, une opinion de lui jusqu'à se regarder
au miroir avec complaisance, et à se croire fait pour la galanterie. Il
avait lu et retenu. Je pense que la conformité d'effronterie et de
talent d'intrigue fit un mariage si bien assorti. Sa sœur était mère de
M. de Lévi, gendre de M. le duc de Chevreuse. Il en sut tirer parti. Sa
fortune, qui lui donna un gouvernement et le grade de lieutenant
général, le rendit impertinent au point de prétendre à tout et de le
montrer. Il en demeura là pourtant avec tous ses charmes, et se fit peu
regretter des honnêtes gens. Sa femme, depuis, a bien fait des
personnages, et à force d'artifices a su marier ses filles hautement, et
bien faire repentir leurs maris de cette alliance.

On a vu (t. IV, p.~87 et suiv.) quel était Vauban à l'occasion de son
élévation à l'office de maréchal de France. Maintenant nous l'allons
voir réduit au tombeau par l'amertume de la douleur pour cela même qui
le combla d'honneur, et qui, ailleurs qu'en France, lui eût tout mérité
et acquis. Il faut se souvenir, pour entendre mieux la force de ce que
j'ai à dire, du court portrait de cette page (87), et savoir en même
temps que tout ce que j'en ai dit et à dire n'est que d'après ses
actions, et une réputation sans contredit de personne, ni tant qu'il a
vécu, ni depuis, et que jamais je n'ai eu avec lui, ni avec personne qui
tînt à lui, la liaison la plus légère.

Patriote comme il l'était, il avait toute sa vie été touché de la misère
du peuple et de toutes les vexations qu'il souffrait. La connaissance
que ses emplois lui donnaient de la nécessité des dépenses, et du peu
d'espérance que le roi fût pour retrancher celles de splendeur et
d'amusements, le faisait gémir de ne voir point de remède à un
accablement qui augmentait son poids de jour en jour.

Dans cet esprit, il ne fit point de voyage (et il traversait souvent le
royaume de tous les biais) qu'il ne prît partout des informations
exactes sur la valeur et le produit des terres, sur la sorte de commerce
et d'industrie des provinces et des villes, sur la nature et
l'imposition des levées, sur la manière de les percevoir. Non content de
ce qu'il pouvait voir et faire par lui-même il envoya secrètement
partout où il ne pouvait aller, et même où il avait été et où il devait
aller, pour être instruit de tout, et comparer les rapports avec ce
qu'il aurait connu par lui-même. Les vingt dernières années de sa vie au
moins furent employées à ces recherches auxquelles il dépensa beaucoup.
Il les vérifia souvent avec toute l'exactitude et la justesse qu'il y
put apporter, et il excellait en ces deux qualités. Enfin il se
convainquit que les terres étaient le seul bien solide, et il se mit à
travailler à un nouveau système.

Il était bien avancé lorsqu'il parut divers petits livres du sieur de
Boisguilbert, lieutenant général au siège de Rouen, homme de beaucoup
d'esprit, de détail et de travail, frère d'un conseiller au parlement de
Normandie, qui, de longue main, touché des mêmes vues que Vauban, y
travaillait aussi depuis longtemps. Il y avait déjà fait du progrès
avant que le chancelier eût quitté les finances. Il vint exprès le
trouver, et, comme son esprit vif avait du singulier, il lui demanda de
l'écouter avec patience, et tout de suite lui dit que d'abord il le
prendrait pour un fou, qu'ensuite il verrait qu'il méritait attention,
et qu'à la fin il demeurerait content de son système. Pontchartrain,
rebuté de tant de donneurs d'avis qui lui avaient passé par les mains,
et qui était tout salpêtre, se mit à rire, lui répondit brusquement
qu'il s'en tenait au premier et lui tourna le dos. Boisguilbert, revenu
à Rouen, ne se rebuta point du mauvais succès de son voyage. Il n'en
travailla que plus infatigablement à son projet, qui était à peu près le
même que celui de Vauban, sans se connaître l'un l'autre. De ce travail
naquit un livre savant et profond sur la matière, dont le système allait
à une répartition exacte, à soulager le peuple de tous les frais qu'il
supportait et de beaucoup d'impôts, qui faisait entrer les levées
directement dans la bourse du roi, et conséquemment ruineux à
l'existence des traitants, à la puissance des intendants, au souverain
domaine des ministres des finances. Aussi déplut-il à tous ceux-là,
autant qu'il fut applaudi de tous ceux qui n'avaient pas les mêmes
intérêts. Chamillart, qui avait succédé à Pontchartrain, examina ce
livre. Il en conçut de l'estime, il manda Boisguilbert deux ou trois
fois à l'Étang, et y travailla avec lui à plusieurs reprises, en
ministre dont la probité ne cherche que le bien.

En même temps, Vauban, toujours appliqué à son ouvrage, vit celui-ci
avec attention, et quelques autres du même auteur qui le suivirent\,; de
là il voulut entretenir Boisguilbert. Peu attaché aux siens, mais ardent
pour le soulagement des peuples et pour le bien de l'État, il les
retoucha et les perfectionna sur ceux-ci, et y mit la dernière main. Ils
convenaient sur les choses principales, mais non en tout.

Boisguilbert voulait laisser quelques impôts sur le commerce étranger et
sur les denrées, à la manière de Hollande, et s'attachait principalement
à ôter les plus odieux, et surtout les frais immenses, qui, sans entrer
dans les coffres du roi, ruinaient les peuples à la discrétion des
traitants et de leurs employés, qui s'y enrichissaient sans mesure,
comme cela est encore aujourd'hui et n'a fait qu'augmenter, sans avoir
jamais cessé depuis.

Vauban, d'accord sur ces suppressions, passait jusqu'à celle des impôts
mêmes. Il prétendait n'en laisser qu'un unique, et avec cette
simplification remplir également leurs vues communes sans tomber en
aucun inconvénient. Il avait l'avantage sur Boisguilbert de tout ce
qu'il avait examiné, pesé, comparé, et calculé lui-même en ses divers
voyages depuis vingt ans\,; de ce qu'il avait tiré du travail de ceux
que dans le même esprit il avait envoyés depuis plusieurs années en
diverses provinces\,; toutes choses que Boisguilbert, sédentaire à
Rouen, n'avait pu se proposer, et l'avantage encore de se rectifier par
les lumières et les ouvrages de celui-ci, par quoi il avait raison de se
flatter de le surpasser en exactitude et en justesse, base fondamentale
de pareille besogne. Vauban donc abolissait toutes sortes d'impôts,
auxquels il en substituait un unique, divisé en deux branches,
auxquelles il donnait le nom de dîme royale, l'une sur les terres par un
dixième de leur produit, l'autre léger par estimation sur le commerce et
l'industrie, qu'il estimait devoir être encouragés l'un et l'autre, bien
loin d'être accablés. Il prescrivait des règles très simples, très sages
et très faciles pour la levée et la perception de ces deux droits,
suivant la valeur de chaque terre, et par rapport au nombre d'hommes sur
lequel on peut compter avec le plus d'exactitude dans l'étendue du
royaume. Il ajouta la comparaison de la répartition en usage avec celle
qu'il proposait, les inconvénients de l'une et de l'autre, et
réciproquement leurs avantages, et conclut par des preuves en faveur de
la sienne, d'une netteté et d'une évidence à ne s'y pouvoir refuser\,;
aussi cet ouvrage reçut-il les applaudissements publics et l'approbation
des personnes les plus capables de ces calculs et de ces comparaisons,
et les plus versées en toutes ces matières qui en admirèrent la
profondeur, la justesse, l'exactitude et la clarté.

Mais ce livre avait un grand défaut. Il donnait à la vérité au roi plus
qu'il ne tirait par les voies jusqu'alors pratiquées\,; il sauvait aussi
les peuples de ruines et de vexations, et les enrichissait en leur
laissant tout ce qui n'entrait point dans les coffres du roi à peu de
chose près, mais il ruinait une armée de financiers, de commis,
d'employés de toute espèce\,; il les réduisait à chercher à vivre à
leurs dépens, et non plus à ceux du public, et il sapait par les
fondements ces fortunes immenses qu'on voit naître en si peu de temps.
C'était déjà de quoi échouer.

Mais le crime fut qu'avec cette nouvelle pratique, tombait l'autorité du
contrôleur général, sa faveur, sa fortune, sa toute-puissance, et par
proportion celle des intendants des finances, des intendants de
provinces, de leurs secrétaires, de leurs commis, de leurs protégés qui
ne pouvaient plus faire valoir leur capacité et leur industrie, leurs
lumières et leur crédit, et qui de plus tombaient du même coup dans
l'impuissance de faire du bien ou du mal à personne. Il n'est donc pas
surprenant que tant de gens si puissants en tout genre à qui ce livre
arrachait tout des mains ne conspirassent contre un système si utile à
l'État, si heureux pour le roi, si avantageux aux peuples du royaume,
mais si ruineux pour eux. La robe entière en rugit pour son intérêt.
Elle est la modératrice des impôts par les places qui en regardent
toutes les sortes d'administration, et qui lui sont affectées
privativement à tous autres, et elle se le croit en corps avec plus
d'éclat par la nécessité de l'enregistrement des édits bursaux.

Les liens du sang fascinèrent les yeux aux deux gendres de M. Colbert,
de l'esprit et du gouvernement duquel ce livre s'écartait fort, et
furent trompés par les raisonnements vifs et captieux de Desmarets, dans
la capacité duquel ils avaient toute confiance, comme au disciple unique
de Colbert son oncle qui l'avait élevé et instruit. Chamillart si doux,
si amoureux du bien, et qui n'avait pas, comme on l'a vu, négligé de
travailler avec Boisguilbert, tomba sous la même séduction de Desmarets.
Le chancelier, qui se sentait toujours d'avoir été, quoique malgré lui,
contrôleur général des finances, s'emporta\,; en un mot, il n'y eut que
les impuissants et les désintéressés pour Vauban et Boisguilbert, je
veux dire l'Église et la noblesse\,; car pour les peuples qui y
gagnaient tout, ils ignorèrent qu'ils avaient touché à leur salut que
les bons bourgeois seuls déplorèrent.

Ce ne fut donc pas merveille si le roi prévenu et investi de la sorte
reçut très mal le maréchal de Vauban lorsqu'il lui présenta son livre
qui lui était adressé dans tout le contenu de l'ouvrage. On peut juger
si les ministres à qui il le présenta lui firent un meilleur accueil. De
ce moment, ses services, sa capacité militaire unique en son genre, ses
vertus, l'affection que le roi y avait mise, jusqu'à croire se couronner
de lauriers en l'élevant, tout disparut à l'instant à ses yeux. Il ne
vit plus en lui qu'un insensé pour l'amour du public, et qu'un criminel
qui attentait à l'autorité de ses ministres, par conséquent à la sienne.
Il s'en expliqua de la sorte sans ménagement.

L'écho en retentit plus aigrement encore dans toute la nation offensée,
qui abusa sans aucun ménagement de sa victoire\,; et le malheureux
maréchal, porté dans tous les cœurs français, ne put survivre aux bonnes
grâces de son maître pour qui il avait tout fait, et mourut peu de mois
après, ne voyant plus personne, consumé de douleur et d'une affliction
que rien ne put adoucir, et à laquelle le roi fut insensible, jusqu'à ne
pas faire semblant de s'apercevoir qu'il eût perdu un serviteur si utile
et si illustre. Il n'en fut pas moins célébré par toute l'Europe, et par
les ennemis même, ni moins regretté en France de tout ce qui n'était pas
financiers ou suppôts de financiers.

Boisguilbert, que cet événement aurait dû rendre sage, ne put se
contenir. Une des choses que Chamillart lui avait le plus fortement
objectées était la difficulté de faire des changements au milieu d'une
forte guerre. Il publia donc un livret fort court, par lequel il
démontra que M. de Sully, convaincu du désordre des finances que Henri
IV lui avait commises, en avait changé tout l'ordre au milieu d'une
guerre, autant ou plus fâcheuse que celle dans laquelle on se trouvait
engagé, et en était venu à bout avec un grand succès\,; puis,
s'échappant sur la fausseté de cette excuse par une tirade de\,:
\emph{Faut-il attendre la paix pour..}., il étala avec tant de feu et
d'évidence un si grand nombre d'abus, sous lesquels il était impossible
de ne succomber pas, qu'il acheva d'outrer les ministres, déjà si piqués
de la comparaison du duc de Sully et si impatients d'entendre renouveler
le nom d'un grand seigneur qui en a plus su en finances que toute la
robe et la plume.

La vengeance ne tarda pas\,: Boisguilbert fut exilé au fond de
l'Auvergne. Tout son petit bien consistait en sa charge\,; cessant de la
faire, il tarissait. La Vrillière, qui avait la Normandie dans son
département, avait expédié la lettre de cachet. Il l'en fit avertir, et
la suspendit quelques jours comme il put. Boisguilbert en fut peu ému,
plus sensible peut-être à l'honneur de l'exil pour avoir travaillé sans
crainte au bien et au bonheur public qu'à ce qu'il lui en allait coûter.
Sa famille en fut plus alarmée et s'empressa à parer ce coup. La
Vrillière, de lui-même, s'employa avec générosité. Il obtint qu'il fît
le voyage, seulement pour obéir à un ordre émané qui ne se pouvait plus
retenir, et qu'aussitôt après qu'on serait informé de son arrivée au
lieu prescrit, il serait rappelé. Il fallut donc partir\,; La Vrillière,
averti de son arrivée, ne douta pas que le roi ne fût content, et voulut
en prendre l'ordre pour son retour, mais la réponse fut que Chamillart
ne l'était pas encore.

J'avais fort connu les deux frères Boisguilbert, lors de ce procès qui
me fit aller à Rouen et que j'y gagnai\,; comme je l'ai dit en son
temps. Je parlai donc à Chamillart\,; ce fut inutilement\,: on le tint
là deux mois, au bout desquels enfin j'obtins son retour. Mais ce ne fut
pas tout. Boisguilbert mandé, en revenant, essuya une dure mercuriale,
et pour le mortifier de tous points fut renvoyé à Rouen suspendu de ses
fonctions, ce qui toutefois ne dura guère. Il en fut amplement dédommagé
par la foule de peuple et les acclamations avec lesquelles il fut reçu.

Disons tout, et rendons justice à la droiture et aux bonnes intentions
de Chamillart. Malgré sa colère, il voulut faire un essai de ces
nouveaux moyens. Il choisit pour cela une élection près de Chartres,
dans l'intendance d'Orléans qu'avait Bouville. Ce Bouville, qui est mort
conseiller d'État, avait épousé la sœur de Desmarets. Bullion avait là
une terre où sa femme fit soulager ses fermiers. Cela fit échouer toute
l'opération si entièrement dépendante d'une répartition également et
exactement proportionnelle. Il en résulta de plus que ce que Chamillart
avait fait à bon dessein se tourna en poison, et donna de nouvelles
forces aux ennemis du système.

Il fut donc abandonné, mais on n'oublia pas l'éveil qu'il donna de la
dîme\,; et quelque temps après, au lieu de s'en contenter pour tout
impôt, suivant le système du maréchal de Vauban, on l'imposa sur tous
les biens de tout genre en sus de tous les autres impôts\,; on l'a
renouvelée en toute occasion de guerre\,; et même en paix le roi l'a
toujours retenue sur tous les appointements, les gages et les pensions.
Voilà comment il se faut garder en France des plus saintes et des plus,
utiles intentions, et comment on tarit toute source de bien. Qui aurait
dit au maréchal de Vauban que tous ses travaux pour le soulagement de
tout ce qui habite la France auraient uniquement servi et abouti à un
nouvel impôt de surcroît, plus dur, plus permanent et plus cher que tous
les autres\,? C'est une terrible leçon pour arrêter les meilleures
propositions en fait d'impôts et de finances.

Il mourut un autre homme de plus haut parage assurément, et de bien
loin, mais bien inférieur en tout le reste. Ce fut M. de Lusignan, de la
branche de Lezay, sortie d'Hugues VII, sire de Lusignan par Simon, son
quatrième fils, vers l'an 1100. À cette époque c'étaient déjà de fort
grands seigneurs, mais dans la maison desquels les comtés de la Marche,
d'Angoulême et d'Eu, ni les couronnes de Chypre et de Jérusalem
n'étaient pas encore entrés. Cette branche de Lezay subsistait seule de
toute cette grande maison, et cette branche même était restreinte en ce
marquis de Lusignan, son frère l'évêque de Rodez et ses deux fils. Il
avait aussi une sœur mariée à M. de La Roche-Aymon. M. de Lusignan était
un fort honnête homme, et qui n'aurait pas été sans talents si l'extrême
misère ne l'avait pas abattu. Il avait été lieutenant des gens d'armes
écossais. M\textsuperscript{me} de Maintenon qui l'avait connu en
province lorsque M\textsuperscript{me} de Neuillant la retira chez elle
en arrivant des îles de l'Amérique, et qui depuis sa fortune voulait
avoir l'honneur de lui appartenir, lui procura quelque subsistance, mais
petitement, à sa manière. Il fut envoyé extraordinaire à Vienne, où on
en fut content, puis à la cour de Lunebourg. Sa femme était Bueil. Son
frère de Rodez fut un étrange évêque.

M. de Lusignan mourut fort pauvre à soixante-quatorze ans, et laissa
deux fils. Le cadet, prêtre avec une petite abbaye, fut grand vicaire de
son oncle, et ne valut pas mieux. L'aîné, marié à une La Rochefoucauld
de la branche d'Estissac, n'a jamais rien fait. S'il n'a point eu
d'enfants, toute cette maison de Lusignan est éteinte\,; car ceux qui en
prennent le nom ne sauraient en montrer de jonction. Les Saint-Gelais
aussi qui s'en sont avisés n'en sont point et ne peuvent le montrer. Le
premier d'eux à qui cette imagination vint est Louis de Saint-Gelais,
baron de La Mothe-Sainte-Heraye, et par sa femme seigneur de Lansac, qui
fut un personnage en son temps, chevalier d'honneur de Catherine de
Médicis, capitaine de la seconde compagnie des cent gentilshommes de la
maison du roi, ambassadeur à Rome en 1554, chevalier du Saint-Esprit en
la seconde promotion 1579, mort en 1589 à soixante-seize ans, dont le
petit-fils fut M. de Lansac, gendre du maréchal de Souvré, mari de la
gouvernante de Louis XIV.

Peu après mourut Pointis, si connu par sa brave et heureuse expédition
de Carthagène, par d'autres actions et par beaucoup d'esprit, de valeur
et de capacité dans son métier. C'était un homme à aller dignement à
tout et utilement pour l'État dans la marine. Mais il n'était plus
jeune, et mourut pour s'être sondé lui-même et blessé. Il s'était
puissamment enrichi et n'avait ni femme ni enfants.

Le chevalier d'Aubeterre le suivit de près. Il avait quatre-vingt-douze
ans dont il abusait pour dire toutes sortes d'ordures et
d'impertinences. Il était le plus ancien lieutenant général de France.
Il s'était démis depuis peu du gouvernement de Collioure, et l'avait
fait donner à son neveu, dont le plus grand mérite était ici d'être le
complaisant et le courtisan des garçons bleus et des principaux commis
des ministres qu'il régalait souvent chez lui, et à l'armée d'être le
plus bas valet de M. de Vendôme qui le fit faire lieutenant général, et
de M. de Vaudemont qui lui valut bien de l'argent qu'il fricassa en
panier percé qu'il était. Ses bas manèges le firent chevalier de l'ordre
en 1724. Son mérite ne l'y aurait pas porté\,; pour sa naissance il n'y
avait rien à dire, surtout dans une pareille promotion. Le plus triste
état que j'aie guère connu était celui d'être sa femme ou son fils. Leur
nom n'est point Aubeterre, c'est Esparbès. Le maréchal d'Aubeterre, mort
en 1628 et maréchal de France en 1620, était gouverneur de Blaye. Il
épousa la fille unique et héritière de David Bouchard, vicomte
d'Aubeterre, chevalier du Saint-Esprit, gouverneur de Périgord, dont
leurs enfants prirent le nom et les armes, mais sans quitter les leurs.
Le chevalier d'Aubeterre, dont je viens de dire la mort, était le
cinquième fils de ce mariage, dont le second fils fut père du chevalier
de l'ordre, duquel aussi je viens de parler. Il commença extrêmement
tard à servir.

\hypertarget{chapitre-xxi.}{%
\chapter{CHAPITRE XXI.}\label{chapitre-xxi.}}

1707

~

{\textsc{Beringhen, premier écuyer, enlevé entre Paris et Versailles par
un parti ennemi, et rescous}} \footnote{Vieux mot qui signifie
  \emph{secouru et délivré}.} {\textsc{.}} {\textsc{- Cherbert à la
Bastille.}} {\textsc{- Duc de Bouillon gagne son procès contre son
fils.}} {\textsc{- Mariage du comte d'Évreux avec la fille de Crosat.}}
{\textsc{- Harlay quitte la place de premier président.}} {\textsc{-
Caractère d'Harlay.}} {\textsc{- Quelques dits du premier président
Harlay.}} {\textsc{- Candidats pour la place de premier président, que
je souhaite au procureur général d'Aguesseau.}} {\textsc{- Pelletier
premier président.}} {\textsc{- Portail président à mortier.}}
{\textsc{- Courson avocat général.}} {\textsc{- Mot ridicule du premier
président sur son fils.}} {\textsc{- Mariage du duc d'Estrées avec une
fille du duc de Nevers.}} {\textsc{- Mort du duc de Nevers\,; sa
famille, sa fortune, son caractère.}} {\textsc{- \emph{Parvulo} de
Meudon.}}

~

Un événement aussi étrange que singulier mit le roi fort en peine, et
toute la cour et la ville en rumeur. Le jeudi 7 mars, Beringhen, premier
écuyer du roi, l'ayant suivi à sa promenade à Marly, et en étant revenu
à sa suite à Versailles, en partit à sept heures du soir pour aller
coucher à Paris, seul dans son carrosse, c'est-à-dire un carrosse du
roi, deux valets de pied du roi derrière, et un garçon d'attelage
portant le flambeau devant lui sur le septième cheval. Il fut arrêté
dans la plaine de Billancourt, entre une ferme qui est sur le chemin,
assez près du bout du pont de Sèvres, et un cabaret dit \emph{le
Point-du-Jour}. Quinze ou seize hommes à cheval l'environnèrent et
l'emmenèrent. Le cocher tourna bride, et remena le carrosse et les deux
valets de pied à Versailles, où dans l'instant de leur arrivée le roi en
fut informé, qui envoya ordre aux quatre secrétaires d'État à
Versailles, à l'Étang et à Paris où ils étaient, d'envoyer à l'instant
des courriers partout sur les frontières avertir les gouverneurs de
garder les passages, sur ce qu'on avait su qu'un parti ennemi était
entré en Artois, qu'il n'y avait commis aucun désordre, et qu'il n'était
point rentré.

On eut peine d'abord à se persuader que ce fût un parti\,; mais la
réflexion que M. le Premier n'avait point d'ennemis, que ce n'était
point un homme en réputation d'argent bon à rançonner, et qu'il n'était
arrivé d'incident de ce genre à pas un de ces gros financiers, fit qu'on
revint à croire que ce pouvait être un parti.

C'en était un en effet. Un nommé Guetem, violon de l'électeur de
Bavière, lors de la dernière guerre qu'il faisait alors avec les alliés
contre la France s'était mis dans leurs troupes, où, passant par les
degrés, il était devenu très bon et très hardi partisan, et par là était
monté au grade de colonel dans les troupes de Hollande. Causant un soir
avec ses camarades, il paria qu'il enlèverait quelqu'un de marque entre
Paris et Versailles. Il obtint un passeport des généraux ennemis et
trente hommes choisis, presque tous officiers. Ils passèrent les
rivières déguisés en marchands, ce qui leur servit à poster leurs
relais. Plusieurs d'eux avaient resté sept ou huit jours à Sèvres, à
Saint-Cloud, à Boulogne\,; il y en eut même qui eurent la hardiesse
d'aller voir souper le roi à Versailles. On en prit un de ceux-là le
lendemain, qui répondit assez insolemment à Chamillart qui
l'interrogea\,; et un des gens de M. le Prince en prit un autre dans la
forêt de Chantilly, par qui on sut qu'ils avaient un relais et une
chaise de poste à la Morlière pour y mettre le prisonnier qu'ils
feraient, mais alors il avait déjà passé l'Oise.

La faute qu'ils firent fut d'abord de n'avoir pas emmené le carrosse
avec Beringhen dedans, tout le plus loin et le plus vite qu'ils auraient
pu à la faveur de la nuit, tant pour éloigner l'avis de sa capture, que
pour le ménager pour le chemin à lui faire faire à cheval et se donner
plus de temps pour leur retraite. Au lieu d'en user de la sorte, ils le
fatiguèrent au galop et au trot. Ils avaient laissé passer le chancelier
qu'ils n'osèrent arrêter en plein jour, et manquèrent le soir M. le duc
d'Orléans, dont ils méprisèrent la chaise de poste. Lassés d'attendre et
craignant d'être reconnus, ils se jetèrent sur ce carrosse, et crurent
avoir trouvé merveilles quand {[}ils{]} virent à la lueur du flambeau un
carrosse du roi et ses livrées, et dedans un homme avec un cordon bleu
par-dessus son justaucorps comme le Premier le portait toujours.

Il ne fut pas longtemps avec eux sans apprendre qui ils étaient, et leur
dire aussi qui il était. Guetem lui marqua toute sorte de respect et de
désir de lui épargner tout ce qu'il pourrait de fatigue. Il poussa même
ses égards si loin, qu'ils le firent échouer. Ils le laissèrent reposer
jusqu'à deux fois\,; ils lui permirent de monter dans la chaise de poste
dont j'ai parlé\,; ils manquèrent un de leurs relais, ce qui les retarda
beaucoup. Outre les courriers aux gouverneurs des frontières, on avait
dépêché à tous les intendants et à toutes les troupes dans leurs
quartiers\,; on avait détaché après eux plusieurs gardes du roi, du guet
même\,; et toute la petite écurie, où M. le Premier était fort aimé,
s'était débandée de tous côtés. Quelque diligence qu'on eût faite pour
garder tous les passages, il avait traversé la Somme, et il était à
quatre lieues par delà Ham, gardé par trois officiers sur sa parole de
ne point faire de résistance, tandis que les autres s'étaient mis en
quête d'un de leurs relais, lorsqu'un maréchal des logis arriva sur eux,
suivi, à quelque distance, d'un détachement du régiment de Livry, puis
d'un autre, de manière que Guetem, ne se trouvant pas le plus fort, se
rendit avec ses deux compagnons et devint le prisonnier du sien.

M. le Premier, ravi d'aise de sa rescousse, et fort reconnaissant
d'avoir été bien traité, les mena à Ham, où il se reposa le reste du
jour, et, à son tour, les traita de son mieux. Il dépêcha à sa femme et
à Chamillart. Le roi, fort aise, lut à son souper les lettres qu'il leur
écrivait.

Le mardi 29, le Premier arriva à Versailles sur les huit heures du soir,
et alla tout droit chez M\textsuperscript{me} de Maintenon, où le roi le
fit entrer, qui le reçut à merveilles et lui fit conter toute son
aventure. Quoiqu'il eût beaucoup d'amitié pour lui, il ne laissa pas de
trouver mauvais que tout fût en fête à la petite écurie, et qu'il y eût
un feu d'artifice préparé. Il envoya défendre toutes ces marques de
réjouissance, et le feu ne fut point tiré. Il avait de ces petites
jalousies, il voulait que tout lui fût consacré sans réserve et sans
partage. Toute la cour prit part à ce retour, et le Premier eut tout
lieu par l'accueil public de se consoler de sa fatigue.

Il avait envoyé Guetem et ses officiers chez lui à Paris attendre les
ordres du roi, où ils furent traités fort au-dessus de ce qu'ils
étaient. Beringhen obtint pour Guetem la permission de voir le roi et de
le mener à la revue ordinaire que le roi faisait toujours de sa maison à
Marly avant la campagne. Le Premier fit plus, car il l'y présenta au
roi, qui le loua d'avoir si bien traité le Premier, et ajouta qu'il
fallait toujours faire la guerre honnêtement. Guetem, qui avait de
l'esprit, répondit qu'il était si étonné de se trouver devant le plus
grand roi du monde, et qui lui faisait l'honneur de lui parler, qu'il
n'avait pas la force de lui répondre. Il demeura dix ou douze jours chez
le Premier pour voir Paris, l'Opéra et la Comédie, dont il devint
lui-même le spectacle. Partout on le courait, et les gens les plus
distingués n'en avaient pas honte, dont il reçut les applaudissements
d'un trait de témérité qui pouvait passer pour insolent. Le Premier le
régala toujours chez lui, lui fournit des voitures et des gens pour
l'accompagner partout, et, en partant, d'argent et des présents
considérables. Il s'en alla sur sa parole à Reims rejoindre ses
camarades, en attendant qu'ils fussent échangés, ayant la ville pour
prison. Presque tous les autres s'étaient sauvés. Leur projet n'était
rien moins que d'enlever Monseigneur ou un des princes ses fils.

Cette ridicule aventure donna lieu à des précautions qui furent d'abord
excessives, et qui rendirent le commerce fatigant aux ponts et aux
passages. Elle fut cause aussi qu'assez de gens furent arrêtés. Les
parties de chasse des princes devinrent pendant quelque temps plus
contraintes, jusqu'à ce que peu à peu toutes ces choses reprirent leur
cours ordinaire. Mais il ne fut pas mal plaisant de voir pendant ce
temps la frayeur des dames, et même de quelques hommes de la cour qui
n'osaient plus marcher qu'entre deux soleils, encore avec peu
d'assurance, et qui s'imaginaient des facilités merveilleuses pour être
pris partout.

Cherbert et six de ses prétendus domestiques furent arrêtés et conduits
à la Bastille. C'était un colonel suisse au service du roi, qui l'avait
quitté pour celui de Bavière, où il était devenu lieutenant général. Le
roi n'avait pas voulu qu'il roulât\footnote{C'est-à-dire qu'il servît à
  tour de rôle.} avec les siens. Il était furtivement revenu, et il fut
pris à Saint-Germain, où il se croyait caché.

L'accommodement de M. de Bouillon avec son fils n'avait pas tenu. Ils
s'étaient rebrouillés\,; ils allèrent plaider à Dijon. Le cardinal de
Bouillon s'y trouva, les rapatria et fit en sorte qu'ils plaidèrent
honnêtement. Le père gagna son procès en plein en fort peu de séjour
qu'il fit en Bourgogne, où le cardinal demeura toujours avec eux.

L'orgueil de cette maison céda immédiatement après au désir des
richesses. Le comte d'Évreux, troisième fils de M. de Bouillon, avait
trouvé dans les grâces du roi, procurées par M. le comte de Toulouse, et
dans la bourse de ses amis, de quoi se revêtir de la charge de colonel
général de la cavalerie, du comte d'Auvergne, son oncle\,; mais il
n'avait ni de quoi les payer ni de quoi y vivre, et M. de Bouillon ni le
cardinal n'étaient pas en état ou en volonté de lui en donner. Il se
résolut donc à sauter le bâton de la mésalliance, et de faire princesse
par la grâce du roi la fille de Crosat, qui, de bas commis, puis de
petit financier, enfin de caissier du clergé, s'était mis aux aventures
de la mer et des banques, et passait avec raison pour un des plus riches
hommes de Paris.

M\textsuperscript{me} de Bouillon, qui vint nous en donner part, nous
pria instamment d'aller voir toute la parentelle nombreuse et grotesque
pour être assimilée aux descendants prétendus des anciens ducs de
Guyenne. Elle nous en donna la liste, et nous fûmes chez tous, que nous
trouvâmes engoués de joie. Il n'y eut que la mère de
M\textsuperscript{me} Crosat qui n'en perdit pas le bon sens. Elle reçut
les visites avec un air fort respectueux, mais tranquille, répondit que
c'était un honneur si au-dessus d'eux qu'elle ne savait comment
remercier de la peine qu'on prenait, et ajouta à tous qu'elle croyait
mieux marquer son respect en ne retournant point remercier que
d'importuner des personnes si différentes de ce qu'elle était,
lesquelles ne l'étaient déjà que trop de l'honneur qu'elles lui
voulaient bien faire, et n'alla chez personne. Jamais elle n'approuva ce
mariage dont elle prévit et prédit les promptes suites.

Crosat fit chez lui une superbe noce, logea et nourrit les mariés.
M\textsuperscript{me} de Bouillon appelait cette belle-fille son petit
lingot d'or.

On gémissait cependant sous le poids des impôts et de l'immensité des
billets de monnaie sur lesquels on perdait infiniment. Malgré cet
accablement public, celui des nécessités de la guerre avait entassé un
grand nombre de nouveaux édits bursaux pendant les vacances du
parlement, qu'il avait été question d'enregistrer à sa rentrée. Harlay,
premier président, parla en cette occasion avec éloquence\,; mais, déchu
de toutes espérances du côté de la cour, il s'y expliqua avec une
liberté dont il n'avait jamais usé jusques alors. Parlant de ce grand
nombre d'édits bursaux qui se présentaient tous à enregistrer, il
s'étendit sur la nécessité de le faire. Il ajouta qu'il n'en fallait
rien craindre pour leur conscience ni pour leur honneur, puisque ce
n'était plus un temps où aucun examen ni aucune remontrance fût
admise\,; qu'il n'était donc point à propos d'entrer dans aucuns détails
sur ces édits, d'en discuter les motifs, les prétextes, l'équité,
puisque le parlement n'était plus chargé de rien de tout cela, mais
seulement de les vérifier en baissant la tête, qui était la seule chose
qui lui fût commandée. Un discours si peu usité ne manqua pas de faire
grand bruit. Le premier président en fut averti. Il en écrivit aux
ministres, et peu de jours après, il tâcha de se justifier auprès du
roi. Partout il fut reçu à merveille, caressé des ministres, fort bien
traité du roi. Il s'en retourna fort content\,; mais, peu après on
commença à se dire à l'oreille que ce cynique ne demeurerait pas
longtemps en place. Il dura pourtant encore quatre mois. Mais à la fin,
il fallut céder pour sortir par la belle porte, en faisant semblant de
vouloir se retirer.

Il convenait à un hypocrite par excellence de sortir de place comme il y
avait toujours vécu. Il fut donc à Versailles demander miséricorde,
comme font les généraux des chartreux à tous leurs chapitres généraux,
mais qui seraient enragés d'être pris au mot et qui ne manquent pas de
prendre les plus justes mesures pour que leur déposition ne soit pas
reçue. Mais ici la chose était décidée sans retour. Il vint donc à
Versailles un dimanche, 10 avril, débarquer dès le matin chez le
chancelier, avec la rage qu'on peut imaginer dans un homme de cette
humeur et de cette ambition, qui avait eu la parole formelle de cet
office de la couronne de la bouche du roi même plus d'une fois, comme je
l'ai raconté à l'occasion des bâtards, qui le voyait dans un autre par
qui il fallait passer même pour sa démission, et qui avait le crève-cœur
de ne pouvoir ignorer qu'il ne l'avait manqué que par la faveur et les
cris de M. de La Rochefoucauld, qui ne s'en était pas caché, en juste
rétribution de ses iniquités à notre égard dans notre procès de
préséance avec M. de Luxembourg.

Harlay, réduit à devenir le suppliant de celui qui jouissait, au lieu de
lui, de cette grande place, mena son fils en laisse dans le désir de le
faire son successeur. Il était conseiller d'État, et j'aurai occasion de
parler ailleurs de cet autre genre de cynique épicurien. De chez le
chancelier, il alla chez le roi qu'il vit en particulier avant le
conseil. Il avait préparé son compliment pour saisir ce moment précieux
de toucher le roi, et d'obtenir sa place pour son fils\,; mais cet homme
si adroit, si artificieux, si prompt et si fécond à la repartie, si
rompu à prendre ses tours et ses détours, se trouva si touché de cette
espèce de funérailles, peut-être encore si piqué, si outré, si confus,
qu'il n'en put proférer une parole, et qu'il sortit du cabinet du roi
plus mal content de soi que de sa démission même. Il eut la faiblesse de
revenir trouver le chancelier et de le conjurer de raccommoder ce qu'il
venait d'omettre. Il ne vit à Versailles que ceux de ses plus intimes
amis qu'il ne peut éviter, et qui eux-mêmes surent bien l'éviter dans la
suite, n'en ayant plus rien à craindre ni à espérer, et s'en retourna, à
Paris plongé dans l'amertume.

Harlay était un petit homme, maigre, à visage en losange, le nez grand
et aquilin, des yeux de vautour qui semblaient dévorer les objets et
percer les murailles\,; un rabat et une perruque noire mêlée de blanc,
l'un et l'autre guère plus longs que les ecclésiastiques les portent\,;
une calotte, des manchettes plates comme les prêtres et le chancelier.
Toujours en robe, mais étriquée, le dos courbé, une parole lente, pesée,
prononcée, une prononciation ancienne et gauloise\,; et souvent les mots
de même, tout son extérieur contraint, gêné, affecté\,; l'odeur
hypocrite, le maintien faux et cynique, des révérences lentes et
profondes, allant toujours rasant les murailles, avec un air toujours
respectueux mais à travers lequel pétillait l'audace et l'insolence, et
des propos toujours composés, à travers lequel sortait Toujours
l'orgueil de toute espèce, et tant qu'il osait, le mépris et la
dérision.

Les sentences et les maximes étaient son langage ordinaire, même dans
les propos communs\,; toujours laconique, jamais à son aise, ni personne
avec lui\,; beaucoup d'esprit naturel et fort étendu, beaucoup de
pénétration, une grande connaissance du monde, surtout des gens avec qui
il avait affaire, beaucoup de belles-lettres, profond dans la science du
droit, et ce qui malheureusement est devenu si rare, du droit public\,;
une grande lecture et une grande mémoire, et avec une lenteur dont il
s'était fait une étude, une justesse, une promptitude, une vivacité de
repartie surprenante et toujours présente. Supérieur aux plus fins
procureurs dans la science du palais, et un talent incomparable de
gouvernement par lequel il s'était tellement rendu le maître du
parlement qu'il n'y avait aucun de ce corps qui ne fût devant lui en
écolier, et que la grand'chambre et les enquêtes assemblées n'étaient
que des petits garçons en sa présence, qu'il dominait et qu'il tournait
où et comme il le voulait, souvent sans qu'ils s'en aperçussent, et
quand ils le sentaient sans oser branler devant lui, sans toutefois
avoir jamais donné accès à aucune liberté ni familiarité avec lui à
personne sans exception\,; magnifique par vanité aux occasions,
ordinairement frugal par le même orgueil, et modeste de même dans ses
meubles et dans son équipage pour s'approcher des mœurs des anciens
grands magistrats.

C'est un dommage extrême que tant de qualités et de talents naturels et
acquis se soient trouvés destitués de toute vertu, et n'aient été
consacrés qu'au mal, à l'ambition, à l'avarice, au crime. Superbe,
venimeux, malin, scélérat par nature, humble, bas, rampant devant ses
besoins, faux et hypocrite en toutes ses actions, même les plus
ordinaires et les plus communes, juste avec exactitude entre. Pierre et
Jacques pour sa réputation, l'iniquité la plus consommée, la plus
artificieuse, la plus suivie, suivant son intérêt, sa passion, et le
vent surtout de la cour et de la fortune.

On en a vu d'étranges preuves en faveur de M. de Luxembourg contre nous.
Quelque temps après sa décision dont notre récusation l'avait exclu, le
roi voulut savoir son avis de cette affaire. Il répondit que les ducs
avaient toute la justice et toute la raison pour eux, et qu'il l'avait
toujours cru de la sorte. Tel est l'empire de la vérité qu'elle tire les
aveux les plus infamants de la bouche même de ceux qui la combattent.
Après ce que ce juge avait fait dans ce procès, pouvait-il lui-même se
déshonorer davantage\,? On a vu (t. Ier, p.~414) avec quelle infamie il
s'appropria le dépôt que Ruvigny, son ami, lui avait confié. De ces
traits publics on peut juger de ce qui est plus inconnu.

Une âme si perverse était bourrelée, non de remords qu'il ne connut
jamais (ou du moins qu'il n'a jamais laissé apercevoir qu'il en eût
senti aucun), mais d'une humeur qui se pouvait dire enragée, qui ne le
quittait point, et qui le rendait la terreur et presque toujours le
fléau de tout ce qui avait affaire à lui. Comme elle ne l'épargnait pas,
elle n'épargnait personne, et ses traits étaient les plus perçants et
les plus continuels. Ce fut aussi une joie publique lorsqu'on en fut
délivré, et le parlement, accablé sous la dureté de son joug, en disputa
avec le reste du monde. C'est dommage qu'on n'ait pas fait un
\emph{Harleana} de tous ses dits qui caractériseraient ce cynique, et`
qui divertiraient en même temps, et qui le plus souvent se passaient
chez lui, en public et tout haut en pleine audience\,: Je ne puis
m'empêcher d'en rapporter quelques échantillons.

Montataire, père de Lassay, que M\textsuperscript{me} la Duchesse fit
faire chevalier de l'ordre en 1724, avait épousé en secondes noces une
fille de Bussy-Rabutin, si connu par son \emph{Histoire amoureuse des
Gaules}, qui le perdit pour le reste de ses jours. Le mari et la femme,
que j'ai connus tous deux, étaient tous deux grands parleurs, et on
disait grands chicaneurs. Ils allèrent à l'audience du premier
président. Il vint à eux à leur tour, le mari voulut prendre la parole,
la femme la lui coupa, et se mit à expliquer son affaire. Le premier
président écouta quelque temps, puis l'interrompant\,: «\,Monsieur,
dit-il au mari, est-ce là M\textsuperscript{me} votre femme\,? --- Oui,
monsieur, répondit Montataire fort étonné de la question. Que je vous
plains, monsieur\,!» répliqua le premier président, haussant les épaules
d'un air de compassion\,; et leur tourna le dos. Tout ce qui l'entendit
ne put s'empêcher de rire. Ils s'en retournèrent outrés, confondus, et
sans avoir tiré du premier président que cette insulte.

M\textsuperscript{me} de Lislebonne, qui outre son rang, sa
considération et son crédit, et celui de ses filles, alla un jour avec
elles à cette audience. Les réponses furent si cruelles qu'elles
sortirent en larmes de colère et de dépit.

Les jésuites et les pères de l'Oratoire sur le point de plaider
ensemble, le premier président les manda et les voulut accommoder. Il
travailla un peu avec eux, puis les conduisant\,: «\,Mes pères, dit-il
aux jésuites, c'est un plaisir de vivre avec vous\,;» et se tournant
tout court aux pères de l'Oratoire\,: «\,et un bonheur, mes pères, de
mourir avec vous.\,»

Le duc de Rohan sortant mal content de son audience, vif et brusque
comme il était, l'avait prié de ne le point conduire, et après quelques
compliments crut avoir réussi. Dans cette opinion il descend le degré,
disant rage et injures de lui à son intendant qu'il avait mené avec lui.
Chemin faisant, l'intendant tourne la tête, et voit le premier président
sur ses talons. Il s'écrie pour avertir son maître. Le duc de Rouan se
retourne, et se met à complimenter pour faire remonter le premier
président. «\,Oh\,! monsieur, lui dit le premier président, vous dites
de si belles choses, qu'il n'y a pas moyen de vous quitter\,;» et en
effet ne le quitta point qu'il ne l'eût vu en carrosse, et partir.

La duchesse de La Ferté alla lui demander l'audience, et, comme tout le
monde, essuya son humeur. En s'en allant elle s'en plaignait à son homme
d'affaires, et traita le premier président de vieux singe. Il la suivait
et ne dit mot. À la fin elle s'en aperçut, mais elle espéra qu'il ne
l'avait pas entendue\,; et lui, sans en faire aucun semblant, il la mit
dans son carrosse. À peu de temps de là, sa cause fut appelée, et tout
de suite gagnée. Elle accourut chez le premier président et lui fait
toutes sortes de remerciements. Lui, humble et modeste, se plonge en
révérences, puis, la regardant entre deux yeux\,: «\, Madame, lui
répondit-il tout haut devant tout le monde, je suis bien aise qu'un
vieux singe ait pu faire quelque plaisir à une vieille guenon.\,» Et de
là tout humblement, sans plus dire un mot, se met à la conduire, car
c'était sa façon de se défaire des gens, d'aller toujours et de les
laisser là d'une porte à l'autre. La duchesse de La Ferté eût voulu le
tuer ou être morte. Elle ne sut plus ce qu'elle lui disait, et ne put
jamais s'en défaire, lui toujours en profond silence, en respect, et les
yeux baissés, jusqu'à ce qu'elle fût montée en carrosse.

Les gens du commun, il les traitait de haut en bas\,; et il ne se
contraignait pas de dire à un procureur, à un homme d'affaires que des
gens de considération amenaient à son audience pour expliquer leur fait
mieux qu'ils ne l'eussent pu eux-mêmes\,: «\,Taisez-vous, mon ami, vous
êtes un bel homme pour me parler\,; je ne parle pas à vous.\,» On peut
croire, après ces sorties, comme le reste se passait.

Il ne traitait guère mieux certains conseillers. Les deux frères
Doublet, tous deux conseillers, et dont l'aîné avait du mérite, de la
capacité et de l'estime, avaient acheté les terres de Persan et de Croï,
dont ils prirent les noms. Ils allèrent à l'audience du premier
président. Il les connaissait très bien, mais il ne laissa pas de
demander qui ils étaient. À leur nom le voilà courbé tout bas en
révérences, puis, se relevant et les regardant comme les reconnaissant
avec surprise\,: «\,Masques, leur dit-il, je vous connais\,;» et leur
tourna le dos.

Pendant les vacances, il était chez lui à Gros-Bois. Deux jeunes
conseillers qui étaient dans le voisinage l'y allèrent voir. Ils étaient
en habit gris de campagne, avec leurs cravates tortillées et passées
dans une boutonnière, comme on les portait alors. Cela choqua l'humeur
du cynique. Il appela une manière d'écuyer, puis, regardant un de ses
laquais

«\,Chassez-moi, lui dit-il, ce coquin-là tout à cette heure qui a la
témérité de porter sa cravate comme messieurs.\,» Messieurs pensèrent en
tomber en défaillance, s'en allèrent le plus tôt qu'ils purent\,; ils se
promirent bien de n'y pas retourner.

Le peu de ses plus familiers, et sa plus intime famille n'en souffraient
pas moins que le reste du monde. Il traitait son fils comme un nègre.
C'était entre eux une comédie perpétuelle. Ils logeaient et mangeaient
ensemble, et jamais ne se parlaient que de la pluie et du beau temps.
S'il s'agissait d'affaires domestiques ou autres, ce qui arrivait
continuellement, ils s'écrivaient, et les billets cachetés avec le
dessus mouchaient\footnote{Voy., sur ce mot, t. Ier, p.~185, note.}
d'une chambre à l'autre. Ceux du père étaient impitoyables, ceux du
fils, qui se rebecquait volontiers, très piquants. Jamais n'allait chez
son père qu'il ne lui envoyât demander s'il ne l'incommoderait point. Le
père répondait comme il eût fait à un étranger. Dès que le fils
paraissait, le père se levait, le chapeau à la main, disait qu'on
apportât une chaise à monsieur, et ne se rasseyait qu'en même temps que
lui. Au départ, il se levait et faisait la révérence.

M\textsuperscript{me} de Moussy, sa sœur, ne le voyait guère plus
aisément ni plus familièrement, quoique dans le même logis. Il lui
faisait souvent de telles sorties à table, qu'elle se réduisit à manger
dans sa chambre. C'était une dévote de profession, dont le guindé,
l'affecté, le ton et les manières étaient fort semblables à celles de
son frère. La belle-fille, très riche héritière de Bretagne, était, avec
toute sa douceur et sa vertu, la victime de tous les trois.

Le fils avait tout le mauvais du père, et n'en avait pas le bon\,; un
composé du petit-maître le plus écervelé et du magistrat le plus grave,
le plus austère et le plus compassé, une manière de fou, étrangement
dissipateur et débauché. Lui et son père s'étaient figuré être parents
du comte d'Oxford, parce qu'il s'appelait Harley. Jamais race si
glorieuse, et glorieuse en tous points, jamais tant de fausse humilité.
Les aventures du premier président avec l'arlequin de la Comédie
italienne, et encore avec Santeuil, et avec bien d'autres, ont été sues
de tout le monde. Ce serait trop que de les rapporter ici, il y en a
pour des volumes.

Tout ce qui dans la robe se crut en passe brigua cette première place du
parlement. Argenson, cet inquisiteur suprême et qui avait tant enchéri
en ce genre sur La Reynie, n'oublia rien pour faire valoir ses services
par les amis importants qu'il s'était faits. Il espéra surtout des
jésuites et de ceux qui leur faisaient leur cour, aux dépens de ce qu'on
nommait ou voulait perdre sous le nom de jansénistes, et qui de fait ou
d'espérance se rendaient cette sorte de chasse si utile. Mais il se
méprit au bon côté. Le roi, accoutumé à savoir par lui tout l'intérieur
des familles et à lui confier beaucoup de petites affaires secrètes, ne
put se résoudre à se passer d'un homme si fin, si habile, si rompu dans
un ministère si obscur et si intéressant. Voysin, appuyé de son adroite
femme que M\textsuperscript{me} de Maintenon aimait beaucoup, approché
d'elle par l'intendance de Saint-Cyr qu'elle lui donna lorsque
Chamillart entra dans le ministère, était le candidat sur lequel on
jetait les yeux depuis longtemps pour toutes les grandes places de sa
portée. De Mesmes, porté par M. du Maine et par quelques valets
intérieurs, se flattait d'arriver. Mais l'heure de ces trois hommes
n'était pas venue.

Celle d'un quatrième était encore plus éloignée pour qui je désirais
cette place, sans avoir jamais eu aucune liaison avec lui. C'était
d'Aguesseau à qui ses conclusions dans notre procès de préséance contre
M. de Luxembourg m'avaient dévoué, et dont la réputation m'encourageait
à prétendre. Il n'avait pour lui que cet appui de sa propre réputation
qui en tout genre effaçait toutes les autres du parlement, et celle de
son père devant laquelle toutes celles du conseil disparaissaient. Je
désirais passionnément le fils à cause de ses conclusions, à son défaut
au moins son père. Celui-ci était fort connu du roi qui le voyait depuis
longtemps dans son conseil des finances. MM. de Chevreuse et de
Beauvilliers l'aimaient et l'estimaient singulièrement. Je les attaquai
tous deux à plus d'une reprise\,; à mon grand étonnement je n'en espérai
rien. Je les fis sonder d'ailleurs pour découvrir ce que ce pouvait être
avec aussi peu de succès. Je m'avisai de dresser une batterie dans
l'intérieur par Maréchal, et par celui-là d'y joindre Fagon, qui pouvait
également et directement atteindre au roi et à M\textsuperscript{me} de
Maintenon. Fagon était heureusement prévenu d'estime pour le procureur
général, et plus heureusement encore, c'était l'estime qui presque
toujours le déterminait, et quand il faisait tant que de vouloir servir,
il savait frapper à propos de grands coups. Mais il craignit que le
soupçon de jansénisme, si aisé à donner et à prendre, et dont le père et
le fils n'étaient pas exempts, ne fît leur exclusion, sans néanmoins se
dégoûter de travailler pour eux. J'agissais donc ainsi par les fentes,
ne pouvant mieux. Mais pour le chancelier avec qui j'étais en toute
portée, et que cette idée de jansénisme n'arrêtait point et l'eût plutôt
poussé, je ne m'y épargnai point, ni lui aussi.

Un mot que je lâchai de mon désir et de mon espérance à l'abbé de
Caumartin, leur ami, alla par lui jusqu'à eux. Le procureur général,
surpris des vues et des démarches d'un homme avec qui il n'avait aucune
sorte de liaison, me manda par l'abbé de Caumartin que, n'espérant rien,
il serait bien fâché d'être mis sur les rangs, avec force remerciements.
Le père m'en fit beaucoup par les galeries, où je le rencontrais souvent
sans m'arrêter à lui avec qui je n'avais aussi pas la moindre liaison,
et par la même raison me conjura de laisser éteindre, ce fut son
expression, le feu que j'avais allumé. Il se trouvait trop vieux et trop
avantageusement placé, pour aller entreprendre un métier pénible dans
lequel il se trouverait tout neuf\,; et pour son fils, il me dit mille
choses qui le barraient, outre que, modeste comme était ce bonhomme si
semblable à ces vertueux magistrats des anciens temps, il le trouvait
plus que très bien placé dans la charge de procureur général. Tout cela
ne me ralentit point, je continuai à pousser ma pointe, intérieurement
satisfait de me sentir aussi vif que le jour même des conclusions.

Lamoignon, porté par Chamillart alors tout-puissant, et par un favori
ardent à ce qu'il voulait, tel que M. de la Rochefoucauld son ami
intime, et qui avait coûté les sceaux au premier président, se pavanait
par avance, tandis que son camarade Pelletier, soutenu du crédit de son
père, était introduit par la chatière de la main de Saint-Sulpice, M. de
Chartres à leur tête, ayant pour adjoints les ducs de Chevreuse et de
Beauvilliers. Cette protection, qui auprès du roi et de
M\textsuperscript{me} de Maintenon avait également le mérite
antijanséniste, l'emporta sur celle des jésuites pour Argenson.

Pelletier ne tenait au roi par rien dont il eût peine à se passer comme
de l'autre. Il avait le même mérite à l'égard du jansénisme, et
M\textsuperscript{me} de Maintenon y alla tête baissée pour l'amour de
M. de Chartres. Les deux ducs, chose rare depuis longtemps, la
secondèrent en cette occasion. Ils étaient demeurés amis intimes de
Pelletier, le ministre d'État retiré\,; et le roi, qui l'avait aussi
toujours aimé, ne résista point au plaisir de lui donner dans sa
retraite la joie de voir son fils premier président, qui était tout ce
qu'il aurait pu lui procurer de plus considérable, s'il était demeuré
contrôleur général et dans tous les, conseils. Pelletier fut donc
choisi.

Sa charge de président à mortier, qui ne lui avait coûté que trois cent
mille livres, fit un autre mouvement dans la robe. La réputation que
Portail s'était acquise dans la charge d'avocat général lui aida
beaucoup à l'emporter. Il en donna cinq cent mille livres qui remplirent
le brevet de retenue d'Harlay\,; et Courson, second fils du président
Lamoignon, fut préféré pour la charge d'avocat général pour apaiser M.
de La Rochefoucauld, et donner quelque consolation au père de n'être pas
premier président. Tous ces messieurs-là reviendront sous ma plume. En
attendant je donnerai une idée de ce nouveau premier président.

Peu de mois avant qu'il le fût, il vint un soir à Versailles chez M.
Chamillart qui, à son ordinaire, était seul à table dans sa chambre avec
quelques familiers, et se déshabillait devant eux en sortant de souper.
Pelletier y vint tout à la fin du souper. Faute de mieux, quelqu'un lui
parla de son fils, aujourd'hui premier président, et le lui loua. Tout
de suite il répondit d'un air dédaigneux\,: «\,que son fils avait trop
de trois choses\,: de biens, d'esprit et de santé\,;» et il répéta plus
d'une fois cette sentence, en regardant la compagnie et cherchant un
applaudissement que personne n'eut la complaisance de lui donner. Un
moment après, il s'en alla comme Chamillart achevait de se déshabiller,
et laissa chacun dans un étonnement et dans un silence qui ne fut rompu
que par des interprétations peu obligeantes. Le premier écuyer et moi
nous étions regardés dans le premier instant. Chamillart nous aperçut,
nous demeurâmes, et nous nous en dîmes notre pensée.

Le cardinal d'Estrées fit le mariage du duc d'Estrées, son petit-neveu,
qui n'avait ni père ni mère, avec une fille du duc de Nevers, lequel ne
survécut pas ce mariage de huit jours. Le cardinal Mazarin avait deux
sœurs\,: M\textsuperscript{me} Martinozzi, qui n'eut que deux filles,
l'une mariée au duc de Modène, et mère de la reine d'Angleterre, épouse
du roi Jacques II, l'autre à M. le prince de Conti, bisaïeule de M. le
prince de Conti d'aujourd'hui, M\textsuperscript{me} Mancini, qui eut
cinq filles et trois fils. Les filles furent\,: la duchesse de Vendôme,
mère du dernier duc de Vendôme et du grand prieur, dont le père fut
cardinal après la mort de sa femme, la comtesse de Soissons, mère du
dernier comte de Soissons et du fameux prince Eugène\,; la connétable
Colonne, grand'mère du connétable Colonne d'aujourd'hui, qui toutes deux
ont fait tant de bruit dans le monde\,; la duchesse Mazarin, qui, avec
le nom et les armes de Mazzarini-Mancini, porta vingt-six millions en
mariage au fils du maréchal de La Meilleraye, et qui est morte en
Angleterre après y avoir demeuré, longues années\,; et la duchesse de
Bouillon, grand'mère du duc, de Bouillon d'aujourd'hui. Des trois fils,
l'aîné fut tué tout jeune au combat du faubourg Saint-Antoine, en 1652.
Il promettait tout. Le cardinal Mazarin l'aimait tellement qu'il lui
confiait, à cet âge, beaucoup de choses importantes et secrètes pour le
former aux affaires, où il avait dessein de le pousser. Le troisième
étant au collège des jésuites, fort envié des écoliers pour toutes les
distinctions qu'il y recevait, se laissa aller à se mettre à son tour
dans une couverture et à se laisser berner\,; ils le bernèrent si bien
qu'il se cassa la tête à quatorze ans qu'il a voit. Le roi, qui était à
Paris, le vint voir au collège. Cela fit grand bruit, mais n'empêcha pas
le petit Mancini de mourir. Resta seul le second, qui est M. de Nevers
dont il s'agit ici.

C'était un Italien, très Italien, de beaucoup d'esprit, facile,
extrêmement orné, qui faisait les plus jolis vers du monde qui ne lui
coûtaient rien, et sur-le-champ, qui en a donné aussi des pièces
entières\,; un homme de la meilleure compagnie du monde, qui ne se
souciait de quoi que ce fût, paresseux, voluptueux, avare à l'excès, qui
allait très souvent acheter lui-même à la halle et ailleurs ce qu'il
voulait manger, et qui faisait d'ordinaire son garde-manger de sa
chambre. Il voyait bonne compagnie, dont il était recherché\,; il en
voyait aussi de mauvaise et d'obscure avec laquelle il se plaisait, et
il était en tout extrêmement singulier. C'était un grand homme sec, mais
bien fait, et dont la physionomie disait tout ce qu'il était.

Son oncle le laissa fort riche et grandement apparenté. Il ne tint qu'à
lui de faire une grande fortune à l'ombre de la mémoire du cardinal
Mazarin, à laquelle très longtemps le roi accorda tout. M. de Nevers fut
capitaine des mousquetaires, dont le roi s'amusait fort. Il eut le
régiment d'infanterie du roi, auquel ce prince s'affectionna toute sa
vie\,; et se l'appropria comme un simple colonel, pour en faire
immédiatement tout le détail par lui-même. Tout cela, au lieu de
conduire M. de Nevers, l'importuna. Il suivit le roi quelques campagnes.
Les troupes et la guerre n'étaient pas son fait, ni la cour guère
davantage. Il quitta ces emplois pour la paresse et ses plaisirs. Il
avait porté la queue du roi le lendemain de son sacre, lorsqu'il reçut
l'ordre du Saint-Esprit des mains de Simon le Gras, évêque de Soissons,
qui, par le privilège de son siège, l'avait sacré en l'absence du
cardinal Antoine Barberin, archevêque-duc de Reims, qui était à Rome. En
conséquence, M. de Nevers fut chevalier de l'ordre, à la promotion de
1661, qu'il n'avait que vingt ans. Il se défit du gouvernement de la
Rochelle et du pays d'Aunis, et il épousa, en 1670, la plus belle
personne de la cour, fille aînée de M\textsuperscript{me} de Thianges,
sœur de M\textsuperscript{me} de Montespan. Il eut, en 1678, un brevet
de duc, qu'il ne tint qu'à lui, dix ans durant, de faire enregistrer. Il
le négligea. Il y voulut revenir quand il n'en fut plus temps, et ne put
l'obtenir. Il fut souvent jaloux fort inutilement, mais jamais brouillé
avec sa femme, qui était fort de la cour et du grand monde. Il ne
l'appelait jamais que Diane. Il lui est arrivé trois ou quatre fois
d'entrer le matin dans sa chambre, de la faire lever, et tout de suite
de la faire monter en carrosse, sans qu'elle, ni pas un de leurs gens à
tous deux, se fussent doutés de rien, et de partir de là pour Rome, sans
le moindre préparatif, ni que lui-même y eût songé trois jours
auparavant. Ils y ont fait des séjours considérables.

Il en eut deux fils et deux filles. L'aînée était mariée depuis sept ou
huit ans avec le prince de Chimay, chevalier de la Toison d'or, de
Charles II, et grand d'Espagne de Philippe V, lieutenant général de ses
armées, qui n'en eut point d'enfants, et qui depuis a été mon gendre.
L'autre fut la duchesse d'Estrées, qui n'a point eu d'enfants non plus.
Les deux fils furent M. de Donzi, fort mal avec son père, qui, par la
duchesse Sforce, sœur de sa mère, a été fait duc et pair pendant la
dernière régence\,; et M. Mancini, qui eut les biens d'Italie. J'aurai
occasion de parler d'eux dans la suite.

M. de Nevers mourut à soixante-six ans. Il s'était fort adonné à Sceaux,
et sa femme encore davantage. Il avait conservé le petit gouvernement du
Nivernais, parce que tout ce pays était presque à lui. Son fils, qui ne
servit presque point, et dont d'ailleurs la conduite avait toujours
déplu au roi, ne put l'obtenir. Il hasarda de se faire appeler duc de
Donzi, après la mort de son père, n'osant prendre le titre de Nevers. Le
roi le trouva si mauvais, qu'il lui fit défendre de continuer à se faire
appeler duc et d'en prendre le titre ni aucune marque. Son père n'avait
qu'un brevet, c'est-à-dire des lettres non enregistrées qui ne pouvaient
passer à son fils.

Avant que de rentrer dans des récits plus importants, je me souviens que
je n'ai point encore parlé de ce qu'on appelait à la cour les
\emph{parvulo} de Meudon, et il est nécessaire d'expliquer cette manière
de chiffre pour l'intelligence de plusieurs choses que j'aurai à
raconter. On a vu (t. Ier, p. 212) l'aventure de M\textsuperscript{me}
la princesse de Conti, pourquoi et comment elle chassa
M\textsuperscript{lle} Choin, qui elle était, et quels étaient ses amis
et l'attachement de Monseigneur pour elle. Ce goût ne fit qu'augmenter
par la difficulté de se voir. M\textsuperscript{me} de Lislebonne et ses
filles en avaient presque seules le secret, nonobstant tout ce qu'elles
devaient à M\textsuperscript{me} la princesse de Conti. Elles
fomentaient ce goût qui les entretenait dans une confidence dont elles
se proposaient de tirer de grands partis dans les suites.

M\textsuperscript{lle} Choin s'était retirée à Paris auprès du petit
Saint-Antoine, chez Lacroix, son parent, receveur général des finances,
où elle vivait fort cachée. Elle était avertie des jours rares que
Monseigneur venait dîner seul à Meudon, sans y coucher, pour ses
bâtiments ou pour ses plantages\,; elle s'y rendait la veille à la nuit,
dans un fiacre, passait les cours à pied, mal vêtue, comme une femme
fort du commun qui va voir quelque officier à Meudon, et par les
derrières entrait dans un entresol de l'appartement de Monseigneur, où
il allait passer quelques heures avec elle. Dans la suite, elle y fut de
même façon, niais avec une femme de chambre, son paquet dans sa poche, à
la nuit, la veille des jours que Monseigneur y venait coucher. Elle y
demeurait sans voir qui que ce soit que lui, enfermée avec sa femme de
chambre, sans sortir de l'entresol, où un garçon du château seul dans la
confidence lui portait à manger.

Bientôt après, du Mont eut la liberté de l'y voir, puis les filles de
M\textsuperscript{me} de Lislebonne, quand il allait des dames à Meudon.
Peu à peu cela s'élargit\,; quelques courtisans intimes y furent admis.
Sainte-Maure, le comte de Roucy, Biron après, puis un peu davantage et
deux ou trois dames, M. le prince de Conti tout à la fin de sa vie.
Alors Mgr le duc de Bourgogne, Mgr le duc de Berry, et fort peu de temps
après M\textsuperscript{me} la duchesse de Bourgogne, furent introduits
dans l'entresol, et cela ne dura pas longtemps sans devenir le secret de
la comédie. Le duc de Noailles et ses sœurs furent admis. Monseigneur y
allait dîner souvent avec les filles de M\textsuperscript{me} de
Lislebonne, souvent après avec elles, et M\textsuperscript{me} la
Duchesse, et quelquefois quelques-uns des privilégiés en hommes et en
femmes, qui s'étendit plus, et toujours avec le même air de mystère qui
dura toujours\,; et c'étaient ces parties secrètes, mais qui devinrent
assez fréquentes, qu'on appelait des \emph{parvulo}.

Alors M\textsuperscript{lle} Choin n'était plus dans les entresols que
pour la commodité de Monseigneur. Elle couchait dans le lit et dans le
grand appartement où logeait M\textsuperscript{me} la duchesse de
Bourgogne quand le roi allait à Meudon. Elle était toujours dans un
fauteuil devant Monseigneur, M\textsuperscript{me} la duchesse de
Bourgogne sur un tabouret\,; M\textsuperscript{lle} Choin ne se levait
pas pour elle\,; en parlant d'elle, elle disait, et devant Monseigneur
et la compagnie\,: «\,la duchesse de Bourgogne\,;» et vivait avec elle
comme faisait M\textsuperscript{me} de Maintenon, excepté qu'elle ne
l'appelait pas \emph{mignonne}, ni elle \emph{ma tante}, et qu'elle
n'était pas à beaucoup près si libre, ni si à son aise là qu'avec le roi
et M\textsuperscript{me} de Maintenon. Mgr le duc de Bourgogne y était
fort en brassière. Ses mœurs et celles de ce monde-là se convenaient
peu. Mgr le duc de Berry, qui les avait plus libres, y était à
merveille. M\textsuperscript{me} la Duchesse y tenait le dé, et
quelques-unes de ses favorites y étaient quelquefois reçues. Mais pour
tout cela, jamais M\textsuperscript{lle} Choin ne paraissait. Elle
allait, les fêtes, à six heures du matin, entendre une messe dans la
chapelle dans un coin toute seule, bien empaquetée dans ses coiffes,
mangeait seule quand Monseigneur ne mangeait pas en haut avec elle, et
il n'y mangeait jamais lorsqu'il couchait à Meudon que le jour qu'il y
arrivait (parce que {[}ce qui{]} en était ne venait que sur le soir), et
jamais ne mettait le pied hors de son appartement ou de l'entresol\,; et
pour aller de l'un à l'autre tout était exactement visité et barricadé
pour n'être pas rencontrée.

On la considérait auprès de Monseigneur comme M\textsuperscript{me} de
Maintenon auprès du roi. Toutes les batteries pour le futur étaient
dressées et pointées sur elle. On cabalait longtemps pour avoir la
permission d'aller chez elle à Paris\,; on faisait la cour à ses amis
anciens et particuliers. Mgr le duc de Bourgogne et
M\textsuperscript{me} la duchesse de Bourgogne cherchaient à lui plaire,
étaient en respect devant elle, et attention avec ses amis, et ne
réussissaient pas toujours. Elle montrait à Mgr le duc de Bourgogne la
considération d'une belle-mère, que toutefois elle n'était pas, mais une
considération sèche et importunée, et il lui arrivait quelquefois de
parler avec autorité et peu de ménagement à M\textsuperscript{me} la
duchesse de Bourgogne, et de la faire e pleurer.

Le roi et M\textsuperscript{me} de Maintenon n'ignoraient rien de tout
cela, mais ils s'en taisaient, et toute la cour, qui le savait, n'en
parlait qu'à l'oreille. Ce tableau suffit pour le présent. Il sera la
clef de plus d'une chose. M. de Vendôme et d'Antin étaient des
principaux initiés.

\hypertarget{chapitre-xxii.}{%
\chapter{CHAPITRE XXII.}\label{chapitre-xxii.}}

1707

~

{\textsc{Duc d'Orléans a un fauteuil à Bayonne, et à Madrid le
traitement d'infant.}} {\textsc{- Origine du fauteuil en Espagne pour
les infants et pour les cardinaux.}} {\textsc{- Étranges abus nés des
fauteuils de Bayonne à M. le duc d'Orléans et à M\textsuperscript{lle}
de Beaujolais.}} {\textsc{- Origine du traversement du parquet par les
princes du sang.}} {\textsc{- Époque où les princesses du sang ont
quitté les housses.}} {\textsc{- Trait remarquable de M. le prince à
Bruxelles avec don Juan et le roi Charles II d'Angleterre.}} {\textsc{-
Ses entreprises de distinction en France.}} {\textsc{- Règlement contre
le luxe des armées peu exécuté.}} {\textsc{- Bataille d'Almanza.}}
{\textsc{- Cilly apporte la nouvelle de la victoire d'Almanza.}}
{\textsc{- Valouse à Marly, de la part du roi d'Espagne.}} {\textsc{-
Bockley apporte le détail, et est fait brigadier.}} {\textsc{- M. le duc
d'Orléans arrive à l'armée victorieuse.}} {\textsc{- Origine de l'estime
et de l'amitié de M. le duc d'Orléans pour le duc de Berwick.}}
{\textsc{- Leurs différents caractères militaires.}} {\textsc{- Grand et
rare éloge du duc de Berwick par M. le duc d'Orléans.}} {\textsc{-
Manquement fatal de toutes choses en Espagne.}} {\textsc{- Siège de
Lerida.}} {\textsc{- La ville prise d'assaut et punie par le pillage.}}
{\textsc{- Le château rendu par capitulation.}} {\textsc{- Joyeuse
malice du roi sur Lerida à M. le Prince.}} {\textsc{- Cilly lieutenant
général.}} {\textsc{- Berwick grand d'Espagne avec les duchés de Liria
et de Xerica en don, une grâce, outre cela, sans exemple en grandesse,
et fait chevalier de la Toison d'or.}}

~

Les généraux des armées partirent chacun pour la leur. M. le duc
d'Orléans s'arrêta à Bayonne pour voir la reine veuve de Charles Ier,
qui lui donna un fauteuil. M. le duc d'Orléans, qui ne l'aurait osé
prétendre, se garda bien de le refuser.

En Espagne, les infants ont un fauteuil devant le roi et la reine. Il
leur est venu de celui des légats \emph{a latere}, qui sont reçus
partout presque comme le `pape en personne, et à qui nos rois ont été
au-devant fort loin, hors de leur ville, jusqu'à Louis XIV
exclusivement, mais qui y envoya Monsieur, qui donna la main au cardinal
Chigi, lequel eut, comme je l'ai marqué (t. II, p.~80), à propos de
l'erreur d'une tapisserie, un fauteuil à son audience du roi. Si les
légats l'ont eu en France, on peut juger si les rois particuliers des
Espagnes le leur disputaient. Ils le donnèrent aussi aux cardinaux qui
ont tant gagné par le grand rang des cardinaux légats, et par la fermeté
de la politique romaine, à porter le leur au plus haut point qu'elle a
pu. Ferdinand et Isabelle, ayant réuni les couronnes particulières
d'Espagne, firent trop d'usage des papes et de la cour de Rome pour
changer ce cérémonial. Philippe Ier, dit le Beau, leur gendre, à qui ces
couronnes devaient toutes arriver, n'eut que celle de Castille, parce
que Ferdinand le Catholique le survécut. Charles-Quint son fils, avant
d'être empereur, recueillit toutes les couronnes de l'Espagne, à celle
de Portugal près. Dès lors il pensait à l'empire, il avait François Ier
pour compétiteur. Il ménageait Rome et n'innova rien au cérémonial de
son grand-père et de sa grand'mère maternels. Philippe II son fils, avec
tous les partis qu'il sut tirer de Rome, n'avait garde d'y rien changer
non plus, et son exemple `a passé en règle à ses successeurs. Il est
même arrivé que plusieurs premiers ministres d'Espagne, avant et depuis
Philippe II, ont été cardinaux, ce qui n'a pas peu contribué à
consolider leur rang en Espagne. Je parlerai en un autre lieu de celui
dont ils y jouissent aujourd'hui\,; mais ce que je viens d'en dire
suffit pour ce que j'ai à expliquer ici.

Ce fauteuil des légats et des cardinaux est l'origine de celui des
infants. Mais en Espagne ils n'ont rien vu par delà ce degré que nous
appelons ici fils de France. Les infants, qui sont nos fils de France, y
ont été fort rares depuis Charles-Quint. À peine y en a-t-il eu d'autres
sous chaque règne que l'héritier de la couronne, et si on excepte le
malheureux don Carlos et un cardinal, le peu qu'il y en a eu a disparu
presque aussitôt que né. Aucun héritier de la couronne n'a été marié du
vivant du roi son père. Je ne compte pas Philippe II, que Charles-Quint
fit roi, qui épousa la reine Marie d'Angleterre, et qui, avant d'être
roi, fut presque toujours séparé du lieu de Charles-Quint, ailleurs en
Europe. Ainsi, en Espagne, il est vrai de dire que, jusqu'à présent, ce
que nous connaissons ici sous le nom de petit-fils de France et de
prince du sang n'y a jamais existé.

La reine douairière d'Espagne, confinée à Bayonne pour ses intelligences
avec l'archiduc, mal aux deux cours, peu comptée d'ailleurs et mal
payée, embarrassée d'un rang qu'elle savait bien n'être pas de fils de
France, mais en approcher fort et s'élever fort au-dessus de celui des
princes du sang, crut pouvoir aider à la lettre pour obliger le neveu,
et peut-être encore plus le neveu et le gendre du roi tout à la fois,
qui allait commander les armées en Espagne, et qui apparemment y
prendrait un grand crédit, au moins celui de la servir ou de lui nuire.
M. le duc d'Orléans, de son côté, hasarda d'accepter ce qui lui fut
offert, parce qu'on aime toujours à se rehausser.

Il n'ignorait pas que le premier fils de France qui ait eu un fauteuil
devant une tête couronnée a été Gaston, qui, étant lieutenant général de
l'État dans la minorité de Louis XIV, profita de l'indigence, des
malheurs, et des besoins de la reine d'Angleterre sa sœur pour ses
enfants et pour elle-même, réfugiés en France après l'étrange
catastrophe du roi Charles Ier, son mari, dont l'exemple et une raison
semblable valut le fauteuil à Monsieur et à Madame, père et mère de M.
le duc d'Orléans, {[}de la part{]} du roi Jacques II et de la reine sa
femme, réfugiés pareillement en France en 1688 par l'invasion et
l'usurpation du prince d'Orange, depuis dit le roi Guillaume III. Mais
il savait aussi que lui-même ne l'avait pu obtenir. On lui avait
seulement souffert, à M\textsuperscript{me} la duchesse d'Orléans, à
Mademoiselle, sa sœur, depuis duchesse de Lorraine, et aux trois filles
de Gaston, de ne voir le roi et la reine d'Angleterre qu'avec
Monseigneur, Monsieur ou Madame, devant qui ils ne prétendaient qu'un
tabouret\,; et comme tout s'étend en France sans autre droit que de
l'oser, les deux autres filles du roi, toujours blessées du rang si
supérieur au leur de leur sœur cadette, se mirent sur le même pied de ne
voir la cour d'Angleterre qu'avec des fils ou des filles de France\,;
puis d'elles, qui étaient princesses du sang par leurs maris, les autres
princesses du sang en ont toujours usé de même. Le roi le souffrait, et
le roi et la reine d'Angleterre n'étaient pas en situation de s'en
plaindre. C'était donc un demi droit, en M. le duc d'Orléans, que cette
prétention telle qu'elle pût être\,; et à l'égard des pays étrangers, il
ne donnait pas la main, et ne rendait pas la visite qu'il recevait des
ambassadeurs, comme faisaient les princes du sang. Les cardinaux
étrangers, même romains, lui écrivaient \emph{monseigneur} et
\emph{altesse royale\,;} et lorsqu'il écrivait aux rois, excepté celui
de France, il ne les traitait point de \emph{sire}, mais de
\emph{monseigneur}. Toutes ces raisons lui parurent bonnes pour ne faire
point de façons sur le fauteuil que la reine douairière d'Espagne lui
fit présenter. Le roi ne le trouva point mauvais, et en Espagne on n'osa
s'en plaindre.

Ce qui en résulta au contraire fut qu'on s'y piqua de ne faire pas moins
qu'à Bayonne, en sorte que don Gaspar Giron, le premier des quatre
majordomes du roi, alla avec des carrosses et des équipages du roi
au-devant de lui jusqu'à Burgos, c'est-à-dire de Madrid comme qui irait
d'ici presqu'à Poitiers, et que sur la route, et partout, il fut reçu en
infant d'Espagne. Il en eut le traitement entier, à la cour, du roi, de
la reine, des infants, des grands et de tout le monde, sans que cela y
ait fait, ni ici, la moindre difficulté\,; mais voici ce que les excès
deviennent. Ils en font naître sans fin, et il vaut mieux le dire ici
tout de suite.

Lorsque la reine veuve du roi Louis Ier d'Espagne, fille de M. le duc
d'Orléans, par conséquent princesse du sang, passa à Bayonne, la reine
douairière d'Espagne trancha toute difficulté, et la traita comme déjà
mariée et comme princesse des Asturies. Elle s'appuyait sur l'exemple de
M\textsuperscript{me} la duchesse de Bourgogne, que, par même raison de
couper court à tout, le roi traita et la fit totalement jouir du même
rang que si elle eût déjà été mariée. Vint après M\textsuperscript{lle}
de Beaujolais, aussi fille de M. le duc d'Orléans, allant épouser
l'autre infant. Sur l'exemple que je viens de rapporter, elle fut
traitée de même\,; mais la duchesse de Duras qui était chargée de sa
conduite, et qui avait mené avec elle la duchesse de Fitz-James sa
fille, depuis duchesse d'Aumont, ne se trouva point, ni sa fille, à
cette séance, parce qu'elle n'avait pas eu ordre de vivre autrement avec
M\textsuperscript{lle} de Beaujolais qu'avec une princesse du sang, et
laissa auprès d'elle sa gouvernante. À la rupture,
M\textsuperscript{lle} de Beaujolais fut renvoyée en France avec sa
sœur, veuve alors du roi Louis Ier. La princesse de Berghes, veuve d'un
grand d'Espagne et la marquise de Conflans, furent envoyées avec les
équipages du roi à Saint-Jean de Luz pour les ramener en France, l'une
comme camarera-mayor de la petite reine, l'autre choisie par
M\textsuperscript{me} la duchesse d'Orléans pour être gouvernante de
M\textsuperscript{lle} de Beaujolais sa fille. M. le duc d'Orléans
n'était plus, et il était régent au premier passage\,; mais M. le Duc
était premier ministre, et quelque chose de plus, et en même temps
prince du sang. La reine douairière d'Espagne ne pouvait plus considérer
M\textsuperscript{lle} de Beaujolais comme mariée et comme infante,
ainsi qu'elle avait fait la première fois. Il n'y avait point eu de
mariage, et elle était renvoyée\,; elle n'était donc plus que princesse
du sang.

Cela embarrassa la reine douairière, qui à la fin se résolut, pour
obliger M. le Duc dans sa puissance (qui toutefois n'y avait pas
seulement pensé), se résolut, dis-je, à donner un fauteuil à
M\textsuperscript{lle} de Beaujolais et à la traiter comme la première
fois, sous prétexte que ses propres malheurs la rendaient sensible à
celui de cette princesse, à qui elle ne le voulait pas appesantir par la
différence du traitement de son premier passage.

Elle habitait une très petite maison de campagne à la porte de Bayonne,
et elle y recevait le monde dans un petit salon, où je l'ai aussi vue,
de plain-pied à un grand et beau jardin. Après les premières embrassades
de la reine douairière à la petite reine et à M\textsuperscript{lle} de
Beaujolais, la reine douairière proposa à la princesse de Berghes
d'aller voir son jardin, et à la duchesse de Liñarez sa camarera-mayor
de l'y mener. Elles étaient averties\,; elles firent dans l'instant la
révérence et entrèrent dans le jardin, après quoi la reine douairière
fit apporter trois fauteuils. La marquise de Conflans y demeura debout
avec les autres dames de la reine douairière. La visite finie, on fit
appeler les deux dames qui étaient au jardin\,; elles ne trouvèrent plus
de fauteuil en rentrant. On était debout et aux embrassades pour prendre
congé.

Par le chemin, M\textsuperscript{lle} de Beaujolais vécut en princesse
du sang. Mais arrivées à Paris, elles trouvèrent que ce fauteuil y avait
fait grand bruit, et que là-dessus les princesses du sang le
prétendaient chez la reine d'Espagne. M\textsuperscript{me} la duchesse
d'Orléans, dont les enfants n'étaient plus petits-fils de France,
trouvait la prétention fort raisonnable, d'autant qu'elle en formait de
plus étranges pour elle-même, jusqu'à ne pas vouloir que les gardes de
la reine sa fille prissent la salle de ses gardes quand elle la venait
voir au Palais-Royal, tandis qu'à Versailles on ne leur disputa pas
d'être mêlés avec ceux du roi, et la droite dans leur salle. Cette
prétention du fauteuil, soutenue de l'autorité d'un prince du sang
pleinement administrateur de l'État, suspendit les visites. Un écrivit
en Espagne, d'où il vint défense à la reine d'Espagne de donner des
fauteuils, même à M\textsuperscript{me} la duchesse d'Orléans sa mère,
qui depuis ne l'a plus vue qu'en particulier, et pas un prince ni
princesse du sang ne l'ont visitée, si ce n'est M. le duc d'Orléans et
M\textsuperscript{lle}s ses sœurs, mais en dernier particulier.

Voilà où conduisent des complaisances mal entendues.
M\textsuperscript{me} la duchesse d'Orléans n'a jamais eu ni prétendu
qu'un tabouret devant les filles de France, même cadettes, même devant
M\textsuperscript{me} la duchesse de Berry sa fille\,; les princesses
filles de Gaston pareillement devant Madame, ainsi que
M\textsuperscript{lle} la duchesse d'Orléans, et M\textsuperscript{lle}
depuis duchesse de Lorraine\,; les princes et les princesses du sang
n'ont jamais eu ni prétendu qu'un siège à dos, sans bras, devant les
filles de Gaston, devant M. et M\textsuperscript{me} la duchesse
d'Orléans, et devant M\textsuperscript{lle} depuis duchesse de
Lorraine\,; et ils veulent un fauteuil devant les têtes couronnées, et
en particulier devant la petite reine d'Espagne, qui, sa couronne mise à
part, est veuve d'un infant d'Espagne, c'est-à-dire d'un fils de France,
puisque, quand Philippe V n'aurait pas eu la couronne d'Espagne, il
serait fils de France, conséquemment son fils petit-fils de France,
lequel remonte à la dignité, au rang, aux traitements de fils de France
par la couronne de son père (et ont été mis et reconnus sur ce pied-là
par Louis XIV, qui leur a envoyé le cordon bleu, dans le moment de leur
naissance, qui ne se donne ainsi qu'aux seuls fils de France, et les a
toujours regardés et traités en tout le reste comme fils de France).
Comment ajuster cela avec ces prétentions de fauteuil, si on ne veut
dire que la couronne d'Espagne ait dégradé les infants d'Espagne du rang
et de la dignité qu'ils ont apportée en naissant, et qui a été anéantie
par la seconde couronne de l'Europe\,? Voilà un paradoxe bien étrange et
toutefois bien littéral.

M. le Prince le héros, que les princes du sang n'accuseront pas d'avoir
manqué de hauteur ni d'entreprises hardies en faveur de leur rang,
témoin le traversement du parquet au parlement, qu'il hasarda à la suite
de M. son père et malgré lui dans la minorité de Louis XIV, et qui leur
est depuis demeuré, auparavant réservé au seul premier prince du sang\,;
la tentative de la housse clouée à son retour de Bruxelles, qu'il ne put
obtenir, d'où les princesses du sang ont quitté leurs housses qu'elles
portaient et avaient toujours portées jusqu'alors comme les duchesses,
et sans prétention à cet égard\,; et bien d'autres choses qui
écarteraient trop\,; M. le Prince, dis-je, pensait bien autrement sur
ces prétentions modernes avec les têtes couronnées. Il était à
Bruxelles, où bien qu'à la merci et au service de l'Espagne, il
maintint, avec la dernière hauteur, son rang, sa préséance, ses
distinctions sur don Juan, gouverneur général des Pays-Bas, bâtard
d'Espagne, et qui commandait les armées avec une hauteur, dans sa cour,
de fils légitime de roi. Charles II, roi d'Angleterre, avait été obligé
de s'y retirer. Il y était aux dépens de l'Espagne, et don Juan en
abusait et le traitait fort cavalièrement. M. le Prince en fut si choqué
qu'il voulut apprendre à vivre à ce superbe bâtard.

Il pria chez lui le roi d'Angleterre, don Juan, les principaux seigneurs
espagnols et flamands, et ce qu'il y avait de plus considérable auprès
de lui et parmi les chefs des troupes, et leur donna un magnifique
dîner. Le repas servi, M. le Prince en avertit le roi d'Angleterre\,;
qui, arrivant dans le lieu du festin avec toute la compagnie, vit une
grande table couverte de mets, un seul fauteuil, un couvert unique et un
cadenas. Voilà don Juan bien étonné, et qui le fut encore davantage
quand il vit M. le Prince présenter la serviette au roi d'Angleterre
pour laver et l'obliger de le faire. Le roi demanda à M. le Prince s'il
ne se mettait pas à table et ces messieurs. M. le Prince, au lieu de
répondre, prit une serviette et se tint debout vers le dos du fauteuil
où le roi d'Angleterre venait de s'asseoir. Aussitôt il se retourna à M.
le Prince pour l'obliger à se mettre à table et à faire apporter des
couverts. M. le Prince répondit que, quand il aurait eu l'honneur de le
servir, il trouverait avec don Juan une table servie pour la compagnie
et pour eux. Ce combat de civilités finit enfin par l'obéissance. M. le
Prince dit que le roi ordonnait qu'on apportât des couverts. Ils étaient
tout prêts et force tabourets aussi, qu'on apporta en même temps. M. le
Prince se mit sur le premier, à la droite du roi d'Angleterre\,; don
Juan, rageant de colère et de honte, sur le premier à gauche, et la
compagnie sur les autres. Voilà un trait bien éloigné de la prétention
du fauteuil. Il fit un honneur infini à M. le Prince, et procura depuis
au roi d'Angleterre les respects que lui devait don Juan, et dont, après
cet exemple si public et si fort parlant à lui, il n'osa plus s'écarter.

À propos de table, le luxe de la cour et de la ville était passé avec
tant d'excès dans les armées qu'on y portait toutes les délicatesses
inconnues autrefois dans les lieux du plus grand repos. Il ne se parlait
plus que de haltes chaudes dans les marches et dans les détachements, et
les repas qu'on portait à la tranchée pendant les sièges étaient non
seulement abondants dans tous leurs services, mais les fruits et les
glaces qu'on y servait avaient l'air des fêtes, avec une profusion de
toutes sortes de liqueurs. La dépense ruinait les officiers, qui, les
uns pour les autres, s'efforçaient à l'envi de paraître magnifiques\,;
et les choses nécessaires à porter et à faire quadruplaient leurs
domestiques et les équipages de l'armée, qui l'affamaient souvent. Il y
avait longtemps qu'on s'en plaignait, ceux même qui faisaient ces
dépenses qui les ruinaient, sans qu'aucun osât les diminuer. À la fin,
le roi fit ce printemps un règlement qui défendit aux lieutenants
généraux d'avoir plus de quarante chevaux d'équipage\,; aux maréchaux de
camp plus de trente\,; aux brigadiers plus de vingt-cinq et aux colonels
plus de vingt. Il eut le sort de tant d'autres faits sur le même sujet.
Il n'y a pays en Europe où il y ait tant de si belles lois et de si bons
règlements, ni où l'observation en soit de si courte durée. On ne tient
la main à aucun, et il arrive que souvent, même dès la première année,
tout est enfreint, et qu'on n'y pense plus dès la seconde.

On a vu (ci-dessus, p.~361) que la révolte de Cahors, qui avait obligé
d'y faire marcher des troupes destinées pour l'Espagne, avait retardé le
départ de M. le duc d'Orléans de huit jours. Ce délai lui coûta cher. Le
duc de Berwick, plus faible en infanterie que les ennemis, et engagé
dans un pays de montagnes, se trouva dans la nécessité de reculer un peu
devant eux pour regagner des plaines où il pût aider sa cavalerie.
Asfeld, qui tout l'hiver avait commandé sur cette frontière, y avait
heureusement, mais très difficilement, pourvu à la subsistance des
troupes. Tout y était donc mangé par les apports qui y avaient été faits
de tous les pays à portée d'en faire\,; et c'est ce qui avait obligé
Berwick de chercher à vivre dans ces montagnes, où les ennemis, fort
éloignés, mais assemblés de bonne heure, forcèrent de marches pour le
venir chercher, et tâcher de le prendre à leur avantage. Le marquis das
Minas, Portugais, commandait leur armée de concert avec Ruvigny, qu'on
appelait milord Galloway, d'un titre d'Irlande que le roi Guillaume lui
avait donné, et qui commandait les Anglais. Enflés de ce mouvement en
arrière, ils suivirent le maréchal de près, qui les attira ainsi dans
les plaines de la frontière du royaume de Valence.

Alors Berwick les eût volontiers combattus\,; mais il savait M. le duc
d'Orléans parti de Madrid pour le venir joindre, qui n'avait fait qu'y
passer et saluer le roi et la reine d'Espagne, et qui faisait toute la
diligence possible pour arriver. Il lui était subordonné de nom et
d'effet. Le roi avait avoué son repentir de lui avoir donné en Italie un
tuteur, qui l'avait perdue e malgré ce prince. Berwick ne voulait pas,
d'entrée de jeu, se brouiller avec un supérieur de cette élévation en
lui soufflant une bataille\,; ainsi il temporisait avec grand dépit de
l'audace des ennemis à l'approcher et à le tâter.

Elle leur crût tellement par la patience du maréchal qu'ils l'imputèrent
tout à fait à faiblesse. Pour en profiter, ils vinrent le chercher
jusque dans son camp. Asfeld, qui en eut le premier avis, l'envoya au
duc de Berwick avec qui il était fort bien, et prit sur soi de faire ses
dispositions de son côté, pour ne perdre pas un moment. Le maréchal fut
aussi diligent du sien, vint au galop voir celles d'Asfeld, les approuva
et ne songea plus qu'à combattre. Le début en fut heureux. Bientôt après
il se mit quelque désordre dans notre aile droite, qui souffrit un
furieux feu. Le maréchal y accourut, le rétablit, et la victoire ne fut
pas longtemps après à se déclarer pour lui. L'action ne dura pas trois
heures. Elle fut générale, elle fut complète. Elle commença tout de bon
sur les trois heures après midi, le 25 avril. Les ennemis en fuite et
poursuivis jusqu'à la nuit, perdirent tout leur canon et tous leurs
équipages avec beaucoup de monde. Il en coûta peu à notre armée\,; et de
gens de marque, le fils unique de Puysieux, qui était brigadier
d'infanterie et promettait beaucoup, avec un esprit orné, et Polastron,
colonel de la couronne.

Tout étant fini, le comte Dohna, qui s'était retiré dans la montagne
avec cinq bataillons, n'ayant ni vivres, ni eau, ni moyen de sortir de
là, envoya au maréchal, trop heureux d'être tous prisonniers de guerre,
qui chargea un officier général d'aller les chercher et les amener à son
camp. Ainsi on eut en tout huit mille prisonniers, parmi lesquels deux
lieutenants généraux, six maréchaux de camp, six brigadiers, vingt
colonels, force lieutenants-colonels et majors, et huit cents autres
officiers avec une grande quantité d'étendards et de drapeaux. Il y eut
treize bataillons entiers.

Cilly, des dragons, maréchal de camp, arriva à l'Étang avec cette bonne
nouvelle, où j'étais et où M\textsuperscript{me} la duchesse de
Bourgogne était venue de Marly, à qui Chamillart donnait une grande
collation. Ma surprise fut extrême lorsqu'en me retournant j'avisai
Cilly. Je jugeai qu'il y avait eu une action heureuse en Espagne. Je lui
demandai à l'instant des nouvelles de M. le duc d'Orléans, et je fus
fort affligé d'apprendre qu'il n'était pas arrivé à l'armée. Chamillart
dit tout bas la nouvelle à M\textsuperscript{me} la duchesse de
Bourgogne. Il me la dit aussi à l'oreille, et aussitôt s'en alla avec
Cilly la porter au roi. Madame accourut aussitôt chez
M\textsuperscript{me} de Maintenon, qui fut fort touchée d'apprendre que
M. son fils n'avait pas joint l'armée. Un musicien qui l'y crut,
accourut le dire à M\textsuperscript{me} la princesse de Conti, qui lui
donna une belle montre d'or qu'elle portait à son côté. Tout ce qui
était à Marly assiégea la porte de M\textsuperscript{me} de Maintenon.
Le roi, transporté de joie, y vint et y conta tout ce que Cilly lui
venait d'apprendre. Le lendemain le duc d'Albe vint à la promenade du
roi, à qui il en avait fait demander la permission, et qui le gracieusa
fort.

Le surlendemain, le même ambassadeur amena au roi Valouse, qui, écuyer
ici du duc d'Anjou, l'avait suivi en Espagne, et y était un de ses
quatre majordomes. Philippe V, averti de la victoire d'Almanza par
Ronquillo, que le duc de Berwick lui avait envoyé du champ de bataille,
avait dépêché Valouse pour venir remercier le roi de ses secours, et du
général qui venait de s'en servir avec tant de gloire.

Bockley, frère de la duchesse de Berwick, arriva le lendemain de Valouse
avec le détail, et en fut fait brigadier. Cilly était parti le 26 avril,
à la pointe du jour, lendemain de la bataille, et il était venu tout
droit ici sans passer à Madrid.

Ce même jour 26, M. le duc d'Orléans joignit l'armée, qui marchait à
Valence par des pays faciles, et qui ne s'éloignaient point de nos
magasins. On sut ce jour-là milord Galloway très dangereusement blessé,
que das Minas l'était aussi, et toute leur armée dispersée. Le duc de
Berwick, avec un gros détachement, alla fort loin recevoir M. le duc
d'Orléans, bien en peine de la réception qu'il lui ferait, et du dépit
qu'il aurait de trouver besogne faite. C'était, après le malheur de
Turin, en essuyer un nouveau bien fâcheux en un autre genre. Tout ce qui
lui était attaché en fut touché, et le public même sembla y prendre
part. L'air ouvert de M. le duc d'Orléans, et ce qu'il dit d'abordée au
maréchal sur ce qu'il était déjà informé qu'il avait fait tout ce qu'il
avait pu pour l'attendre, le rassurèrent. Il y joignit de justes
louanges\,; mais il ne put s'empêcher de se montrer fort touché de son
malheur, qu'il avait tâché d'éviter par toute la diligence imaginable,
et par ne s'être pas même arrêté à Madrid autant que la plus légère
bienséance l'aurait voulu. Enfin le prince, persuadé avec raison qu'il
n'avait pu être attendu plus longtemps par l'attaque des ennemis dans le
camp même du maréchal, et le maréchal à l'aise, ils ne furent point
brouillés\,; et cette campagne jeta entre eux les fondements d'une
estime et d'une amitié qui ne s'est depuis jamais démentie.

Ce n'est pas qu'ils fussent tous deux souvent de même avis. Le prince
était entreprenant et quelquefois hasardeux, persuadé qu'un attachement
excessif à toutes les précautions arrache des mains beaucoup d'occasions
glorieuses et utiles\,; le maréchal, au contraire, intrépide de cœur,
mais timide d'esprit, accumulait toutes les précautions et les
ressources, et en trouvait rarement assez. Ce n'était pas pour
s'accorder. Mais le prince avait le commandement effectif, et le
maréchal une probité si exacte que, content d'avoir contredit et disputé
de toutes ses raisons et de toute sa force un avis qui passait malgré
lui, il concourait à le faire réussir, non seulement sans envie, mais
avec chaleur et volonté, jusqu'à chercher des expédients nouveaux pour
remédier aux inconvénients imprévus, et à mettre tout du sien, comme
s'il eût été l'auteur du conseil qui s'exécutait nonobstant toute
l'opposition qu'il y avait faite.

C'est le témoignage que M. le duc d'Orléans m'a rendu de lui plus d'une
fois, et bien rare d'un homme nouvellement orné d'une grande victoire,
et naturellement opiniâtre et attaché à son sens. Mais, comme ce prince
me l'a souvent dépeint, il était doux, sûr, fidèle, voulant surtout le
bien de la chose, sans difficulté à vivre, vigilant, actif, et se
donnant, mais quand il était à propos, des peines infinies. Aussi M. le
duc d'Orléans m'a-t-il dit souvent que, encore que leurs génies se
trouvassent souvent opposés à la guerre, Berwick était un des hommes
qu'il eût jamais connus avec qui il aimerait mieux la faire\,; grande
louange, à mon avis, pour tous les deux.

J'avais un chiffre particulier que M. le duc d'Orléans m'avait donné en
partant, et lui et moi, nous chiffrions et déchiffrions nous-mêmes, et
ne nous écrivions en chiffre que par des courriers. Je lui proposai de
cueillir au moins de grands fruits de cette grande défaite, et le
dessein de laisser Berwick en Aragon avec une médiocre armée, et de s'en
aller avec le reste joindre le marquis de La Floride sur la frontière de
Portugal. Les ennemis n'y avaient ni magasins ni troupes, et le roi de
Portugal n'était pas en état de résister. Je pressai donc M. le duc
d'Orléans de profiter d'une conjoncture qui ne se retrouverait plus pour
s'illustrer par la conquête facile d'un royaume, délivrer l'Espagne de
ce côté-là de guerre et d'ennemis en l'agrandissant d'un pays si utile,
et de la mettre en état de finir la guerre, en portant la campagne
suivante toutes ses forces en Aragon, sans avoir plus de jalousie par
derrière. C'était en effet le moyen certain de terminer la guerre
d'Espagne en deux campagnes. On peut juger en passant quel eût été cet
avantage, quelles ses suites et quelle gloire pour le prince qui
l'aurait exécuté. Le malheur fut que l'exécution était impossible.

M. le duc d'Orléans me manda que ma proposition en elle-même était bonne
et solide pour une armée de non mangeants et de non buvants\,; que, dans
toute la longue route à travers les provinces d'Espagne, il n'y avait
magasin ni provision de quoi que ce fût, ni étapes réglées, ni moyen
aucun d'y suppléer\,; que, s'il y avait quelques provisions en Aragon
pour la subsistance des troupes, et encore successives, ce n'était qu'à
force d'industrie\,; que les chaleurs qui commençaient à se faire
sentir, et qui allaient devenir excessives, ajoutaient une nouvelle
difficulté à ce dessein que le manquement de toutes choses rendait
impossible\,; mais qu'il allait travailler à faire en sorte que ces
obstacles fussent levés pour l'année suivante, et à si bien profiter de
l'avantage que le duc de Berwick venait de remporter, qu'on pût
affaiblir assez l'armée d'Aragon, la campagne suivante, pour se porter
en nombre suffisant sur la frontière de Portugal, et y exécuter, à la
vérité plus difficilement alors, par les précautions qu'ils pourraient
avoir prises, ce que ce défaut aurait rendu aisé cette année.

À cela il n'y avait point de réplique. En Aragon, la disette de tout
était même telle qu'avec une armée victorieuse et en liberté d'agir ce
fut un chef-d'œuvre de l'industrie de pouvoir former le siège de Lerida,
après avoir battu encore plusieurs fois les ennemis en détail et en
petits corps, et pris plusieurs petites places. Achevons tout de suite
cette campagne d'Espagne. Les difficultés en furent si grandes qu'il
fallut, en attendant, s'amuser à nettoyer l'Aragon des petites places et
des postes, tandis que Bay prenait Ciudad-Rodrigo et d'autres places
vers le Portugal, amassa force drapeaux et étendards, et eut enfin près
de quatre mille prisonniers. Après des peines et des longueurs infinies,
la tranchée fut ouverte devant Lerida la nuit du 2 au 3 octobre. Asfeld
s'y chargea des vivres et des munitions, et M. le duc d'Orléans, qui m'a
dit souvent que c'était le meilleur intendant d'armée qu'il fût possible
de trouver, sans que ce pénible détail l'empêchât de ses fonctions
militaires, M. le duc d'Orléans, dis-je, se chargea lui-même de tous les
autres détails du siège, rebuté des difficultés qu'il rencontrait dans
chacun. Il fut machiniste pour remuer son artillerie, faire et refaire
son pont sur la Sègre, qui se rompit et qui ôta la communication de ses
quartiers. Ce fut un travail immense. Son abord facile, la douceur avec
laquelle il répondait à tout, la netteté de ses ordres, son assiduité
jour et nuit à tous les travaux, surtout aux plus avancés de la
tranchée, son exactitude à tout voir par lui-même, sa justesse à
prévoir, et l'argent qu'il répandit dans les troupes et qu'il fit donner
du sien aux officiers qui se trouvaient dans le besoin, le firent adorer
et donnèrent une volonté qui fut le salut d'une expédition que tout
rendit si difficile.

C'était après Barcelone le centre et le refuge des révoltés, qui se
défendirent en gens qui avaient tout à perdre et rien à espérer. Aussi
la ville fut-elle prise d'assaut le 13 octobre, et entièrement
abandonnée au pillage pendant vingt-quatre heures. Elle était remplie de
tout ce que les lieux à sa portée y avaient pu retirer. On n'y épargna
pas les moines qui animaient le plus les habitants. La garnison se
retira au château, où les bourgeois entrèrent avec elle. Ce château tint
encore longtemps\,; enfin il capitula le 11 novembre, et le chevalier de
Maulevrier en apporta la nouvelle au roi le 19.

Chamillart l'amena sur les huit heures avant que le premier gentilhomme
de la chambre fût entré. Le roi les fit venir à l'instant à son lit\,;
il fut si content de cette nouvelle qu'il envoya éveiller Madame et
M\textsuperscript{me} la duchesse d'Orléans pour la leur apprendre.

Ils sortirent cinq à six cents hommes, et pouvaient tenir encore
quelques jours\,; et tant devant la ville que devant le château, M. le
duc d'Orléans n'eut pas plus de sept à huit cents hommes tués ou
blessés. L'armée ennemie n'était qu'à deux lieues de Lerida, lorsque le
château se rendit, faisant contenance de venir le secourir. Das Minas,
blessé à Almanza, en avait repris le commandement\,; Galloway,
extrêmement blessé, était hors d'état d'agir. Après une campagne si
longue et si difficile, il n'y eut plus moyen de rien entreprendre\,; et
quelque désir que M. le duc d'Orléans eût d'aller faire le siège de
Tortose, il fallut le remettre à l'année suivante.

M. le Prince, mais surtout M. le Duc, et un peu M. le prince de Conti,
voyaient avec grande jalousie la gloire de M. le duc d'Orléans. Ils
étaient surtout piqués de la conquête de Lerida, dont M. le Prince, tout
grand et hardi capitaine qu'il était, avait levé le siège, et une autre
fois encore le comte d'Harcourt. M. le Duc et M\textsuperscript{me} la
Duchesse ne se contenaient pas, et M. le Prince s'échappait volontiers.
J'eus le plaisir d'entendre le roi adresser la parole là-dessus à M. le
Prince à son dîner, puis à M. le prince de Conti avec une joie maligne
qui jouissait de leur embarras. Il vanta l'importance de la conquête, il
en expliqua les difficultés, il loua M. le duc d'Orléans, et leur dit
sans ménagement que ce lui était une grande gloire d'avoir réussi où M.
le Prince avait échoué. M. le Prince balbutia, lui qui tenait si
aisément et si volontiers le dé. J'étais vis-à-vis de lui, et je voyais
à plein qu'il rageait. M. le prince de Conti, auprès de qui j'étais,
plus doux et plus circonspect, ne prenait pas plus de plaisir à cette
conversation, qui, de la part du roi, fut allongée. M. le prince de
Conti ne dit que quelques mots pour ne pas demeurer dans le silence, et
laissa le poids à M. le Prince, qui, avec tout son esprit et ses grâces
(car il en avait beaucoup dans la conversation), se tira au plus mal de
celle-là. Elle ne put durer qu'une partie du dîner, étant aussi peu
soutenue d'une part\,; mais le roi qui ne voulut rien affecter, et qui
se plaisait à les mortifier, se tourna, sur la fin, à M. de Marsan,
presque derrière sa chaise, et lui reparla du succès de M. le duc
d'Orléans qui avait été l'écueil du comte d'Harcourt. Marsan n'en était
pas à cela près pourvu que le roi lui parlât, et qu'il pût lui
barbouiller quelque chose. Il chercha donc à faire sa cour et à parler,
et renouvela le dépit et l'embarras de M. le Prince qui n'ouvrit pas la
bouche, mais à qui l'impatience, sortait par les yeux et de toute sa
contenance. Cette scène, je l'avoue, me divertit beaucoup. Cela fit du
bruit à la cour et dans le monde\,; j'eus regret que M. le Duc ne s'y
trouvât pas.

Le roi fit Cilly lieutenant général en le renvoyant, et permit au duc de
Berwick d'accepter la grandesse que le roi d'Espagne lui accorda, tant
pour lui que pour celui de ses fils qu'il lui serait libre de choisir.
Elle fut de la première classe. Pour ajouter l'utile à l'honneur, le roi
d'Espagne établit cette grandesse sur les villes et territoires de Liria
et de Xerica dans le royaume de Valence conjointement, dont il lui fit
présent. C'était un domaine de quarante mille livres de rente du domaine
de la couronne, qui avait fait autrefois l'apanage des infants d'Aragon.

Cette grâce très justement méritée était sans exemple\,:

1° On a déjà vu que le père et le fils ne sont jamais grands tous deux à
la fois, le père eût-il plusieurs grandesses, à moins que le fils n'eût
succédé à sa mère morte qui en aurait eu une de son chef, où qu'il jouît
de celle de sa femme qui lui en aurait apporté une\,; 2° la grandesse
passe toujours à l'aîné, et d'aîné en aîné, et ne fut jamais laissée au
choix du père\,; 3° {[}ce{]} qui n'est pas sans exemple, mais qui en a
fort peu, est le don de la terre et d'un domaine aussi distingué. J'ai
profité de l'exemple des deux qui sont sans exemple. Je remets ailleurs
à expliquer ce qui fit que le duc de Berwick désira le choix entre ses
enfants pour la grandesse. Le roi d'Espagne crut que ce n'était pas
encore assez, il le fit chevalier de la Toison d'or\footnote{Passage
  omis par les anciens éditeurs depuis \emph{On a déjà vu}.}.

\hypertarget{chapitre-xxiii.}{%
\chapter{CHAPITRE XXIII.}\label{chapitre-xxiii.}}

1707

~

{\textsc{Différence du gouvernement de la Castille et de l'Aragon, l'un
plus despotique que la France, l'autre moins que l'Angleterre.}}
{\textsc{- Explication curieuse.}} {\textsc{- Philippe V abolit les lois
et les privilèges de l'Aragon et de ses dépendances, et les soumet aux
lois et au gouvernement de Castille.}} {\textsc{- Deux partis proposés
par Médavy pour les troupes restées avec lui en Italie, tous deux bons,
tous deux rejetés.}} {\textsc{- Traité pour le libre retour des troupes
en abandonnant l'Italie.}} {\textsc{- Duc de Mantoue, dépouillé sans
être averti, se retire précipitamment à Venise.}} {\textsc{- Contraste
étrange de la fortune des alliés de Louis XIII et de ceux de Louis
XIV.}} {\textsc{- Médavy à Marly\,; sa récompense.}} {\textsc{- Arrivée
de Vaudémont à Paris et à la cour.}} {\textsc{- Chambre de la Ligue.}}
{\textsc{- Vaudémont et ses nièces\,; leur union, leur intérêt, leur
cabale, leur caractère, leur conduite.}} {\textsc{- Étrange découverte
de M\textsuperscript{me} la duchesse de Bourgogne sur
M\textsuperscript{me} d'Espinoy.}} {\textsc{- M\textsuperscript{me} de
Soubise\,; son caractère\,; son industrie.}}

~

Le roi d'Espagne profita de l'état où la bataille d'Almanza et ses
suites venaient de mettre les affaires d'Aragon, et de la leçon que ses
peuples lui avaient donnée de l'inutilité de sa considération et de ses
bontés pour eux, pour se les attacher. Rien de plus différent que le
gouvernement de la Castille et que celui de l'Aragon et des royaumes et
provinces annexés à chacune de ces couronnes En Castille le gouvernement
est despotique plus encore que nos derniers rois ne l'ont rendu en
France.

Ils y ont du moins conservé quelques formes, et communiqué à d'autres le
pouvoir de rendre des arrêts, qui sans aller plus loin s'exécutent. Il
est vrai que nos rois sont les seuls juges de leurs sujets\,; qu'il ne
se rend de jugement souverain qu'en leur nom, que ceux qui se prononcent
peuvent être arrêtés et réformés par eux, qu'ils peuvent
évoquer\footnote{Voy., les notes de la fin du volume sur les évocations,
  l'enregistrement des édits et le droit de remontrances.} aussi à eux
toutes les affaires qu'ils jugent à propos, pour les juger, ou seuls, ou
avec qui il leur plaît, ou les renvoyer à qui bon leur semble. Il est
encore vrai que les enregistrements nécessaires de leurs édits et
déclarations ne sont rien moins à leur égard que l'emprunt de l'autorité
des parlements qui enregistrent pour que l'exécution s'ensuive, mais
uniquement une manifestation publique de ces édits et déclarations dont
l'enregistrement sert, et à la publier dans les juridictions
inférieures, et à demeurer en note dans les registres du parlement, pour
que les juges s'en souviennent, et que, tant eux que les juges
inférieurs, conforment leurs jugements à cette volonté des rois déclarée
à eux, et par eux, à tous leurs sujets par cet envoi que
l'enregistrement ordonne qui sera fait aux tribunaux inférieurs des
instruments qui la contiennent et qu'eux-mêmes viennent d'enregistrer.
Il est vrai encore que les remontrances des parlements ne sont en effet
que des remontrances et non des empêchements, parce qu'en France il n'y
a qu'une autorité unique, une puissance unique, qui réside dans le roi,
de laquelle et au nom duquel émanent toutes les autres. C'est une autre
vérité que les états généraux mêmes ne se peuvent assembler que par les
rois, qu'ils n'ont dans leur assemblée aucune puissance législative, et
qu'à l'égard des rois, ils n'ont que la voix consultative et la voix de
représentation et de supplication. C'est ce que toutes les histoires et
toutes les relations des états généraux montrent avec évidence. La
différence d'eux aux parlements est que le corps représentatif de tout
l'État mérite et obtient plus de poids et plus de considération de ses
rois qu'une cour de justice, ou que plusieurs ensemble, quelque relevée
qu'elle puisse être. Qu'il est vrai que ce n'est que depuis plusieurs
siècles que les états généraux en sont réduits en ces termes, surtout
quant aux impositions, et qu'il ne l'est pas moins que jamais les
parlements n'ont eu plus d'autorité que celle dont ils jouissent. Je
m'étendrais trop si je voulais traiter ici de certaines formes
nécessaires pour les affaires majeures qui regardent la couronne même,
ou les premiers particuliers de l'État. Ce sont d'autres sortes de
formes majeures comme les affaires majeures qui les exigent, et dont
Louis XIV même, qui a porté son autorité bien au delà de ce qu'ont fait
tous ses prédécesseurs, n'a pas cru se devoir départir, ni de son aveu
même pouvoir les omettre. Toujours demeure-t-il constant que l'autorité
de nos rois a laissé subsister ce qui vient d'être exposé.

En Castille, rien moins\,: les cortès ou états généraux ne s'y
assemblent plus par ordre des rois que pour prêter les serments que le
roi veut recevoir, ou qu'il veut faire prêter au successeur de sa
couronne. Il ne s'y agit de rien de plus depuis des siècles. La
cérémonie et la durée des cortès ne tient pas plus d'une matinée. Pour
le reste il y a un tribunal qui s'appelle le conseil de Castille, dont
la juridiction supérieure s'étend sur toutes les provinces soumises à
cette couronne, qui n'ont chez elles que des tribunaux subalternes qui y
ressortissent, avec une dépendance bien plus soumise que n'en ont les
nôtres à nos parlements. Ce conseil de Castille est tout à la fois ce
que nous connaissons ici sous le nom de parlement et du conseil des
parties\footnote{Voy., sur le \emph{conseil des parties}, t. Ier,
  p.~445.}\,; et le chef de ce tribunal, qui n'a point de collègues
comme les présidents à mortier à l'égard des premiers présidents ici,
est tout à la fois ce que nous connaissons ici sous le nom de chancelier
et de premier président, du prodigieux état duquel j'ai dit un mot en
parlant de la dignité des grands d'Espagne. C'est donc lui qui, avec ce
conseil, jugé en dernier ressort tout ce qui dépend de la couronne de
Castille, et qui de plus est le supérieur immédiat en de certaines
choses avec le conseil, seul en plusieurs autres, de tous les membres,
non seulement de tous les tribunaux inférieurs de la Castille, outre
qu'il l'est avec le conseil de ces tribunaux chacun en corps, mais il
l'est de tous les régidors et de tous les corrégidors, qui ont à la fois
toutes les fonctions des intendants des provinces, des lieutenants
civils, criminels, et de police et de prévôts des marchands, comme nous
parlons ici\footnote{Voy., note à la fin du t. III, p.~442.}.

Mais toute cette puissance et toute cette autorité disparaît chaque
semaine devant celle du roi. Toutes les semaines le conseil de Castille
en corps vient chez le roi, son chef à sa tête, dans une pièce de son
palais destinée à cela, à jour et heure marquée. Le roi s'y rend peu
après et y entre seul. Il y est reçu à genoux de tout le corps qu'il
fait asseoir sur des bancs nus et couvrir, après qu'il est lui-même
assis et couvert dans son fauteuil sous un dais. En retour à droite, sur
le bout du banc le plus près de lui, est le chef de ce corps, ayant à
son côté celui des conseillers choisi pour faire ce jour-là rapport de
ce que le conseil a jugé depuis la dernière fois qu'ils sont venus chez
le roi\footnote{Passage omis par les précédents éditeurs depuis \emph{Il
  y est reçu}.}. Il a les sentences à ses pieds, dans un sac, et il en
explique sommairement le fait, les raisons des parties, et celles qui
ont déterminé le jugement. Le roi, qui les approuve d'ordinaire, signe
la sentence qui ne devient arrêt qu'en ce moment\,; sinon il ordonne au
conseil de la revoir et de lui en rendre compte une autre fois, ou il
renvoie l'affaire à des commissaires qu'il choisit, ou à un autre
conseil, comme celui des finances, des Indes ou autre pareil.
Quelquefois il casse la sentence, rarement, mais il le peut, et cela est
quelquefois arrivé\,; il rend de son seul avis un arrêt tout contraire,
qui s'écrit là sur-le-champ, et qu'il signe. Il n'entre point dans tout
ce qui est procédure ou interlocutoire\footnote{On appelait jugement
  \emph{interlocutoire}, dans l'ancien droit français, un arrêt qui ne
  décidait point la question\,; le tribunal se bornait à ordonner une
  plus ample information sur quelques points.}, à moins qu'il n'ait reçu
des plaintes, et qu'il veuille en être informé, mais seulement dans les
décisions. Ainsi il est vrai de dire que ce conseil de Castille, si
suprême, n'a que voix consultative, et de soi ne rend que des sentences,
et que c'est le roi seul qui juge et décide de tous les procès et les
questions.

Après cette séance, qui ne va guère à deux heures, le roi se lève\,;
tous se mettent à genoux, et il sort de la pièce où il les laisse. Dans
la joignante, il trouve ses grands officiers et sa cour. Le chef du
conseil de Castille le suit, je dis chef parce que c'est ou un président
ou un gouverneur, et j'en ai expliqué la différence en parlant de la
dignité des grands\,; ce chef, dis-je, le suit. Le roi s'arrête dans une
des pièces de son appartement où il trouve un fauteuil, une table avec
une écritoire et du papier à côté\,; et vis-à-vis, tout près, un petit
banc nu de bois, fort court. L'accompagne ment du roi passe outre et
l'attend dans la pièce voisine. Il se met dans le fauteuil, et le chef
sur ce petit banc nu, et là il lui rend compte du conseil même et de
tout ce qui passe par lui seul\,; sur quoi il reçoit ses ordres. Cela
fait, il retourne d'où il était venu, et le roi en même temps passe
outre, trouve sa cour dans la pièce voisine, qui le suit jusqu'à la
porte de son cabinet\footnote{Passage omis par les précédents éditeurs
  depuis \emph{Après cette séance}.}.

Ce conseil enregistre les mêmes choses que fait ici le parlement, mais
sans jamais y faire obstacle, et, s'il y a quelque remontrance ou
observation à faire, il prend son temps lorsqu'il va au palais. Alors il
s'explique ou par le chef ou par un des conseillers, quelquefois après
la séance par ce chef tête à tête et de quelque façon que ce soit\,; la
volonté du roi entendue, il est obéi sans délai, et sans plus lui en
parler. Il consulte assez souvent le conseil avant de faire certaines
choses, avec liberté d'en suivre après l'avis ou non. Il est donc
difficile de pousser plus loin et l'effet et l'apparence du despotisme.

En Aragon, c'est tout le contraire pour cette couronne et pour toutes
les provinces qui en dépendent. Les lois qui y sont en vigueur ne
peuvent recevoir d'atteinte, le roi ne peut toucher à aucun privilège
public ni particulier. Les états généraux y sont les maîtres des
impositions dans toutes leurs parties, qui refusent presque toujours ce
qu'on y voudrait ou innover ou augmenter, et ils ont la même délicatesse
sur tout ce qui est édits et ordonnances, qui ne peuvent être exécutés
non seulement sans leur consentement, mais sans leur ordre. Le tribunal
suprême réside à Saragosse, qui est pour l'Aragon et tout ce qui en
dépend, comme est le conseil de Castille dans ce royaume et ses
dépendances. Le chef de ce tribunal qui, comme en Castille, est un
grand, et peut aussi être un homme de robe, avec moins de consistance
alors, est tout un autre personnage que le président ou le gouverneur du
conseil de Castille. Il se nomme, non le justicier, niais le
\emph{justice}, comme étant lui-même la souveraine justice. Il ne peut
être ni déposé, ni suspendu, ni écorné en quoi que ce soit. Il préside
également au tribunal suprême et aux états quand ils sont assemblés, et
qui quelquefois s'assemblent ou par lui ou d'eux-mêmes, sans que le roi
puisse l'empêcher. C'est dans ces états assemblés que le nouveau roi
prête le serment entre les mains du justice, qui lui dit, étant assis et
couvert, cette formule mot à mot et lentement, tout haut, en sorte que
toute l'assemblée l'entende\,: \emph{Nous qui valons autant que vous,
vous acceptons pour notre roi, à condition du maintien de tous nos
droits, lois et prérogatives\,; sinon, non}. Voilà un étrange compliment
à recevoir pour une tête couronnée\,; et, en Aragon, ils ont toujours
tenu parole tant qu'ils ont pu, et l'ont pu presque toujours. Ce
justice, en absence des états, les représente seul, et fait, en partie
seul, en partie avec le conseil, ce que feraient les états s'ils étaient
assemblés, auxquels il en doit compte, et leur est soumis en tout. Il a,
comme les états, une grande jalousie d'empêcher que le roi n'étende son
autorité au préjudice de la leur en quoi que ce soit, et de part et
d'autre en petit, ils ressemblent fort, quoique dans une autre forme, au
roi et au parlement d'Angleterre. C'est aussi ce qui a si souvent armé
l'Aragon, la Catalogne, etc., contre ses princes et c'est ce que le roi
d'Espagne prit cette année son temps d'abolir.

Il éteignit la dignité et toutes les fonctions de ce fâcheux justice, il
abolit les états, il supprima tous les droits et prérogatives, il cassa
toutes les lois, il changea le tribunal suprême, il asservit l'Aragon et
toutes les provinces qui en dépendent, les mit en tout et partout sur le
pied de la Castille, il y étendit les lois de ce royaume, et il abrogea
tout ce qui y pouvait être contraire. Ce fut un grand et utile coup
frappé bien à propos, et qui mit toutes ces provinces au désespoir et en
furie. Le bonheur de l'issue des armes a soutenu ce qu'elles avaient
tant aidé à établir. L'Aragon, la Catalogne et toutes les provinces
dépendantes de cette couronne ont fait l'impossible pour alléger au
moins ce joug. Philippe V est demeuré, avec grande raison, inébranlable,
et les choses sont demeurées jusqu'à présent dans la forme où il les mit
dans ce temps-là.

Le parti était pris dès l'hiver de n'essayer point de rentrer en Italie.
Médavy y était resté avec les troupes que M. le duc d'Orléans marchant
avec son armée à Turin lui avait laissées en Lombardie, et avec
lesquelles il remporta une victoire en même temps que se donna la
bataille de Turin, qui en aurait réparé les malheurs, si, comme M. le
duc d'Orléans le voulut, il avait mené son armée en Italie, au lieu de
la ramener dans les Alpes et dans le Dauphiné. Médavy se maintint avec
ses troupes sans que les ennemis osassent l'attaquer\,; il tenait
Mantoue et quantité d'autres places.

Ne renvoyant point de troupes en Italie, il restait deux partis à
prendre, que Médavy proposa tous deux, et du succès de celui des deux
qu'on voudrait prendre il répondit. Le premier, et celui que Médavy
appuyait le plus, était celui de se cantonner en Lombardie, d'y
abandonner à leurs propres forces les places qui ne s'y pourraient
couvrir, de conserver les principales possibles, surtout Mantoue, de les
bien munir toutes, et de se tenir sur la défensive en Lombardie, où la
subsistance ne pouvait manquer sans aucun autre secours, et fatiguer les
ennemis par les courses de nos garnisons, et par la nécessité des
sièges, les amuser ainsi en attendant les événements, et les empêcher de
songer à venir nous attaquer chez nous, délivrés de toute guerre en
Italie.

L'autre parti était de marcher avec sa petite armée, par les pays
vénitiens et ecclésiastiques, très neufs et très abondants, droit au
royaume de Naples, qui se maintenait encore, mais qui ne pouvait que
tomber bientôt s'il n'était secouru en lui-même\,; ou par la diversion
d'Italie, si on était en état et en volonté d'y en tenter quelqu'une.
C'était au moins conserver à l'Espagne Naples et Sicile, et ne pas tout
perdre à la fois en ne prenant aucun de ces deux partis, dont chacun des
deux était très praticable. Mais il était écrit que les ténèbres dont
nous étions frappés s'épaissiraient de plus en plus, et que le nombre et
l'énormité de nos fautes entassées les unes sur les autres en Italie, la
campagne dernière, seraient comblées par celle de son entier abandon.

Pour ce dernier parti, on eut peur d'offenser un pape faible et une
république infidèle qui avait toujours favorisé ouvertement les
Impériaux, et un pape qui, bien que de mauvaise grâce, n'avait osé
résister à leurs volontés. Ces deux si médiocres puissances sentaient
bien alors la faute qu'elles avaient faite, et trouvaient les Impériaux
devenus de beaucoup trop forts\,; mais cette même raison les tenait en
crainte, je n'oserais dire et nous avec eux. Le trajet était court,
facile, sans obstacle quelconque à appréhender, et toujours dans
l'abondance, et Naples et Sicile étaient sauvées. On en eût été quitte
pour des cris de politique et pour des excuses de même sorte. On s'en
fit des monstres\,; on aima mieux regarder tout d'un coup Naples et
Sicile comme perdues.

L'autre parti fut considéré comme trop hasardeux.

On fit à l'égard de Médavy et de ses troupes coupées d'avec la France,
comme ces mères, tendres jusqu'à la sottise, qui ne veulent pas laisser
aller leurs enfants faire ou essayer fortune par des voyages de long
cours, dans la crainte de ne les revoir jamais. On oublia la conduite
des grands rois et des grands capitaines qui, après les plus désespérés
revers, se sont roidis à se soutenir contre la fortune, et par un léger
levain sont parvenus, à force de courage, d'art, de savoir se passer, se
cantonner, se maintenir, à changer la face des affaires et à en sortir
heureusement et glorieusement.

Vaudemont avait le commandement d'honneur\,; Médavy, qui portait tout le
poids, l'avait en effet. Le Milanais ne rapportait plus à Vaudemont
l'autorité ni l'argent qui le rendaient grand, depuis le malheur de
Turin. Il avait des sommes immenses qu'il ne voulait pas hasarder. On a
vu ici ses perfides manèges du temps de Catinat et de Villeroy. Il avait
mieux couvert son jeu pendant celui de Vendôme, en qui toute la
confiance et l'autorité était passée et avec lequel il avait
principalement songé à se lier. La mort de son fils unique semblait
avoir rompu ses chaînes\,; M. le duc d'Orléans, qui avait eu les yeux
fort ouverts sur sa conduite dans le peu qu'il eut à l'examiner, me dit
au retour en avoir été fort content.

Pour moi, j'avais toujours sur le cœur ce chiffre fatal qu'il nia avoir,
et qu'il m'a toujours paru impossible qu'il n'eût pas, dont j'ai parlé
(t. V, p. 229), et qui a été si funeste. Je ne sais si, quand il serait
enfin devenu fidèle, un gouvernement si mutilé et le commandement
apparent de troupes abandonnées ne lui parut pas une charge trop
pesante, et, supposé ses anciennes liaisons, s'il ne se défia pas de ses
souplesses dans des conjonctures si délicates de cette décadence. Il
sentait sa partie si bien faite en France, qu'il s'en promettait tout,
et la suite a montré qu'il ne se trompait pas, et qu'il n'y a manqué que
des chimères insoutenables. Il était dans la première considération du
roi\,; ses nièces et le maréchal de Villeroy avant sa chute lui avaient
acquis Chamillart sans mesure. Monseigneur, tel qu'il était, mené par
ses nièces, était à lui. M\textsuperscript{me} de Maintenon, il la
tenait par Villeroy avant sa disgrâce, qui n'y fut même jamais avec
elle, par Chamillart, et par le ricochet de Vendôme qui faisait agir M.
du Maine, auprès d'elle. Enfin il avait le gros du monde par ces
cabales, par toute la maison de Lorraine, par tout ce qui avait servi en
Italie, comblé par lui de politesse, gorgé d'argent du Milanais, et
charmé de la splendeur, car c'est peu dire de la magnificence, dont il
vivait.

Il appuya donc si faiblement tous ces deux partis, qu'il les décrédita
par cela même qu'il avait un intérêt apparent de désirer qu'on prît
celui de soutenir en Lombardie, qui lui en conservait le commandement et
ce qui restait de son gouvernement du Milanais\,; et son bonheur, aidé
de sa cabale, fut tel, que le roi lui sut le meilleur gré du monde de
cette faiblesse d'appuyer, comme étant plus sincère qu'intéressé. Enfin,
dans le besoin où on était de troupes, bonnes et vieilles, on ne
considéra pas où elles seraient le plus utiles pour occuper l'ennemi et
l'éloigner de nos frontières, on ne se frappa que de l'idée de sauver
celles-ci et de les employer dans nos armées.

Vaudemont fut donc chargé de négocier, de concert avec Médavy, le libre
retour de nos troupes et de leur suite, leur retraite en Savoie, la
route qu'elles tiendraient, et tout ce qui regardait leur marche et leur
subsistance en payant, et en abandonnant tout ce que nous tenions en
Italie. On peut juger s'il eut peine à être écouté et à conclure un
traité si honteux pour la France, et si utile et si glorieux à ses
ennemis. Tout fut donc arrêté de la sorte, et le général Patay fut livré
pour otage à Médavy pour marcher avec lui jusqu'à ce que toutes nos
troupes et leur suite fût arrivée en Savoie. C'est ce que Médavy eut la
douleur de recevoir ordre d'exécuter.

Tout y fut fait assez à la hâte pour ne se donner pas le loisir d'en
avertir le malheureux duc de Mantoue à temps, dont les places, l'État et
Mantoue même furent remis aux troupes de l'empereur. Le duc de Mantoue
se retira en diligence à Venise avec ce qu'il put emporter de meilleur,
et envoya sa femme, dont il n'eut point d'enfants, en Suisse pour ne se
revoir jamais. Le dessein était qu'elle allât en Lorraine\,: rien
n'était plus naturel\,; mais M. de Lorraine était trop à l'empereur pour
oser recevoir chez lui sans la permission de ce prince l'épouse d'un
allié de la France, dépouillé à ce titre, et pour avoir si longtemps mis
l'empereur dans le plus grand embarras par avoir reçu les François dans
Mantoue.

Louis XIII avait conservé, et deux fois rétabli à main armée dans les
États de Mantoue et de Montferrat, le père et le grand-père de ce duc de
Mantoue, et la première des deux en personne, où sa capacité militaire
et sa valeur personnelle qui le couvrit de gloire, jointes à la fidélité
de sa protection dans des temps si difficiles, lui mérita toute celle
des héros au célèbre pas de Suse vis-à-vis du fameux Charles-Emmanuel et
de l'armée autrichienne, comme je l'ai plus amplement remarqué (t. Ier,
p.~63). Ce ne fut donc pas une satisfaction légère pour une maison aussi
implacable que la maison d'Autriche s'est toujours piquée si utilement
de l'être, de se voir enfin maîtresse du duché et de la ville de Mantoue
et du Montferrat, et de faire sentir au souverain dépouillé tout le
poids de sa vengeance, et à la France celui de sa faiblesse, dont les
alliés, chassés et proscrits par l'empereur en criminels, se trouvaient
partout réduits à chercher de lieu en lieu des asiles, et à subsister de
ce que la France, qui n'avait pu les soutenir, leur pouvait donner,
contraste étrange entre Louis XIII et Louis XIV. Crémone, Valence, en un
mot tout ce que nous tenions en Italie fut livré aux Impériaux, qui
furent si jaloux de cette gloire qu'ils ne voulurent jamais souffrir que
ce que nous tenions de places du duc de Savoie lui fût immédiatement
remis, mais qu'ils s'opiniâtrèrent à les recevoir eux-mêmes pour que ce
prince, qui en cria bien haut, ne les pût recevoir que de leurs mains.

Sur la fin d'avril, Vaudemont et Médavy arrivèrent à Suse avec près de
vingt mille hommes tant des troupes du roi que de celles du roi
d'Espagne. Le 9 mai, c'est-à-dire le lendemain du détail de la bataille
d'Almanza apporté par Bockley, Médavy arriva à Marly, et vint saluer le
roi dans ses jardins, dont il fut très bien reçu, après quoi il le
suivit chez M\textsuperscript{me} de Maintenon où il demeura une heure à
lui rendre compte d'un pays et d'un retour qu'il devait entendre avec
une grande peine. Le gouvernement de Nivernais venait de vaquer tout à
propos\,; le roi le lui donna sans qu'il le demandât, quoiqu'il eût
celui de Dunkerque, mais il l'avait acheté. On le fit repartir au bout
d'un mois pour aller commander en chef en Savoie et en Dauphiné, avec
deux lieutenants généraux et deux maréchaux de camp sous lui, et le
traitement de général d'armée, quoique aux ordres du maréchal de Tessé
qui y était déjà. Il eut de plus douze mille livres de pension. Le roi
lui dit que c'était en attendant mieux, parce qu'il avait cru le
gouvernement de Nivernais de trente-huit mille livres de rente, et qu'il
se trouvait n'en valoir que douze mille. Ces grâces, contre l'ordinaire,
ne furent enviées de personne, et chacun y applaudit avec grande raison.

Le prince de Vaudemont ne tarda pas après Médavy. Il s'arrêta dans une
maison à quelques lieues de Paris, qu'un fermier général lui prêta, où
M\textsuperscript{lle} de Lislebonne et M\textsuperscript{me} d'Espinoy
ses nièces l'allèrent attendre, d'où elles le menèrent loger chez
M\textsuperscript{me} de Lislebonne leur mère et sa sœur, près des
filles de Sainte-Marie, de la rue Saint-Antoine, à l'hôtel de Mayenne,
maison précieuse aux Lorrains pour avoir appartenu au fameux chef de la
Ligue dont ils lui ont chèrement conservé le nom, les armes et
l'inscription au-dessus de la porte, et où est une chambre dans laquelle
furent enfantées les dernières horreurs de la Ligue, l'assassinat
d'Henri III et le projet de l'élection solidaire de l'infante d'Espagne
et du fils du duc de Mayenne pour roi et reine de France, en les mariant
et en excluant à jamais Henri IV et toute la maison de Bourbon. Cette
chambre s'appelle encore aujourd'hui la \emph{chambre de la Ligue}, dont
rien n'a été changé depuis par le respect et l'amour qu'on lui porte. Ce
fut là que, sous prétexte de repos, M. de Vaudemont acheva de se
concerter avec sa sœur et ses nièces.

Il y reçut quelques familiers, s'en alla coucher à l'Étang une nuit, et
le lendemain il salua le roi avant dîner à Marly, passant de chez
M\textsuperscript{me} de Maintenon chez lui après sa messe. Le roi le
fit entrer dans son cabinet et le reçut comme un homme qui avait rendu à
lui et au roi son petit-fils les plus grands services, et qui, en
dernier lieu, lui avait sauvé vingt mille hommes par le traité qu'il
avait fait avec le prince Eugène, pour les ramener en sûreté, en lui
livrant toute l'Italie. On lui avait réservé un logement à Marly et on
lui prêta celui du maréchal de Tessé à Versailles, lors absent, comme je
l'ai dit ailleurs.

Il faut maintenant se souvenir de ce que j'ai dit en divers endroits de
ce bâtard de Charles IV, duc de Lorraine, dont il avait si parfaitement
hérité l'esprit, l'artifice, la fourberie et l'infidélité, et en qui de
plus on ne doutait pas que l'âme du fameux Protée n'eût passé, si on
pouvait s'arrêter aux fables et à la folie de la métempsycose. Il faut
aussi avoir présent ce que j'ai dit (t. III, p.~195 et suiv.) de ses
nièces et de leur position également solide et brillante à la cour, de
leur union entre elles deux et leur habile mère, c'est peu dire, allons,
ce n'est pas trop, jusqu'à l'identité, en laquelle Vaudemont fut en
quart. Outre l'amitié soigneusement cultivée par le commerce de lettres,
soutenue par les grandes vues, l'intérêt de cette union était double,
celui de la grandeur, du crédit, de la considération, et celui de
l'intérêt depuis que, par la mort du fils unique de Vaudemont, ses
nièces étaient devenues ses uniques héritières. Ce fut donc à tant de
grands objets tout à la fois qu'ils butèrent.

J'ai expliqué comment ils se comptèrent très assurés de Chamillart, de
M. du Maine, de M\textsuperscript{me} de Maintenon, de Monseigneur. Ils
pouvaient aussi être certains de M\textsuperscript{lle} Choin et de
M\textsuperscript{me} la Duchesse, et de ce qui, en hommes, approchait
le plus confidemment de Monseigneur. Tessé leur avait préparé les voies
auprès de M\textsuperscript{me} la duchesse de Bourgogne, et ne leur
avait rien laissé ignorer de ce qui les pouvait instruire de ce côté-là.
M. de Vendôme était à eux et le groupe de la maison de Lorraine. Le roi
anciennement prévenu par le maréchal de Villeroy, du temps de sa grande
faveur, et entretenu depuis dans la même opinion par les puissants
appuis que je viens de nommer, ils avaient de plus la grâce de la
nouveauté, et ce lustre étranger dont le Français s'éblouit jusqu'à
l'ivresse, et qui leur réussit au delà de ce qu'ils pouvaient espérer.

Le roi fit à Vaudemont les honneurs de Marly comme il s'était plu à les
faire à la princesse des Ursins. Il avait affaire à un homme qui savait
répondre, s'exclamer, admirer, tantôt grossièrement, tantôt avec
délicatesse, par un même artifice. Il ordonna au premier écuyer une
calèche et des relais pour que Vaudemont le suivît à la chasse, et
{[}lui dit{]} de l'y accompagner. Il arrêta souvent sa calèche à la
sienne pendant les chasses, en un mot, ce fut un second tome de
M\textsuperscript{me} des Ursins. Tout cela était beau, mais il en
fallait faire usage pour le rang et pour les biens.

M\textsuperscript{me} de Lislebonne avait l'esprit habile, et tout
tourné pour faire un grand personnage dans sa maison, si elle eût vécu
au temps de la Ligue. Sa fille aînée avait un air tranquille et
indifférent au dehors, avec beaucoup de politesse, mais choisie et
mesurée, et avec les pensées les plus hautes, les plus vastes et tout le
discernement et la connaissance nécessaire pour ne les rendre pas
châteaux en Espagne, avait naturellement une grande hauteur, de la
droiture, savait aimer et haïr, moins de manège que de ménagements et de
suite, infatigable avec beaucoup d'esprit, sans bassesse, sans
souplesse, mais maîtresse d'elle-même pour se rabaisser quand il `était
à propos, et assez d'esprit pour le faire même avec dignité, et en faire
sentir le prix à ceux dont elle avait besoin, sans les blesser, et se
les rendre favorables.

Sa sœur avec peu d'esprit, souple, et assez souvent basse, non faute de
cœur et de hauteur, mais d'esprit, l'avait tout tourné au manège avec
une politesse moins ménagée que sa sœur, et un air de bonté qui faisait
aisément des dupes. Elle savait servir et s'attacher des amis.

Leur vertu et leur figure étaient d'ailleurs imposantes\,; l'aînée, très
simplement mise et sans beauté, inspirait du respect\,; la cadette,
belle et gracieuse, attirait\,; toutes deux fort grandes et fort bien
faites\,; mais\,; à qui avait du nez, l'odeur de la Ligue leur sortait
par les pores\,; toutes deux point méchantes pour l'être, et se
conduisant au contraire de manière à en ôter le soupçon, mais, lorsqu'il
y allait de leurs vues et de leur intérêt, terribles.

Outre ces dispositions naturelles, elles en avaient bien appris de deux
personnes avec lesquelles elles furent intimement unies, les deux de la
cour les plus propres à instruire par leur expérience et leur genre
d'esprit. M\textsuperscript{lle} de Lislebonne et le chevalier de
Lorraine étaient de toute leur vie tellement un, qu'on ne doutait pas
qu'ils ne fussent mariés. On a vu en son lieu quel homme était le
chevalier de Lorraine. Il était, par conséquent, dans la même union avec
M\textsuperscript{me} d'Espinoy. C'est ce qui les avait si fort liés
avec le maréchal de Villeroy, l'ami intime et très humble du chevalier
de Lorraine, et c'était par le maréchal de Villeroy que le roi si jaloux
de tout ce qui approchait Monseigneur, non seulement n'en avait point
conçu contre ces deux sœurs, mais avait pris confiance en elles, était
bien aise de ce commerce si intime de son fils avec elles, et leur
manquait en tout une considération si distinguée, qui dura la même après
la mort de Monseigneur\,; d'où il faut conclure que les deux sœurs, au
moins la cadette, firent toute leur vie auprès de Monseigneur le même
personnage secret à l'égard du roi, que le chevalier de Lorraine se
trouva si bien toute sa vie de faire auprès de Monsieur, qu'il gouverna
toujours. C'était un exemple qu'il était à portée de leur confier, et
elles de suivre, et dont le maréchal de Villeroy put être aussi
quelquefois le canal.

Il les avait mises de même dans la confiance de M\textsuperscript{me} de
Maintenons dont j'avancerai ici un trait étrange qui n'arriva que
depuis, que je sus le lendemain du jour qu'il fut découvert, et qui
montrera combien avant était cette confiance. M\textsuperscript{me} la
duchesse de Bourgogne s'était acquis une telle familiarité avec le roi
et avec M\textsuperscript{me} de Maintenon, que tout en leur présence
elle furetait leurs papiers, les lisait, et ouvrait jusqu'à leurs
lettres. Cela s'était tourné en badinage et en habitude. Un jour, étant
chez M\textsuperscript{me} de Maintenon, le roi n'y étant pas, elle se
mit à paperasser sur un bureau, tout debout, à quelques pas d'où
M\textsuperscript{me} de Maintenon était assise, qui lui cria plus
sérieusement qu'à l'ordinaire de laisser là ses papiers. Cela même
aiguisa la curiosité de la princesse qui, toujours bouffonnant mais
allant son train, trouva une lettre ouverte, mais ployée entre les
papiers, où elle vit son nom. Surprise, elle lut une demi ligne, tourna
le feuillet, et vit la signature de M\textsuperscript{me} d'Espinoy. À
cette demi ligne, et plus encore à la signature, elle rougit et devint
interdite. M\textsuperscript{me} de Maintenon qui la voyait faire, et
qui apparemment ne l'en empêchait pas, comme elle l'aurait pu si
absolument elle l'eût voulu, ne fut pas apparemment fâchée de la
découverte. «\,Qu'avez-vous donc, mignonne, lui dit-elle, et comme vous
voilà\,! Qu'avez-vous donc vu\,?» Voilà la princesse encore plus
embarrassée. Comme elle ne répondait point, M\textsuperscript{me} de
Maintenon se leva et s'approcha d'elle comme pour voir ce qu'elle avait
trouvé. Alors la princesse lui montra la signature.
M\textsuperscript{me} de Maintenon lui dit\,: «\,Eh bien\,! c'est une
lettre que M\textsuperscript{me} d'Espinoy m'écrit. Voilà ce que c'est
que d'être si curieuse\,; on trouve quelquefois ce qu'on ne voudrait
pas\,;» puis prenant un autre ton\,: «\,Puisque vous l'avez vue, madame,
ajouta-t-elle, voyez-la tout entière, et si vous êtes sage,
profitez-en\,;» et la força de la lire d'un bout à l'autre. C'était un
compte que M\textsuperscript{me} d'Espinoy rendait à
M\textsuperscript{me} de Maintenon des quatre ou cinq dernières journées
de M\textsuperscript{me} la duchesse de Bourgogne, mot à mot, lieu par
lieu, heure par heure, aussi exact que si elle, qui n'en approchait
guère, ne l'eût pas quittée de vue\,; dans lequel il était fort question
de Nangis et de beaucoup de manèges et d'imprudences. Tout y était
nommé, et ce qui est plus surprenant qu'une telle instruction même,
c'était de signer une lettre de cette nature, et pour
M\textsuperscript{me} de Maintenon de ne l'avoir pas brûlée
sur-le-champ, ou du moins enfermée. La pauvre princesse pensa s'évanouir
et devint de toutes les couleurs. M\textsuperscript{me} de Maintenon lui
fit une forte vesperie\footnote{Réprimande.}, lui fit voir que ce
qu'elle croyait caché était vu par toute la cour, et lui en fit sentir
les conséquences. Sans doute qu'elle lui en dit bien davantage, mais
M\textsuperscript{me} de Maintenon lui avoua que lorsqu'elle lui avait
parlé plusieurs fois, c'était par science, et qu'il était vrai que
M\textsuperscript{me} d'Espinoy et d'autres encore étaient chargées par
elle de suivre secrètement sa conduite, et de lui en rendre un compte
exact et fréquent.

Au partir d'un lieu si fâcheux, la princesse n'eut rien de plus pressé
que de gagner son cabinet, et que d'y appeler M\textsuperscript{me} de
Nogaret qu'elle appelait toujours sa petite bonne et son puits, et de
lui conter toute sa déconvenue, fondant en larmes, et dans la furie
contre M\textsuperscript{me} d'Espinoy qu'il est aisé d'imaginer.
M\textsuperscript{me} de Nogaret la laissa s'exhaler, puis lui remontra
ce qu'elle jugeait à propos sur le fond de la lettre, mais surtout elle
lui conseilla très fortement de se garder sur toutes choses de rien
marquer sur M\textsuperscript{me} d'Espinoy, et lui représenta qu'elle
se perdrait si elle lui témoignait moins de familiarité et de
considération qu'à l'ordinaire. Le conseil était infiniment salutaire,
mais difficile à pratiquer. Cependant M\textsuperscript{me} la duchesse
de Bourgogne, qui avait confiance en l'esprit et en la science du monde
et de la cour de M\textsuperscript{me} de Nogaret, en quoi elle avait
grande raison, la crut, et se conduisit toujours avec
M\textsuperscript{me} d'Espinoy de même qu'auparavant, en sorte qu'elle
n'a jamais pu être soupçonnée d'en avoir été découverte. Le lendemain
M\textsuperscript{me} de Nogaret, avec qui nous étions intimement
M\textsuperscript{me} de Saint-Simon et moi, nous le conta à tous deux
précisément comme je viens de l'écrire.

Ce trait honteux et affreux, surtout pour une personne de cet état et de
cette naissance, montre à découvert jusqu'à quel point, et par quels
intimes endroits, les deux sœurs, celle-ci surtout, tenaient directement
au roi et à M\textsuperscript{me} de Maintenon, et tout ce qu'elles s'en
pouvaient promettre, surtout avec l'infatuation dont
M\textsuperscript{me} de Maintenon ne se cachait pas pour les
préférences et le rang de la maison de Lorraine.

Du côté de Monseigneur, leur règne sur son esprit était sans trouble.
M\textsuperscript{lle} Choin, sa Maintenon de tous points, excepté le
mariage, leur était dévouée sans réserve. Elle n'oubliait pas que
M\textsuperscript{me} de Lislebonne et ses filles devant tout, leur
subsistance, leur introduction dans l'amitié de Monseigneur, le
commencement de leur considération à M\textsuperscript{me} la princesse
de Conti, elles n'avaient pas balancé de la lui sacrifier sans y avoir
été conduites par aucun mécontentement, mais par la seule connaissance
du goût de Monseigneur, et l'utilité d'avoir seules d'abord avec lui la
confiance de leur commerce après la sortie de M\textsuperscript{lle}
Choin de la cour. Elle avait été trop longtemps témoin aussi de cette
confiance et de cette amitié de Monseigneur pour ces deux sœurs, chez
qui il allait presque tous les matins passer en tiers une heure ou deux
avec elles, pour se heurter à elles, pour ne leur demeurer intimement
unies, et M\textsuperscript{me} la Duchesse dont l'humeur égale et gaie,
et la santé toujours parfaite la rendit toujours la reine des plaisirs,
chez qui Monseigneur s'était réfugié, chassé par le mésaise que
l'aventure de la Choin d'abord, l'ennui ensuite et l'humeur de
M\textsuperscript{me} la princesse de Conti avait dérangé de chez elle,
et réduit aux simples bienséances, M\textsuperscript{me} la Duchesse,
dis-je, qui n'avait ni humeur ni jalousie, et à qui cette habitude et
cette familiarité de Monseigneur à venir chez elle n'était pas
indifférente pour le présent contre les fougues et les sorties de M. le
Duc et de M. le Prince même, et moins encore pour le futur, n'avait
garde de choquer ces trois personnes, les plus confidentes et les plus
anciennes amies de Monseigneur.

Toutes quatre étaient donc, à l'égard de ce prince et de beaucoup
d'autres choses communes entre elles, dans une intelligence qui ne se
refroidit jamais en rien, s'aidant en tout avec un parfait concert les
unes les autres, quittes après la mort du roi, si Monseigneur eût
survécu, à se supplanter réciproquement pour demeurer les maîtresses
sans dépendance de personne, mais en attendant unies au dernier point,
et tenant sous leur joug commun le peu d'hommes en qui le goût de
Monseigneur, ou leur industrie auprès de lui, pouvait avoir quelques
suites.

L'autre personne des instructions de qui M\textsuperscript{lle} de
Lislebonne et M\textsuperscript{me} d'Espinoy tirèrent de grands secours
fut l'habile M\textsuperscript{me} de Soubise. Elle était sœur de la
princesse d'Espinoy, belle-mère de celle-ci, et dans toute l'union
possible\,; avec plus d'esprit qu'elle n'en paraissait, soutenu de tout
ce que l'art du manège, de l'intrigue et de la beauté, aiguisé des
besoins de l'ambition la plus vaste et la plus cachée, et soutenu de
tout ce que la politique, la fausseté, l'artifice, ont de plus profond.
Ses appas l'avaient initiée dans la connaissance la plus intime de
l'intérieur du roi, dans laquelle elle était sans cesse entretenue par
le commerce qui s'était conservé entre eux, et dont elle sut tirer de si
utiles partis. Livrée au roi par ambition, tant que la dévotion rie
l'arrêta pas, contente de la faveur, dès que cette dévotion la répudia,
elle sut mettre le roi à son aise, et se servir de cette dévotion même
pour maintenir son crédit, sous prétexte de ne pas ouvrir les yeux à son
mari, qui les avait si volontairement fermés, par la différence qu'il en
sentirait et par l'époque de cette différence.

Elle sut gagner M\textsuperscript{me} de Maintenon, et se servir jusque
de sa jalousie du goût que le roi lui conservait, en lui offrant une
capitulation dans laquelle la nouvelle épouse se crut heureuse d'entrer.
Elle fut de la part de M\textsuperscript{me} de Soubise de ne jamais
voir le roi en particulier que pour affaire dont M\textsuperscript{me}
de Maintenon aurait connaissance\,; d'éviter même ces particuliers,
quand les billets pourraient y suppléer\,; de le voir même à la porte de
son cabinet, quand elle n'aurait qu'un mot court à dire\,; de n'aller
presque jamais à Marly, pour éviter toute occasion\,; de choisir les
voyages les plus courts, et de n'y aller qu'autant qu'il serait
nécessaire pour empêcher le monde d'en parler\,; de n'être jamais
d'aucune des parties particulières du roi, ni même des fêtes de la cour
que lorsque, étant fort étendues, ce serait une singularité de n'en être
pas\,; enfin, que demeurant souvent à Versailles et à Fontainebleau où
ses affaires, sa famille, sa coutume qu'il ne fallait pas changer aux
yeux de son mari, la demandaient, elle n'y chercherait jamais à
rencontrer le roi, mais se contenterait, comme toutes les autres dames,
de lui faire sa cour à son souper assez souvent (où même, ni au sortir
de table, elle trouvait fort à propos que le roi ne lui parlât point,
non plus qu'il avait accoutumé de parler aux autres). De son côté,
M\textsuperscript{me} de Maintenon lui promit service sûr, fidèle,
ardent, exact dans tout ce qu'elle pourrait souhaiter du roi pour sa
famille et pour elle-même\,; et de part et d'autre elles se sont toutes
deux tenu parole avec la plus scrupuleuse intégrité.

Rien aussi ne convenait plus à l'une et à l'autre. M\textsuperscript{me}
de Maintenon se délivrait de toute inquiétude par celle-là même qui lui
en aurait donné de continuelles et d'impossibles à parer, et il ne lui
en coûtait que de la servir en toutes choses qui n'allaient point à les
renouveler, et qui d'ailleurs lui étaient parfaitement indifférentes, et
entièrement à part de tout ce qu'elle pouvait souhaiter. En même temps
elle se donnait des occasions de plaire au roi, au lieu de l'importuner
de jalousie, en se montrant amie, servant celle qui lui en aurait pu
donner, et pour qui le goût du roi, qui ne s'est jamais ralenti, s'était
tourné en bonne amitié et en considération du premier ordre.
M\textsuperscript{me} de Soubise, par cette adresse, secondait la
dévotion et les scrupules du roi, le mettait à l'aise avec elle, et
cultivait cette affection dans l'autre tour qu'elle avait pris, qui n'en
recevait que plus de force, et à l'égard de M\textsuperscript{me} de
Maintenon, elle sentait bien qu'elle lui donnait des fiches pour de
l'argent comptant qu'elle en retirait\,; que sa lutte contre elle serait
presque toujours inutile au point où en étaient les choses entre le roi
et elle, sûrement funeste enfin\,; au lieu qu'avec cette conduite elle
fortifiait son crédit direct auprès du roi de tout celui de
M\textsuperscript{me} de Maintenon, qu'autrement elle eût eue contre
elle à bannière levée. Les mêmes raisons les firent convenir encore de
ne se voir jamais sans une nécessité à laquelle rien ne pourrait
suppléer, et les billets mouchaient entre elles comme avec le roi. Telle
était la situation solide de M\textsuperscript{me} de Soubise qu'elle
avait eu l'art, en saisissant l'occasion si délicate de la dévotion du
roi et de la rupture qui y était si conséquente, de faire succéder à une
situation très hasardeuse.

La conduite domestique était menée avec la même sagesse et la même
adresse. M. de Soubise n'avait eu de jalousie de sa femme que celle
qu'il avait jugé utile de n'avoir point. Il était né pour être un
excellent intendant de maison et un très bon maître d'hôtel\,; il avait
encore la partie d'un admirable écuyer. Être à la cour et ne rien voir,
il avait trop d'esprit pour le croire praticable aux yeux du monde\,; il
avait donc pris le parti d'y aller rarement, de ne parler au roi que de
sa compagnie des gens d'armes, dont, dans les vacances de charges et
dans la manutention ordinaire, il sut tirer des trésors, de servir
longtemps et bien à la guerre, et du reste se tenir enfermé dans sa
maison à Paris, à y voir peu de monde, tout appliqué à ses affaires et à
son ménage, et laisser sa femme à la cour se mêler du grand, des grâces
et des établissements de sa famille. C'est le partage qui subsista,
entre eux toute leur vie.

M\textsuperscript{me} de Soubise, trop avisée pour ne pas sentir la
fragilité du rang que sa beauté avait conquis, n'était occupée qu'à le
consolider. Elle songea à l'appuyer de la maison de Lorraine, tout
indignée qu'elle en fût, du moment que par le mariage du prince
d'Espinoy, son neveu, elle vit jour à s'unir avec M\textsuperscript{me}
de Lislebonne et ses filles. M\textsuperscript{me} d'Espinoy, sa sœur,
qui lui était très soumise (car rien de plus impérieux dans sa famille
que cette femme qui en faisait tout l'appui), sa sœur, dis-je, qui
d'abord pour percer par le jeu s'était fort adonnée à la cour de
Monsieur, avait si bien fait la sienne au chevalier de Lorraine qu'elle
était devenue son amie intime\,; et je me souviens que, tout jeune
encore, désirant une cure vacante auprès de la Ferté qu'il nommait par
son abbaye de Saint-Père-en-Vallée, je l'eus dans l'instant par le
prince d'Espinoy avec qui j'étais continuellement alors.
M\textsuperscript{me} de Soubise, qui ne négligeait rien, avait tâché de
s'accrocher par là au chevalier de Lorraine et par lui aux Lislebonne.
Ce fût tout autre chose quand le mariage de son neveu fut fait\,: leur
esprit d'intrigue et d'ambition se rapportait\,; elles connaissaient
réciproquement leurs allures\,; elles sentirent combien elles se
pouvaient être réciproquement utiles\,; elles se lièrent peu à peu, et
bientôt l'union devint intime. Elle se resserra dans la suite par
l'alliance et la communauté d'intérêts\,; elle dura autant que leur vie,
et passa aux enfants de M\textsuperscript{me} de Soubise devenus de
grands maîtres à son école, et desquels les deux sœurs tirèrent dans les
suites l'usure de ce que d'abord elles avaient mis de leur part.

\hypertarget{note-i.-histoire-et-condamnation-de-b.-de-fargues.}{%
\chapter{NOTE I. HISTOIRE ET CONDAMNATION DE B. DE
FARGUES.}\label{note-i.-histoire-et-condamnation-de-b.-de-fargues.}}

Saint-Simon raconte comment Fargues fut arrêté dans sa maison de Courson
par les huissiers du parlement, et sur un ordre du premier président de
Lamoignon, amené à Paris, condamné à mort, et exécuté. Il ajoute que le
premier président, qui avait dirigé la procédure, s'enrichit par la
confiscation d'une partie des biens de Fargues. Ce récit, qui incrimine
la mémoire de Guillaume de Lamoignon, renferme plusieurs erreurs\,; et,
sans entrer dans une discussion approfondie, il suffira d'opposer à
Saint-Simon, ou plutôt à Lauzun, l'autorité d'un contemporain, témoin
impartial, qui avait bien connu B. de Fargues, et qui donne les détails
les plus précis sur sa condamnation et sur son supplice.

Rappelons d'abord que le fait dont il s'agit se passa au commencement de
l'année 1665, longtemps avant la naissance de Saint-Simon. Cet écrivain
cite comme unique autorité le duc de Lauzun, personnage célèbre par ses
intrigues, sa vanité, l'éclat de sa fortune et de sa chute. On voit tout
de suite quelle confiance mérite un pareil témoignage. Aussi, lorsqu'en
1781, La Place publia, dans le premier volume de ses \emph{Pièces
intéressantes et peu connues pour servir à l'histoire}, le récit de
l'arrestation et de la mort de B. de Fargues, emprunté textuellement aux
Mémoires encore inédits de Saint-Simon, la famille de Lamoignon réclama,
et produisit des pièces qui établissaient que Fargues n'avait pas été
condamné par le parlement de Paris, mais par l'intendant d'Amiens et par
d'autres commissaires délégués par le roi, et que le premier président
n'avait obtenu la terre de Courson qu'en 1668, en sa qualité de seigneur
de Bâville, dont relevait Courson.

Nous ajouterons à cette réfutation celle qui résulte du récit d'Olivier
d'Ormesson, qui écrivait son \emph{Journal} au moment même où B. de
Fargues fut arrêté et exécuté.

«\,Le dimanche 29 mars 1665, je reçus des lettres de la condamnation de
Fargues, et qu'il avait été pendu, le vendredi à cinq heures du soir, à
Abbeville. Cette fin extraordinaire m'oblige de dire que Fargues était
né de petite condition, dans Figeac en Languedoc\,; qu'ayant épousé la
sieur du sieur de La Rivière, neveu de M. de Bellebrune, il avait été
major d'Hesdin, dont M. de Bellebrune était gouverneur\,; et qu'au mois
de janvier 1658, le sieur de Bellebrune étant mort, il forma le dessein
de se rendre maître de cette place. Étant venu à Paris, il offrit à M.
de Palaiseau, gendre de M. de Bellebrune, de le servir pour lui
conserver le gouvernement, et lui demanda le nom de ses amis dans la
place, lequel M. de Palaiseau lui donna, et en même temps il offrit à M.
le comte de Moret, auquel ce gouvernement était donné, de l'argent et
son service. Mais en ayant été fort peu accueilli, il partit devant,
disant que c'était pour lui préparer toutes choses\,; et étant dans la
place, il s'en rendit le maître, ayant chassé tous les amis de M. de
Palaiseau et de M. de Moret, et ayant écrit à M. le maréchal
d'Hocquincourt pour lui livrer cette place. M. d'Hocquincourt, avec son
régiment qui était sur la frontière, s'y retira\,; et je me souviens
qu'étant en Picardie\footnote{Olivier d'Ormesson était alors intendant
  de Picardie.}, le lieutenant-colonel de ce régiment vint de la cour
m'apportant des ordres, et témoignait vouloir servir la cour contre le
maréchal, et néanmoins, sitôt qu'il eut joint son régiment, il le
débaucha et se retira à Hesdin.

«\,Lorsque, par la paix\footnote{Paix des Pyrénées (1659).}, la ville
d'Hesdin fut rendue au roi, je la reçus et y fis entrer le régiment de
Picardie. Je parlai à Fargues de toute sa conduite. Il me dit que sitôt
qu'il était entré dans Hesdin, il avait écrit en quatre endroits pour
négocier\,: à la cour, par l'entremise de Carlier, commis de M. Le
Tellier, qui y fit deux voyages, et enfin par sa femme, qui prit cette
occasion pour aller à Hesdin et se rendre auprès de son mari\,; au
maréchal d'Hocquincourt, qui ne manqua pas de se venir jeter dans
Hesdin\,; mais Fargues prit si bien ses précautions avec lui qu'il n'en
fut jamais le maître, et ne lui permit jamais ni d'y être le plus fort
ni de parler à un homme en particulier\,; {[}enfin il négocia{]} avec M.
le prince et avec les Espagnols, dont il reçut des troupes qu'il fit
camper dans le faubourg de Saint-Leu, sans que jamais il souffrît deux
officiers de ses troupes entrer ensemble dans la ville.

«\,Le roi, en avril 1658, marchant avec son armée pour faire le siège de
Dunkerque, fit semblant de vouloir assièger Hesdin, et le bruit en
courait. Il passa à la vue de cette place, croyant que sa présence
ferait quelque soulèvement dans la place\,; mais Fargues me dit que
sachant qu'il ne serait point assiégé, il jugea qu'il n'avait qu'à se
défendre d'une révolte\,; qu'il avait assemblé toute sa garnison, et
leur ayant dit que le roi venait pour les assièger, que pour lui il
était résolu de se défendre, et qu'il laissait la liberté à ceux qui
voudraient de sortir\,; que tous lui avaient juré de mourir avec lui, et
que, profitant de cette disposition, il avait mis ces troupes dans les
dehors, et était demeuré dans la place, craignant seulement un coup de
main et d'être assassiné\,; que M. le maréchal d'Hocquincourt
escarmoucha avec la cavalerie, et que depuis il n'avait songé qu'à ses
fortifications, et à maintenir l'ordre et la police dans sa place\,; que
La Rivière et lui étaient dans des chambres séparées aux deux bouts
d'une salle commune, dans laquelle il y avait un corps de garde de
pertuisaniers\,; que jamais l'un ne dormait que l'autre ne fût
éveillé\,; qu'ils n'allaient jamais en un même lieu ensemble\,; et enfin
Fargues m'ayant expliqué sa conduite, fait voir ses magasins, il me
parut homme de tête et de grand ordre, et chacun convient qu'il a
soutenu sa révolte avec beaucoup d'habileté, n'ayant ni naissance, ni
condition, ni charge, ni considération qui le distinguât pour se
soutenir.

«\,L'on dit que, durant son procès, il a dit souvent qu'il n'avait
commis qu'une seule faute, qui était de s'être laissé prendre. Il a
déclaré, après son jugement, qu'il entretenait commerce avec
Saint-Aulnays, et qu'il le pressait de se retirer en Espagne.

«\,Cette condamnation porte pour vol, péculat, faussetés et
malversations commises au pain de munition\footnote{Ce ne fut donc pas
  pour meurtre, comme le dit Saint-Simon, que Fargues fut condamné à
  mort.}, etc. Chacun a renouvelé à cette occasion les anciennes
histoires de penderie de M. de Machault, et que celui-ci ne dégénérera
point d'un nom si illustre.\,»

Ce fut en effet l'intendant de Picardie Machault qui condamna Fargues.
Il avait été nommé tout exprès pour cette exécution, dont ne voulut pas
se charger son prédécesseur Courtin. «\,L'affaire de Fargues, écrit
Olivier d'Ormesson, qui tenait ces détails de Turenne\footnote{\emph{Journal
  d'Olivier d'Ormesson}, IIe partie, fol.~87 verso.}, est l'occasion de
ce changement\footnote{Machault fut transféré de l'intendance de
  Champagne à celle de Picardie.}\,; car M. de Machault va pour le juger
souverainement, et M. Courtin l'avait refusé.\,»

Olivier d'Ormesson, après avoir rappelé que Fargues fut pendu à
Abbeville le vendredi 27 mars, continue ainsi\,: «\,L'on remarquait
qu'ayant été conduit à Hesdin, il avait été mis dans la prison avec les
mêmes fers et dans le même lieu où il avait retenu prisonnier le nommé
Philippe-Marie, qui était un officier qui avait voulu soulever la
garnison contre lui, lors de sa révolte\,; qu'un soldat qu'il avait
obligé d'être bourreau et de pendre un homme, avait été le sien et
l'avait pendu. L'on convenait aussi qu'il avait entendu la lecture de sa
condamnation avec beaucoup de fermeté\,; qu'il avait baisé trois fois la
terre remerciant Dieu\,; qu'il avait aussi baisé trois fois sa potence,
et qu'il était mort avec courage et fort chrétiennement.\,»

Il résulte de ces détails si précis, écrits au moment même des
événements, par un témoin impartial et bien informé, que le premier
président de Lamoignon n'a été pour rien dans le procès et la
condamnation de B. de Fargues.

\hypertarget{note-ii.-opposition-de-la-noblesse-aux-honneurs-accorduxe9s-uxe0-quelques-familles.}{%
\chapter{NOTE II. OPPOSITION DE LA NOBLESSE AUX HONNEURS ACCORDÉS À
QUELQUES
FAMILLES.}\label{note-ii.-opposition-de-la-noblesse-aux-honneurs-accorduxe9s-uxe0-quelques-familles.}}

Saint-Simon parle souvent dans ses Mémoires des tentatives de familles
nobles pour obtenir des privilèges particuliers, tabouret à la cour,
entrée en carrosse dans les châteaux royaux, ce qu'on appelait alors
\emph{les honneurs du Louvre}, rang de princes étrangers, etc. Ces
efforts pour s'élever au-dessus de la noblesse ordinaire provoquèrent
une très vive opposition, surtout au mois d'octobre 1649. Saint-Simon en
parle. Nous réunirons ici plusieurs passages du Journal inédit de
Dubuisson-Aubenay\footnote{Ms.~Bibl. Maz. H, 1719, in-fol.}, qui indique
avec précision tous les détails de cette petite révolution de cour.
Attaché au secrétaire d'État Duplessis-Guénégaud, Dubuisson-Aubenay est
comme le Dangeau de la Fronde\,: il retrace minutieusement les cabales
qui agitèrent la cour de 1648 à 1653\,; il parle aussi de l'opposition
qu'à la même époque les ducs et pairs firent aux prétentions de
certaines familles qui affectaient le rang de princes étrangers.

«\,Lundi 4 octobre (1649), la reine étant au cercle, le maréchal de
L'Hôpital lui a présenté le mémoire ou requête de toute la noblesse de
la cour opposante aux tabourets, de la poursuite desquels les sieurs de
Miossens\footnote{César-Phoebus d'Albret, comte de Miossens, dans la
  suite maréchal de France. Saint-Simon en parle avec détails à l'année
  1714.} et de Marsillac\footnote{François de La Rochefoucauld, alors
  prince de Marsillac, duc de La Rochefoucauld après la mort de son
  père. C'est l'auteur des \emph{Maximes}.} voulaient bien se
déporter\,; mais les princes qui la portaient ont voulu que l'affaire
allât jusqu'au bout. Enfin elle est échouée tout à fait ou remise à une
autre fois. Les comtes de Montrésort\footnote{Claude de Bourdeille,
  comte de Montrésor, un des principaux agitateurs de la Fronde.} et de
Béthune\footnote{Hippolyte de Béthune, né en 1603, mort en 1665.}, qui
n'avaient point encore parlé, y ont paru, et le premier a parlé à la
reine d'une façon de longtemps préméditée. Il y avait une lettre
circulaire aux gouverneurs et grands seigneurs de toutes les provinces,
toute prête à être signée, et envoyée de la part des opposants, qui
avait été dressée en l'assemblée chez le marquis de Sourdis.

«\,Les ducs et pairs s'assemblent chez le duc d'Uzès, et les princes
autres que du sang chez M. de Chevreuse.

«\,Mardi matin, 5 octobre, encore assemblée de la noblesse opposante,
que l'on appelle \emph{anti-tabouretiers}, chez le marquis de Sourdis,
lui absent, et son fils, le marquis d'Alluye, présent.

«\,Jeudi 7, la noblesse opposante aux tabourets s'assemble encore chez
le marquis d'Alluye, en l'hôtel de Sourdis.

«\,L'opposition des ducs et pairs contre la principauté de la maison
Bouillon La Tour continue, et la plainte des maréchaux de France contre
le vicomte de Turenne, de ce qu'il a fait ôter les bâtons de maréchal de
France de son carrosse\footnote{Saint-Simon revient souvent sur les
  prétentions de la maison de Bouillon. Voy., principalement t. V,
  chap.~XVII.}.\,»

Après avoir dit que les assemblées de la noblesse continuèrent le
vendredi 8 et le samedi 9, sans entrer dans aucun détail,
Dubuisson-Aubenay parle avec plus d'étendue de celle qui se tint le 11
octobre\,:

«\,Il y a eu grand bruit. Le marquis d'Alluye, fils du marquis de
Sourdis d'Escoubleau, absent, a voulu faire sortir de chez lui les
Besançon\footnote{Les seigneurs, dont il s'agit ici, étaient de la
  famille du Plessis-Besançon.}, disant qu'ils n'étaient pas
gentilshommes. Ceux-ci ont menacé l'autre de coups de bâton. Le sieur
d'Amboise, ci-devant gouverneur de Trin\footnote{Trino, petite ville de
  Piémont, au N. O. de Casal.} en Piémont, puis de Lagny-sur-Marne
durant le siège de Paris, a été admonesté de s'en retirer, quoiqu'il ait
eu pour père un maître des requêtes, et qu'il ait les armes de
l'ancienne maison d'Amboise, qui est de six pals\footnote{Bandes
  perpendiculaires sur l'écu.} d'or et de gueules\,; ce qu'il a fait
doucement. Le prince de Condé avait prié du commencement quelques-uns de
ses amis de n'y pas aller\,; à la fin il les y a envoyés lui-même. Le
bruit des Besançon fut dès samedi.

«\,Dimanche après midi l'assemblée fut chez le maréchal de L'Hôpital, et
aussi ce jourd'hui lundi depuis huit heures jusques après dix, que fut
apporté le brevet de la reine, par lequel elle abolit tous tabourets,
entrées au Louvre et autres privilèges, concédés à qui que ce soit
contre les formes ordinaires depuis l'an 1643 et durant la régence. On a
voulu délibérer si l'on se contenterait de ce brevet, et s'il ne fallait
pas une déclaration du roi enregistrée au parlement, et les uns étaient
d'un avis, les autres d'un autre\,; mais le maréchal d'Estrées, l'un des
présidents (car les maréchaux de France\footnote{«\,Auparavant qu'il y
  en eût, c'étaient les chevaliers des ordres qui présidaient, entre
  autres le comte d'Orval\,; et le vieux marquis de La Vieuville, aussi
  chevalier des ordres, s'étant relâché à laisser passer le comte de
  Montrésor devant lui sous protestation que cela ne préjudicierait au
  rang, il en a été repris par le comte d'Orval\,; mais lesdits
  chevaliers des ordres du roi, comme ils précèdent tous gentilshommes
  même gouverneurs de province, aussi cèdent-ils aux officiers de la
  couronne, comme sont les maréchaux de France.\,» (\emph{Note de
  Dubuisson-Aubenay}.)} y président, et les sieurs de Maulevrier, Brèves
et de Villarceau servent de greffiers), ayant dit que l'heure était
passée, est sorti et beaucoup de noblesse avec lui. Les autres sont
demeurés en colère, disant qu'ils voulaient délibérer et qu'ils
n'avaient que faire de ceux qui s'en allaient de la sorte. Mais le comte
de Montrésor les a apaisés disant que jusqu'alors ils n'avaient rien
fait que de bien\,; qu'ils ne devaient donc pas finir par désordre et
précipitation\,; que l'on attendît à demain que l'assemblée fût légitime
et complète pour achever leur délibération\,; ce qui a été fait, et on
nomma douze commissaires d'entre eux pour examiner l'affaire.

«\,Mardi 12, l'assemblée de la noblesse continue pour la dernière fois.
Le brevet de révocation des brevets des tabourets et entrées en carrosse
dans le logis du roi, donnés à la comtesse de Fleix\footnote{La comtesse
  de Fleix était fille de la marquise de Senecey, gouvernante de Louis
  XIV. Saint-Simon parle (t. Ier, p.~70 et 350 de cette édition) de
  Gaston de Foix, fils de la comtesse de Fleix.} de la part de la reine
comme à une veuve de la maison de Foix, à la demoiselle de
Brantes-Luxembourg\footnote{Marie-Louise-Claire-Antoinette, fille de
  Léon d'Albert, sieur de Brantes et duc de Piney-Luxembourg.}, et aussi
à M. de Bouillon comme prince étranger, a été reçu. On a voulu faire
passer que dorénavant toutes les concessions n'auraient d'effet qu'après
l'enregistrement des brevets du roi, même majeur, au parlement. La
pluralité des voix au contraire l'a emporté. L'assemblée ainsi s'est
rompue, et l'archevêque d'Embrun, jadis abbé de La Feuillade, y est venu
la haranguer de la part de son corps. Celui de la noblesse ira, dit-on,
les remercier, et remerciera aussi tant les ducs et pairs que les
princes qui ont épaulé ladite noblesse. Là-dessus le comte de Miossens,
sous-lieutenant des gens d'armes du roi, demanda qu'il fût fait un
décret que dorénavant en France on ne reconnût plus aucuns princes que
ceux du sang, et que les autres fussent réduits aux purs rangs de la
noblesse.

«\,Mercredi 13, se tient encore assemblée chez le maréchal de L'Hôpital
par la noblesse, où elle a résolu la députation vers la reine et M. le
cardinal pour les remercier du brevet de révocation ci-dessus, et donner
part aux ducs et pairs assemblés chez le duc d'Uzès, et aux princes
étrangers chez le duc de Chevreuse, de la conclusion de leur assemblée
et de tout ce qui s'y est passé.

«\,Le comte de Miossens est aussi allé remercier la reine de ce qu'elle
lui promet toit qu'il ne se ferait aucune concession de cette nature
durant la régence qu'il n'y eût part\,; et qu'elle lui donnait cependant
et dès à présent sa charge de maître de la garde-robe de M. le duc
d'Anjou, de laquelle il a pris possession à l'heure même près de ce
petit prince, et en outre douze mille livres d'appointements.\,»

\hypertarget{note-iii.-uxe9vocations-enregistrement-droit-de-remontrances.}{%
\chapter{NOTE III. ÉVOCATIONS\,; ENREGISTREMENT\,; DROIT DE
REMONTRANCES.}\label{note-iii.-uxe9vocations-enregistrement-droit-de-remontrances.}}

Saint-Simon parle souvent, et notamment page 422 de ce volume, des
évocations, du droit d'enregistrement et de remontrances. Il ne sera pas
inutile de préciser pour le lecteur moderne le sens de ces expressions.

Les évocations étaient des actes de l'autorité supérieure qui enlevait
la connaissance d'une affaire aux juges naturels pour l'attribuer à un
autre tribunal. Tantôt c'était le souverain, tantôt c'étaient les
tribunaux supérieurs qui évoquaient le jugement d'un procès. Les
évocations étaient souvent un moyen de favoriser un personnage en le
renvoyant devant un tribunal où il avait plus d'influence. Aussi la
célèbre ordonnance de Moulins, rendue en 1566, déclare-t-elle qu'une
évocation ne pourrait avoir lieu qu'en vertu d'une ordonnance du roi
contresignée par les quatre secrétaires d'État. On autorisait les
parlements à faire des remontrances pour s'opposer provisoirement à
l'exécution de l'ordonnance d'évocation, et, provisoirement, la partie
en faveur de laquelle avait été prononcée l'évocation devait se
constituer prisonnière.

Le droit d'enregistrement est un exemple frappant des abus qui se
glissent à la faveur d'un mot ou d'un usage, et qui peu à peu deviennent
lois constitutives d'un État. De la coutume de transcrire sur des
registres les actes royaux est venue la prétention du parlement
d'exercer sur ces mêmes actes un contrôle qui se traduisait quelquefois
par le refus de l'enregistrement. Il fallait alors que le roi vînt en
personne au parlement pour forcer les magistrats de transcrire la loi
sur leurs registres. Il est nécessaire de rappeler les origines et les
vicissitudes de cette prétention des parlements.

Avant le règne de saint Louis, il n'est pas question de registres sur
lesquels on inscrivît les ordonnances des rois ou les arrêts des
tribunaux. On les écrivait sur des feuilles de parchemin que l'on
roulait et que l'on déposait dans le trésor des chartes. Pour constater
l'authenticité d'un acte, on ne disait pas qu'il avait été enregistré ou
inscrit sur les registres du parlement, mais qu'il avait été placé dans
le dépôt des actes publics (\emph{depositus inter acta publica}).
Étienne Boileau, prévôt de Paris sous le règne de saint Louis, fut le
premier qui fit transcrire sur des registres les actes de sa
juridiction. Le parlement de Paris fit faire, vers le même temps, un
recueil de ses arrêts, connu sous le nom d'Ohm, et qui a été publié dans
le recueil des \emph{Documents inédits relatifs à l'histoire de France}.
Au commencement du XIVe siècle, le même corps fit dresser un registre
des ordonnances royales qui devaient servir de règle à ses jugements.
L'ordonnance, après avoir été lue en présence de la cour, était
transcrite sur les registres du parlement. Dès 1336, on trouve au bas
d'une ordonnance de Philippe de Valois la formule suivante\,: «\,Lu par
la chambre et enregistré par la cour de parlement dans le livre des
ordonnances royales.\,» (\emph{Lecta per cameram, registrata per curiam
parliamenti in libro ordinationum regiarum}.)

De cet usage de la transcription sur ses registres, le parlement passa,
au commencement du XVe siècle, au droit de soumettre à son contrôle et
même de rejeter une ordonnance royale. Pendant les troubles du règne de
Charles VI, le parlement, devenu permanent, prétendit qu'il avait le
droit de refuser l'enregistrement d'une ordonnance royale\,; il la
frappait ainsi de nullité et n'en tenait aucun compte dans ses arrêts.
Même sous Louis XI, en 1462, le parlement de Paris refusa d'enregistrer
un don fait par le roi au duc de Tancarville\,; il fallut un ordre
exprès de Louis XI pour l'y contraindre. Dans la suite, toutes les fois
que la royauté rencontra dans le parlement une résistance de cette
nature, elle en triompha par une ordonnance spéciale, et alors, en
mentionnant l'enregistrement, on ajoutait la formule\,: \emph{Du très
exprès commandement du roi}. Souvent même, pour vaincre l'opposition des
parlements, les rois allèrent y tenir des lits de justice, où ils
faisaient enregistrer les ordonnances en leur présence.

Le droit de remontrances était étroitement lié à celui d'enregistrement
et datait du même temps. Avant de céder aux ordres du roi, le parlement
lui adressait de très humbles remontrances, pour lui exposer les motifs
qui l'avaient engagé à surseoir à l'enregistrement de tel ou tel édit.
L'ordonnance de Moulins, tout en reconnaissant au parlement le droit de
présenter des remontrances, déclara qu'elles ne pourraient surseoir à
l'exécution des édits. Même réduit à ces limites, ce privilège des
parlements parut encore redoutable à Louis XIV. Par sa déclaration du 24
février 1673, il régla la forme dans laquelle devaient être enregistrés
les édits et lettres patentes émanés de l'autorité royale. Le parlement
ne conservait le droit de remontrances que pour les actes qui
concernaient les particuliers. Ainsi jusqu'à la fin du règne de Louis
XIV, le droit de remontrances sur les matières politiques resta
suspendu\,; mais la déclaration du 15 septembre 1715 la rendit aux
parlements, et les lettres patentes du 26 août 1718 en réglèrent
l'usage.

\end{document}
